\environment envCfgThesisDescriptions
\environment envCfgThesisSynonyms

\startenvironment envCfgThesisCommands

% quote such as a book with author and year parameters, without specific page reference
\define[2]\aQuote{\color[magenta:8]{\tt{\small{(#1, #2)}}}}

% quote such as a book with author and year parameters, with specific page reference
\define[3]\aQuoteB{\color[magenta:8]{\tt{\small{(#1, #2, p.#3 )}}}}

% website quote, author and year
\define[2]\aQuoteW{\color[magenta:8]{\tt{\small{(#1, #2, web)}}}}

% source quote, just the year
\define[1]\aQuoteY{\color[magenta:8]{\tt{\small{(#1)}}}}

% personal information quotes, just year
\define[1]\aQuoteP{\color[magenta:8]{\tt{\small{(own Source, #1)}}}}

% the quote of authors within the text flow
\define[1]\aQuoteInTextA{\color[magenta:8]{\tt{\small{#1}}}}

% the quote of a title within the text flow
\define[1]\aQuoteInTextT{\color[magenta:8]{\tt{\small{\quote{#1}}}}}

\define[1]\cited{\quote{\em#1}}
\define[1]\bracket{ [{\ss{\small#1}}]}

\define[2]\abbrNew{\color[magenta:8]{\em#1}\bracket{#2}\abbreviation{#2}{#1}}
\define[2]\abbrNewInVi{\abbreviation{#2}{#1}}

\define[1]\abbr{\color[magenta:8]{\em#1}}	%here we could probably link to the full abbreviation?
\define[1]\abbrFull{\color[magenta:8]{\em\infull{#1}}}

\define[1]\aLocation{\from[#1]\footnote[#1]{location: \goto{\url[#1]}[url(#1)]}}

% just display the link in the text and generate a footnote, hence link must have been registered already
\define[1]\aLink{\from[#1]\footnote[#1]{website: \goto{\url[#1]}[url(#1)]}}

% register a new link, display it in the text and generate a footnote
\define[1]\aLinkNew{\useURL[#1][#1][][#1]\from[#1]\footnote[#1]{website: \goto{\url[#1]}[url(#1)]}}

% register a new link, display a generic "link" surrounded by brackets in the text and generate a footnote
\define[1]\aLinkNewG{\useURL[#1][#1][][link]\small\bracket{\from[#1]}\footnote[#1]{website: \goto{\url[#1]}[url(#1)]}}

% display a generic "link" surrounded by brackets in the text and generate a footnote, thus link must have been registered already
\define[1]\aLinkG{\small\bracket{\from[#1]}\footnote[#1]{website: \goto{\url[#1]}[url(#1)]}}

% just generate a foornote. link ist not displayed in text but only in footnote, hence it must have been registered already!
\def\doLinkF[#1]#2%
{%
	\iffirstargument
		\goto{#1}[url(#2)]
		\footnote[#1]{website: \goto{\url[#2]}[url(#2)]}
	\else
		#1
		\footnote[#1]{website: \goto{\url[#1]}[url(#1)]}
	\fi
}
\define\aLinkF{\dosingleempty\doLinkF}


% register a link url and generate a footnote
\define[1]\aLinkNewF{\useURL[#1][#1][][#1]\footnote[#1]{website: \goto{\url[#1]}[url(#1)]}}

\define[1]\aLinkE{\inframed[frame=off, background=color, backgroundcolor=green:4]{\bf#1 }}


\define[1]\aLocationE{\inframed[frame=off, background=color, backgroundcolor=blue:4]{\bf#1 }}

%\define[1]\toTranslate{\inframed[frame=off, background=color, backgroundcolor=red:4]{\bf#1 }}

\def\toTranslate{\dosingleempty\doTranslate}

\def\doTranslate[#1]#2%
{%
	\iffirstargument
		\TransLa{{\ss#2}\bracket{#1}}
		\color[magenta:8]{\em#2}\bracket{#1}
	\else
		\coupledTransLa{{\ss#1} <> translation missing}
		\color[magenta:8]{\em#1}\bracket{translation missing}
	\fi 
}

\define[1]\toMark{\startAMark{\bf#1}\stopAMark}
\define[1]\aKeyword{{\bf\em #1}}
\define\refMissing{\color[red:8]{\tt{\small{REF MISSING }}}}
\define[1]\refMissingSrc{\color[red:8]{\tt{\small{REF MISSING(#1)}}}}

\define[1]\toLink{\inframed[frame=off, background=color, backgroundcolor=yellow:2]{\bf#1 }}
\define[1]\isCorrect{\inframed[frame=off, background=color, backgroundcolor=magenta:4]{\bf #1(?) }}

% \define[]\audio
% \define[]\video
% \define[]\media

\define[]\spaceHalf{\blank[0.5cm]}

\define[1]\roundBoxedMax{%
\framed
[
	frame=on, 
	corner=0, 
	width=local,
	align={width, verytolerant, nothyphenated},
	frameoffset=1pt,
   	framecolor=magenta:7,
	background=color,
   	backgroundcolor=black, 
	backgroundoffset=1pt, 
   	foreground=color,
	foregroundcolor=white,
	rulethickness=3pt,
	%offset=4pt,
	framedepth=2pt,
	backgrounddepth=2pt,
]
{ #1 }
}

\def\textBoxedRoundMax{\dosingleempty\doTextBoxedRoundMax}

\def\doTextBoxedRoundMax[#1]#2%
{%
	\iffirstargument
		\roundBoxedMax{#2} \blank[#1cm]
	\else
		\roundBoxedMax{#2}
	\fi 
}

\stopenvironment