\startcomponent component_NarratingInquiries_SaoPauloDiaries_DayAndNight
\product product_Thesis
\project project_MasterThesis

% definitions and macros
\environment envThesisAllEnvironments

\define[]\subsectionSaoPauloDiariesDayAndNight{\subsection[narratinginquiries_saopaulodiaries_dayandnight]{Day//Night//Streets}}

\define[]\subjectDayOne{\subject{Day One}}
\define[]\subjectNightOne{\subject{The Night}}
\define[]\subjectDayTwo{\subject{Day Two}}

\starttext

\startmode[tocLayout]
\subsectionSaoPauloDiariesDayAndNight
The section is a narrative of two days and a night experience on the streets of São Paulo.
\subjectDayOne
Introduction to several spots in the centre.
\subjectNightOne
Looking for a place to sleep.
\subjectDayTwo
Visiting more places, a emergency case and the final visit of CONDEPE.
\stopmode

\startmode[draft]
\subsectionSaoPauloDiariesDayAndNight
Today, \Ma and I are going to realize I our two days journey on the streets. We meet at \Ay, its early in the afternoon. \Ma intention is to introduce me to the lived space of the people in street situation, that I experience how to organize daily life on the streets, how people support each other and how people are adversely affected by public policies, institutions and agents.

\spaceHalf

\inright{heavy rain and its effect on the streets}
This week São Paulo is going to be \aLinkMediaNew[hit by a cold weather front]{http://bit.ly/mQxbEJ}\toMark{13/07/2010}. It is already cold today, probably 15 to 20 degrees during the day but at night, temperature will drop to 10 to 12 degrees. It will start to rain as well.

Some days later heavy rain is basically stopping the movement in the city for one week. \aLinkMediaNew[Newspapers report impassable roads]{http://bit.ly/nrKPY7} \toMark{14/07/2010} caused by floods throughout the city.

\startPersonal
The cold weather has severe effects on the movement of the people on the streets. Those that are in streets situation will leave the lower places of the centre in order ascend the steep hills up to \aLocationNew{Avenida Paulista}{http://osm.org/go/M@y3Wjuw}. Looking at the city topographically, \Se or \Republica are located in the valleys of the surrounding hills. Once heavy rain hits the city, those lower places are much more prone to floods then the upper areas on the hills, where \AvPaulista is located for instance. Thus heavy rains forces people to leave the lower areas and ascend upwards in order to avoid being flooded away and in order to find a more or less secure and dry place to sleep. 

Even though those urban floods are not comparable to river floods, during the period of heavy rain small rivers pop up everywhere on the side walks and the streets because the city is sealed by constructions and water is searching its way wherever possible, accumulating in streams that make it impossible to sleep on the ground. 
\stopPersonal

\subjectDayOne

But for now, we start our journey downwards, descending \aLocationNew{Rua  Carmelias}{http://osm.org/go/M@ziKy3jL--} until it hits \aLocationNew{Rua Frederico Alvarenga}{http://osm.org/go/M@ziK8Ucu--}. There, at the corner we meet one of \Mas friends. He is sitting there at the corner most of the day, almost everyday. Right now he is sleepy and does not talk much. We do not stay long, \Ma is just asking how his friend is doing and then we continue. \Ma says that he knows him since he hits the streets, years back, and since then the guy always stayed at that corner every day. 

We are heading towards the \Tendas. They are not far away, just two streets, located below \aLocationNew{Viaduto 25 de Marco}{http://osm.org/go/M@ziLozVF--} where \aLocationNew{Avenida Rangel Pestana}{http://osm.org/go/M@ziLozVF--} traverse a branch of \aLocationNew{Rio Tietê}{http://osm.org/go/M@zie@K}\aLinkNewF[Rio Tietê at Wikipedia]{http://bit.ly/pvlcQg} that crosses through the northern part of the city's central area.

\spaceHalf

\inright{loosing means to generate income}
We arrive at the junction where \aLocationNew{Rua Dom Pedro II and Avenida Rangel Pestana}{http://osm.org/go/M@ziK3nYF--} are crossing and meet another friend of \Ma. He tells us that the \abbr{GCM} has just taken all his possessions, his bag and all the goods he was selling on the streets because he could not show a permission as street vendor when \abbr{GCM} has been checking him. Thus he has just lost all means to generate income.

\spaceHalf

\inright{entering the Ţendas}
We cross the street and entering the \toTranslate[Tendas]{tents}. \Tendas is the name of an area converted to a public service that receives people in street situation. The \toTranslate[Tenda de convivência]{Tenth of gathering} at Parque Dom Pedro is a service provided by the Secretaria Municipal de Assistência Social of the Prefeitura de São Paulo \aLinkNewF{http://bit.ly/qImvjC}.

The \Tendas at Parge Dom Pedro consists basically of a huge tent, open to one side, packed with people sitting on banks and chairs, in front of a TV. The backmost part is occupied for showers and toilets. Outside the big tent, groups of people are sitting everywhere, some of them talking, some just quite. The atmosphere is rather depressed. We talked to a social worker who told us that they are also offering workshops from time to time, such as \toMark{nochmal in meine aufzeichnungen schauen}. The \Tendas are open during the day from 8h in the morning until 9h in the evening and closed for the night.  

\startARemark{woher kommen die leute nochmal? alberque oder strasse oder beides?}\stopARemark
\startARemark{mehr über die tendas in meinen aufzeichnungen}\stopARemark

\startPersonal
The \Tendas are located at a spot of the centre that is usually not much frequented, except by cars that rush over the three lane avenue that is passing above them in order to traverse the river. It is noisy and the air is polluted by massive traffic that is circulating all day long from the centre to the eastern regions of São Paulo and vice versa. The traversals at \aLocationNew{Parque Dom Pedro}{http://osm.org/go/M@ziLozVF--}, thus above the \Tendas, eventually connect the centre with Brás and the \aLocationNew{Radial Leste}{http://osm.org/go/M@zoEwXb-} highway that head towards the eastern margins of the city. 

Especially the notion of artesian workshops becomes contradictory when we thought about the friend we met just before, who is actually living from vending artesian goods on the streets and who has been expelled by the police for doing so. 
\stopPersonal

According to \Ma, the \aKeyword[tendas]{streets+places+institutional+tendas} are just another way of keeping people in their miserable situation because they do not provide a  single proposal to sustainably improve the situation of the people. For \Ma they are solely a justification for public institutions, civil and police agents, to banish people from the central commercial areas and send them here because the city wants clean the commercial centre of all unwanted subjects. What awaits them at the \Tendas is the just a TV, some workshops, food and sanitation. \Ma said that the chemical sanitation is dirty like hell and that he would not take a shower here anyway.

\startDialog
\tell{\Ma}{Look what people can do here. Nothing. They just sit in front of the TV all day long and wait until the place closes its doors at night. Tomorrow they will be here again but how can they improve their situation then?}
\stopDialog

\spaceHalf

\inright{heading towards Sé}
We leave the the \Tendas after a while and head towards \Se. \Ma says that a research has been released at the end of last year \bracket{2009} which determines the number of people in street situation in São Paulo between 13000 und 14000. He says that this number is way to low. According to the estimation of the very people on the streets, the number of could be between 20000 and 25000. 

\spaceHalf

\inright{official census about people in street situation}
The study conducted by \abbrNew{Fundação Instituto de Pesquisas Econômicas}{FIPE} \toTranslateT{Fundação Instituto de Pesquisas Econômicas}{Institute of Economic Studies Trust} says that 13666 people in São Paulo are considered being in street situation, half of them staying in \Albergues \bracket{7079}, the other half staying on the streets \bracket{6587} \aQuoteB{Schor and da Costa Vieira}{2009}{4} \footnote{ \abbr{FIPE} is private not-for-profit institute that supports research of the Faculty of Economy, Administration and Accountancy of the University of São Paulo (FEA-USP) (\aLinkNewNoF[O que é a Fipe]{http://bit.ly/rlQyOQ})}.

\startARemark{\Ma complains that those numbers are wrong}
because....
\stopRemark

\startDialog
\tell{\Ma}{You know what? Albergues are like human deposits. You have to wake up at 5 o'clock in the morning, you get a coffee and then they kick you out. You can only come back at night, punctual, depending on the place you are. Sometimes at 9 p.m sometimes later. The only thing you do there is sleeping. You cannot leave your stuff there because others will steal it and they don't allow you to keep your stuff there. There is nothing where you can deposit your stuff. And can you imagine that people stealing from others in the same situation?}
\tell{\Ma}{Sometime pereople make noise all night long, how can you sleep then? If you don't obey the rules they kick you out immediately, its like prison in there, but imagine people that lived on the streets for 10 years or so, how can one force them to obey those rules? Their life on the streets changed their behaviour, you cannot force them to follow rules that did not exist on the streets. I preferred to stay on the streets instead of being \quote{home} punctual, leaving my ID there and always being afraid that someone will come to take my stuff away. Its a human deposit. People are not empowered there, they are just taken from the streets for the night but left in their miserable situation. During the day you have to hit the streets anyway. Then people are just waiting to get back at night, that's all what happens when you go to the Albergue. It doesn't change anything.}
\tell{\Ma}{And how can it? Albergues are packed up. Mostly men are received there, a few of them are for mixed gender or families. But men and women are then always separated strictly. An what does the Prefeitura? They even close Albergues, alone last year \toMark{X} of them have been closed, all together about \toMark{1200} places, \toMark{X at Y, X at Y and X at Y}. One of them, Cirineu is located to the opposite of the Camerá Municipal. The politicians did not want to see the miserable reality in front of their faces so they just closed it. When new Albergues are inaugurated they are far away from the centre. People have to take public transport there, \toMark{for example to} and once they are there they won't come back to the centre because there is no work for them, their social network is not function there, as is does here in the centre and if they can't earn money they can't afford public transport back here. Its a convenient way for the city to expel people from the central areas and clean them up as positive side effect.}
\tell{\Ma}{Let's go there to Cirineu there at the Camerá Municipal, the one they closed some month ago, eliminating \toMark{X places}}
\stopDialog

%\startPressCoverage{Press is also relating about closed Albergues in central areas.}
%\tellPress{}{}
%\tellPress{}{}
%\tellPress{}{}
%\stopPressCoverage

So we are heading to the \aKeyword[Cirineu]{streets+places+institutional+albergue+Cirineu}, the \Albergue opposite to the \toTranslate[Camara Municipal]{Municipal Chamber} at the corner of \aLocationNew{Viaduto Jaceguai and Rua Santo Amaro}{http://osm.org/go/M@ziIyW@E--}. 

We flow through the centre meeting people at every corner. The centre is the lived space that \Ma knows inside out. Literally every corner, every small street, every blind, wall, canopy, roof and loophole that protects from rain, wind and that protects one self from others. \Ma says that he never slept alone, they always stayed in small groups, with his \aKeyword[familía]{streets+actors+streets+família} that meant protection and reliance. 

Here a street where \Ma slept with others for three years under a shop's canopy, every night until the place was secured with a lattice so that no one could sleep there any more once the shop closed its doors in the evening and the lattice shielded the canopy. 

We already crossed \aLocationNew{Viaduto do Chá}{http://osm.org/go/M@ziJvMDT--} and enter \aLocationNew{Rua Barão de Itapetininga}{http://osm.org/go/M@ziJ5YHu--} where we meet another friend of \Ma with his wife. This friend is not in street situation but works as a social worker for the public service whose agenda are the people in street situation. He tells us that he just has quit his job at the service because it was no longer possible for him to support what has been required from him in terms of forcing people away from the streets. \Ma and he are talking a long time because they have not seen each other for month. 

Then we leave them and turn to \aLocationNew{Anhangabaú}{http://osm.org/go/M@ziJpyD2--} and from there towards \aKeyword[Cirineu]{streets+places+institutional+albergue+Cirineu}, floating through the overground pedestrian way connecting the centre with the bus terminal \aLocationNew{Terminal Bandeira}{http://osm.org/go/M@ziI9Gin--}. Here the same picture, everywhere friends and known faces \Ma is acquainted with, sitting in between the constant stream of people heading to the metro stations, the bus terminals or elsewhere. 

From there we are climbing up the road to \aLocationNew{Viaduto Jaceguai }{http://osm.org/go/M@ziIyW@E--}. Its getting dark, must be around 7 p.m. We just take a look at the place, there is not much to see, just the history is important. We continue, \Ma would like to show me one of those \Albergues that receives families and mixed gender. Its located at \aLocationNew{Rua São Domingos}{http://osm.org/go/M@ziIlROe--}. At the entrance just an old woman waiting to get inside. We are staying here just a short while. 

It is already dark now and the area is shady so we decide to return to the centre, to  \aLocationNew{Praca Ouvidor Pacheco e Silva}{http://osm.org/go/M@ziI@zJh--} for a short break and for organizing something to eat. Once we arrived there we are already 4 or 5 hours floating through the area.

\startDialog
\tell{\Ma}{Do you see how the city is constructed. Look here, do you see this walls? They shield the ventilators that are blowing warm air from the inside of the building. The shop just constructed those walls because people were sleeping on the lattices above the ventilation system. There has always been a warm stream of air. Especially with temperatures like today you would have seen the lattices full of people because its the only warm place here outside. Now they put those walls there to prevent that have an incentive to come and stay here. They think its not good for their business, but look, there so many people still sitting here and freezing today.}
\tell{\Ma}{And have you seen those shops on the opposite of the Faculty of Rights? They installed those massive lattices there to prevent people to sleep below their canopies. But this is public space, they privatized public space by making it inaccessible and try to draw us out from here by preventing us to stay in areas that protects. And the Prefeitura is happy because now this private shop owners are doing the dirty business of installing stuff that is supposed to make our life even more difficult. The Prefeitura does not even need to do this because the people are doing it already. There at Sé you have already seen that the cathedral is entirely fenced. They fence a church that is supposed help everyone. Its a property of the church but this church is as worse as the Prefeitura. They only care about the tourists coming there to shoot photos but what is with those that used to sleep there, at those walls?}
\tell{\Ma}{You see, the centre used to support us in our daily struggle but now there is not even water in the dwells at Sé or República and only one public toilet in the whole area. We will go there tomorrow morning after we woke up. The city is ripped off every infrastructure that supported us. Have you seen benches somewhere? If there are benches they are too narrow to sleep on them. But most of the public spaces do not have any bench, and where are we supposed to wash our self or go to toilet. The public opinion is that we piss everywhere but what would you do without toilet living on the streets. And even to the bars you cannot go because you can use their toilets only if you buy something. Every citizens is struck by this situation but we are the guilty ones.}
\stopDialog

\Ma says we have to wait until 9 p.m. because then \aKeyword[citizens]{streets+actors+groups+citizens} and \aKeyword[grupos de sopa]{streets+actors+groups+grupos de sopa}\toTranslateT{Grupos de Sopa}{Soup Kitchen} will arrive and distribute food. Till then we sit here on a low wall, around us other people, mostly men, waiting for the arrival of food. Its cold, we are freezing.

\Ma says that when you life on the street you need to learn from where you can get food for free, without money. Water is another problem because there are no public wells left in the centre, thus those that distribute food mostly also bring water. The distribution takes place at different spots in the central area, often in the evening. 

There are other places such as courtyard of an \aKeyword[abandoned villa]{streets+places+abandoned+villa} alongside the \Minocao at \aLocationNew{Rua Apa}{http://osm.org/go/M@ziHwnfo--}\footnote{\Minocao is the name of the elevated highway Presidente Costa e Silva and covers a large extend of \aLocationNew{Avenida São João}{http://osm.org/go/M@ziNBCw}, leading from the centre to the West, to \locMissingSrc{Barra Funda}. Some impressions of the \Minocao can be found at \aLinkNewNoF[Minhocão, 24 horas]{http://blogs.estadao.com.br/olhar-sobre-o-mundo/minhocao-24-horas/}. The Minhocão is open for traffic during the day but is closed at night and on Sundays all day long. There is much to say about that street, but this has to wait till another time.}. \Ma says that the people in street situation below the Minhocão are separated from those of the centre but also from those at Luz, they do not have much contact. We decide to go to Luz later on but that we spare \Minocao for tonight. 

\startPersonal
I perceived the \Minocao as even rougher then the centre. It is hard for me to say, its probably the heavy car traffic and those traffic jams all day long, the pollution, the noise. Even it is not that long from the centre, its a different world already. I had never contact with people living there and those I met in the centre never went there as well.
\stopPersonal

Around 9 p.m. the first van is entering the pedestrian area around \aLocationNew{Praca Ouvidor Pacheco e Silva}{http://osm.org/go/M@ziI@zJh--}. Its a van of a \aKeyword[grupos de sopa]{streets+actors+groups+grupos de sopa}. They stop and one guy on top of the bed is distributing half litre plastic bottles of water to everyone that is approaching the van. \Ma and I are taking two because we didn't drink since the afternoon. We also get some bred and returning to our place. \Ma says 

\startDialog
\tell{\Ma}{Look how fast everything goes. They are not allowed any more to do this here.}
\stopDialog

After some minutes the van has nothing left and is leaving the area. On the opposite of the place, another car is stopping, this time a father with his two kids. He opens the trunk of his car and takes out boxes with cups of warm food. The movement of the people is starting again. The family is passing the plastic cups with warm noodles, sauce and some meat to everyone. \Ma says that this family is doing this every week, always on the same day, at the same time. We talk with them shortly before they leave and they say that they had prepared around 80 portions at home for tonight. As fast as they arrived they leave. Their portions are distributed fast as well and within a couple of minutes all cups are gone. \Ma says that those that do not go to the \Tendas or the \Refeitorios have now probably got their first and last meal of the day.

\startPersonal
I feel a bit strange not eating the meat in my sauce but I leave it for \Ma.  I would never complain about it because we were both hungry and no one would throw away stuff the other could eat. It is ridiculous to prohibit the self-organisation of food distribution. That people freely organise the distribution implies that they care about the situation in the city and that their practice is pragmatic and self-determined. They simply come to the place where the people are. Nothing more and nothing less.
\stopPersonal

\startARemark{some days later I found several newspaper articles about the same topic.} They are arguing from the standpoint of the owners of commercial businesses in the central areas and from the standpoint of the cities public agents.

übersicht hier hin und nochmal auf die tendas eingehen.

\stopARemark

Seeing the car leaving we are also leaving the place, saying goodbye to everyone around us and heading towards \Republica.

\subjectNightOne
\reference[narratinginquiries_saopaulodiaries_dayandnight_lookingforaplacetosleep]{}Looking for a place to sleep.
\subjectDayTwo
Visiting more places, a emergency case and the final visit of CONDEPE.

\addReference
{
Schor, S.M. \& da Costa Vieira, M.A., 2009. Principais Resultados do Censo da População em Situação de Rua da Cidade de São Paulo. Available at: \goto{\hyphenatedurl{http://www.prefeitura.sp.gov.br/cidade/secretarias/upload/chamadas/2\_1275339508.pdf}} [url(http://www.prefeitura.sp.gov.br/cidade/secretarias/upload/chamadas/2_1275339508.pdf)].
}

%---------------------------------------------------
% gets only displayed in unfinshed mode

\showImperfection

%---------------------------------------------------
\stopmode

\stoptext

\stopcomponent