\startcomponent component_Methodology_WhatShouldIDo_TheoryVsPraxis
\product product_Thesis
\project project_MasterThesis

% definitions and macros
\environment envThesisAllEnvironments
\environment envCfgThesisImages 

\define[]\subsectionPraxis {\subsection[methodology_whatshouldido_theoryvspraxis]{Theory//Versus//Practice}}

\starttext

\startmode[tocLayout]

\subsectionPraxis

Describes the used methods in praxis.

\stopmode

\startmode[draft]

\startKeywords
deductive, inductive, empirical, theoretical, qualitative, quantitative, participatory, tyranny, hierarchies, authority, emancipation, self-determination
\stopKeywords

\startAReminder[wie kommen theorie und praxis zusammen] ich muss noch erklären wie die theorie und die praxis zusammenkommen. in der praxis geht es um forderungen aufgrund von verschiedenen situation, in der theorie geht es um eine bertachtung der forderungen aus einer meta perspektive. das zusammenkommen ist dann die erweiterung der meta ebene und/oder die erweiterung der praktischen ebene und/oder die zukünfige nutzung des erzeugten wissens.
\stopAReminder

\startAReminder[wissenerzeugung und inhalte erzeugung] is theorizing also die from wie wissen generiert wird, und content production ist der ausdruck und die ausrichtung die dieses wissen annimmt?
\stopAReminder

\startAReminder[das kapitel muss größtenteils nochmal überarbeitet werden]\stopAReminder

In order construct a methodological framework based on the \bracket{subjective} motivations and demands \refMissingSrc{motivation-demand chapter} formulated above, I first want to explore available schemes of scientific research approaches and balance reasons for and against them. Knowledge about their content and form will permit me to

\spaceHalf

\textBoxedRoundMax[0.4]{determine the methodological approach(es) embodied in the thesis research action(s)}

\textBoxedRoundMax[0.4]{determine the direction this thesis is further veering towards}

\textBoxedRoundMax[0.4]{argue for the chosen research action(s) and by doing so induce transparency in order to comprehend the decisions made to realize this thesis}

As principle scientific scheme I have chosen the \quote{inductive-deductive} model of \refMissingSrc{inductive-deductive model}

Due to the fact this thesis is finally composed of theory and practice (action), the extend and relevance of those aspects could be be determined first.

\imgThesisTheroy

Several important aspects have been identified, mainly during the course of research actions in São Paulo, thus it can already be said that the shape of this thesis is strongly influenced by the experience made during those actions. The following factors have been identified and may therefore determine the concrete outcome and notion of the thesis: 

\startitemize
\item observation, empiricism and theory
\item induction and deduction
\item quality and quantity
\stopitemize

Those aspects will be aligned to the considerations that have been made earlier about \quote{Motivation and Demands} of this thesis.

\subject{Inductive}

\startKeywords
deductive, inductive, reflective, narrative
\stopKeywords

\startAReminder 
in diesem abschnitt stellt sich die frage was neben induktiv und deduktiv noch benannt werden sollte, da ja keine strukturierten daten erhoben wurden. vielleicht kann das kapitel auch wech?
\stopAReminder

\startADefinition
{\bf inductive} procedures originate from particular situations, the local, which is supposed to be verified against a theory. Usually, a corresponding theory already exists and no new theory is supposed to be formulated. The local is always a snapshot of a particular situation which cannot be easily generalized by default. The conclusions drawn from the local can be used to adjust, support or object existing theories. Therefore, the local is described and represented by empirical data which is supposed to be representative and reliable enough to be considered valid for the contention or adjustment of the corresponding theory.
\stopADefinition

\startADefinition
{\bf deductive} procedures originate from theory, the general, which is supposed to be translated to the local, to particular situations, where it can be validated and proved. On the local it shall be verified whether the proposed theory is valid or not or if it needs adjustments. On the local this is done with empirical data which is supposed to be representative and reliable enough to evaluate the correspondingly formulated theory. 
\stopADefinition

\subject{Empirical}

\startKeywords
empirical, theoretical, objective, observation, daily life, systematic, unsystematic, reproducibly, randomly
\stopKeywords

It has already been mentioned that the thesis structure and outcome is impacted by the experience made in São Paulo. This experience is composed of particular situations, events and narratives, bounded to the local context and the local people, not to the whole city but just a tiny fraction of it. Therefore it has to be seen in how far the situation(s) narrated in this thesis can and should be generalized, where it is reasonable and where unnecessary. 

\spaceHalf

\inright{empiricism: objective and reproducible}
One aspect that will be referred to later on in more detail, when introducing different types of research action(s), is the aspect of objectivity and reproducibility of observed situations and made experiences. Those notions are normally referred to particular types of research inquiries which are grounded on empiricism.

\startADefinition
In contrast to theory, \inright{empricism: a brief definition} empiricism observes situations and draws conclusions that cannot be explained and derived in a formal or logic manner, such as a mathematical formula. In order to derive conclusions, empiricism is supposed to be performed in a systematic and objective manner, grounded on \toMark{(reliable and analysable)} data and information, \toMark{ref missing}, the base of \toMark{western} science. This shall distinguish empiricism from observing randomly occurring daily life experiences, which are claimed to be non-producible in a systematic manner \toMark{ref missing}. Empiricism means therefore systematic inquiry of the daily life.
\stopADefinition

\spaceHalf

\inright{observation: open and non predictable}
In stark contrast, other types of research actions rely on observations that do not claim to be systematic and which question the objectivity of gathered data and reproducibility of situations. Those observations may occur when neither the concrete local context is known in advance by the research agent, nor the outcome of the research is clearly defined. Therefore, necessary knowledge has to be gained in the first place in order to formulate frame of research and action(s). This knowledge is composed of rather daily situations, realities and conditions, which are not predictable beforehand and which affect and construct the frame research is bounded by.

\spaceHalf

\inright{open local observations and narratives do not fit with empiricism}
Therefore, this thesis can not be considered empirical because the narratives of particular situations, experiences and observations made in São Paulo had to be made in an unsystematic manner due to the fact that the whole local context was unknown to me. Hence, the unawareness off the city of São Paulo, the thousands of realities it represents, the complexity of its social and political structures, the whole universe called São Paulo, automatically implies that unsystematic observation and participation is one of the viable options that allow to dig a bit deeper than just to scratch on the surface. 

For the sake of completeness, it should be mentioned that another option would have been the participation in an existing institutional project. For me this would have been antipodal to the intended non-hierarchical/non-authoritarian approach due to the fact that an existing project already represents a fixed frame, hierarchies are already established and existing (institutionalized) social networks have to be utilized. The fixed frame leaves little space for navigation and by that the thesis could have been degenerated to the mere achieving of the title and the fulfilment of project goals, no matter if reasonable and justifiable or not. Thus, project based thinking and working would have prevailed. This type of thinking was supposed to be pushed aside for and during this thesis.

I had to search for one way out of many that was viable for me and the ones I got to be acquainted with on that discovery. The role as observer and participator will be illustrated in more detail later on as well.

\spaceHalf

\inright{relevance of theoretical versus observed and experienced information}
In the end, if one argues about relevance in the context of this thesis, personal observation and participation contemplates those complex realities that are rather invisible in daily life and public discourse, while theory can primarily be used to translate those realities into another form. Thus, talking about those realities is relevant for the thesis, because then, this thesis and its information could become another mosaic that composes the struggle of those people that facilitated the insight in and cognition of the city of São Paulo and their realities and by that impacted and shaped the current form of this thesis. Relevance for the thesis means relevance for the people and vice versa. The question of relevance is crucial here because research actions were cooperative work and cannot be divided into research agent and research target. Thus research outcome, apart from putting everything into the form of a thesis, is cooperative work as well, thus work we have done together, and which is relevant for us. 

{\em In other words, this thesis has been shaped from and is drawn on concrete but unsystematic experiences, observations and actions, thus cannot be considered as a thesis of empirical origin or with empirical focus.}

\subject{Qualitative}

\startKeywords
qualitative, quantitative
\stopKeywords

While being in São Paulo, it became clear over the course of the first two month that moving through the city means learning how the city functions. This meant to learn how to use public transport, how to use and impropriate the city in order to loose the fear of getting lost. Further on it meant to impropriate the slang of the streets and finally to get in touch with the people, in this case with the people that live and work on the streets. 

What became evident was the fact that new dimensions of the city emerged after social relations had been established. Those dimensions are partly invisible in daily life (more on this later on) and can often just be perceived if one spends hours together on the streets, hanging around or walking from one spot to another and by that circling through the city, hearing many stories, perceiving many situations, such as this emblematic one:

\startCitation
ein gutes beispiel
\stopCitation

The experience of being on the streets, embedded in and being a part of a network of people, lead to a more comprehensive understanding of the city, its dimensions and realities. This comprehension also means that any research based on quantitative measures, would have failed or at least led to wrong assumptions because any questions and situations imagined before hand would have been non-relevant (in the sense as stated in this chapter), due to the lack of understanding of the city, the situation of the people and a biased European perspective. 

\startADefinition
{\bf quantitative} \inright{data sets, normalisation, indicators, statistics, calculus, theory generation, theory prove}
 {\em methods generate quantities, thus relatively large amounts of normalized and standardized data sets. These sets can for example be generated by conducting surveys, a classical method where copies of a set of questions are distributed to many people, probably several hundreds or more, and where an analysis of the received answers is performed afterwards, when all surveys had been re-collected. Analysis can be done statistically for example, because surveys are usually standardized to a set of questions (every participant receives the same questions so to speak) that are related to one or more indicators. An indicator is a \toMark{formal} representation of a specific domain of interest: if a survey asks about your housing condition (rent, squatted, property, homeless, etc), a related indicator could represent the domain of \quote{housing-types} or something else. Thus quantitative approaches are about (relatively) large amounts of data and information, normalisation, indicators, statistics and calculus. Quantitative approaches inherently generalize the made observations because collected data sets usually contain large numbers of entries and are often considered representative (though not all-encompassing), thus generalizable.
}

{\bf qualitative} {\em }  \inright{non standardized, interpretative, comprehensive, theory extension, theory adoption}
{\em methods generate comprehensive data, thus sets of information that could represent a particular context in order to understand its meaning in a comprehensive, holistic or qualitative manner. This data is not meant for statistical analysis even though it can be related to indicators as well, but due to the small size of data-sets equipped with a large amount of non standardized context information, statistical analysis renders not feasible. Therefore, analysis is often done in interpretative ways. Referring to the quantitative example, qualitative data would not only ask about the housing type, but probably also about the materials it is constructed of, the neighbourhood, the age  and once an answer doesn't fit  (i.e. no house but the streets) the process of information gathering can be adjusted to the new situation. In addition, those information may be gathered during an interview or a talk, thus involves inter-personal communication and personal opinions. Qualitative approaches do not generalize because the number of data sets, though comprehensive, are not sufficient to draw a general picture from it, maybe just a small picture from a small spot. Therefore, qualitative information and analysis can be used to extend or adjust existing knowledge by new facets.
}
\stopADefinition

\spaceHalf

\inright{quantitative shortcomings}
Quantitative approaches require access to a relatively large group of people and other resources of knowledge. Several drawbacks emerge from that necessity: 

\spaceHalf

\inright{enforced power hierarchies due to distance and impersonality: definition shapers still rule information providers}
The relative distance and impersonality to people, which is immanent due to the fact that a large group is addressed. This impersonality and distance can be seen as a hierarchy between the research agent and the research participants. The participants are solely perceived as information providers and not as participants (in the sense discussed in the \toLink{participation chapter}). This form of participation just means outsourcing the information generation process while keeping the powerful position of defining (probably wrongly) which type of information is necessary or relevant. Here we enter the question of power again which is supposed to be dissolved in the course of this research.

\spaceHalf

\inright{dissolution of power hierarchies requires genuine participation}
Dissolution of power hierarchies means that people are involved in shaping the whole process of research. This would mean that decisions about the relevance of research, the type of research objectives, the provided information and the usage thereafter are decided together with and by the people. In order to accomplish this situation one would require access and involvement in the peoples networks and communities. This accomplishment would require much more time then granted by the institutional framework this thesis is embedded in and is therefore impossible to achieve.

\spaceHalf

\inright{generalization leads to elision of the margins}
The motivation of gathering large sets of data is usually to be able to generalize analysis, by proving previously formulated hypotheses or by creating new theories. Due to the fact that collected data sets seldom represents the whole spectrum of variants and possibilities, generalization do not cover all aspects of the area of interest, which could be no problem in certain areas, such as the measurement of rainfall, but should be taken in mind when actions are socially motivated and generalization may lead to the exclusion of the margins of society. Thus, generalization is problematic for this research because no one shall be elided.

\spaceHalf

\inright{lack of local knowledge leads to wrong perceptions}
A quantitative approach applied for this thesis research, such as a survey, that had been designed before hand, during preparation of the research, without the concrete knowledge of the city, would have been composed of assumptions that led astray. 

\spaceHalf

\inright{qualitative shortcomings and benefits}
The previously mentioned drawbacks are party dismantled by going qualitative rather than quantitative. Qualitative approaches can have similar shortcomings than quantitative ones, depending on the type of approach and the motivation of its application. 

The question of information gathering in the qualitative context can lead to similar, wrongly made assumptions than already explained above, due to the fact that local knowledge must be gathered somehow. Thus, a qualitative interview based on wrong perceptions and lack of local knowledge can lead to the same misleading assumptions than a wrongly designed quantitative survey. 

On the other hand, qualitative approaches do not aim to be generalizable, hence inherent to the access to local knowledge is the process of becoming part of the peoples networks and communities, on a much smaller scale, probably just on interpersonal or group level. Thus decisions and discussions about research, its relevance, its objectives and so forth can be mediated directly by and with the people. This process also requires time and trust between each other but could open other viewpoints on research for all. 

If we return to the notions of hierarchy and emancipation, one could argue that the processes of becoming acquainted with people, building relations and being together, already provides plenty of space for acting in a non-hierarchical, non-authoritarian and emancipatory manner. This possibilities are of course not absolute nor are they vested due to the fact that power hierarchies pervade all aspects of social life of individuals and groups, thus are always visible or perceivable \toMark{ref missing?}.

However, having the possibility to observe and take part in a genuine participatory manner opens up spaces that would not have been reached just through plain non-genuinely-participatory approaches. A note from São Paulo that certainly impacted the way I see the relevance and significance of my research, may illustrate this claim.

\startCitation
After I met \toMark{Matheus} one day at \toMark{Ay Carmela (a self-organized centre)}, we fixed a date when he was going to show me the streets where he lived for 15 years, in the centre of São Paulo. First we planned to spend one week on the streets but later reduced that time to two days. What I have seen and experienced in those days is barely comparable to informations gathered through semi-personal surveys or interviews which would have failed to transport the intensity and manifoldness of situations. This trip on the streets gave me answers to questions that I wouldn't have been able to formulate in advance and once it passed it wasn't necessary to ask any more because everything had already been unfold \aQuoteP{2010}.
\stopCitation

\spaceHalf

\inright{the thesis foundation in terms of empirical and qualitative categories}
Lets summarize the initial attributes of this thesis research actions:

\startitemize[3]
\item The concrete experience and the gained local \toMark{or endogenous?} knowledge finally proposed the research action approach and theme for this thesis. {\em This thesis research focus and approaches are determined by local knowledge, observations and experience.}
\item The process of gathering local knowledge imply that this thesis is drawn on observed and experienced, thus, qualitative information which are further theoretically contemplated. 

\startAReminder
der nächste satz ist falsch und muß neu formuliert werden
\stopAReminder

{\em This thesis is inherently empirical and inductive.}


\subject{Participative}

\startKeywords
emancipatory, grounded theory, participatory observation, action research, hierarchies
\stopKeywords


\startitemize[3]
\item As qualitative approaches mostly those are taken into account that possess a portion of genuine participation in the sense of emancipatory and self-determined practice. {\em This thesis is mainly based on observation and action}
\item More 
\stopitemize

\showImperfection

\stopmode

\stoptext

\stopcomponent
