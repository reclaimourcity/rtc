\startcomponent component_HowToRead
\product product_Thesis
\project project_MasterThesis

% definitions and macros
\environment envThesisAllEnvironments

\define[]\chapterHowToRead{\chapter[howtoread]{Read//This//Text}}

\starttext

\startmode[tocLayout]
\chapterHowToRead

describes how to read this thesis
\stopmode

\startmode[draft]
\chapterHowToRead

Before I begin I would like to lay out how this thesis is structured. I did not use a common document or thesis structure. This thesis rather reshapes the flow of my research. It is divided in three main sections that follow the short \gotoTextMark[Introduction]{intro} to what you as reader could expect when continue reading.

\spaceHalf
\inright{guiding questions}
The flow of this document conforms to three \guidingQuestions that lay at the heart of my research. Those \guidingQuestions are: \WhoAmI \WhatDoIWant \WhatShouldIDo

I used those questions because this thesis is a narration of social movement's struggle therefore I am attempting to narrated my thesis from the \standpoint of social movements. Social movements use similar questions to position themselves in the world they are struggling in. Their main \guidingQuestions are: \WhoAreWe \WhatDoWeWant \WhatShouldWeDo

I am asking \gotoTextMark[Who am I?]{methodology_whoami} in order to determined the ground on which my research is based. The answer to this question is to one extend a subjective and personal definition of a self conception, a personal \gotoTextMark[list of demands and motivations]{methodology_whoami_motivation} about my research. To another extend its the \gotoTextMark[determination of my approach to research actions]{methodology_whoami_actionresearch} in São Paulo. 

The second question asked, \WhatDoIWant, draws on the answers of the first and proposes concrete \gotoTextMark[aims, intentions and objectives]{methodology_whatdoiwant} of my research actions. 

The answers to third question asked, \WhatShouldIDo, \gotoTextMark[determines approaches]{methodology_whatshouldido} that could help to realize the proposed aims, intentions and objectives.

\spaceHalf
\inright{section one}
Those three \guidingQuestions compose the \gotoTextMark[first section]{methodology_intro} of this thesis that could be conceived as the \methodology section. In short it determines why I am would like to realized research actions, what it means for me, how I approach those actions in order to be finally able to experience and obtain insights from the social struggle of streets in São Paulo, as part of the streets, the space this thesis is mainly remaining in.

\spaceHalf
\inright{section two}
My insights and experiences are then narrated in the second section in form of \gotoTextMark{diaries from the streets}. Those narrations and diaries could be conceived as the empirical work of this thesis. My \narrations are mainly the product of those approaches proposed in the first section. 

\spaceHalf
\inright{section three}
The third section is about \gotoTextMark[theorizing ]{theorizing_intro} from the what I have learned on the streets, that what is transported in my narrations and diaries. \Theorizing is also the product of those approaches proposed in the first section. The third section could be conceived as the theoretic section. I conceive it as an extension of my narrations rather than an introduction or framing of the \quote{topic}. Its content is driven by the experiences and insights from my research actions, therefore it composes the last section of this thesis.

After having briefly presented the thesis structure I would like to further draw on its layout. This thesis utilizes a particular text formatting and colouring style in order to visually determine the meaning of certain words and text passages.

\spaceHalf
\inright{the margin for indicating text passages}
On each page one can perceive a larger margin that should provide space for taking notes on a printed version and that also serves to indicate text passages by short descriptions

\spaceHalf
\inright{the footer links back to the table of content}
Each page contains a header and footer. The footer is composed of the page number at its right edge and the current section name you are currently reading at its left edge. At the begin of the section name there is a link that leads back to the \gotoTextMark[table of content]{tocComplete}, named \goto{Start}[tocComplete].

\spaceHalf
\inright{formats and meanings}
Apart from structural page elements, words and text passages may have distinct meanings that are expressed by various types of text formatting.

\spaceHalf
\inright{translations}
Words that need to be translated are indicated as \toTranslate[to translate]{zum übersetzten}. All translations are listed under the section \gotoTextMark[Translations]{translations}.

\spaceHalf
\inright{abbreviations}
Words that are abbreviated are indicated as \abbrNew{An Abbreviation}{AA} when they occur for the first time. Later on they are either used in their abbreviated form: \abbr{AA} or in the full form: \abbrFull{AA}. All abbreviations can be found under the section \gotoTextMark[Abbreviations]{abbreviations}

\spaceHalf
\inright{keywords}
Keywords are indicated as \aKeyword[How To Read]{how to read}. A complete list of keywords can be found in the \gotoTextMark[Index]{index}

\spaceHalf
\inright{footnotes}
Footnotes to certain text passages or words are simply indicated as \footnote{an example of a foodnote}

\spaceHalf
\inright{document links}
A link to another location within this document is indicated as \gotoTextMark[Introduction]{intro}. 

\spaceHalf
\inright{web links}
A link to an online sources such as a website is indicated as \aLinkNew[the thesis blog ]{https://rtc.noblogs.org}. All utilized links to online sources are listed at \gotoTextMark[Links//General]{refLinks}. Web links produce a footnote with the link address.

\spaceHalf
\inright{locations links}
A link to a position at \aLinkNew[openstreetmap]{http://openstreetmap.org} is indicated as \locSe. Locations are mainly used throughout my narrations. All utilized location links are listed at \gotoTextMark[Links//Locations]{refLinksLocations}. Location links produce a footnote with the location address.

\spaceHalf
\inright{objectives}
Throughout the text I formulated objectives for my research actions. Objectives are indicated as 

\spaceHalf

\textBoxedRoundMaxObj{definition of objectives}

\spaceHalf

\inright{definitions}
Throughout the text I define and determine what is important for my research actions. Those definitions are indicated as

\spaceHalf

\textBoxedRoundMaxDef{a definition or determination of something relevant}

\spaceHalf

\inright{questions}
Throughout the text I asked questions that should help me to structure what I am doing. Questions are indicated as 

\spaceHalf

\textBoxedRoundMaxQuestion{asking of a question}

\spaceHalf

\inright{important sections}
Sections that I would like to highlight as a reminder for instance are indicated as

\spaceHalf

\textBoxedRoundMax{something to highlight}

\spaceHalf

\inright{narrating subjective thoughts}
My subjective thoughts and feelings are indicated throughout my narrations as

\startPersonal
A personal thought or a feeling.
\stopPersonal

\spaceHalf

\inright{narrating dialogs}
Dialogues between people are indicated throughout my narrations as 

\startDialog
\tell{He}{says something}
\tell{I}{say something}
\tell{You}{say something}
\stopDialog

\spaceHalf

\inright{citations}
Citations occur throughout the whole text. Complete text passages taken from a particular source are indicated as

\startCitation
Citations occur throughout the whole text. Complete text passages taken from a particular source are indicated as this.
\stopCitation

\spaceHalf
\inright{references}
References to complete text passages are indicated as \aQuoteB{author}{year}{X} for a text reference, \aQuoteW{author}{year} for an online reference.

References to authors, titles or short text passages within paragraphs are indicated as \aQuoteInText{a short text passage}, \aQuoteInTextT{a title}, \aQuoteInTextA{an author}.

References are structured according their meaning. All bibliographic references such as books, texts, and the like are listed under \gotoTextMark[Reference//Bibliography]{refRefBib}. All references to used media content such as journal and newspaper articles are listed under \gotoTextMark[Reference//Media Coverage]{refRefMedia}. All sources that I utilized to find bibliographic content are listed under \gotoTextMark[Reference//Content//Sources]{refRefBibSources}. All sources I utilized to find journal and newspaper content are listed under \gotoTextMark[Reference//Media//Sources]{refRefMediaSource}

%---------------------------------------------------
% gets only displayed in unfinished mode

\showImperfection

%---------------------------------------------------
\stopmode

\stoptext

\stopcomponent
