\startcomponent component_Methodology_WhatShouldIDo_MethodParticipantObservation
\product product_Thesis
\project project_MasterThesis

% definitions and macros
\environment envThesisAllEnvironments

\define[]\subsectionActionsPA {\subsection[methododology_whatshouldido_methodparticipantobservation]{Methods//Partici[pating//pant]//Obser[ver//vation]}}

\starttext

\startmode[tocLayout]

\subsectionActionsPA

This section describes \quote{Participant Observation} and \quote{Participating Observer} as a knowledge production methodology. The concept of \quote{Participant Observation} is extended by the active part of the Observer role which turns the Observation into active Participation. Both concepts are embedded in a \quote{Action Research} framework.

\stopmode

\startmode[draft]

\subsectionActionsPA

\startKeywords
method, participant, observation, observer, participating, narration, movement content, standpoint, action research, roles, 
\stopKeywords

What did I do during my research action(s) in São Paulo? How did I approach \gotoTextMark[\abbrFull{AR}]{methodology_whoami_actionresearch}.

At the beginning, my \aKeyword{role} has been certainly that of a \aKeyword[passive observer]{role+passive observer}. When I went for instance for the first time \bracket{after two weeks being in the city or so} to a workshop of the \aKeyword[aRUAssa collective]{collectives+aruassa}\aLinkNewF{https://rtc.noblogs.org/post/2010/05/31/oficina-de-v-deo-workshop-film-making-with-mobile-devices/} \bracket{which actually has not given itself that name then}, I merely observed what the collective was doing, what it was discussing and planning. I also talked with the collective about the reasons why I was there, in São Paulo, what the aim of my research was supposed to be, primarily to introduce myself. For me this was a normal behaviour because we did not know each other and I felt myself still an alien in the city, had to cope with my Portuguese skills, thus had to find my way. 

Then there has been a shift at a point I cannot exactly determine any more. A shift from the role of the \aKeyword[passive observer]{role+passive observer} to the role of the \aKeyword[participating observer]{role+participating observer}, which explains itself most obviously for me when I stayed with \Matheus for two days on the streets, where we organized our place to sleep, discussed and talked with people and when I experienced, unfiltered and raw, what it means to stay on the streets but where I also started to understand how to read the city, its excluding architecture, its institutions for the good and for the bad.
 
A third shift occurred from the role of a \aKeyword[participating observer]{role+participating observer} to the \aKeyword[participant observing]{role+participant observing} or maybe the \aKeyword[activist observing]{role+activist observing}. I can also not define when this third shift happened, but one situation that probably represents this shift is the interview\aLinkNewF{https://rtc.noblogs.org/post/2010/10/11/entrevista-ocupacao-avenida-ipiranga-pt/} that {\bf we} conducted as aRUAssa collective in one of the occupations in the centre of the city. Thus I identified myself with the collective as active part of it, sharing its attitude and intention.

Those three roles have always been present during that period, in different intensities though. Towards the end, there has been probably not much left from the role of the \aKeyword[passive observer]{role+passive observer}, because I knew my people, I participated in the flow of their activities, I was much more confident then at the beginning, being capable of actually contributing and being active. 

I would also say that the adoption of different roles dependent on the context I was located in, can be traced back to the fact that I participated with \gotoTextMark[different intensity]{methodology_whoami_actionresearch_engagementintensity} in the \aKeyword[lived space]{space+lived} of different people, movements and collectives but was and could not become  \gotoTextMark[fully engaged]{methodology_whoami_actionresearch_engagement} in them. I would have needed much more time, one of the limitations of my research actions that has been \gotoTextMark[noted out earlier]{methodology_whoami_experience_limitingfactortime}. Thus...

\spaceHalf

\placefigure[force]{Capturing my roles in action research.}
{
\textBoxedRoundMaxDef
{
I describe my approach to \aKeyword{action research} as a participating and collaborative approach, where I assume the role of a participating observer and participant observing \bracket{or activist observing}, two roles that I assume depending of the context I am located in.
}
}

The written reproduction of my approach to \aKeyword{action research} consists of several practices. 

One is the narrative. Narrating stories of experiences, situations, insights, incidents and occurrences can be as beautiful as in the \aQuoteInTextT{Street Corner Society}, a study of an mostly Italian immigrant district in Boston in the early nineteen fortieth.

\spaceHalf

\startCitation
The liquor traffic of prohibition provided many of the prominent racketeers of today with their business experience and financial resources. In the early years of prohibition there were  a large number of small liquor dealers in active competition. Prices fluctuated, and spheres of operation were not clearly defined. Competition often led to violence. 

As time went on, some of the more skilful, energetic, and daring of the dealers gained in financial status and power, so that they were able to push a number of smaller independents out of business and extend their control over others.This combinati9on movement continued steadily and, in Eastern City, reached its height shortly before repeal under the leadership of a man who became known as \quote{the Boss} \aQuoteB{Whyte}{1993}{111}. 
\stopCitation

\spaceHalf

My narrations will hopefully reflect the fact that I assumed different roles, by being much broader in scope, not focusing on one particular action but expressing a wider range of experiences I made and insights I gained. By no means my narrations will reach literary quality because I am not an experienced writer nor an experienced social researcher and by far not proficient enough in English or Portuguese. 

My narrations will be complemented with all kinds of \gotoTextMark[movement, street and miscellaneous content]{methodology_whatshouldido_toolstheorizing} that is useful to draw a broad but dense picture of the \gotoTextMark[themes]{methodology_whatdoiwant} determined as \gotoTextMark[relevant]{methodology_whoami_relevance} from the \aKeyword[standpoint]{standpoint+streets} of the streets and its people. Thus...

\spaceHalf

\placefigure[force]{Capturing the scope of narrations and content emerging from action research.}
{
\textBoxedRoundMaxDef
{
The scope of narrations will cover a wider range of experiences and insights, complemented by additional movement, street and miscellaneous content in order to produce a broad but dense picture of relevant themes.
}
}

\spaceHalf

Another practice to reproduce my approach to \abbr{AR} in written form is the incorporation of content not narrated by me. Even though my narrations are already affected by the experiences and insights I gained through \gotoTextMark[actions and genuine participation]{methodology_whoami_actionresearch_genuineparticipation} with my people, I also have access and make use of content produced by the people \bracket{an by other related local sources}, thus \gotoTextMark[movement content and theorizing]{methodology_whoami_experience_peoplesknowledge} enters the scene here, affecting and complementing my content and theorizing, allowing to align my thesis and research actions to the standpoint of the streets in São Paulo. This standpoint is not representative to \quote{the} streets because I did not have contact with everyone on the streets nor did I visit every meter of streets. My standpoint is still rooted in the streets, a partial one of many others. I would like to give a short example.

\spaceHalf

\startCitation
When we passed by the front \aNewLocation[Cathedral da Sé]{http://osm.org/go/M@ziKS_1G--}, the massive cathedral an the south-western corner of \aNewLocation[Praca da Sé]{http://osm.org/go/M@ziKciPa--}, the central place of downtown São Paulo, I saw a cathedral completely fenced by two meters high iron lattice, shielding public space around the building from the people. Public space that prior to that has been used as shelter by the people in street situation, now transformed into closed space to get rid of unwanted subjects \aQuoteP{2010}.
\stopCitation

\spaceHalf

I wrote this example from the standpoint of the streets. I could also have written that the lattice finally solved the problem of homeless people around the cathedral, but I did not because I narrated the situation from the standpoint of the people I passed by with and that are affected by it immediately because they lost a location where they found shelter. I also used the terminology of the people from the streets that call themselves \toTranslate[pessoas em situacão de rua]{people in street situation} instead of homeless or street person.

\spaceHalf

\placefigure[force]{Capturing the standpoint of narrations and content emerging from action research.}
{
\textBoxedRoundMaxDef
{
The standpoint of narrations and content reproducing my experiences and insights is that of the people I participated and collaborated with.
}
}

\spaceHalf

\startARemark{die referenzen und beschreibungen zu participant observing und particpating observer fehlen noch}
\startitemize
\item \aQuote{Aggarwal}{2000}
\item \aQuote{Cattaneo}{2006}
\item \aQuote{Rappaport}{2008}
\item \aQuote{Schöne}{2003}
\item \aQuote{Smith}{1997}
\item \aQuote{Whyte}{1993}
\stopitemize
\stopARemark

\startARemark{die schwerpunkte the erzählungen fehlen noch}
\startitemize
\item right to the city
\item self-determination
\item participation
\item citizenship
\stopitemize
\stopARemark

\addReference
{
Aggarwal, R., 2000. Traversing Lines of Control: Feminist Anthropology Today. {\em The ANNALS of the American Academy of Political and Social Science}, 571(1), p.14 -29. Available at: \goto{\hyphenatedurl{http://ann.sagepub.com/content/571/1/14.abstract}} [url(http://ann.sagepub.com/content/571/1/14.abstract)] [Accessed March 24, 2011].
}
\addReference
{
Cattaneo, C., 2006. Investigating neorurals and squatters’ lifestyles: personal and epistemological insights on participant observation and on the logic of ethnographic investigation. {\em CattaneoAthenea Digital}, 10, p.16-40. Available at: \goto{\hyphenatedurl{http://redalyc.uaemex.mx/pdf/537/53701002.pdf}} [url(http://redalyc.uaemex.mx/pdf/537/53701002.pdf)] [Accessed May 21, 2011].
}
\addReference
{
Rappaport, J., 2008. Beyond Participant Observation: Collaborative Ethnography as Theoretical Innovation. {\em Collaborative Anthropologies}, 1, p.1-31. Available at: \goto{\hyphenatedurl{http://muse.jhu.edu/journals/collaborative_anthropologies/v001/1.rappaport.html}} [url(http://muse.jhu.edu/journals/collaborative_anthropologies/v001/1.rappaport.html)] [Accessed July 22, 2011].
}
\addReference
{
Schöne, H., 2003. Participant Observation in Political Science: Methodological Reflection and Field Report. {\em Forum Qualitative Sozialforschung / Forum: Qualitative Social Research}, 4(2). Available at: \goto{\hyphenatedurl{http://www.qualitative-research.net/index.php/fqs/article/viewArticle/720/1558}} [url(http://www.qualitative-research.net/index.php/fqs/article/viewArticle/720/1558)] [Accessed March 8, 2011].
}
\addReference
{
Smith, M.K., 1997. Participant observation and informal education. {\em the encyclopedia of informal education}. Available at: \goto{\hyphenatedurl{http://www.infed.org/research/participant_observation.htm}} [url(http://www.infed.org/research/participant_observation.htm)] [Accessed March 21, 2011].
}
\addReference
{
Whyte, W., 1993. {\em Street corner society : the social structure of an Italian slum 4th ed.}, Chicago: University of Chicago Press.
}

\showImperfection

\stopmode

\stoptext

\stopcomponent
