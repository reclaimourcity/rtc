\startcomponent component_ReturningToLefebvre
\product product_TheRightToTheCity
\project project_MasterThesis

% definitions and macros
\environment envCfgPaperLanguageEn
\environment envRttcUrl
\environment envRttcText

\starttext

\section[returning]{Returning to Lefebvre: the right to the city}

In contrast to \harvey who approaches the \rttc through a political economy of space (Kipfer et al, 2008, p.08), for example in terms of newly invented financial tools necessary for large scale urbanization, \hlef \quote{promoted the developed of a comprehensive theory of the production of space}  (Schmid in Kipfer et al, 2008, p.08). Concepts such as the \globalcity and urban displacement processes are immanent to \lefs conceptualization of the contemporary (western, post-war, industrialized) city, which has been the Paris of his epoch (Stanek, 2008, p.72).

\Segregation and \centrality \inright{segregation and centrality} are two contradictory processes at first glance, that define the \productionofspace and the reproduction of social relations in the contemporary city. 

In the city, everything is separated, segregated and (socially) unconnected, the spaces specialized, functions, labour and society are divided (Stanek, 2008, p.71). Urban space as social product and without specific content means \centrality: a space of \quote{association and encounter of whatever exists together in one space at the same time}, a space where dominant power is centralized, a space where oppositional struggle against the dominant spacial and social form is centralized, thus urban space is the simultaneity of everything that can be brought together at one point (Kipfer et al, 2008, p.291). 

\Centrality \inright{centrality and the global city} as seen from the perspective of the holistic city, thus the city perceived as an entity on a macro scale, means the assimilation of the urban, peri-urban and rural, with its urban cores, the peripheral layers that surround the city and the (contiguous but also globally spread) rural areas that are affected by the city, because the city itself concentrates power and produces the highest power: the decision (Kipfer et al, 2008, p.291). This decision produced in a city leaks into the global \spaceofflows and implies decision making and resulting effects on global scale, thus \hlefs \centrality of a city can easily be translated to the concept of the \globalcity.

\quote{\em The ideal city, the New Athens, is already there to be seen in the image which Paris and New York and some other cities project. The centre of decision-making and the centre of consumption meet. Their alliance on the ground based on a strategic convergence creates an inordinate centrality. We already know that this decision-making centre includes all the channels of information and means of cultural and scientific development. Coercion and persuasion converge with the power of decision-making and the capacity to consume. Strongly occupied and inhabited by these new Masters, this centre is held by them. Without necessarily owning it all, they possess this privileged space, axis of a strict spatial policy. Especially, they have the privilege to possess time. Around them, distributed in space according to formalized principles, there are human groups which can no longer bear the name of slaves, serfs or even proletarians. What could they be called? Subjugated, they provide a multiplicity of services for the Masters of this State solidly established on the city.} (Lefebvre, 1996, in Kipfer et al, 2008, p.291)

\Segregation and \fragmentation of the urban space means \centrality on the ground level. 

\quote{\em [...] And yet everything (“public facilities,” blocks of flats, “environments of living”) is separated, assigned in isolated fashion to unconnected “sites” and “tracts”; the spaces themselves are specialized just as operations are in the social and technical division of labour} (Levebfre in Stanek, 2008, p.71)

Urban space \inright{elimination of the difference} becomes \gentrified, \revitalized and \transformed, land becomes parcelled, the city is zoned and segregated (Stanek, 2008, p.72). Those fragments the city consists of are \homogenized, thus every difference within them is eliminated which leads to hierarchization of urban spaces and within society (Stanek, 2008, p.72). \lef differentiates between two forms of \difference, {\em minimal} and {\em maximal difference} (Kipfer, 2008, p.203). 

Kipfer (2008, p.202-208, 296) \inright{minimal difference} notes that, \mindiff (or \indifference) produces the capitalist form of life and is produced by \quote{\em appropriating and monopolizing the urban space as a productive force, [...] commodified festivity, racialized suburban marginality, multiculturalized ethnicity, [...] parcelized social spaces planned in vulgar modernist fashion, [...] alienations of property, individualism and group particularism}. The city of \mindiffs is the city of resorts, university compounds, gated communities, slums, exclusive districts or working class quarters, but also the city of \estdiffs, the partiarchal family, the reproduction of \quote{established} norms (discursively fixed identities or according to \quote{natural} distinguishing characteristics) onto women, immigrants, the poor, the old, the ill and their expulsion to dedicated (often peripheral) urban regions (favelas in Brasil, Banlieus in France, social hotspots in Germany), public housing tracts or specialized institutions.

To the \inright{maximal difference} contrary, \maxdiff is \proddiff, thus fully lived forms of plurality and individuality, an articulated identity based on rich social relations and not affected by any form of \indifference. It is the quest for a unalienated, festive, creative, self-determined, fully lived urban society (Kipfer, 2008, 203) that is not forced into a space that was produced only for the purpose of discrimination (Kipfer et al, 2008, 293).

\Maxdiff \inright{transformation of minimal to maximal difference} is incompatible with \mindiff (Kipfer, 2008, 203). Due to this incompatibility, a transformation of \mindiffs to \maxdiffs will result in the \rttc. How this transformation can be achieved remains to be seen. Existing spaces of (at least intended) \maxdiff and alternative practices such as social centres, squats, or self managed enterprises have not reached sufficient weight so far, thus, according to \lef, a \quote{\em transformation can be achieved only by [more] social struggles for political self-determination and a new spatial centrality, which help liberate difference from the alienating social constraints produced by capital, state, and patriarchy.} (Lefebvre, 1970 in Kipfer et al, 2008, p.08)

The \rttc \inright{the right to the city according to \hlef} is therefore complemented by the {\em right for difference}. These rights are not of normative nature, thus rights granted by institutions (the right or obligation to vote) which do not prevent social, economical and cultural exclusion (Gilbert and Dikeç, 2008, p.258), but rights that area undetachable human properties, \quote{defined and redefined by political action, social relations [...] and the sharing of space} (Gilbert and Dikeç, 2008, p.258,259).  The continuous re-negotiation of those rights essentially means the active participation in societies self-management (Gilbert and Dikeç, 2008, p.260) where \quote{\em each time a social group refuses passively to accept its conditions of existence, of life or of survival, each time such a group attempts not only to learn but to master its own conditions of existence.} (Lefebvre, 1996 in Gilbert and Dikeç, p.260).

\stoptext

\stopcomponent
