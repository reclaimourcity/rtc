% global environments rtc
\environment envRTC
\environment envUrlRTC
\environment envCmdRTC
\environment envTextRTC

% global environments paper
\environment envPaperTerma
\environment envTextPaperTerma
\environment envUrlPaperTerma
\environment envImagesCommonPaperTerma

% local environment
\environment envRupTextRttc
\environment envRupUrlRttc
\environment envRupRefRttc

% start text
\starttext
%\showlayout
\dontleavehmode

%...........................................................................................................................
% start coverpage
%...........................................................................................................................

% coverpage setup+++++++++++++++++++++++
%
% switch background to cover page color
\setupbackgrounds[page][background=color,backgroundcolor=gray:8]

% change link color and styke for coverpage
\setupinteraction[state=start,color=gray:1, click=yes, style=slanted]

% coverpage content++++++++++++++++++++++

% cover page uses centered text
\startalignment[center]

% text is going to be bright due to dark background
\startcolor[gray:1]

\blank[1cm]
{\ssb \myTitle}
\blank[2cm]
\startlines
{\ssa \myTermaInstitute}
at
{\ssa \myTermaUniversity}
\stoplines
\blank[2cm]
{\bold \ssa \myTermaCourse}

\myTermaModule
\blank[1.5cm]
{\bold lectured by}

{\em \myProf}
\blank[2.5cm]
{\bold written by}

{\em \myMe}
\blank[1.5cm]
{\bold released on}

{\em \myDateToday}
\blank[1.5cm]
{\bold licensed under}

{\em\myLicenceText}
\blank[1cm]
% cc-by-sa image
\myLicenceImage

\stopcolor
\stopalignment

%...........................................................................................................................
% start new page: toc
%...........................................................................................................................
\page

%++++++++++++++++++++++++++++++++++++++
% toc setup 

% change link color and style for table of content
\setupinteraction[state=start,color=black, click=yes, style=normal]

% switch background back to normal (white color) and fill header/footer area from now on
\setupbackgrounds[page][background=color, backgroundcolor=white]
\setupbackgrounds[header, footer][background=color,backgroundcolor=gray:8]

%++++++++++++++++++++++++++++++++++++++
% toc content

% insert auto generated table of content
\completecontent

%...........................................................................................................................
% start new page::introduction >> text this starts with page #1
%...........................................................................................................................
\page[1]

%++++++++++++++++++++++++++++++++++++++
% introduction::setup

% start page numbering from now on
\setuppagenumbering[state=start]

% enable headers/footers from now on
\setupfooter[state=start, color=gray:1, style=small, strut=yes]
\setupheader[state=start, color=gray:1, style=small]

\setupheadertexts
        [  {\myTitle}]    []

\setupfootertexts
        [  {\getmarking[chapter]}]    [{page \pagenumber/\lastpagenumber } ]
\setuppagenumbering[location=]

% change link color and style for table of content
\setupinteraction[state=start,color=darkgreen, click=yes, style=slanted]

\chapter[objectives]{Objectives}
%++++++++++++++++++++++++++++++++++++++
% foreword

\startTwoColKey
The aim of this paper is to give an overview about the \rttc. The \rttc is a concept that is part  of the discourse about contemporary urban development with its accompanied and resulting social inequalities. It is referred to in institutional discourse on international level by UN-Habitat/UNESCO, on city level in form of statutes that intends to grant more rights to city dwellers (such as in Brazil and Mexico) but emerges also in various forms in the struggle of (urban) social movements and grassroots groups all over the world. Therefore it seems to be interesting to take a closer look to its roots that has been initially planted by the French sociologist and philosopher \from[wikiEnLefebfre].
\nextTwoColKey
\stopTwoColKey
\startTwoColKey
Due to its complexity, it is not the aim of this paper to comprehensively lay out the \rttc to its full extend and in all its possible forms and facets. The paper focus on the basic concepts and takes a further look at contemporary interpretations of scholars such as \from[wikiEnHarvey] and \from[wikiEnCastells]. Hence, the \rttc is used here primarily to describe the development of the \quote{capitalist city}, its negative effects for its inhabitants and the potential utopias that could be made possible, in contrast to other approaches such as the \quote{creative city} developed by urban theorist like \from[wikiEnFlorida]. 
\nextTwoColKey
\stopTwoColKey
\startTwoColKey
In order to relate the \rttc to concrete urban development processes, another concept, the \globalcity, is going to be examined. The \globalcity serves as the frame to take a look of four concrete urban development projects in various cities (two German, One Turkish and one Brazilian city), that fit all into the category of \waterfront \revitalization. The selected examples shall conceptualize the impacts of large scale urban development projects on the urban and social structure of a city. Similar to the \rttc, the \globalcity can examined from different points of views. Here, it is intended to focus on the global city as driver of globalization which will be related directly in the notion of the \quote{capitalist city}.
\nextTwoColKey
\stopTwoColKey
\startTwoColKey
The paper will finally return to an analysis of the \rttc related to the \worldcity with its given examples and its potential meaning in terms of citizenship, the shaping of the city and the production of urban space, driven from below.
\nextTwoColKey
\stopTwoColKey

\chapter[intro]{Introduction}
%++++++++++++++++++++++++++++++++++++++
% introduction::content

\startTwoColKey
The explicit struggle for {\em right to the city} has been one of the major urban struggles in the last couple of years. 
\nextTwoColKey
\stopTwoColKey
\startTwoColKey
Nowadays, cities are the living ambient for the majority of the world's population, from small towns to mega- and global cities. Simultaneously to the cities growth (and sometimes their shrinking as well), their literally \quote{explosion} as it can be observed in contemporary megacities and metropolitan regions, its citizens lost any possibility to shape the city they life in according to their ideas, wishes and needs. Decisions are made by a political elite, often driven by economic principles and reasoning, without or with very limited possibilities of participation by the citizens.
\nextTwoColKey
{\em powerlessness of the citizen}
\stopTwoColKey
\startTwoColKey
The shape and structure of the cities is changing, driven by or enforcing processes such as {\em gentrification, segregation,} or {\em rural-urban-international-migration}. Many western cities experience an expulsion and substitution of long time residents from their quarters and neighbourhoods by a new, young, dynamic and relatively rich elite. Mega-events such as the Football Worldcup or the Olympic Games are the justification for massive urban infrastructure (housing, transport, leisure) investment and the revitalization of whole city regions, in the centres and peripheries, which usually force many residents to leave (in the best case \quote{voluntarily} or by eviction in the worst case) the area due to increasing living costs or their \quote{inappropriate} social background and appearance. 
\nextTwoColKey
{\em urban processes and development}
\stopTwoColKey
\startTwoColKey
These and other factors and processes are often accompanied by high levels of repression in guise of police or private security agents which execute state led decisions and defend private interests, against the will of the citizen. 
\nextTwoColKey
\stopTwoColKey
\startTwoColKey
Despite this development, a rising number of cities are scenes of an emerging struggle for the right to shape and influence the development of and the life in the cities, driven by the often excluded, marginalized or discriminated people that live in them. 
\nextTwoColKey
{\em the right to the city}
\stopTwoColKey

\chapter[rttc]{The Right to the city}

\startTwoColKey
The \rttc can be referred back to the French sociologist and philosopher \from[wikiEnLefebfre]. Several of his writings are dedicated to and develop a concept of \rttc, among them {\em Le droit à la ville (The right to the city)} (Lefebvre, 1968), {\em Le droit à la ville suivi de Espace et politique} (Levebfre, 1974), {\em The production of space} (Lefebvre, 1991) or {\em Writings on cities} (Lefebvre, 1996).
\nextTwoColKey
\hlef {\em and his writings about the \rttc}
\stopTwoColKey
\startTwoColKey
\hlefs concept of the \rttc can be reckoned as a highly abstract framework that aims to conceptualize a critique of the development of the (industrialized, western) city that neglects (the majority) of its inhabitants. With the \rttc he provides a conceptualization of a {\em right} that each citizen is supposed to possess. 
\nextTwoColKey
{\em an abstract framework}
\stopTwoColKey
\startTwoColKey
The abstractness of this conceptualization makes it actually difficult to fully grasp its meanings and corresponding real world consequences (Purcell, 2002, p.99) and leaves much space for theorization and interpretation, which is clearly visible in the different discourses about the \rttc, between the points of views of institutional agents and agencies and those of urban (social) movements which began to incorporate the \rttc as a fundamental claim of their struggles. These differences will be briefly explored and laid out later on. For now, let us see how the \rttc is formulated.
\nextTwoColKey
\stopTwoColKey
\startTwoColKey
According to \hlef, the city consists of (social) spaces and (social) practices that are shaped, accessed and carried out by its inhabitants, which in turn are shaped by those spaces and practices, thus, the city is
\nextTwoColKey
{\em interdependent formation of space and human}
\stopTwoColKey
\startTwoColKey
\quote{\em [...] man’s most successful attempt to remake the world he lives in more after his heart’s desire. But, if the city is the world which man created, it is the world in which he is henceforth condemned to live. Thus, indirectly, and without any clear sense of the nature of his task, in making the city man has remade himself.}(Park, 1967 in Harvey, 2008)
\nextTwoColKey
\stopTwoColKey
\startTwoColKey
The concept of space in the context of \lefs work refers to various forms of perceptions of space. According to him, the spaces that shape the city are not only of physical or geographical nature but can be considered as {\em perceived}, {\em conceived} and {\em lived spaces} (Purcell, 2002, p.102).
\TwoColKey
{\em multidimensional spatiality}
\stopTwoColKey
\startTwoColKey
This multidimensional spatiality is important in the context of the \rttc because urban space cannot just be divided and reduced to its individual components in order to be analyzed separately. This would lead to \quote{the analytic destruction of space} (Prigge, 2008, p.47).
\TwoColKey
\stopTwoColKey
\startTwoColKey

\setupDescrTable
% head on every page, stretch columns
\bTABLE[split=repeat]
\bTABLEbody
\bTR
  \bTC  perceived space \eTC
  \bTC  conceived space \eTC
  \bTC  lived space \eTC
\eTR
\bTR
  \bTC  perceived space is the physical space one encounters when moving through the city. the urban reality that interlinks the daily reality, thus, the urban network of routes that links daily routine, the private life, leisure, work  \eTC
  \bTC  conceived space is space loaded with meanings and concepts \eTC
  \bTC  lived space is perceived and conceived space, the personal experience of space and social life, the physical space charged with symbols and imagination of its {\em inhabitants}, its {\em users}, its {\em producers} \eTC
\eTR
\eTABLEbody
\eTABLE
\nextTwoColKey
{\em perceived, conceived and lived spaces according to Henri Lefebvre}
\stopTwoColKey
\startTwoColKey
% hier fehlt eine referenz bei foucalt
Foulcaut  noted that {\em space} had always been just referred to a {\em geography}: the living space, the city, the state, something given, natural, land, surface. The reduction of space to something physical neglects the fact that, here particularly in the urban context, spatial structures symbolize the invisible social relations, roles, hierarchies and powers that are distributed in the physical or geographical (urban) space (Prigge, 2008, p.47) thus, 
\TwoColKey
{\em space is not only geographically determined}
\stopTwoColKey
\startTwoColKey
\quote{\em the urban is [...] pure form; a place of encounter, assembly, simultaneity. This form has no specific content, but is a center of attraction and life. It is an abstraction, but unlike a metaphysical entity, the urban is a concrete abstraction, associated with practice [. . .] What does the city create? Nothing. It centralizes creation. Any yet it creates everything. Nothing exists without exchange, without union, without proximity, that is, without relationships. The city creates a situation, where different things occur one after another and do not exist separately but according to their differences. The urban, which is indifferent to each difference it contains, . . . itself unites them. In this sense, the city constructs, identifies, and sets free the essence of social relationships [...] }(Lefebvre, 2003, in Prigge, 2009, p.49)
\TwoColKey
\stopTwoColKey
\startTwoColKey
If then the city is shaped by its inhabitants and vice versa, the deciphering of (urban) spaces can reveal the {\em spatial practices} of a society while \quote{\em spatial practice of a society secretes that society's space} (Lefebvre, 1991, in Prigge, 2008, p51).
\TwoColKey
{\em a society's spatial practice}
\stopTwoColKey
\startTwoColKey
Due to this interdependency of space, spatial practice and citizen, the production of (urban) space reflects the (social) conditions a society lives in. At the same time, the (social) conditions of a society reflect the way how (urban) space is produced. 
\TwoColKey
\stopTwoColKey
\section[globalcity]{The Deciphering of Global Cities}
\startTwoColKey
\boxframed
{
The deciphering of (urban) spaces under contemporary, thus capitalistic, conditions, reveals the processes, roles, hierarchies and power structures that \hlef also perceived in the urban development of Paris in the 60's of the last century (Harvey, 2008 and Holm, 2010). A Paris that strongly segregates citizens according to a social-economic logic that leads to their expulsion to remote ghettos, the {\em Banlieus} (Holm, 2010). \lef also noted the extinction in distinction between urban and rural spaces (or the town and the city) due to the creation of {\ integrated and transnational spaces} (Harvey, 2008), a concept which is further developed in the context of \globalcities (or \worldcities) and their prominent role as backbone of globalisation. According to Harvey, \hlef predicted that the city and its development is central for the survival of capitalism and the future (but also contemporary) space of political (and) class struggle (Harvey, 2008).
}
\TwoColKey
{\em urbanization necessary for the survival of capitalism}
\stopTwoColKey
\startTwoColKey
\boxframed
{
The \globalcity is a further abstraction of spatiality and the city that is already detached from notions such as {\em urban} or {\em rural}. The \globalcity may occupy a geographical space but its function can only be understood in the context of a network of interconnected,  thus {\em integrated} (global or world) cities. These interconnections are of {\em national and transnational} scope and need no longer be of physically nature, in form of streets, highways or railways, instead they are purely virtual. Virtual networks which allow instant information and data exchange, with the (global) cities as its nodes and hubs. \from[wikiEnCastells] describes this setup as the\spaceofdataflows rather then the \spaceofplaces even though the city itself is the interface between \spaceofflows and \spaceofplaces and won't dissolve into the virtual network (Castells, 2004, p.85)
}
\TwoColKey
{\em the global city}
\stopTwoColKey
\startTwoColKey
\boxframed
{
\from[wikiEnSassens] developed the concept of \globalcities in the her book \quote{\em The global city: New York, London, Tokyo (1991)}. She assessed that \globalcities are the strategic command and control sites of global economy which particularly agglomerate {\em key services} of the so called F.I.R.E. sectors: finance, insurance, real estate; those services serve an international customer base and are oriented to the world markets. They are also sites of production (of innovations) and offer (generate) the necessary market in order to trade them. Her primary focus on advanced services and innovations characterizes the internationalisation of production, the organisation of the division of labour on a global scale (Fröbel et al, 1980, in Beaverstock et al, 2009), i.e. production processes (often depending on cheap labour) are spatially (often globally) separated from design, innovation and management processes (Sassen, 1994 in Taylor et al, 2009), thus \globalcities are the prime sites of the knowledge elite accumulating there. 
}
\TwoColKey
{\em Saskia Sassen's global city}
\stopTwoColKey
\startTwoColKey
% space of flows and space of places
\setupDescrTable

\bTABLE[split=repeat]
\bTABLEbody
\bTR
  \bTC  space of places \eTC
  \bTC  space of flows \eTC
\eTR
\bTR
  \bTC  space of places are those places where people live in, where the daily life is experienced, which are bound to a locality, with physical and symbolic meanings, with form and function. Places are the symbols of the dominant power structures such as media, economy, technology or institutions, that exist and act in the \quote{space of flows}. The \quote{space of places} seems to have a similar notion as Lefebvre's \quote{lived spaces}. \eTC
  \bTC  spaces of flows can exist everywhere, they have no special characteristics, they are {\em non-places}. Its flows can be perceived as flows of information, capital, symbols, technology, images, social interactions, etc. The city is the interface between the local \quote{space of places} and the global \quote{space of flows}, thus is interlinks people or activities, in distinct geographical regions, worldwide, simultaneously. On the other hand, the city is one of the nodes that builds the global, electronically interconnected, network of spaces of flows, thus network of (global) cities.  \eTC
\eTR
\eTABLEbody
\eTABLE
\TwoColKey
{\em spaces of places and flows according to Manuel Castells (2004, p.83)}
\stopTwoColKey
\startTwoColKey
\definitionframed
{
{\em Advanced producer services} are those services that are produced, invented, traded and exchanged in and for the global economy. They are injected to and extracted from the global \spaceofflows and by that rely solely on the network of digital communication infrastructure, thus, their core can only consist of information and data. Those services operate on instruction, advice, planning, interpretation, strategy, knowledge, creativity, culture (Taylor et al, 2009), finance, real estate, or insurance (Sassen, 1991). 
}
\TwoColKey
{\em advanced producer services}
\stopTwoColKey
\startTwoColKey
\boxframed
{
Her work also represents a shift from the existing concepts of the \worldcity at the time. \from[wikiEnFriedmann] and \from[wikiEnWolff] (Friedmann and Wolff, 1982) described their \worldcity as command and \quote{control centres of capital flows in world economy} (Beaverstock et al, 2009). Their \worldcity has been mainly described as intense concentration of formal powers represented by major (international) corporate headquarters and institutions. \sassens \globalcity shifted the focus from the concentration of solely formal power that control (international) capital flows, to the agglomeration of services and innovations which enforce an internationalisation of production (Beaverstock et al, 2009). 
}
\TwoColKey
\stopTwoColKey
\startTwoColKey
\boxframed
{
Similar observations (of {\em institutional} power structures) have been made 3 decades earlier as well. \from[wikiEnPeterHall] described the \worldcity of his decade as
\nl
\quote{\em "[...] centres of political power, both national and international, and of the
organizations related to government; centres of national and international trade
and all kinds of economic activity, acting as entrepôts for their countries and
sometimes for neighbouring countries also." }(Hall, 1966)
}
\TwoColKey
{\em Peter  HAll's World City}
\stopTwoColKey
\startTwoColKey
\boxframed
{
The initial concept of the \globalcity has been further developed by the \from[gawc], an academic think tank on cities in globalization at \from[loughborough]. The early focus on concentration of advanced producer services as the indicator for {\em global cityness} has been substituted by measurement of \quote{network connectivity} as key indicator. According to Taylor (2009), \quote{network connectivity} defines how well a city is connected to the global network of cities, thus the global economy. It is not just the number of advanced producer service firms or offices a city possesses but the level of {\em connectivity} these sites generate by being interconnected with other sites in other cities.  The level of connectivity is translated into a rooster of world cities whose highest level defines pure world cities which have the highest level of advanced producer service connections to other cities, on a global scale, thus, they are the most important {\em command and control sites} of global economy, those nodes with the highest capacity in the global {\em space of flows}. 
}
\TwoColKey
{\em GaWC's World City Rooster}
\stopTwoColKey
\startTwoColKey
\boxframed
{
The accomplishment of \worldcity status leads to a phenomena emblematic for contemporary globalization: the competition among cities on a global scale (Taylor et al, 2009). On the one hand, cities try to construct their global image in order to attract more services, more investment, more highly skilled information workers; on the other hand, cities try to attract for example mega-events such as Olympic Games, Football World Championships, conventions and other spectaculars. Those events are often the trigger for large scale (gigantic) urban development projects which might completely change the face of the city in a relatively short time frame, i.e. {\em revitalization} of neighbourhoods and central city areas that result in the implicit {\em displacement} of old established neighbourhoods by emerging gentrification and the accompanied increase of living costs or direct {\em displacement} of neighbourhoods and slums by \quote{slum clearance} through \quote{beautification} projects (Greene, 2003, p.163). Further on, the implemented urban development schemes are then exported and replicated worldwide. One of the prominent examples is the so called \from[barcelonaModel] which has been realized as {\em flagship urban development project} for the Olympic Games of Barcelona in 1992. It attracted massive private investments and caused among others \quote{the revival of the real estate market [which] was rapid and voracious, from the Olympic nomination in October 1986 to the middle of 1990 [...] The market price of new and previously-built housing between 1986 and 1992 grew, respectively, 240\% and 287\%} (Brunet, 1995, p.17). The \quote{Barcelona Model} serves as a blueprint for contemporary urban development in the context of mega events, most current for the preparations of the Olympic Games that will be held in \from[rioOlympicGames] in 2016 (Fox, 2010).
}
\TwoColKey
{\em urban development and mega events}
\stopTwoColKey
\section[waterfront]{Waterfronts everywhere}
\startTwoColKey
\boxframed
{
Other large scale urban development schemes have been adopted worldwide as well (for European projects, see for example {\em Les Cahiers no 147(2007)} ), such as the so called \waterfrontdevelopment in port cities (Butuner, 2006, p.3), the \revitalization the degenerated port areas (or \brownfields, the general term for degenerated industrial areas). \Waterfront \revitalization is used here as emblematic example for a \quote{ genuine urban revolution} (Bruttomesso, 1993, p.10). Many of the world's contemporary large and big cities are port cities due to the former importance of a settlement's proximity to the water and the port as trigger for the emergence of transportation, trading, manufacturing or military activities and by that also for physical/geographical expansion and residential growth. The contemporary transformation of the waterfront, which represents the interface between the city and the water, is therefore a new  \quote{transformation in [the city's] physical layout, function, use and social pattern} (Butuner, 2006, p.3).
}
\TwoColKey
{\em waterfront revitalization}
\stopTwoColKey
\startTwoColKey
\boxframed 
{\Revitalization of those often run down or abandoned industrial \waterfronts and port areas that are the last evidence of the downfall of (heavy) industries and manufacturing in inner city regions, serves several purposes: from a pragmatically point of view, degenerated inner city \brownfields (from ports to fabrics) are often the last vast and unused spaces available in inner city regions. The \waterfront and port areas are of special interest due to their prime location on to the water (rivers or the sea) and their proximity to the inner city. A revitalized and spectacular \waterfront has therefore the potential to boost the local tourism industry, to control the resident and living structure by attracting a more exclusive class of citizens, businesses and services but also to improve the \quote{image} of the city on national and global scale and to position the city's \quote{label} in the {\em globalcityness} competition (Butuner, 2006, p.3). Hence, a \revitalized \waterfront is evidence of the  transformation of an industrial city to a post-industrial city, whose new mode of production is based on services, culture or tourism. 
}
\TwoColKey
{\em \waterfront \revitalization can be observed globally in Amsterdam, Barcelona, London, Venice, Sidney, Toronto, Bangkog, Honh Kong, Shanghai, Tokyo, Cape Town, Buenos Aires (Bruttomessi, 1993), Havana or Bilbao (Marshall, 2001)
}
\stopTwoColKey
\startTwoColKey
\boxframed
{
Since its early realizations in (post industrial) cities such as Boston or San Francisco (Marshall, 2001, p.118, 119) in the late 60's, early 70's of the 20th century, \waterfront \revitalization has been used as a generic and standardized tool (Butuner, 2006, p4 and Bruttomessa, 1993) that has been and still is applied in many cities in the global north and south.
}
\TwoColKey
\stopTwoColKey
\startTwoColKey
\boxframed
{
Even though the local context and times always differ, similar transformations in a city's physical and social structure can be observed as a result of \waterfront \revitalization:
}
\TwoColKey
\stopTwoColKey
\startTwoColKey
\boxframed
{
{\bf social \segregation}

due to the fact that the revitalized waterfront may become an exclusive life, work, leisure and tourism zone, that relies on a correspondingly exclusive real estate, consumption and service infrastructure, accessible only for those who can afford it (Höhmann, 2010; Niedt, 2006, p.110) 
}
\boxframed
{
{\bf social \exclusion and \surveillance} 

in order to guarantee that unwanted subjects, from homeless people to skaters, are not frequenting the area and disturb its shiny and exclusive image (IVV, 2010, p19).
}
\boxframed
{
{\bf \gentrification} 

of the revitalized waterfront, the displacement of current residents (if existent) by a wealthier class due to increased living costs (Bischof, 2007, p.64; Hogan, 2006, p.31, Niedt, 2006, p.110) 
}
\boxframed
{
{\bf \newbuildgentrification}

that affects neighbouring quarters. No direct displacement of (old) residents takes place on the previously not inhabited \waterfront but the new lifestyle that emerges after \revitalization may leak to neighbouring quarters and cause \gentrification there (Davidson and Less, 2005 p.1168, 1169).
}
\TwoColKey
{\em some effects of \waterfrontdevelopment}
\stopTwoColKey
\section[waterfrontExample]{The waterfront in development}
\startTwoColKey
\boxframed
{
Contemporary \waterfront revitalization projects are realized for example in Hamburg, Cologne, Rio de Janeiro or Istanbul and are discussed very briefly in the following paragraphs. 
}
\TwoColKey
\stopTwoColKey
\startTwoColKey
\boxframed
{
\from[wikiEnHamburgs] \from[hafencity] represents a large scale neoliberal inner city urbanisation project, realized in an once fallow \from[hafencityPortArea]. Until the 1990's, all port activities had been shifted from the inner city edge of the river Elbe, the southern edge, to the northern edge and left large areas of fallow land (Bauriedl, 2007). 
}
\boxframed
{
The \hafencity is designed to push the \quote{label Hamburg} on the international market of competing (european and world) cities (Bischoff et al, 2009; Hafencity, 2010, p.4). The \hafencity shall not only strengthen the city's competitiveness but strives to gain recognition as the inner city urban development blueprint for the 21th century European city (Hafencity Masterplan, 2006, p.8). It is supposed to attract and profile the creative and digital, thus information based, industries (Hafencity Masterplan, 2006, p.14). The focus on advanced producer services is expressed with the intended construction of 1 million m² of office space, which could serve for 35 to 50 thousand working places and could attract further public and private investment (Hafencity Masterplan, 2006, p.6). 
}
\TwoColKey
{\em Hamburg's \hafencity}
\stopTwoColKey
\startTwoColKey
\boxframed
{
Mixed (here: demand driven) forms of living, tailored for different needs, are promoted, with a special focus on up-scale housing (Hafencity Masterplan, 2006, p.14). Besides promoting (exclusive) living on the waterfront, the \hafencity tries to gain recognition as tourist attraction with an \from[hafencitySeaport] and as the site of high-end leisure amenities and landmark architecture, such as the gigantic Opera building, the so called \from[hafencityElbphilharmonie], one of the modules of the waterfront development toolkit that has been previously applied for example at Sydney's waterfront, which is well-known worldwide for (or represented by) its \from[wikiEnSydneyOpera] building. 
}
\boxframed
{
The real impact of the \hafencity on Hamburg's social and urban structure remains to be seen. The \hafencity is still work in progress, with an intended construction period of 25 years. Already visible today is a beginning segregation of local inhabitant structures, where \from[wikiEnCondos] are the pre-dominant form of housing while social and affordable housing is almost not apparent (Statistikamt Nord, 2007, p.3-4, Statistikamt Nord, 2008, p.231-232).
}
\TwoColKey
\stopTwoColKey
\startTwoColKey
\boxframed
{
Another example of \waterfront \revitalization in Germany can be witnessed in \from[wikiEnColognes] \from[rheinauhafen]. The \rheinauhafen, once an inland transportation, stock turnover and trading hub(Dietmar, Rackoczy, 2002), well located in the old city centre, is now splurging with gigantic landmark architecture (like Hamburg's \from[hafencityElbphilharmonie]) in form of three \from[wikiDeKranhaus] buildings. The harbour experienced its decline in the 70's of the 20th century. In 1976, the city of Cologne finally decided to put it out of service until 2000 in order to revitalize the area (Blenck et al, 2001).  Until then, all port  activities had to be shifted to newly constructed harbours outside the city, to its northern and the southern extensions (Blenck et al, 2001). Despite the decline of the port, the area has never been abandoned completely. Several single cultural projects had been realized in the 1990's such as the {\em Immhof Schokoladenmusem} in 1993 or the {\em German Sport and Olympic Museum} in 1999 (Dietmar, Rackoczy, 2002). The initial contest for \revitalization proposals was hosted in \from[rheinauhafen1992] (Modernes Köln, 2010).
}
\boxframed
{
The \rheinauhafen is designed to \from[rheinauhafenIntegrate] with a mix of housing, office space, amenities and other consumption infrastructure (RVG, 2010). Hence, the \rheinauhafen is supposed to revaluate and promote the inner city and transform the area into a lively district. 
}
\boxframed
{
The waterfront is promoted almost completely as exclusive living space: the \from[kranhausPandion] shall finally be composed of 135 luxury condos while the former warehouse \from[rheinauhafenSiebengebirge] and the newly constructed  \from[rheinauhafenWohnwerft] have already sold their complete stock of apartments (RVG, 2010, RVG, 2010). Even though rented apartments are offered as well, their potential renters are limited to those who can afford high rents and living space beyond 100 square meters (Immonet, 2010).
}
\TwoColKey
{\em Colognes's \rheinauhafen}
\stopTwoColKey
\startTwoColKey
\boxframed
{
Comfortable living shall be guaranteed by Cologne's largest underground parking deck (Stredich, 2010) or the privately owned \from[rheinauhafenSporthafen] which is operated by Cologne's Marina Club (Pesch, 2008). Security and protection against unwanted subjects (IVV, 2010, p.18) is enforced by private security agencies, surveillance infrastructure and the living in gated communities (here: entrance permitted by a janitor) (Immonet, 2010). 
}
\boxframed
{
Besides housing, the \rheinauhafen is offering large amount of office space for advanced producer services. Its main aim is to attract the creative and digital industries: a hub of hightech, information based, companies is already emerging, composed of branch offices of international companies such as {\em Electronic Arts} or {\em Microsoft} (RVG, 2010).
}
\boxframed
{
Similar to the \hafencity, the \rheinauhafen is work in progress, although in its last state. A segregated inhabitant structure can already be observed: social and affordable housing is absent while expensive and large apartments for a correspondingly potent clientèle are the norm (Höhmann, 2010). Potential effects such as \newbuildgentrification in neighbouring quarters like the \from[wikiDeSeverinsviertel], due to the clustering of hightech, creative, cultural and tourism industries and upscale housing, remains to be seen. Even though the \rheinauhafen occupies a huge area of public space, its realization mainly focus on the construction of a new elitist neighbourhood that hosts hightech industries, housing for the hightech working force and correspondingly exclusive leisure amenities.
}
\TwoColKey
\stopTwoColKey
\startTwoColKey
\boxframed
{
Several \waterfront development projects have been realized in \from[wikiEnIstanbul] since the 1980's. In contrast to port cities in post industrialized countries, where inner city ports turned into abandoned \brownfields, ports in Turkish cities are still active (Butuner, 2006, p.04). One of those ports that is intended to be \revitalized is the \from[galataPort] which is located in the historic \from[wikiEnGalata] quarter.
}
\boxframed
{
\from[galataPort] witnessed its decline as trading hub in the 1980's when it became unsuitable for trucks and vessels due to the increase of inner city traffic. Its mode of operation was therefore reduced to passenger transport only. Since then, the port couldn't fulfil this function (Butuner, 2006, p.08). 
}
\boxframed
{
An attempt to turn it fully operational is the \from[galataPortProject]. The intended \revitalization of the port shall convert the area into a leisure zone that represent a prestigious tourist entry to the city (Butuner, 2006, p.08). A modern cruise terminal shall emerge where, according to official estimations, an estimated 12 million tourists could arrive through the water ways over the course of a year (Butuner, 2006, p.08). The necessary consumption infrastructure is expected to boost commercial and tourist industries. Hence, the general aim of \from[galataPortProject] is the (re)definition of Istanbul's image on national and international scale in order to develop a new identity known for its cultural, tourist and commercial strength (Shafik \& Steele, 2010, p.07).
}
\TwoColKey
{\em Istanbuls \waterfront: the Galata Port Project}
\stopTwoColKey
\startTwoColKey
\boxframed
{
Due to the fact that the project has not been realized so far, it remains to be seen how it would effect the \from[wikiEnGalata] area. The project follows the logic of \waterfront \revitalization as catalyst for the construction of Istanbul's global image and the development of a strong tourist and commercial industry. From a contemporary real estate point of view, \quote{the long-awaited Galataport project is the most important factor driving prices exponentially up in the neighbourhood} (Kalkavan, 2010). In contrast to the institutional point of view, residents and NGO's are criticizing "\quote{the content and scope of the project} (Butuner, 2006, p.08) as destructive. An urban transformation and renewal policy, as it has been set-up in 2005, grants full authorization to municipalities in order to initiate and realize urban development projects (Tan, 2007, p.08). This power has already been used in other districts, such as the Sululuke district for example, to evict and displace the local Gypsy community and demolish their houses under the umbrella of urban development (Tan, 2009).
}
\TwoColKey
\stopTwoColKey
\startTwoColKey
\boxframed
{
The last example is going to take a look into the near future. \from[wikiEnRio] is going to be one of the main sites that host the \from[rio2014] and \from[rio2016] in 2014 and 2016 respectively. In the course of the preparation of those spectaculars, one large scale urban development project will be the total transformation of the old \from[rioPortArea]. The construction of the port has been part of the urban reform that took place in Rio between 1903 and 1906. Pollutive industries settled in the port area but until the 1940's it turned into a non-place. Industries shifted to neighbouring quarters such as São Cristovão, along the rails, and finally to the peripheries of the city. (Ferreira Santos, 2005) The main port activities can nowadays be found in the \from[rioPortoSepetiba] (da Silva, 2009). With the construction of several main roads, the port area has been completely separated from central area (Ferreira Santos, 2005).
}
\boxframed
{
The deteriorated port areas are nowadays the site of the struggle of city dwellers, urban social movements and the speculative real estate industries. Since the 1950's, the relocation of port activities, the loose of economic and industrial importance to São Paulo, the loose of status as capital and the legal prohibition to settle in the central area enforced by \from[wikiEnFavela] cleaning and demolition of \from[corticios], left vast abandoned spaces and buildings in the central area (da Silva, 2009).
}
\TwoColKey
{\em the old port in Rio de Janeiro}
\stopTwoColKey
\startTwoColKey
\definitionframed
{
 \from[corticios] are the predecessors of \from[wikiEnFavela]. They are the places and agglomeration of houses where the excluded live, the oppressed, all those who don't mix with the bourgeoisie (Azevedo, 1890).
}
\TwoColKey
{\em cortiços and favelas}
\stopTwoColKey
\startTwoColKey
\boxframed
{
With the announcement of two mega events, the port area is intended to be reintegrated into the cities structure. The revitalization of the area is concentrated around several cores: Transport; Technology, Communication; Habitation; Environment; Tourism; Culture; Leisure.
}
\TwoColKey
\stopTwoColKey
\startTwoColKey
\boxframed
{
These urban interventions follow the \from[barcelonaModel] (Barcelona Bulletin, 2009), thus the logic of \worldcityness. The city is supposed to be transformed into a city of spectacular that offers exclusive living, world class tourist amenities such as an international museum (Ferreira Santos, 2005) and a shift from the informal, tertiary, sector to advanced producer sector (Prefeitura Rio, 2010). In order to accomplish these goals, among others, city dwellers in risk areas are supposed to be resettled (Rio Negócios, 2010b) while urbanization projects shall take place in various favelas (Rio Negócios, 2010a). 
}
\TwoColKey
\stopTwoColKey
\startTwoColKey
\boxframed
{
A change in the cities social and urban structure has already been addressed by \from[rioSocialMovementMoradia], \from[occupationsRio] and favela inhabitants (NPC, 2010). A constant conflict with the agents of real estate speculation and the police is emerging (Pela Moradia, 2010) due to intimidations and intended displacements from the central areas (Marques, 2010)
}
\boxframed
{
\quote{\em We see: 130 favelas that are scheduled to be removed until the Olympic Games. Millions of evictions and removals are necessary for the construction of three big highways (Transcarioca, TransWest e TransOlympic). All 73 units of Metro land, all located in regions with infrastructure, will be sold in order to make space for the new Metro lines instead of using them for public housing. The port zone, where around 70\% of the land is public, has been targeted by the Olympic plans as well, in order to enforce the \gentrification of the region. Security policies, including UPP's (unidades da polícia pacificadora - peace making police units), are prioritising the creation of pacified zones (and walls) around Olympic infrastructure and the tourist routes to and at revitalized areas.} (Marques, 2010)
}
\boxframed
{\tfx 
original quote: \quote{\em Vejamos: estão previstas remoções de 130 favelas até as Olimpíadas. Para a construção de 3 grandes vias rodoviárias (Transcarioca, Transoeste e Transolímpica) serão necessários milhares de despejos e remoções. Os 73 terrenos do Metrô, todos em áreas com infraestrutura, ao invés de usados para habitação popular, serão vendidos para fazer caixa para o metrô prometido ao COI. A Zona portuária carioca, onde cerca de 70\% do solo é público, também entrou nos planos Olímpicos, para reforçar o projeto de aburguesamento da região. A política de segurança, o que inclui as UPPs, tem como prioridade criar zonas de paz (e de muros) nos entornos dos equipamentos esportivos, nas vias de acesso dos turistas a esses equipamentos e nas áreas valorizadas ou em vias de valorização.} (Marques, 2010)
}
\TwoColKey
\stopTwoColKey

\section[worldcityHowsHow]{What is a worldcity anyway}

\startTwoColKey
 \boxframed
{
Even though the abstract manifestations of the \worldcity are difficult to grasp, in daily reality one can nonetheless translate the abstract to the real. The examples of \waterfront development have shown that not only so called \worldcities such as New York or London are affected by or enforce (mega) urban development processes and models, but that smaller cities and cities in the global south are replicating them as well, as it has been seen in the case of Cologne and Rio de Janeiro.
}
\TwoColKey
\stopTwoColKey
\startTwoColKey
\boxframed
{
In another oversimplified \worldcity sketch one could draw connections from the concentration of advanced producer services that reside within a city, to the necessity of skilled and ultra mobile knowledge workers that organize these services and produce innovations. The services and workers to attract, demand a safe city and high class cultural amenities, probably their own districts or quarters, perhaps newly constructed as part of (mega) {\em urbanisation projects} that \revitalize \brownfields or displace long established quarters, perhaps {\em gentrified} or {\em cleaned} from \quote{unwanted} subjects. Due to the effects of the city on its hinterland (the integration of urban and rural spaces but also the transnational effects caused in other world regions), a rising stream of migrants on national and international level can be observed, which are caused by devastating changes in rural areas provoked by i.e. large scale \from[wikiEnAgribusiness], commanded and controlled from and within the network of (world) cities. Arriving in the city, there exist often no access to the city's {\em lived space}, i.e. due to the lack of necessary education, different languages or valid papers and therefore the denial of access to learning facilities or work which would help to guarantee a decent living. The result is further {\em precarization} of living conditions i.e. in the peripheries of the cities or informal settlements; the necessity to work informally with corresponding institutional repression; a rising fraction of population dumped on the streets. 
}
\TwoColKey
\stopTwoColKey
\startTwoColKey
\boxframed
{
This sketch is by far not complete nor does it claim to be perfectly accurate. It is obvious that different cities in different geographical regions are sometimes shaped by similar, sometimes by different processes. \Castells notes that each city is always connected, up to a certain extend, to the global \spaceofflows or the global network of (world) cities, thus each city possesses a smaller or larger fraction of \worldcityness and by that experiences often similar effects and transformations of their \livedspaces (Castells, 2004).
}
\TwoColKey
{\em world city or worldcityness}
\stopTwoColKey
\section[accumulation]{Accumulation of Dispossession}
\startTwoColKey
%\boxframed
%{
Returning to the \rttc, David Harvey describes the coherence of contemporary urban development processes as \quote{accumulation of dispossession}. In \from[newLeftReviewRttc] (2008) he argues that
\quote{\em the global urbanization boom has depended, as did all the others before it, on the construction of new financial institutions and arrangements to organize the credit required to sustain it. [...] urbanization, we may conclude, has played a crucial role in the absorption of capital surpluses, at ever increasing geographical scales, but at the price of burgeoning processes of creative destruction that have dispossessed the masses of any right to the city whatsoever.} (Harvey, 2008)
\TwoColKey
{\em accumulation of dispossession}
\stopTwoColKey
\startTwoColKey
A small part of the city's inhabitants benefit from the (planned) changes that reshape the city's structure. For the remaining majority, the social conditions are worsening, because they are dispossessed from the freedom to freely live in, move around, use the city, dispossessed from their homes, quarters and neighbourhoods, their social networks, culture, work, from the right to shape the city, thus dispossessed from any \rttc. 
\TwoColKey
\stopTwoColKey
\startTwoColKey
This total transformation of the city engendered a new lifestyle, a lifestyle of consumerism in all possible variants, which turned quality of life and the city itself into a commodity while the defence of property became of major political interest. These \quote{new} norms can be identified in the massive spatial fragmentation of the city, in the {\em segregation} of the society and the displacement of its members, may it be due to {\em gentrification}, repression, state led {\em requalification and revitalization} or complete renewal of city quarters, which all feed into the \quote{\em accumulation by dispossession [which] lie at the core of urbanization under capitalism} (Harvey, 2008). {\em Displacement and dispossession} take thus place in {\em perceived spaces} but also in {\em conceived and lived spaces} where a dispossession of any sense of affiliation to a (however defined) society takes place.
\TwoColKey
\stopTwoColKey
\section[returning]{Returning to Lefebvre: the right to the city}
\startTwoColKey
In contrast to \harvey who approaches the \rttc through a political economy of space (Kipfer et al, 2008, p.08), for example in terms of newly invented financial tools necessary for large scale urbanization, \hlef \quote{promoted the developed of a comprehensive theory of the production of space}  (Schmid in Kipfer et al, 2008, p.08). Concepts such as the \globalcity and urban displacement processes are immanent to \lefs conceptualization of the contemporary (western, post-war, industrialized) city, which has been the Paris of his epoch (Stanek, 2008, p.72).
\TwoColKey
\stopTwoColKey
\startTwoColKey
\Segregation and \centrality are two contradictory processes at first glance, that define the \productionofspace and the reproduction of social relations in the contemporary city. 
\TwoColKey
{\em segregation and centrality}
\stopTwoColKey
\startTwoColKey
In the city, everything is separated, segregated and (socially) unconnected, the spaces specialized, functions, labour and society are divided (Stanek, 2008, p.71). Urban space as social product and without specific content means \centrality: a space of \quote{association and encounter of whatever exists together in one space at the same time}, a space where dominant power is centralized, a space where oppositional struggle against the dominant spacial and social form is centralized, thus urban space is the simultaneity of everything that can be brought together at one point (Kipfer et al, 2008, p.291). 
\TwoColKey
\stopTwoColKey
\startTwoColKey
\Centrality as seen from the perspective of the city as a whole means the assimilation of the urban and rural, the peripheral elements that surround the city because the city itself concentrates power and produces the highest power: the decision (Kipfer et al, 2008, p.291). The decision of a city in the global \spaceofflows means decision making on global scale, thus \hlefs \centrality can easily be translated to the concept of the \globalcity.
\TwoColKey
{\em centrality and the global city}
\stopTwoColKey
\startTwoColKey
\quote{\em The ideal city, the New Athens, is already there to be seen in the image which Paris and New York and some other cities project. The centre of decision-making and the centre of consumption meet. Their alliance on the ground based on a strategic convergence creates an inordinate centrality. We already know that this decision-making centre includes all the channels of information and means of cultural and scientific development. Coercion and persuasion converge with the power of decision-making and the capacity to consume. Strongly occupied and inhabited by these new Masters, this centre is held by them. Without necessarily owning it all, they possess this privileged space, axis of a strict spatial policy. Especially, they have the privilege to possess time. Around them, distributed in space according to formalized principles, there are human groups which can no longer bear the name of
slaves, serfs or even proletarians. What could they be called? Subjugated, they provide a multiplicity of services for the Masters of this State solidly established on the city.} (Lefebvre, 1996, in Kipfer et al, 2008, p.291)
\TwoColKey
\stopTwoColKey
\startTwoColKey
\Segregation and \fragmentation of the urban space means \centrality on the ground level. 
\TwoColKey
\stopTwoColKey
\startTwoColKey
\quote{\em [...] And yet everything (“public facilities,” blocks of flats, “environments of living”) is separated, assigned in isolated fashion to unconnected “sites” and “tracts”; the spaces themselves are specialized just as operations are in the social and technical division of labour} (Levebfre in Stanek, 2008, p.71)
\TwoColKey
\stopTwoColKey
\startTwoColKey
Urban space becomes \gentrified, \revitalized and \transformed, land becomes parcelled, the city is zoned and segregated (Stanek, 2008, p.72). Those fragments the city consists of are \homogenized, thus every difference within them is eliminated which leads to hierarchization of urban spaces and within society (Stanek, 2008, p.72). \lef differentiates between two forms of \difference, {\em minimal} and {\em maximal difference} (Kipfer, 2008, p.203). 
\TwoColKey
{\em elimination of the difference}
\stopTwoColKey
\startTwoColKey
Kipfer (2008, p.202-208, 296) notes that, \mindiff (or \indifference) produces the capitalist form of life and is produced by \quote{\em appropriating and monopolizing the urban space as a productive force, [...] commodified festivity, racialized suburban marginality, multiculturalized ethnicity, [...] parcelized social spaces planned in vulgar modernist fashion, [...] alienations of property, individualism and group particularism}. The city of \mindiffs is the city of resorts, university compounds, gated communities, slums, exclusive districts or working class quarters, but also the city of \estdiffs, the partiarchal family, the reproduction of \quote{established} norms (discursively fixed identities or according to natural distinguishing characteristics) onto women, immigrants, the poor, the old, the ill and their expulsion to dedicated (often peripheral) urban regions (favelas in Brasil, Banlieus in France, social hotspots in Germany), public housing tracts or specialized institutions.
\TwoColKey
{\em minimal difference}
\stopTwoColKey
\startTwoColKey
To the contrary, \maxdiff is \proddiff, thus fully lived forms of plurality and individuality, an articulated identity based on rich social relations and not affected by any form of \indifference. It is the quest for a unalienated, festive, creative, self-determined, fully lived urban society (Kipfer, 2008, 203) that is not forced into a space that was produced only for the purpose of discrimination (Kipfer et al, 2008, 293).
\TwoColKey
{\em maximal difference}
\stopTwoColKey
\startTwoColKey
\Maxdiff is incompatible with \mindiff (Kipfer, 2008, 203). Due to this incompatibility, a transformation of \mindiffs to \maxdiffs will result in the \rttc. How this transformation can be achieved remains to be seen. Existing spaces of (at least intended) \maxdiff and alternative practices such as social centres, squats, or self managed enterprises have not reached sufficient weight so far, thus, according to \lef, a \quote{\em transformation can be achieved only by [more] social struggles for political self-determination and a new spatial centrality, which help liberate difference from the alienating social constraints produced by capital, state, and patriarchy.} (Lefebvre, 1970 in Kipfer et al, 2008, p.08)
\TwoColKey
{\em transformation of minimal to maximal difference}
\stopTwoColKey
\startTwoColKey
The \rttc is therefore complemented by the {\em right for difference}. These rights are not of normative nature, thus rights granted by institutions (the right or obligation to vote) which do not prevent social, economical and cultural exclusion (Gilbert and Dikeç, 2008, p.258), but rights that area undetachable human properties, \quote{defined and redefined by political action, social relations [...] and the sharing of space} (Gilbert and Dikeç, 2008, p.258,259).  The continuous re-negotiation of those rights essentially means the active participation in societies self-management (Gilbert and Dikeç, 2008, p.260) where \quote{\em each time a social group refuses passively to accept its conditions of existence, of life or of survival, each time such a group attempts not only to learn but to master its own conditions of existence.} (Lefebvre, 1996 in Gilbert and Dikeç, p.260).
\nextTwoColKey
{\em the right to the city according to \hlef}
\stopTwoColKey


%(261)
%We have tried to advance three main arguments on the nature of the right to the city
%and its relation to contemporary debates and issues around citizenship. First, as
%Balibar’s remarks on the LOV imply, the advancement of a right to the city calls for
%major changes in the structural dynamics that produce urban space. Unless the forces
%of the free market, which dominate—and shape to a large extent—urban space,
%are modified, the right to the city would remain a seductive but impossible ideal for
%those who cannot bid for the dominated spaces of the city; those, in other words,
%who cannot freely exercise their right to the city.


%lesen (207)
% alternative micropractices (centri sociali, squats, self-managed enterprises) have provided only
%uneven counterweight to the forces of neo-imperial capitalism.Whether this indicates
%that in metropolitan centers of decision-making, power is based not only on coercion
%but on persuasion, as Lefebvre suspected long ago, we cannot really know without
%linking the subjectivities of radical politics to an urbanized problematic of hegemony.



\chapter[conclusion]{Conclusion}

\startTwoColKey
The concept of the \rttc according to \hlef and its further development shows clearly that this {\em right}, which is supposed to be immanent to every person, represents an utopian model of society which will be completely different in its modes of organisation, city shaping, its forms of interpersonal communication and attitudes, probably in everything we have learned an accepted as \quote{naturally} given or as norm. If we consider the world and its societies as it is, the \rttc won't be accomplished by mere waiting and the acceptance of institutionalised appropriation of this concept, which, by its core concept, shall lead to a self-organized society. This excludes a normative granting of rights, as it has been laid out previously, by default.
\nextTwoColKey
\stopTwoColKey
\startTwoColKey
It can already been seen in contemporary discourse, mainly on institutional level such as the \quote{\em Right to the city} discourse of UNESCO/UN-Habitat, that the used concept of the \rttc is often condensed to a right that is again granted to the citizen by a however defined city administration:
\nextTwoColKey
\stopTwoColKey
\startTwoColKey
\quote{\em The right to the city signifies societal ethics cultivated through living together and sharing urban space. It concerns public participation, where urban dwellers possess rights and cities—city governments and administrations—possess obligations or responsibilities. Civil and political rights are fundamental, protecting the ability of people to participate in politics and decision-making by expressing views, protesting and voting. The exercise of substantive urban citizenship thus requires an urban government and administration that respects and promotes societal ethics. It also demands responsibilities of citizens to use and access the participatory and democratic processes offered.} (Brown, 2009, p.17)
\nextTwoColKey
{\em management of social transformations}
\stopTwoColKey
\startTwoColKey
Another often cited example is the Brazil City Statute from 2001. Even though not as comprehensive as the \rttc, it grants, among others, (homeless) city dwellers the right to make use of (occupy) empty buildings that actually do not serve any social function if abandoned since years, and which force the city to invest in the city to improve the living conditions of its citizens (da Silva, 2009). In the light of massive repression against urban social movements, occupations and favelas in Brazilian cities alone in 2010, its is obvious that those rights merely exist on paper and have no impact in daily reality (two of 4 big squatted buildings that have been occupied in October 2010 by around 3800 people in the centre of São Paulo, are already evicted and its people expulsed back to the peripheries of the city; the intended \revitalization of the centre of Rio de Janeiro for the comming mega events and its already beginning repression has already been lain out briefly). 
\nextTwoColKey
\stopTwoColKey
\startTwoColKey
The analysis of \worldcityness, the city's function in the global \spaceofflows that determines globalization, and the resulting processes visible in the city's urban and social structure may have indicated (to a certain extend) how the contemporary city is shaped and that the \rttc cannot be achieved under contemporary (capitalistic) conditions.
\nextTwoColKey
\stopTwoColKey
\startTwoColKey
How the \rttc can be accomplished then, by the citizens on the streets, remains to be seen. The social struggle that \hlef reckoned in order to achieve the \rttc to its full extend is slightly visible in the struggle of urban social movements, campaigns and grassroots groups. Even though they are already focusing on the concept of the \rttc in various flavours, only a few go currently beyond the claim (often still addressed to the cities governments) for decent housing or the right to work in the centre. 
\nextTwoColKey
\stopTwoColKey
\startTwoColKey
\quote{\em the question of what kind of city we want cannot be divorced from that of what kind of social ties, relationship to nature, lifestyles, technologies and aesthetic values we desire. The right to the city is far more than the individual liberty to access urban resources: it is a right to change ourselves by changing the city}.(Harvey, 2008)
\nextTwoColKey
\stopTwoColKey

%...........................................................................................................................
% references
%...........................................................................................................................

%++++++++++++++++++++++++++++++++++++++
% reference::content

\chapter[ref]{References}
\nl
\myRefs

%...........................................................................................................................
% links
%...........................................................................................................................

%++++++++++++++++++++++++++++++++++++++
% links::content

\chapter[links]{Links}
\nl
\myLinks

% end text
\stoptext