\startcomponent component_Methodology_WhoAmI_Relevance
\product product_Thesis
\project project_MasterThesis

% definitions and macros
\environment envThesisAllEnvironments

\define[]\sectionRelevance{\subsection[methodology::whoami::relevance]{Relevant for whom and what?}}

\starttext

\startmode[tocLayout]

\sectionRelevance

This section argues about the relevance of \quote{relevance in research} and contests its usual meaning arguing that research can be relevant even if it is not a breaking new topic but that it is relevant if it is relevant for the people that collectively conduct research. Here, four questions regarding relevance are formulated which are going to be tackled in the following chapters.

\stopmode

\startmode[draft]

\sectionRelevance

\inright{ relevance of research and social struggle}
In order to begin, the question of \quote{relevance} shall be contested as well as the question of \quote{feasibility}, the form of publication and access of knowledge or the frame in which this thesis has been written. From my point of view, those points are all interrelated and interdependent. 

In order to construct viable methodology approaches and argumentation lines, I would like to take a set of questions into account, each of them related to the notion of \quote{relevance}. Those questions haven been posed by Don Mitchell in {\em What Makes Justice Spatial? What Makes Spaces Just?} \aQuoteB{Brown et al}{2007}{8} as follows:

\startCitation
\inright
{ 
\startitemize[packed, fit]
\sym{} how becomes?
\sym{} why should?
\sym{} what goals?
\sym{} for whom? 
\stopitemize
}
Mitchell’s work goes beyond calls for social or political \quote{relevance} in research and practice by reminding us that determinations of relevance always unfold in a historical and professional matrix. With Lynn Staeheli, he has written that calls for relevance in professional practice \quote{cannot be separated from questions about why research should be relevant, how research becomes relevant, the goals of research (including political goals), and the intended audiences and beneficiaries of research} (Staeheli and Mitchell 2005: 357). Those questions of why, how, what and for whom also lie at the heart of any movement for justice.

\stopCitation

Those questions may serve as a solid base for further explorations in order to determine the motivation(s) and realization of this thesis. The answer to them may affect the way how individual elements of this thesis are going to be shaped, but may also determine the thesis' abstract layers in terms of applied theories, objectives or the utilized language.

\spaceHalf
\startQuestionBack 
\startitemize[3]
\item Why should research be relevant?
\item How does research becomes relevant?
\item What are the goals of research?
\item Who is the audience and who are the beneficiaries?
\stopitemize
\stopQuestionBack 

\subject{Actions}

\startKeywords 
participant observation, participating observer, grounded theory, action research 
\stopKeywords

After all those reflections on attitudes, a thesis self conception and determinations of \quote{relevance} aspects, I would like to describe those research action principles that has been followed in São Paulo and guided me through the writing of this thesis.

As already mentioned in the introductory chapter \refMissingSrc{Methodology Introduction}, I came to São Paulo without prior knowledge of the city and without concrete but diffuse ideas about the purpose of my stay. So to say, the city drove me into the direction which is finally illustrated by this thesis. Most experiences and information approached me in a non-systematic manner because I was not looking for particular events nor situations or people, but tried to absorb and keep hold of everything.

\spaceHalf

\inright{participant observation (PO) }
For me, this premise seems optimal for conducting \aKeyword{Participant Observation }(PO). Even though PO serves as the basic framework for realizing research action(s), it does not appear to completely grasp the actions intention and their realization. As explained in more detail later on, PO can be seen as a method to \bracket{passively} observe scenes and situations. Therefore I would like to extend the concept of Participant Observation with the concept of \inright{participating observer} the \aKeyword{Participating Observer}, who actively participates in concrete action(s) and events as well, and by doing so, has an active impact on the form of those observation(s) that he or she wants to keep note of. 

\startARemark
bin noch nicht sicher ob Grounded Theory teil der arbeit ist
\stopARemark

\inright{grounded theory (GT)}
A portion of \aKeyword{Grounded Theory} (GT) is apparent here as well, because GT does not pre-suppose anything when starting research and incorporates all available types of information in the formulation of a research objective, thus in a sense, it is driven by (or grounded on) every imaginable type of action(s) and observations, on the streets, to the newspapers and the internet.

Finally, I would like to fuse those three approaches, \aKeyword{Participant Observation}, \aKeyword{Participating Observer} and \aKeyword{Grounded Theory}, under the umbrella of \aKeyword{Action Research} (AR). The intentions and objectives inherent to AR are to a certain extend the intentions and objectives of this thesis. Especially the contention of the notion of \quote{knowledge} and its \quote{production}, that is articulated through AR, match the defined thesis' attitude to upgrade \quote{universally accepted} knowledge production mechanism driven by scholarly and academic discourse with the knowledge produced by the people, on a genuine participative and emancipatory manner. Hence for me, AR is  the hook for practical and theoretical research action(s).

\startARemark 
hier soll noch ein diagram rein dass das ganze etwas verdeutlicht
\stopARemark

The next section \refMissingSrc{Action Research chapter} is supposed to shed light on the concept of Action Research.

\subject{Self Conception}

A further conceptualization of the given themes may help to determine and construct the red line that is going to guide through this work. 

\spaceHalf
\startQuestionBack 
\startlines
{\bf Why should my research be relevant?}
\stoplines
\stopQuestionBack 
It should be relevant if it can be embedded in the struggle of the people. It is another tool which can be deployed in order to be part of the struggle it is talking about instead of being solely an academic analysis of the situation.

\startQuestionBack 
\startlines
{\bf How does my research becomes relevant?}
\stoplines
\stopQuestionBack 
its is open accessible - it is embedded and respects the local context - is or becomes part of the struggle of the people - not just academic work 

\startQuestionBack 
\startlines
{\bf What are the goals of my research?}
\stoplines
\stopQuestionBack 
no judgement of people, no analysis of people, no proposal for people, together with the people, neglecting of academic framework - in terms of time

\startQuestionBack 
\startlines
{\bf Who is the audience and who are the beneficiaries?}
\stoplines
\stopQuestionBack 
The audience can be 

\showImperfection

\stopmode

\stoptext

\stopcomponent
