\startcomponent component_NarratingInquiries_SaoPauloDiaries_Missing
\product product_Thesis
\project project_MasterThesis

% definitions and macros
\environment envThesisAllEnvironments

\define[]\subsectionSaoPauloDiariesMissing{\subsection[narratinginquiries_saopaulodiaries_missed]{Missing//Links}}

\define[]\subjectSaoPauloDiariesOcas{\subject{Social Center//Ocas}}
\define[]\subjectSaoPauloDiariesPadreDeCha{\subject{Padre de Chá}}
\define[]\subjectSaoPauloDiariesRepublica{\subject{Around//Praça de República}}
\define[]\subjectSaoPauloDiariesGrajau{\subject{Zona//Sul}}

\define[]\subjectSaoPauloDiariesColoquio{\subject{Primeiro//Colóquio//Território//Autônomo}}
\define[]\subjectSaoPauloDiariesSolidarity{\subject{Solidarity//Guerreiro//Urbano}}
\starttext

\startmode[tocLayout]
\subsectionSaoPauloDiariesMissing
What did I miss and why

\subjectSaoPauloDiariesOcas
\subjectSaoPauloDiariesPadreDeCha
\subjectSaoPauloDiariesRepublica
\subjectSaoPauloDiariesGrajau
\subjectSaoPauloDiariesColoquio
\subjectSaoPauloDiariesSolidarity

\stopmode

\startmode[draft]
\subsectionSaoPauloDiariesMissing

Many things do not leave tracings in the map of narrations. This is mainly due to time constraints and due to my own, sometimes mainly unstructured, practice especially during the course of writing. However, I would like to briefly document some occasions that I am referring to at different locations in this text. Even though they are just sparkling around for now, relatively unconcrete and incomplete, I hope they will enter the scene finally when transforming this offline text into it online counterpart. Some of the sparkles can be seen as \gotoTextMark[random shots]{narratinginquiries_saopaulodiaries_random}.

\subjectSaoPauloDiariesOcas
\reference[narratinginquiries_saopaulodiaries_ocas]{}
I spend much time at \locOcas, a \aKeyword[cultural space]{streets+places+cultural+ocas} in \locBras. \Ocas has been, similar to \Ay, a place of encounter, it has often be the place of departure for me an \Ju when we were heading towards the centre. It is also an important place for the struggle of the \peopleInStreetSituation because there, regular \aKeyword[workshops]{streets+actions+workshops} are taking place, mainly for expressing what the life on the streets individually means for the people, in form of \aKeyword[poetry]{streets+actions+workshops+poetry} or \aKeyword[photo shooting]{streets+actions+workshops+photo shooting} for instance. \Ocas is also a point of reference because it publishes a monthly \aKeyword[street journal]{streets+actions+journal}, often composed of pictures and texts of the people. The journal is a source of income for street vendors, as I know it from Europe as well. \Ocas has also been the place where \aruassa hold several reunions to discuss the direction of the ongoing classes and workshops. Above all, meeting people there, discussing, departing into the city, offered many insights to São Paulo.

\subjectSaoPauloDiariesPadreDeCha
\reference[narratinginquiries_saopaulodiaries_padredecha]{}

At \locPadreDeCha, I met with \Gi who \gotoTextMark[invited me to come there]{narratinginquiries_penaforte_goodby}. \PadreDeCha provides food for people in street situation, likewise \Penaforte. \Gi played a small concert that day. I also met \Jur, a philosopher who is coming everyday. Later that day he showed me the centre from his philosophical point of view.

\subjectSaoPauloDiariesRepublica
\reference[narratinginquiries_saopaulodiaries_republica]{}

Not including anything about the time I spent with \Ju is particular sad. We spend much time at night at the commercial area at \locRepublica that turns completely once the shops are closed. Another world of streets busyness emerges once its getting dark. Different people gathering then, all known to each other, making their way of living then. When we stayed there sometimes for hours, drinking small pins of cheap Cachaça, \Ju showed me how people organize there, how they are running their businesses, hiding from the always present police, its car and foot patrols.

\subjectSaoPauloDiariesGrajau

One Sunday at the begin of June at \locGrajau, zona sul of São Paulo: the \aLinkNew[Rede de Comunidades do Extremo Sul de São Paulo]{http://redeextremosul.wordpress.com/} goes on the streets to reclaim it with culture, breakdance, rap, free radio, in order raise awareness about the issues the communities in this part of the city have to face. Their public action took place at \aLocationNew[Avenida Belmira Marin]{http://osm.org/go/M@yl3i9QU--} that is usually blocked by traffic. It is the main street which connects the even more remote communities and brings daily traffic and traffic jams from outside São Paulo into the city.

\startPersonal
I stayed in contact with the people of the Rede because I asked them if it would be possible for my research actions to take place in their area. Finally it was not possible. The people had very good reasons because they would not have had time to accompany me. All of them were busy in their struggle and as it has been prevailing on the streets in the centre as well, without people introducing one to those that life there, its becomes difficult for someone not living there, for an outsider, to get to know the place and its people. Anyway, this day and our contact later on was super nice, as much as the people I met there and spoke with.
\stopPersonal

\subjectSaoPauloDiariesColoquio
\reference[narratinginquiries_saopaulodiaries_coloquio]{}

At the end of October, the \aLinkNew[Primeiro Colóquio Território Autônomo]{http://territorioautonomo.wordpress.com/} took place at the Federal University in Rio de Janeiro. This meeting was an invitation for social movements, activists and scholars to discuss and propose spaces of cooperation in social struggles in order to converge the academic spaces and spaces of struggle. The colloquium offered lectures, from a libertarian perspective \bracket{my point of view just to heavy to digest}, and group work. 

Its main question could not be answered but a wish that has been formulated stated that academia must come closer to the spaces where struggles are taking place, in the centres of the cities, at its peripheries, and that it should not solely remain its its university complexes, campuses and theories. What impressed me was the narration of one of the organizers of the occupation \quote{Quilombo das Guerreiras} that is located close to the main bus terminal. She mapped the various housing occupations in Rio, their forms of organization, from purely horizontal to hierarchical, from anarchistic to occupations of workers.

Some sources providing further information probably provide information for a better understanding: \aLinkNew[a mapping of housing occupations in the central are of Rio de Janeiro]{http://bit.ly/n4q9M7}\aQuote{da Silva}{2009} and the website \aLinkNew[Pela Moradia]{http://pelamoradia.wordpress.com} is informing about what is happening in Rio these days, while the city is transforming its shape in preparation of the the Olympic Games (2016) and the World Cup (2014). Audio recordings are also available at \aLinkNew[Sem Teto - Bewegung im Zentrum rio de Janeiros]{http://semteto.noblogs.org/}.

The following call for solidarity has been issued by the \quote{Guerreiro Urbano Occupation} at the begin of December 2010 and gives an impression of what Rio is intending in its central areas in order to clean up the city for the spectacles to come.

\subjectSaoPauloDiariesSolidarity

Solidarity with the Guerreiro Urbano (Urban Warrior) Occupation

The “Urban Warrior Occupation“ is a collective of about 50 families which are organizing themselves to meet the need for decent housing in the center of Rio de Janeiro. After seven months of meetings, they occupied a public building on 1 November. The building, which had been abandoned for 20 years, is officially the property of the National Social Security Institute (INSS – Instituto Nacional do Seguro Social1), and was once a hotel.  It has enough space to help tens of families realize their dream of decent housing. The occupation was carried out by homeless workers (sem tetos) with the help of various sympathizers and social movements. Through direct action, these “sem tetos” tried to force the state apparatus to enforce the laws that the state’s own structures neglect. Under Brazil’s Federal Constitution (Article 6), every property must fulfill a “social function” (Article 182), and the population must participate in the processes of urban planning and management (Estatuto da Cidade – Law Number 10.257/2001, Articles 2, 4, 39 and 45). As soon as the “sem tetos” entered the abandoned building, they began to clean and reorganize it. This is the real revitalization(2) that we can expect from those who tried, despite all violence and injustice at the hands of the state and business interests, to construct a new, socially just urban space. On the next day the “sem tetos“ were displaced by the Federal Police, who acted without any legal documents and without any kind of identification, though several occupants were obliged to provide their IDs. The Federal Police acted violently and illegally, and their actions were undoubtedly illegal. Tens of families were thrown in the middle of the street, on a rainy morning.  Most of them had no place to go and had to stay at relatives houses, friends’ houses or even in the street.

International support and solidarity is needed. Please sign the manifesto or send solidarity greetings to the local support group: Comitê de Solidariedade as Ocupações Sem Teto, pelamoradia@gmail.com

For more information please read the appendix. More information in
Portuguese:  \aLinkNew{http://pelamoradia.wordpress.com}

(1)Based on the amount of abandoned buildings that this federal institute own, some members of the sem teto movement consider the INSS to be a group of large landowners who “act like the landowners against whom the Brazilian rural landless movement fights.”

(2)The central area of the city of Rio de Janeiro is passing through a “new wave” of urban restructuration. Following the demand of real estate capital and state interests in the mega-events that will take place in the city over the next 6 years (the 2014 World Cup and 2016 Olympics), the port area is receiving massive public investments to transform the area (abandoned by the state since the decline of the city's port activity in the early twentieth century) into a new space for real estate investment. This project depends on the expulsion of the poor population of the area. The police action in the so called “Pacifying Police Units“ (UPPs - Unidades de Polícia Pacificadora”), the rising cost of living and the destruction of housing alternatives for poor families are some of the strategies adopted by the state to accomplish this task. Moreover, the three state spheres (federal, state and municipal power) are implementing a joint “revitalization” plan for the area (as though the population living there now were not even "living beings"). All the popular and social movements recognize that the “Porto Maravilha Project” is, actually a plan to “revitalize” the profit of national and international real estate and tourist industry capital. On the 13 of December, there will be a demonstration in solidarity with the "Sem Teto" occupations in the center of Rio de Janeiro.


\addReference
{
da Silva, C., 2009. As ocupações de prédios vazios e o esvaziamento do centro da cidade do Rio de Janeiro. {\em ChiqdaSilva}. Available at: \goto{\hyphenatedurl{http://bit.ly/n4q9M7}} [url(http://bit.ly/n4q9M7)] [Accessed November 5, 2010].
}

%---------------------------------------------------
% gets only displayed in unfinished mode

\showImperfection

%---------------------------------------------------
\stopmode

\stoptext

\stopcomponent
