\startcomponent component_Conclusion
\product product_Thesis
\project project_MasterThesis

% definitions and macros
\environment envThesisAllEnvironments

\starttext

\startmode[tocLayout]
\chapter[conclusion]{Resumé}

Contains the thesis' conclusion and further suggestions.
\stopmode

\startmode[draft]
\chapter[conclusion]{Résumé}

This is the end. The end of this work. When looking back now my résumé is twofold. This work has been a struggle, it revealed limits. What I found especially striking has been my turn of perceiving this work as a project and accept it as a process instead. In a sense this thesis became subject to its own research. It opened my perspectives about conceiving knowledge, how my embeddedness in my social environment affects the way I select the approaches to research, how difficult it is to stick to my personal attitude in an environment that functions according to other principles. I am glad I made this experience because I feel now a bit better prepared to continue leaving new tracings on new maps. I also know now even more that I will continue in one way or another to be involved in social struggle. I always found it enriching for my personal practice, for my being. São Paulo allowed me learn how to decipher a city, the city, how to look in order to understand what I see.

Being together with the people in São Paulo revealed a glimpse about the multitude of worlds existing on the streets, about the multitude of people, struggles and ideas, but also about the multitude of process that make life unlivable. The time in the city showed me my limits bluntly, trashed some of my ideals. But perceiving a ground level of chaos, of uncontrollability, unpredictability, also made room for imagining concrete utopia to self-determinately shape the city and by that ones own life. This sounds a bit strange having in mind that many people struggle out of necessity and not out of free will. 

Having seen the massive urban housing movements, the cooperation in urban and land struggle, the ideas behind those struggles, the people that shape those struggles, I fell that many people care to act, that many people struggle to overcome what is not acceptable, what keeps us down, what exploits us.

Even though this work is finished in a couple of lines, I intend to strengthen the connections to São Paulo even if we are geographically separated. Informing, revealing the invisible, can be done on all levels, locally and globally. This work could have been broader, could have narrated more, but for now I accept its form. Certainly, the time to come will be a continuation, thus this thesis reached already one goal: not being made just for the sake of an academic title. It  reached another goal because it allowed the establishment of relations and interlinks us and our struggles, independent of the physical space we remain.

\stopmode

\stoptext

\stopcomponent
