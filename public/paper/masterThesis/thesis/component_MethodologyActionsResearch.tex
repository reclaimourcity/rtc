\startcomponent component_MethodologyActionsResearch
\product product_Thesis
\project project_MasterThesis

% definitions and macros
\environment envThesisAllEnvironments

\define[]\subsectionActionsResearch {\subsection[methododology::actions::research]{Action//Activist//Research}}

\defineregister[MAR][MARS]
\setupregister[MAR][coupling=yes, indicator=yes, referencing=on, interaction={pagenumber, text}, alternative=A, n=2, balance=yes]

\define[1]\aKeyMarVi{\MAR{#1}{\color[magenta:8]{\em#1}}}
\define[1]\aKeyMarInVi{\MAR{#1}}

\define[2]\aKeyMarViSee{{\seeMAR{#1}{#2}\color[magenta:8]{\em#1}}}
\define[2]\aKeyMarInViSee{\seeMAR{#1}{#2}}

\definesynonyms[translation][translations][\infull]
\setupsynonyms[style=cap]
         
\setuptyping[
	margin=1cm,
	before=\startAWordMapBack,
	after=\stopAWordMapBack,
	tab=5,
	option=commands]

%\enablemode[draft]
%\disablemode[tocLayout]

\starttext

\startmode[tocLayout]

\subsectionActionsResearch

This section describes the concept of Action Research applied as theoretical and practical framework of this thesis.

\stopmode

\startmode[draft]

\subsectionActionsResearch

\startKeywords
action, activist, action research, activist research, framework, participatory, transformation
\stopKeywords

I determined \aKeyMarInVi{Action Research}\abbrNew{Action Research}{AR} as the overall methodological framework for this thesis. 

In a sense, the word \abbrFull{AR} by itself seems already matching the way I intend to do research. This may be a relatively weak justification but nonetheless it nearly hits the mark. \abbr{AR} is research that emerges from within the movement, from within a struggle, playing an active role in the movement, in the situation and in its intended transformation \aQuoteB{Morell}{2009}{40}. 

\spaceHalf

\inright{workers inquiry}
\aKeyMarVi{Workers Inquiry}, the subjective workers view on and analysis of the situation in the fabric, of being exploited, and their knowledge about needs and the necessary transformation according to those needs, is one instance of \abbrFull{AR} that emerged from within a movement, conducted by the activists themselves. In Italy for example, visible in the \aKeyMarInVi{Operaismo}\toTranslate[Workerism]{Operaismo} movement that originated from debates in the journal \aKeyMarInVi{Quaderni Rossi}\toTranslate[Red Notebook]{Quaderni Rossi} in the early sixties of the 20th century \aQuoteW{Wildcat}{1995} \aLinkNewF{http://www.wildcat-www.de/wildcat/64/w64opera.htm} and which led to the workers' struggle at the end of that decade.

\spaceHalf

[...]\startCitation
sie waren durch ihre Untersuchungen auf kommende Kämpfe vorbereitet, hatten die Probleme innerhalb der Fabrik analysiert, hatten die Arbeiterdiskussion verfolgt, um die Arbeiterforderungen auf die Flugblätter schreiben zu können und auf Versammlungen als politische Linie durchzusetzen. Sie hatten gelernt, »daß es schon Kämpfe gibt, bevor sie offen ausbrechen«. \aQuoteW{Wildcat}{1995}
\stopCitation

\spaceHalf

If one thus looks at social movements and collectives and their organized struggle, one can perceive the production and articulation of their own knowledge and analysis referred of their reality, in order to develop the necessary step(s) to transform it. Here, knowledge emerges from within the struggle and is supposed to facilitate it \aQuoteW{Barker, Cox}{2001}.

\spaceHalf

\inright{academic versus movement knowledge}
We then see the difference of knowledge produced: \aKeyMarVi{academic knowledge} means theorizing about a situation by trying to derive generalized and abstract assumptions, analysed from a claimed neutral observant perspective; \aKeyMarVi{movement knowledge} of struggle, formed by the direct necessity to change a situation that constraints the individual well being and freedom, thus is formulated from a subjective position according to personal or collective needs, more practical in nature and destined to be applied for a concrete transformation \aQuoteW{Gramsci in Barker and Cox}{2001}.

\spaceHalf

\startCitation
In this perspective, movement theorizing is an aspect of the work that people do as they try to create institutions (movements) that will enable them (indirectly, through a change in the social order) to meet needs that are not currently being met.\aQuoteW{Barker, Cox}{2001}.
\stopCitation

\spaceHalf

\startAReminder
hier können noch andere beispiele rein: Transborder Immigrant Tool, Feminist Theory, Interface Journal

außerdem ein mapping der bewegungen in São Paulo: Rede Extrema Sul, MNPR, MNCR, MTST, MST, Ay Carmela, Indymedia, 

Außerdem wäre noch eine mapping von Tools nötig
Flickr + Fotos, Journale + Zeitschriften, Webseiten + Blogs + Portale, Zines + Flyer + Zeitschriften, Software, Videos, eigenpersonale Medien also, 

\stopAReminder

\spaceHalf

I wanted to draw some attention to those examples because I perceive \abbr{AR} as a valid approach of conducting research, even if this research originates from the people and their realities and struggle, or better, exactly for that reason. The knowledge produced there is the result of a intellectual work, which has for me the same significance as knowledge produced in journalistic work or the academic space. 

Looking at the time in São Paulo, I would say that \abbr{AR} has been the consequential way of being together with the people from the streets, collaborating with them and participating in their actions. This was possible because I left the academic space I am bounded to \bracket{due to the fact I write this thesis in an academic framework} during my time in the city.

I also choose \abbr{AR}, because I think that the knowledge and conceptualisation of the situation on the streets, produced by the very people from the streets, is an enrichment for academic work in the sense that it could engage people to become more active, contest the status the academic space represents and by that introduce other ways of thinking, theorizing and adopting it. Therefore I make plenty of use of the information provided by the people, along with my personal and collective experiences and academic theories.

Certainly, the form of knowledge creation, distribution and adoption is different from the norm of academic knowledge, but \aQuoteInTextA{Barker and Cox} nicely describe this difference as complementary rather then contradictory when they write that theorizing in struggle

\spaceHalf

\startCitation
can be usefully understood as theoretical because it is not simply a product of \quote{the situation} or \quote{folk culture}, but is rather a process of ongoing intellectual engagement, whose results [...] shift over time. [...] [it] is grounded in the process of producing \quote{social movements} against opposition. It is always to some extent knowledge-in-struggle, and its survival and development is always contested and in process of formation. Its frequently partial, unsystematic and provisional character does not make it any the less worth our attention, though it may go some way towards explaining why academic social movements theory is too often content with taking the 'cream off the top', and disregarding - or failing to notice - everything that has to happen before institutionalized social movement theorizing appears in forms that can be easily appropriated. \aQuoteW{Barker and Cox}{2001}
\stopCitation

\spaceHalf

Continuing with the justification of choosing \abbr{AR} as main methodological framework for this thesis, I found the work of \aQuoteInTextA{Morell} quite helpful. \aQuoteInTextA{Morell} conceptualized different tendencies in his article \aQuoteInTextT{Action research: mapping the nexus of research and political action}\aQuoteY{2009} that can be related to \abbrFull{AR} and of whom some have been briefly described right now. In his mapping, five main tendencies are presented

\startitemize
\item {\ss participative-collective methods} 
\item {\ss producing alternative content} alternative content can be any form of publication and media that transport knowledge 
\item {\ss developing strategic thinking for broad political processes }
\item {\ss building relationships and networking connections}
\item {\ss opening knowledge}
\stopitemize

\placefigure[location=here]{Five main tendencies Action Research is directed to, according to \aQuoteB{Morell}{2009}{23,24}}
{
\starttyping
				/BTEX\tfa{Developing Strategic Thinking}/ETEX

/BTEX\tfa{Building Relations}/ETEX
/BTEX\nl/ETEX
			/BTEX\tfa\ss\bf\color[green]{ACTION RESEARCH}/ETEX
/BTEX\nl/ETEX
/BTEX\tfa{Participative\\Collaborative}/ETEX      /BTEX\tfa{Opening Knowledge}/ETEX
/BTEX\nl/ETEX
		/BTEX\tfa{Producing Alternative Content}/ETEX
	
\stoptyping
}
%\writetolist[figures]{

Comparing those tendencies to my research, three of them are very present because they reflect my attitude on the form of my research

\startitemize
\item participative observing and participating observer
\item open knowledge
\item building solidarity and connections
\item \bracket{to a small extend} producing alternative content
\stopitemize

\placefigure[location=here]{The tendencies of this thesis research in the framework of Action Research}{
\starttyping
				/BTEX\tfa{Participative\\Collaborative}/ETEX

/BTEX\tfb{Opening Knowledge}/ETEX
/BTEX\nl/ETEX
				/BTEX\tfa\ss\bf\color[green]{THESIS RESEARCH}/ETEX

						/BTEX\tt\small{Producing Alternative Content}/ETEX

/BTEX\tfb{Building Relations}/ETEX

\stoptyping
}

\startAReminder

One justification of AR is that one must be there on the local level, one must see and experience the local space in order to understand the situation (see Tsing, Friction)

This knowledge tries to answer questions such as - What do we want? Who are we? What do we need to do?

\stopAReminder

\abbr{AR} is said to be favourable for research that intends to work for a social transformation of an inequitable situation \aQuoteB{}{}{}. 

This may not be the only area of application because \abbr{AR} is not bounded to a specific attitude but rather represent a set of tools that can be used for basically any purpose, thus \abbr{AR} is applied in corporations \aQuoteB{}{}{}, as well as in social movements research \aQuoteB{}{}{} or by activists themselves \aQuoteW{Periferies Urbanes}{2010}\aLinkNewF{http://periferiesurbanes.org/?p=165&lang=en}.

Hence, \abbrFull{AR} is a organic framework consisting of various concrete methodologies, that depend on the context it is applied in. 

One attribute of \abbr{AR} is its participative research approach \aQuoteB{}{}{}. 

\startAReminder

im empierischen teil kann ich mit einer karte der bewegungen und orte in são paulo beginnen

außedem eine unterteilung in situation, aktion und organisation?

\stopAReminder

What remained completely out of reach is a \aKeyMarVi{collaborative writing process}. Before I continue I just want to cite a fraction of a definition of contemporary \aKeyMarVi{collaborative ethnography} \bracket{even though I do not consider me an ethnographer} because I would say that this definition, still very formal in expression, in its core fits, to the approach that I intend(ed) to use, which 

\spaceHalf

[...]\startCitation 
deliberately and explicitly emphasizes collaboration at every point in the ethnographic process, without veiling it—from project conceptualization, to fieldwork, and, especially, through the writing process. Collaborative ethnography invites commentary from our consultants and seeks to make that commentary overtly part of the ethnographic text as it develops. In turn, this negotiation is reintegrated back into the fieldwork process itself. \aQuoteB{Lassiter in Rappaport}{2008}{1}
\stopCitation

\spaceHalf

Thus, here its is again, that contradiction between academic and activist researcher. For a collaborative writing process, which would return its results back to the people as well, was no space. And due to the fact that activist knowledge is fluid and progressing, further reflections on the once produced content and the resulting action would be necessary, which means that cycles of collaborative writing, reflection, action and re-writing would be required. Here, mainly time constraints but also the constraints of the academic space come into play again. As already mentioned in the introduction of this \refMissingSrc{methodology chapter}, the plain time I spend together with the people has been basically three month, thus it was not even possible to start thinking about a collaborative writing process. This process could have started after those initial month' of participating and experiencing, but at this time, I had already to return to Germany. From my point of view, a collaborative writing and feedback process would have required several month, especially when thinking about the conditions this process would have been realized under.

Being back in Germany means that the direct contact to the people has been lost. The possibility to hang around with them, to meet here and there, in the city, in cultural and political spaces, somewhere on the streets. Contact is therefore only possible through digital communication which makes a collaborative writing process impossible for me. We use the \aKeyMarVi{Internet}, \aKeyMarVi{Email}, \aKeyMarVi{Facebook} or \aKeyMarVi{Orkut}, but for more than one way communication we are not prepared. Even if we would, double effort had to be put in translation work because in my case it is fundamentally important to communicate in Portuguese, which would mean to translate all produced content into English at least as well, if necessary also into German. This situation would be the optimum but my current reality does not provide space and time for such as effort. A deadline must be kept. The alternative would have been to stay longer but just plain visa issues would not even allow that. 

So, what to do then? Basically that what is done usually, I write just by myself in order to keep the deadline for this thesis, I will try to translate important parts into Portuguese afterwards, make everything accessible online on the \aLinkF[thesis blog]{https://rtc.noblogs.org}  and will try to keep the contacts alive, try to organize some events on grassroots level to distribute information about the São Paulo experience. 

\subject{translations}
\placeTransLa

\subject{figures}
\placelist[figure][criterium=all]

\subject{keywords}
\placeMAR

\subject{reference}

\startREF
\nl%
Barker, C. \& Cox, L., 2001. “What have the Romans ever done for us?” Academic and activist forms of movement theorizing. Available at: \goto{\hyphenatedurl{http://www.iol.ie/\~mazzoldi/toolsforchange/afpp/afpp8.html}} [url(http://www.iol.ie/\~mazzoldi/toolsforchange/afpp/afpp8.html)] [Accessed July 18, 2011]. 
\nl%
Morell, M.F., 2009. Action research:
mapping the nexus of research and political action. {\em Interface: a journal for and about social movements}, 1(1), p.21-45. Available at: \goto{\hyphenatedurl{http://interfacejournal.nuim.ie/wordpress/wp-content/uploads/2010/11/interface-issue-1-1-pp21-45-Fuster.pdf}} [url(http://interfacejournal.nuim.ie/wordpress/wp-content/uploads/2010/11/interface-issue-1-1-pp21-45-Fuster.pdf)] [Accessed May 20, 2011]. 
\nl%
Rappaport, J., 2008. Beyond Participant Observation: Collaborative Ethnography as Theoretical Innovation. {\em Collaborative Anthropologies}, 1, p.1-31. Available at: \goto{\hyphenatedurl{http://muse.jhu.edu/journals/collaborative_anthropologies/v001/1.rappaport.html}} [url(http://muse.jhu.edu/journals/collaborative_anthropologies/v001/1.rappaport.html)] [Accessed July 22, 2011]. 
\nl%
Wildcat, 1995. Renaissance des Operaismus? {\em Wildcat}, 64, p.99-110. Available at: \goto{\hyphenatedurl{http://www.wildcat-www.de/wildcat/64/w64opera.htm}} [url(http://www.wildcat-www.de/wildcat/64/w64opera.htm)] [Accessed July 22, 2011]. \nl%
\stopREF

\stopmode

\stoptext

\stopcomponent
