\startcomponent component_NarratingInquiries_SaoPauloDiaries_SomeTalks
\product product_Thesis
\project project_MasterThesis

% definitions and macros
\environment envThesisAllEnvironments
\environment envCfgThesisImages

\define[]\subsectionSaoPauloDiariesSomeTalks{\subsection[narratinginquiries_saopaulodiaries_sometalks]{Encounters//Talks//Streets}}

\define[]\subjectAy{\subject{Around Ay Carmela}}
\define[]\subjectCrack{\subject{Meeting again}}
\define[]\subjectPCC{\subject{PCC}}
\define[]\subjectVeganLunch{\subject{Collective Vegan Lunch}}
\define[]\subjectRecycling{\subject{Recycling}}

\starttext

\startmode[tocLayout]
\subsectionSaoPauloDiariesSomeTalks

\subjectAy
discovering the centre around Ay Carmela

\subjectCrack
Meeting the young guy again.

\subjectPCC
PCC in the centre.

\subjectVeganLunch
Socializing at Ay Camela

\subjectRecylcing
Recycling at Ay Carmela

\stopmode

\startmode[draft]

\subsectionSaoPauloDiariesSomeTalks

I would like to start with a citation by \aQuoteInTextA{Cleisa Moreno Maffei Rosa} taken off her book \aQuoteInTextT{Vidas de Rua}\toTranslateT{Vidas de Rua}{Street Lifes}:

\startCitation
No entato, ficavam à mercê do controle burocrático exercido pelos órgãos governamentais e, no limite, reiteravam aões imediatistas ligadas à higiene e alimentação, destituídas de conteúdo de natureza socioeducativa. Algumas organizações procuravam sobreviver sem verbas públicas, mas - a duras penas - recorriam a apoio de grupos solidários.

Não havia pesquisas, estudos ou levantamentos atualizados sobre essas questões, nem mesmo programas alternativos quo apontassem para autonomia das pessoas attendidas nos serviços e particpação nas decisões institucionais - elementos fundamentais à conquista da cidadania da população de rua \aQuoteB{Rosa}{2005}{174}.\footnote{In the meantime we remain at the mercy of bureaucratic control exerted by governmental bodies and, at maximum, reiterate the immediate need of hygiene and alimentation, stripped-off content of socio-educative nature. Some organisations are seeking to survive without public aid, but - with heavy legs -  run after the support of solidary groups. There exist no research, study or contemporary inquiry about those questions, no alternative programs that are directed to the autonomy of the people received at public services or to the participation in institutional decisions - fundamental elements for conquering citizenship for the street population.}
\stopCitation

\Ma lent me this book when we made up \gotoTextMark[our plan to hit the streets]{narratinginquiries_saopaulodiaries_sundaystreetscentreplan}. It is full of stories from people living in street situation, content from the standpoint of the streets, narrations so detailed that I do not need to intend to do the same. I have taken the quote above because it seeks for ways to practice autonomy, self-determination, participation and citizenship. The following narrations shall give a small insight into self-determined actions that are already practised, on small scale though, but with direct impact on various forms of struggle. However, the aspect of violence is also part of those narrations because violence in its various facets is part of the daily realities on the streets. It should not be neglected because it has an impact on social struggle.

\subjectAy

\reference[narratinginquiries_saopaulodiaries_sundaystreetscentregibson]{}Today is the last Sunday of the month, its end of July. I went to \Ay for \aKeyword[collective vegan lunch]{streets+actions+collective lunch}\aLinkContentNewF[collective vegan lunch at \Ay]{http://ay-carmela.birosca.org/node/334} that is organized every last Sunday of the month and met \Ma there. 

We met the first time at the free mapping festival \aKeyword[você está aqui, mas por quê - um festival de mappeamento livre]{streets+actions+festivals+festival de mappeamento livre}\aLinkContentNewF[você está aqui, mas por quê - um festival de mappeamento livre]{http://ay-carmela.birosca.org/node/433}\aLinkContentNewF[i am here but why - a free mapping festival]{https://rtc.noblogs.org/post/2010/07/25/i-am-here-but-why/} in June. \Ma knows the streets of city inside out because he was in street situation more than a decade and just recently, some month ago, managed to leave the streets and moved into an apartment. 

He is super engaged in the struggle of the people because as he said, he constantly fears that people will continue to die on the streets because they lack everything and are prone to many forms of violence on the streets. He refers to the massacre of 15 people in street situation in August 2004, killed by death squats of the military police and private security agents \aQuoteB{Dossiê}{2009}{63-64}\footnote
{
The mentioned Dossier is collaborative work of various civil society organisation and non governmental institutions that meticulously maps the illegal execution of people by institutional and state agents, by the police on the streets or in prisons for instance
}\aQuoteB{o Trecheio}{2009}{1}\footnote
{
\aKeyword[O Trecheiro]{streets+actions+journals+o trecheiro} is a street journal published by \aKeyword[Rede Rua]{streets+actors+groups+rede rua} and includes reports and articles written by \aKeyword[people in street situation]{streets+standpoints+in street situation}
} but also to the current \toTranslate[higienista]{cleansing} policy of the city that aims to \refMissingSrc{clean the central areas} \bracket{such as Sé, República or Luz} off the people in street situation by expelling and forcing them through agents of civil service and \abbr{GCM} into social institutions such as the \gotoTextMark[tendas at Parque Dom Pedro]{narratinginquiries_saopaulodiaries_dayandnight}.

\imgOTrecheiroMassacre

\spaceHalf

\inright{metting \Ma and \Gi for a trip through the centre}
\Ma offered me a trip through the city, to show me around, literally, in order to get in touch with people in street situation but also to experience what is meant by {\bf being} in street situation, how we then have to organize our day and night, our food and shelter. While making this plan, here at the entrance of \Ay, a friend of \Ma is coming around, together with a teenager. It's \Gi, with whom I will spend some time \gotoTextMark[here]{narratinginquiries_saopaulodiaries_penaforte} and \gotoTextMark[there]{narratinginquiries_saopaulodiaries_padredecha}. \Gi says that he is accompanying the young guy because he cannot take care of himself. They are heading towards \locMissingSrc{Parque Dom Pedro} and \Gis stop and talk with us makes him quickly anxious continue their way. 

\Ma, like he is always doing, is introducing me, telling \Gi what I am doing in São Paulo, that we are making plans to stay on the streets. \Gi like he is always doing is super kind and says that he thinks that it is important that people outside Brazil get to know what is going on in São Paulo. \Gi does not stay long because his friend urges him to continue. \Gi says that he is dedicated to him today because he is still a very young guy, just recently entered the streets and he needs someone who shows him around, shows him the places that he should know. Before \Gi and his friend leaves we agree on meeting again later this afternoon, at the place where \Ma is living now.

\spaceHalf

\inright{at \Mas place}
\reference[narratinginquiries_saopaulodiaries_sundaystreetscentreplan]{}Shortly afterwards we are heading towards \Mas place. After arrival we fixed our plan first of all. \Ma said that we should stay at least one week on the streets, moving through different areas of the city, staying in the centre but also moving to the east, Brás, Mooca, Belem. 

\startPersonal
I felt a bit uncomfortable because I just got acquainted to the city and did not feel ready for such a long trip.
\stopPersonal

Our compromise is finally just an introduction, \gotoTextMark[two days on the streets]{narratinginquiries_saopaulodiaries_dayandnight}. Then, \Ma gave me a some material that he has been collecting about the struggle on the streets. 

\startARemark{hier können fotos von den sachen die \Ma mir gegeben hat rein} 
plus erklärungen dazu

\startitemize
\item ausweis zum kongress der catadores
\item handschriftliche aufzeichnungen
\item buch "vidas de rua"
\item ...
\stopitemize

\stopARemark

He says that he is also a militant of the \abbrNew{Movimento Nacional da Populacão de Rua}{MNPR}, the \gotoTextMark[movement of the population in street situation]{go to text mark missing},  and that he is now taking some time off in order to organize his own life and because he is fearing repression by police, that according to him, is observing active members of \abbr{MNPR}.

\startARemark{hier fehlt noch was zu MNPR}\stopARemark

I asked him what he thinks about the \aKeyword[right to the city]{streets+aims+right to the city} but for him another question is much more important. The question of \aKeyword[access to the city]{streets+aims+access to the city}. Being formerly in street situation, this means access to educational facilities, access to decent work, access to a decent housing, access to participation in the cities decisions. For him, being on the streets means not having access at all, to nothing, not even food, being totally excluded. 

Once \Gi arrived we talked about the perception \quote{the society} has about the people in street situation and other marginalized groups

\startDialog
\tell{\Ma}{Do you know how they call us in the media, what society thinks of us? They call us \toMark{Noia, Vagabundo, Zumbi}, what else...?}
\tell{\Gi}{\toMark{Ladrão, Bandito}...}
\tell{\Ma}{...Bicho, they call us bicho! You see? This is what people think of someone living on the streets, someone not human, an animal...}
\stopDialog

\spaceHalf

\inright{walking to the centre}
After a while we decide to take a walk back to \Ay and talk with people in \locMissingSrc{the surrounding area}. \Ma says that it is already late afternoon and probably a good time to meet some people there because they are usually arriving around that time in order to secure their space for the night. 

We enter \aLocationNew{Rua do Carmo}{http://osm.org/go/M@ziKw9eb--}\footnote{Rua do Carmo is one of those roads that cross the ancient settlement area of São Paulo. Impression and pictures from that time can be found at \aLinkNewNoF[São Paulo de garoa]{http://vivipara.blogspot.com/2010/09/militao-paisagens-e-brasileiros.html}} and pass an \aKeyword[abandoned construction site]{streets+places+abandoned+construction site}, an unfinished high-riser\footnote{More information about this abandoned build are available at \aLinkNewNoF[São Paulo abandonada]{http://www.saopauloantiga.com.br/edificio-rua-do-carmo-93/}}, maybe 20 floors high, its red bricks exposed, no windows, missing walls, on the third floor clotheslines packed with laundry to dry and in front of it, on the small court leading to the street, some children playing while a few people are leaving and entering the improvised ramp into the interiors of the building.

On the other side of the street some \aKeyword[blocked up and abandoned houses]{streets+places+abandoned+house} as well.  A back road is passing behind a colonial building at the corner. There we are heading towards, to the back road. Its the place where people are declaring their space for the night to come, on the stairs to the side entrances to the church.
%\aLocationNew{Igreja Nossa Senhora do Carmo}{http://osm.org/go/M@ziKxBat--}
\spaceHalf

\inright{meeting four guys}
\reference[narratinginquiries_saopaulodiaries_sundaystreetscentrefourguys]{}There we meet a group of four guys. Neither \Gi nor \Ma know them but we are getting in touch with them right away. Two are already massively drunken, another one, probably the oldest is hectically talking and standing still, the fourth one, the youngest, is sitting with the other two on the stairs. \Ma introduces us and asks them to tell me a bit about their situation, telling them that I am in the city to get to know about the struggle on the streets. 

\startPersonal
For me this kind of situations have always been uncomfortable. I always feel somehow exposed as a stranger, somebody different, not belonging to the place or the people \bracket{which is how it is anyway}, even though my clothes are in similar bad conditions then of most of those I meet, even though I speak Portuguese and even though I its my demand for transparency that the people know directly why I am there at that particular moment. Here my role is that of a \aKeyword[passive observer]{streets+roles+passive observer} and this sometimes leads to arguing and reluctance.
\stopPersonal

\reference[narratinginquiries_saopaulodiaries_sometalks::hecticguy]{}The hectically guy is asking immediately:

\startDialog
\tell{He}{Does he understand Portuguese.}
\tell{Me}{Sure I do.}
\tell{He}{Then tell me, what does a guy from the first world do here in the third world? Why are you here? Don't you have problems to solve and analyse in your country?}
\stopDialog

\startAReminder{ein teil fehlt noch}
\stopAReminder

The young guys interrupts and wants to know if I speak English as well.

\startAReminder{wie war nochmal der name von ihm?}
\stopAReminder

After affirming he is asking me questions about Germany in rough and broken English slang. He says that he has been to South Germany for a month on a trip with his religious youth group. He says that he is not from São Paulo, just arrived a month ago and went immediately to the streets. The hectic guy is interrupting us, asking:

\startDialog
\tell{He}{Do you believe in god?}
\tell{Me}{No, I don't.}
\tell{He}{Whooo, Irmãos, did you hear that, he is not believing in God. You are not a good person if you don't believe in god.}
\tell{Me}{So what, I believe in something else...}
\tell{He}{Whooo, did you guys hear that, he is not believing in God.}
\stopDialog


Luckily, \Ma came to safe me but now he has to start arguing with the hectic guy:

\startDialog
\tell{\Ma}{Irmão, I lived on the streets for many years and I have never seen you in this area. You don't wanna tell me what I have to believe in or he.}
\stopDialog

\Gi finally calmed everything down with his conciliatory way of arguing, telling the hectic guy that it does not matter in what we believe and that one religion is not better then another. What counts that we show solidarity for each other. Point.

The hectic guy, now calming down, but ready to start another ritual. This time he offers everybody to drink from a bottle they are sharing. In the meantime a car stops right at our side, the driver asking through the window

\startDialog
\tell{\Ma}{E aí irmãos, what's up? How are things going? Anything's going on around?}
\stopDialog

The guys tell him that everything is relaxed, nothing special is going on. This seems sufficient for him and he drives away. Once the car is out of sight the hectic guy asks:

\startDialog
\tell{\Ma}{Irmãos, I wanna buy a new bottle. I need five Reais. Alemão \bracket{that is me}, can you spare some Reais?}
\stopDialog

I gave him the five Reais note I was carrying with me, the others said they do not have money. The hectic guys takes them and disappears. We did not see him that afternoon again. Then \Ma has to leave as well, \Gi and me stay a bit longer. \Gi is talking to the two drunken guys and the young guy starts to talk with me again. After a while he says

\startDialog
\tell{He}{Alemão, I just wanna ask you if you can give me five Reais. Look, its nothing personal. I just wanna tell you that we need Crack now and it's better for you to leave because it will become urgent soon and you probably don't want to be here then.  It's really  important now because we had our last stone already some hours ago. So I ask you, do you have five Reais now?}
\tell{Me}{Sorry, but I gave my money the other guy....}
\tell{He}{Look, its really important for us now. How long would it take for you to go home get the money?}
\tell{Me}{I life too far away. It would take 2 hours or so to come back...}
\tell{He}{Ok, then better you get off soon and if you could get any money it would really help us much...}
\stopDialog

\Gi is approaching me and we decide to leave because the situation won't become favourable for us, so its better to say good by. The young guy then called out to us:

\startDialog
\tell{He}{Come around the other day, alemão, we are always here, around this time.}
\stopDialog

While we are slowly walking towards \locMissingSrc{Sé} and then further on to \locMissingSrc{República} \Gi said:

\startDialog
\tell{\Gi}{It was good that we left, it was not safe any more, they turn crazy when they need crack. And the next time you better don't give them money. The one guy never came back and probably also went to get some crack. One stone is just five Reais.}
\stopDialog

This was the last time I saw \Gi for a while, until I run into him some weeks later at \gotoTextMark[Praca da Sé]{narratinginquiries_saopaulodiaries_penaforte}.

\subjectCrack

Two or three weeks after we met the \gotoTextMark[four guys]{narratinginquiries_saopaulodiaries_sundaystreetscentrefourguys} at a back road of \aLocationNew{Rua do Carmo}{http://osm.org/go/M@ziKw9eb--}, I pass there nearby. I hear somebody shouting

\startDialog
\tell{}{Alemão, come here!}
\stopDialog

Its the young guy we met back then, sitting there, close to the metro entrance of \locMissingSrc{Sé}, together with a bunch of other people. We greet each other and he starts immediately:

\startDialog
\tell{He}{Alemão, I go back to my home town. I can't stand this city any more. Since I arrived here I stayed on the streets, but I cannot stand it any more. I did not find work, Crack is killing me. I'm finished with this city, I have to leave for now, getting back my life.}
\tell{Me}{Good news! When do you wanna leave?}
\tell{He}{I take the bus next Tuesday. How long do you stay?}
\tell{Me}{Till November probably.}
\tell{He}{Then we'll meet again. My plan is to return to São Paulo in October or so.}
\tell{Me}{Look, take good care and who knows, probably we'll meet again...}
\stopDialog 

We never met again.

\subjectPCC

\inright{PCC and the streets}
On another day I meet \Ma again. We are talking about the \gotoTextMark[four guys we met]{narratinginquiries_saopaulodiaries_sundaystreetscentrefourguys} at a back road of \aLocationNew{Rua do Carmo}{http://osm.org/go/M@ziKw9eb--}. \Ma asks if I have an idea who that guy in the car has been, who was stopping and asking the crowd there. I say that he is probably police or something. \Ma says that this guy was no police but that he was patrolling for the \abbrNew{Primeiro Comando do Capital}{PCC}. He was patrolling and asked the people on the street what they have seen, if something happened, if police was around. \Ma asks me if I every heard that  loud fireworks somewhere, not during football games, but just when I was in the centre. He said that those fireworks are the signals for new deliveries arriving at the \toMark{bocas de fumo} and the one patrol we say probably belongs to one of them.

\spaceHalf

\inright{PCC attacks ROTA}
We continued to talk about \abbr{PCC}. Just some days ago, on the 31th of July, online and offline mass media \aLinkMediaNewF[Folha de São Paulo: ]{http://bit.ly/pTNkSJ},\aLinkMediaNewF[Último Segundo: ]{http://bit.ly/d01dWz},\aLinkMediaNewF[Estadão: ]{http://bit.ly/rnA6q0} reported that \abbr{PCC} launched an assault on a captain of the \abbrNew{Rondas Ostensivas Tobias de Aguiar}{ROTA}, a special unit of the \abbrFull{MP} of São Paulo. Two men tried to kill a captain of the \abbr{ROTA} in front of his house when he left in the morning. One day later, during the night from Sunday to Monday, again two men shot at the head-quarter of the \abbr{ROTA} that is located right beside \locParqueDaLuz, close to the city's \locMainTrainStation at \locLuz. Several journals report later on, that within 2 days after the last attack, 7 or 8 suspects have been killed by \abbr{MP} in São Paulo \bracket{\aLinkMediaNew[Radio Agência NP]{http://bit.ly/a8ciGl}, \aLinkMediaNew[Carta Capital]{http://bit.ly/d3gLpl}}. 

\spaceHalf

\startPersonal
\inright{who is PCC?}
The other day, I brought the July issue of \aQuoteInTextA{Caros Amigos}, inside an article about \abbr{PCC } called \aQuoteInText{Por dentro do PCC}\footnote{\toTranslate[Por dentro do PCC]{From within PCC}}\footnote{a shortened version of the article is available online as well \aLinkMediaNewF[Caros Amigos: Por Dentro do PCC]{http://bit.ly/o6NOs9}}. The article renders an interview with two anthropologist, Karina Biondi and Adalton Marques, who have conducted anthropological studies inside the prison system in Brazil\footnote{Karina Biondo: Junto e Misturado: uma ethnografia do PCC}\footnote{Adalton Marques: Crime, proceder, convívio-seguro: um experimento antropológico a partir de relacões enter ladrões}. In its introduction, the article states:
\stopPersonal

\startCitation
\language[pt]A Caros Amigos conversou com os dois antropólogos sobre os princípios e a organização do PCC, essa facção criminosa tão grande quanto pouco compreendida pela população do Estado com a maior população carcerária do Brasil \aQuoteB{Delamnto and Moncau}{2010}{36}\footnote{Caros Amigos spoke with the two anthropologists about the principles and the organisation of PCC, this criminal fraction, so large but little understood by the population of the state \bracket{of São Paulo} with the largest number of prisoners in Brasil.}
\stopCitation

\startPersonal
Me personally, I know little as well, my knowledge is not even partial, that's why I am seeking for more standpoints. The notion of \aKeyword[crime]{streets+standpoints+crime} and \aKeyword[jail]{streets+standpoints+jail} is omnipresent in the narrations of the people, in the narrations of the streets but also as issues for actions, for instance as a theme discussed by the \gotoTextMark[aRUAssa collective]{narratinginquiries_saopaulodiaries_aruacca_ideas} for a small film project. 
\stopPersonal

\startARemark{das ist noch überhaupt nicht gut} und muss auf jeden fall noch überarbeitet werden.
\stopARemark

\startPersonal
\abbr{PCC} is one reality of São Paulo I stumbled across but I do not aim to argue in favour or against it. For me, the situation is too complex to understand but I also do not want to neglect it because I have the feeling it is relevant to think about the reasons and manifestations of what is called crime, what is called the prison system, the police system, because they all produce the city \bracket{the lived space of society}, and they are produced by the city and affect social struggle in turn. 

\startARemark{wie kann die überleitung hier aussehen?} crackolândia, polizei, straße, crime?
\stopARemark

This may probably be their truth but I am seeking other standpoints. Then the picture becomes blurry and gets different notions:
\stopPersonal

\startCitation
As principais avenidas de São Paulo nunca estão desertas. Não posso enumerar os motivos que levam as pessoas a ganhar as ruas durante a madrugada, mas um deles conheço bem: é o dia de visita nas cadeias. À minha direita, reconheço essa motivação em duas mulheres que dividem o peso de uma grande sacola, provavelmente cheia de alimentos a serem entregues ao parente preso. Eu nunca havia notado esse tipo de movimentação antes da prisão do meu marido [...]\footnote{The main streets of São Paulo are always in motion. I cannot count all the motives that drive the people onto the streets at dawn, but one I know good enough: the visiting day at the prisons. To my right I recognize this motive in two women sharing the weight of a heavy bag, probably full of foot that they will deliver to an imprisoned relative. I have never noticed this type of movement before the imprisonment of my husband [...]}\footnote{This section of the book has been published in an interview with Karina Biondi \aLinkMediaNewF[Carta Capital: Fechado Com O Comando]{http://www.cartacapital.com.br/sociedade/fechado-com-o-comando}. I decided to cite it because it describes invisible facets of São Paulo better then I could do.} \aQuote{Biondi}{2010} in \aQuoteW{Huberman}{2010}.
\stopCitation

\spaceHalf

\inright{social movements affected by \quote{crime}}
Returning to \quote{the crime}, \Ma talks about another invisible face of the streets. He explained that at some of the place we visited, houses have been converted from occupations of social movements to illegal apartment complexes. He says that occupations of social movements are sometimes infiltrated by \quote{the crime} \bracket{according to his notion}. In those cases social movements are slowly drawn out by drug dealers for instance. Once the building has been entirely overtaken, rooms are prepared for renting and the building, initially occupied by social movement due to a lack of affordable housing, serves as an illegal apartment complex, generating profit. 

We have been in such a house, a former an industrial building, several floors high. Once this building was overtaken, walls have been brought up inside, establishing new rooms on each floor, ready to rent. In that house, probably 5 apartments are residing on each floor. The one where we have been was not very large, one room, a kitchen and sanitation.

Organized occupations are attempting to prevent such a development by defining and applying rules of conduct within their buildings. The \gotoTextMark[interview we made with people]{text mark missing interview} of a hotel occupied by \abbrNew{Frente da Luta por Moradia}{FLM} at \locMissingSrc{Avenida Ipiranga} is bringing this matter up as well. 

\spaceHalf

\reference[narratinginquiries_sometalks_presspcc]{}
\startPressCoverage{Press coverage about the attacks of PCC and the PCC in general}

\stopPressCoverage

\subjectVeganLunch

\inright{at Ay Carmela again}
Every last Sunday a month, the \aKeyword[collectively cooked vegan]{streets+actions+collective lunch} lunch at \Ay is my favourite place to be. For me such a day is important in various ways. 

First and foremost the lunch offers space for socializing. Its an event  open for everybody, starting at noon and organized by the \aKeyword[Ay Camela collective]{streets+actors+collectives+ay carmela}. Is a place for meeting friends and to get in touch with other persons. For me as a stranger this is important. The lunch is also meant to support the payment of bills of the space.

I personally love the place anyway because this is where I am coming from, what I consider important to organize and maintain, from an activist perspective. 

At \Ay I also \gotoTextMark[met friends like \Ma and \Gi]{narratinginquiries_saopaulodiaries_sundaystreetscentregibson} or the \gotoTextMark[aRUAssa collective]{text mark missing to arruassa meeting}. 

Besides those important reasons \bracket{personally spoken}, the organization of \Ay as a self-determined space is already an action of self-determination. 

\startCitation
O Espaço Ay Carmela! é um centro político-cultural autogestionário mantido por grupos, movimentos e indivíduos autônomos da cidade de São Paulo. Um lugar de construção de ações e conhecimentos coletivos, além de um pólo de produção, reunião e dispersão de informações, saberes e transformações. O Ay Carmela! é localizado no centro de São Paulo, próximo ao marco zero. E é mais uma forma de afirmar que o centro é nosso, das pessoas, de quem vive e circula por essa cidade e não do capital, das corporações ou do estado. \aQuoteW{Ay Carmela}{2010}\aLinkContentNewF[Ay Carmela: Sobre]{http://ay-carmela.birosca.org/Sobre}\footnote{Ay Carmela! is a self-determined cultural political centre maintained by groups, movements and autonomous individuals of the city of São Paulo. A place to construct collective ideas and actions, a pole to produce, assemble and disperse informations, knowledge and transformations. Ay Carmela! is located in the centre of São Paulo, close to the mark zero. It is another form to affirm that the centre is ours, that it belongs to the people that live in and move through this city and not to the capital, the corporations or the state.}
\stopCitation

\spaceHalf

\inright{who is Ay Carmela?}
\Ay offers infrastructure and space for collectives and movements to meet and organize. Even though it has to pay rent and bills, it seeks to balance them independent from institutional support, in a self-organized manner, by conducting events such as the lunch for instance \bracket{as one example of much more that has to be done}. 

I mention \Ay and the monthly lunch in particular because I would like to pick out three examples of usages and organization of the space: the \aKeyword[organic market]{streets+actions+organic market} organized by \abbrNew{Movimento dos Trabalhadores Rurais sem Terra}{MST}\footnote{\toTranslate[Movimento dos Trabalhadores Rurais sem Terra]{Movement of Landless Rural Workers} }\aLinkSourceContentNewF[MST]{http://www.mst.org.br/} during the lunch, \aKeyword[recycling]{streets+actions+recycling} by a collective of \aKeyword[Catadores]{streets+actors+collectives+catadores} and meeting place of the \aKeyword[aRUAssa film collective]{streets+actors+collectives+arruassa}, the latter two described elsewhere \gotoTextMark{narratinginquiries_saopaulodiaries_sometalks_recycling} \gotoTextMark{narratinginquiries_saopaulodiaries_aruacca}.

\imgAyPanoramaNightFestivalMapa

\startARemark{kurze erklärung von mst fehlt noch}
\stopARemark

\spaceHalf

\inright{mini-feira by MST}
\reference[narratinginquiries_saopaulodiaries_minifeira]{}People of the \Mst\aKeywordInVi{mst} \toTranslate[assentamento]{settlement} \aKeyword[Irmã Alberta]{streets+places+settlements+irmã alberta}\aKeywordInVi{mst+settlement+irmã alberta}, located in Peŕus at the fringes of the city of São Paulo, started to establish an \aKeyword[mini-feira]{streets+actions+mini-feira}\toTranslateT{mini-feira}{mini market} during vegan lunch \aLinkContentNewF[MST: Mini-Feira]{http://www.mst.org.br/node/10157}. 

\startCitation
O MST (Movimento dos Trabalhadores Rurais Sem Terra), por meio do assentamento Irmã Alberta, de Perus, na Grande São Paulo, estará no Ay Carmela vendendo produtos que foram produzidos no assentamento, no próximo domingo (27/6).

Serão verduras, legumes, frutas entre outros produtos que, além de serem fruto da luta pela terra, possuem qualidade (são orgânicos e cultivados sem agrotóxicos) e ótimos preços. Ou seja, você poderá comprar produtos saudáveis, baratos e contribuir com a luta popular brasileira \footnote{At next Sunday (27/6) the \abbr{MST} settlement Irmã Alberta in Perus, São Paulo metropolitan area, will be at Ay Carmela to sell products that have been produced by the settlement. Vegetables and fruits will not only be sold at good prices, they are of good quality (organic and cultivated without pesticides) and are the fruits of the struggle for land. Thus, you can buy healthy and cheap products and contribute to the popular struggle in Brasil.} \aQuoteW{Organização Popular Aymberê}{2010}\aLinkContentNewF[MST: Mini-Feira]{http://www.mst.org.br/node/10157}\aLinkSourceContentNewF[Organização Popular Aymberê]{http://www.opaymbere.wordpress.com/}.
\stopCitation

Usually a woman supported by several others arrived by car in the morning, bringing their products. They set up several small tables on which they put mainly organic vegetables, herbs, fruits\footnote{Such as feijão, milho or mandjoca} and coffee cultivated and produced in their settlement. 

All products are strictly organic, thus cultivated without \toTranslate[agrotóxicos]{pesticides} and \abbrNew{gene modified}{GM} plants. At the \aKeyword[mini-feira]{streets+actions+mini-feira}, 500 gram of organic coffee costs about 5 Reais, a similar price as one has to pay in cheap supermarkets \footnote{Bom Dia, for instance} for non-organic coffee. Organic coffee \bracket{as well as organic products in general} are luxury products, often only available at more expensive supermarkets \footnote{Pão de Açúcar, for instance} where it costs about 3 or 4 times as much as at the \aKeyword[mini-feira]{streets+actions+mini-feira}. \footnote{Prices can also be compared online, for instance at \aLinkNewNoF[nacional.com.br]{http://bit.ly/rqfzXn}.}

\spaceHalf

\startPersonal
\inright{political actions}
I perceive Vegan lunch and Mini-Feira as political actions. One purpose is to maintain space, Ay Carmela through donations by offering and collectively preparing lunch, the settlement through selling food. The reason that the mini-feira can be held here, is \bracket{among others} the availability of free space provided by the place Ay Carmela. At these days, people that are engaged in social struggles in the city frequent the place, but neighbours of the surrounding area are coming around as well, probably not all for lunch but for buying organic food. They may have just seen one of the distributed flyers on the streets in the neighbourhood and usually probably just pass by.
\stopPersonal

\imgTheFlyerAyAlmoco

\startPersonal
The practised modes of production and distribution are self-determined. Distribution is self-organized, directly brought by the settlement \bracket{the producer}, without intermediate dealers, not generating profit. The gathered money is used to maintain the spaces necessary to allow this practices. 

The organic food produced by the \aKeyword[Irmã Alberta]{mst+assentamento} settlement is a luxury product accessible through luxury supermarkets. At those Sundays it is shown on a small scale that healthy food is not supposed to be a matter of affordability and monetary accessibility, that those practices of production and distribution could benefit much more people as the common modes profit oriented production and distribution.
\stopPersonal

\spaceHalf

\inright{open university at the settlement Irmã Alberta}
Returning to the \aKeyword[mini-feira]{streets+actions+mini-feira} once more. One month before I had to leave São Paulo, the people of the \aKeyword[Irmã Alberta]{mst+assentamento} settlement are proposing to everyone interested to visit their settlement in order to participate in giving lessons in domains of personal knowledge and skills. They say that many of those living in the settlement will not be able to enter university because they may lack the necessary school degrees or simply cannot afford to travel by public transport to the campus everyday. Thus the settlement came up with the idea of an open university at the settlement. They said that many of them would like to learn English for example. They therefore proposed an open day for everybody to come and decide together with the people how classes could be organized for specific areas of interest, what topics are of interest, what topics could be provided, in what frequency are classes supposed to be conducted. I personally feel pity that I could not participate in that proposal because my time in the city was converging towards its end.

\startPersonal
One could categorized all those actions as informal work and organizing, thus a practice not according to legal rules. This would lack a large part of what is actually done here. The spaces involved here are self-organized, in the city centre as well as at the fringes of the city. The practice of these actions do determine a way of \bracket{self} organizing without the notions of profit making and exploitation by freely agreeing on the terms of what could be done and how it could be done. 

This practice means exchange between the urban space and the space at the fringes. It includes the access to healthy and affordable food in urban space and access to knowledge and education at the fringe, according to non-discriminatory agreements made among those that are interested in pushing this practice forward. Exchange is also meant physically, by visiting each other, working collectively with each other. 

It also shows that concrete geographical places are necessary to develop such a praxis. 
\stopPersonal

\subjectRecycling
\reference[narratinginquiries_saopaulodiaries_sometalks_recycling]{}

Another short narration. I went earlier to \Ay today, it is mid-morning. Inside I meet a man and a woman that are separating trash and bundle it to large packages. They are members of a \aKeyword[collective of Catadores]{streets+actors+collectives+catadores}. The self-organised space that \Ay represents is collectively used to gather waste in there and \aKeyword[process it further: separate it, bundle it up and transport it away]{streets+actions+recycling}. 

\imgCatadoresAyOne

\startPersonal
The existence of self-organized and free space allows collective organisation. The \gotoTextMark[mini-feira]{narratinginquiries_saopaulodiaries_minifeira} is based on free agreements between the space and the collective and represents yet an additional possibility that strengthens self-determined practice. The \aKeyword[aim of space and action]{streets+aims} are similar here as well. Its a proposal to \aKeyword[act collectively, in solidarity]{streets+aims+collectivity}, to organise the struggle for \aKeyword[self-determined work and life]{streets+aims+self determination}, instead of acting competitively, exploiting one another. This is not a difficult praxis. Even though affordable or free space is rare in a city like São Paulo \bracket{despite the many abandoned buildings}, where it is available, people and groups can start to practice and experience similar ways as the collective of catadores, the mini-feira and open university of \abbr{MST} or the vegan lunch of \Ay already do.
\stopPersonal

\toTranslate[Catadores de Lixo]{waste pickers} are often organized in the social movement of recyclers \abbrNew{Movimento Nacional dos Catadores de Materiais Recicláveis}{MNCR}\footnote{\toTranslate[Movimento Nacional dos Catadores de Materiais Recicláveis]{National Movement of Collectors of Recyclable Materials}}. Even though \Mncr \aLinkSourceContentNewF[MNCR]{http://www.mncr.org.br} is a national wide movement, it is organized in small and independent units on the streets in many cities, by that realizing an collective approach to work, self-determined, independent of class and political parties.

\startCitation
Acreditamos na prática da ação direta popular, que é a participação efetiva do trabalhador em tudo que envolve sua vida, algo que rompe com a indiferença do povo e abre caminho para a transformação da sociedade.

Desenvolvemos nossas ações na busca de uma sociedade mais justa e melhor para todos. Buscamos a organização de nossa categoria na solidariedade de classe, que reúne forças para lutarmos contra a exploração buscando nossa liberdade. Esse princípio é diferente da competição e do individualismo, busca o apoio mútuo entre os companheiros(as) catadores(as) e outros trabalhadores.

Lutamos pela autogestão de nosso trabalho e o controle da cadeia produtiva de reciclagem, garantindo que o serviço que nós realizamos não seja utilizado em beneficio de alguns poucos (os exploradores), mas que sirva a todos.

Nesse sentido organizamos bases orgânicas do Movimento em cooperativas, associações, entrepostos e grupos, nas quais ninguém pode ser beneficiado às custas do trabalho do outro. \aQuoteW{MNCR}{2008}\aLinkContentNewF[MNCR: O que é o Movimento ]{http://www.mncr.org.br/box_1/o-que-e-o-movimento} \footnote{We believe in the practice of popular direct actions, that is, effective participation of the worker in all spheres hat affect his life, something that breaks the indifference of the people and opens a way for the transformation of society. We develop our actions in search of a more just society, better for everyone. We are seeking to organize according to our terms of solidarity of the class, that unites power for our struggle against the exploitation that chases our freedom. This principle differs from competition and individualism. It seeks the mutual help between companions, catadores and workers. We struggle to self-determining our work and for the control of the means of production of recycling, guaranteeing that our service is not utilized to benefit a few (the exploiters) but benefits all. In this sense we organize the movement's organic bases as co-operations, associations, depots and groups in which nobody can benefit at the costs of the work of others.}
\stopCitation

\imgCatadoresEntrposto

\Catadores massively shape the image of the city, of its central areas. They pull large and self-made two-wheel trolleys, packed with materials collected from the streets, piled up two meters high. Trolleys full of material hold in place by tight ropes, ultra heavy, pulled by just one person \bracket{often men}, moving slowly through pedestrian areas, through the heavy traffic on packed streets, always moving on the outer right lane, a trolley the size of a small car, collecting stuff society has no use for any more.

When \Ma and I have been \gotoTextMark[looking for a place to sleep]{narratinginquiries_saopaulodiaries_dayandnight_lookingforaplacetosleep} one night, we asked a \Catador who was already sleeping in his trolley parked close to a wall, for \aKeyword{papelão} to share. His trolley was full of collected cardboards. His trolley was his bed for the night.

Recyclable material is everywhere. While \Ju and I are \gotoTextMark[on our way]{narratinginquiries_saopaulodiaries_republica_streetlife} towards \locRepublica, crossing the pedestrian area at \aNewLocation[Rua Barão de Itapetininga]{http://osm.org/go/M@ziJ2Vp4--} early at night around 7 p.m. when commercial business is closing, we always see piles of waste, cardboards, plastic bags, the daily residuals of consumption, thrown on the street. \Catadores are then gathering there, collecting, separating, piling up all the stuff they can make use of, a nightly ritual, the area occupied by trolleys and \Catadores, still working while everybody else is going home or is just arriving for nightly entertainment.

\startPersonal
I did not have much contact with \Catadores, only at those few occasion where we showed solidarity and met in a particular situation on the streets. They are workers, the ones we met are \Catadores in street situation. A bit of their available movement content is reproduced here and no in depth insights besides those few mentioned can be narrated.
\stopPersonal

\imgMncrDireitoACidadeOne
\imgMncrDireitoACidadeTwo
\imgMncrDireitoACidadeThree

\addReferenceMedia
{
caros amigos
}

\addReferenceMedia
{
carta capital
}

\addReferenceMedia
{
úlitmo segundo
}

\addReferenceMedia
{
estadão
}

\addReferenceMedia
{
folha de são paulo
}

\addReferenceMedia
{
radio agência NP
}

\addReference
{
Dossiê, 2009. Mapas do extermínio: execuções extrajudiciais e mortes pela omissão do Estado de São Paulo. Available at: \goto{\hyphenatedurl{http://www.acatbrasil.org.br/down/DOSSIE\_pena de morte final.pdf}} [url(http://bit.ly/oLtVlG)] [Accessed August 24, 2011].
}

\addReference
{
o Trecheiro, 2009. Notícias do Povo da Rua. {\em o Trecheiro}, (180). Available at: \goto{\hyphenatedurl{http://www.rederua.org.br/pub/otrecheiro/2009/180\_trecheiro\_agosto\_2009.pdf}} [url(http://bit.ly/mXpSTr)] [Accessed August 24, 2011].
}

\addReference
{
Moncau, G. \& Delmanto, J., 2010. Por dentro do PCC. {\em Caros Amigos}, (160). Available at: \goto{\hyphenatedurl{http://carosamigos.terra.com.br/index\_site.php?pag=revista\&id=145\&iditens=690}} [url(http://bit.ly/o6NOs9)] [Accessed August 26, 2011].
}

\addReference
{
Huberman, B., 2010. Fechado com o Comando. {\em CartaCapital}. Available at: \goto{\hyphenatedurl{http://www.cartacapital.com.br/sociedade/fechado-com-o-comando}} [url(http://www.cartacapital.com.br/sociedade/fechado-com-o-comando)] [Accessed August 28, 2011].
}

\addReference
{
Organização Popular Aymberê, 2010. Assentados vendem alimentos no centro de São Paulo. {\em MST - Movimento dos Trabalhadores Sem Terra}. Available at: \goto{\hyphenatedurl{http://www.mst.org.br/node/10157}} [url(http://www.mst.org.br/node/10157)] [Accessed August 28, 2011].
}

\addReference
{
Biondi, K., 2010. {\em Junto e misturado : uma etnografia do PCC}, São Paulo  SP: Editora Terceiro Nome.
}

\addReference
{
Rosa, C., 2005. {\em Vidas de rua}, São Paulo: Editora Hucitec  ;Rede Rua.
}

%---------------------------------------------------
% gets only displayed in unfinshed mode
\showImperfection

%---------------------------------------------------

\stopmode

\stoptext

\stopcomponent
