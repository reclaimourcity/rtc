\startcomponent component_Introduction
\product product_Thesis
\project project_MasterThesis

% definitions and macros
\environment envThesisAllEnvironments
\environment envCfgThesisImages

\define[]\chapterIntroduction{\chapter[intro]{Introduction}}

\starttext

\startmode[tocLayout]
\chapterIntroduction

Contains the thesis' introduction. The following draft tries to visualize the red line of the thesis.

\imgThesisRedLine

\stopmode

\startmode[draft]
\chapterIntroduction

Writing this introduction feels a bit strange. Its is an introduction to my master thesis but at the same time it is a retrospect, not only about the time I spend in São Paulo, as the thesis title is already revealing, but also a retrospective on what I did since then until this very moment, an imagination of this time in guise of a \rhizomaticMap, \aQuoteInText{a map that must be produced, constructed, a map that is always detachable, connectible, reversible, modifiable, and has multiple entryways and exits and its own lines of flight} as \aQuoteInTextA{Deleuze and Guattari} write in \aQuoteInTextT{A Thousand Plateaus}.

\spaceHalf

\inright{the thesis as rhizomatic map}
The tracings that I put on this map are unordered occurrences, therefore this thesis has also no strict order in the academic sense. This thesis is map-like, not tree like, because it has no real root. \inright{an entry point}One of the entry points to this thesis is my arrival in São Paulo, without a fixed schedule nor a predetermined path. I just started somewhere, unstructured, left tracings here and there in order to discover what I should do in the city, to reverse, modify or abandon the ideas I had in mind. I finally had first to discover that I arrived at a particular point in time, not at a beginning, nor an end, not at the beginning of a particular research project but somewhere in the middle, a \quote{somewhere in the middle} that is called São Paulo.

\startCitation
Those things which occur to me, occur to me not from the root up but rather only from somewhere about their middle. Let someone then attempt to seize them, let someone attempt to seize a blade of grass and hold fast to it when it begins to grow only from the middle. \aQuoteB{Deleuze and Guattari}{1987}{23}
\stopCitation

What I have seen and experienced, the insights I gained, my narrations that compose a fraction of this thesis, that represent some of the traces on the map, are my attempt to grasp the streets of the city, the organization and struggle of the people, the spaces that are created in a volatile urban world. Those spaces I entered are present in this thesis, they are my \aQuoteInText{somewhere in the middle}. My narrations are random, as the situation that  occurred to me, the situations I entered \aQuoteInText{somewhere in the middle}. They represent different conjunctions on the map.

Until now, I have been somewhat unconcrete about the topic of this thesis. To remind me it is called: \quote{\myTitle}.


\spaceHalf

\inright{introducing or nor?}
By leaving my \rhizomaticMap for a while, I could begin with an introduction to the topic. I could now present known and established facts, statistical data about the size of São Paulo, its number of inhabitants, the disparity between the rich and the poor, the number of people working in the so called informal sector or the number of people living in so called informal housing situations. By doing so I would determined the root of my research most likely in social sciences, conceptualized in predetermined hierarchical academic imagination.

\imgSaoPauloSatellite

What would this imagination reveal about the tracings on my \rhizomaticMap, about the struggle fought by people on the streets, the struggle that is supposed to compose parts of my map, my thesis, unconditioned? I guess they would reveal as much as this satellite photo above. It gives an impression of a size that I could never grasp, imagine, understand, explain, while being on the ground level, at zoom factor 1, at the streets of the city. Therefore I waive those numbers. \aQuoteInTextA{Deleuze and Guattari} would say:

\startCitation
Even in the realm of theory, especially in the realm of theory, any precarious and pragmatic framework is better than tracing concepts, with their breaks and progress changing nothing. Imperceptible rupture, not signifying break. The nomads invented a war machine in opposition to the State apparatus. History has never comprehended nomadism, the book has never comprehended the outside. \aQuoteB{Deleuze and Guattari}{1987}{24}
\stopCitation

\spaceHalf

\inright{another entry point}
Another entry point to my map are the imagination of the aims of urban struggle in São Paulo. Those aims are manifold and their accomplishment is in full progress. No start and no end here again, but much for extending my map. Tracings of proposals and answers to questions that determine social struggle: \WhoAreWe \WhatDoWeWant \WhatShouldWeDo. As it happened the way it did, my approaches to research and writing this thesis have been heavily influenced by those questions. I transformed them to ask \WhoAmI \WhatDoIWant \WhatShouldIDo.

Those questions are probably prevailing this text and they may indicate that it has been equally important for me to understand that my thesis is not a product or a project but that it is a processes, perceivable as part of the struggle of the people on the streets in São Paulo, that it is a small part of those struggles that seek to overcome the structural inequalities and discrimination that exists in the city. While reversing, modifying or abandoning the ideas I had in mind I learned that my attitudes towards research cannot be different from the attitudes prevailing in emancipatory struggle and also not different from my personal attitude. I suppose my \rhizomaticMap, my thesis as such, turned into a subject of research and by that into a subject of struggle.

\spaceHalf

\inright{one more entry point}
Here I eventually left hierarchical academic imagination, transformed my \rhizomaticMap by turning to movement imagination, a new entry point, by turning around to view my research as a process from the first sparkle of an idea passing through the times of struggles \bracket{in the city and in front of the laptop so to say} and not ending with the last lines in this document.

\spaceHalf

\inright{the last entry point}
The last entry point to my \rhizomaticMap leads to theoretical tracings that shall benefit struggle and that shall benefit research as part of struggle. A couple of my theoretical tracings aim to reflect on research as such and are arranged around the themes of \partialKnowledge formulated from a particular \standpoint, as well as \movementTheorizing as formulated from the \standpoint of movements. 

A couple of other theoretical tracings are arranged around the themes embedded in social struggles in São Paulo. Those themes are manifold and only a few will enter my \rhizomaticMap. Tracings are arranged around the themes of \participation and the \rightToTheCity as imagined by \HLefebvre, extended by \politics and \police as imagined by \JRanciere.

My \rhizomaticMap, this work, is not intended as an analysis of a particular situation, its is a subjective narration from my \standpoint and from the \standpoint of the streets. In a sense, it is also an experiment, an opportunity to learn and reflect, to propose new imaginations. This is what the next pages are supposed to do, to narrate, nothing more, and having said this I would like to finish for now in order to start. 

\addReference
{
Deleuze, G. \& Guattari, F., 1987. {\em A thousand plateaus : capitalism and schizophrenia 11th ed.}, Minneapolis: University of Minnesota Press.
}

\stopmode

\stoptext

\stopcomponent
