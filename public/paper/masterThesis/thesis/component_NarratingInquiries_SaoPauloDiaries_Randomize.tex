\startcomponent component_NarratingInquiries_SaoPauloDiaries_Randomize
\product product_Thesis
\project project_MasterThesis

% definitions and macros
\environment envThesisAllEnvironments
\environment envCfgThesisImages
\define[]\subsectionSaoPauloDiariesRandom{\subsection[narratinginquiries_saopaulodiaries_random]{City//Entropy}}

\starttext

\startmode[tocLayout]
\subsectionSaoPauloDiariesRandom
Random impressions.
\stopmode

\startmode[draft]
\subsectionSaoPauloDiariesRandom

\imgPrestesMaiaFacade

\imgOccupationLuz

\imgBuildingLuz

\imgMstcOne

\imgMstcTwo

\imgMohinoOne

\imgMohinoTwo

\imgMohinoThree

\imgMathAtViadutoDoCha

\imgGrajau

\imgMinhocaoSantaCecilia

\imgMinhocaoConsolocao

\addReference
{
FLM, 2010. Famílias do MSTC continuam ocupação no Parque D. Pedro, após uma semana. {\em Frente de Luta por Moradia}. Available at: \goto{\hyphenatedurl{http://www.portalflm.com.br/noticias/familias-do-mstc-continuam-ocupacao-no-parque-d-pedro-apos-uma-semana/282}} [url(http://www.portalflm.com.br/noticias/familias-do-mstc-continuam-ocupacao-no-parque-d-pedro-apos-uma-semana/282)] [Accessed September 12, 2011].
}

%---------------------------------------------------
% gets only displayed in unfinished mode

\showImperfection

%---------------------------------------------------
\stopmode

\stoptext

\stopcomponent