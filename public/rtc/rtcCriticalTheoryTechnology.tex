%{


\item who provides the server to put the mapping portal online and guarantees maintenance over a certain period
\item who adjusts the software to the needs of the project
\item how can homeless people utilize mobile phones or the web-platform in order to participate and generate information if we think about the little money they have, the high costs for sms and voice calls, the lack of energy and battery recharging possibilities, illiteracy rate, the access to computer (and here not only the physical access but also skill access, how to use the mapping platform, etc )
\item who is inserting data to the map over a period of time
\item who verifies the data over a period of time


(The usage of those technologies follows a certain logic which is finally immanent to them: due to the fact that we use machines, all data that is going to be processed must be turned machine readable. Thus, the process of feeding the machine with information must follow a certain pattern in order to allow to convert human input into digital information. )

Some keywords are already brought up in the given citation: {\em neutrality of technology} and {\em technical elite}. Other aspects are {\em division of labor} and forms organization, here {\em project based} approaches

{\bold neutrality of technology} if it is argued that technological tools can be adjusted to individual needs and used differently as intended by its creator, then it is also implicitly argued that technology is neutral, thus it just needs a few tweaks in order adapt it and change its purpose. But if one looks a bit closed and deconstructs certain forms technology, one can see that technology is constrained to function in the way the society, that developed it, functions. In the case of a mapping projects that utilizes mobile phones and online mapping platform this is also the case. 

{\em mobile phones:} only function where mobile networks are available. All network operators work according to capitalistic logic, thus a network is only available where profits can be made. In case of Brazil, the costs for sending SMS or for voice calls are extremely high [LINK], around RXX for an SMS and RXX for a voice call, with further complications if one needs to call different states. A mobile phone needs electricity to function but a person that lives on the street has often no possibilities to recharge the phone's batteries. Even if he or she lives in an Albergue, often no electricity plugs are available, thus recharging must be done in open places such as cultural centers. Other issues such as registration with the operator only with valid documents and buying a phone in the first place represent further constraints, but not in absolute terms as this can be solved by cheaply buying for example second hand phones and prepaid cards in various spots in the city.

{\em internet:} the internet 

{\bold digital inclusion/exclusion} the utilization of mobile phones and the Internet require in addition to the physical access to this technologies and devices also a certain set of skills, the most important one is literacy, other skills are basic usage of computers in general and the Internet.


{\em the mapping process}

{\bold technical elite}
%}