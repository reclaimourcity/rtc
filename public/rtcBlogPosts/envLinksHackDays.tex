\startenvironment envLinksHackDays

\define[]\myLinks%
{%
\startREF%
Casa da Cultura Digital, 2010. Casa da Cultura Digital. {\em Casa da Cultura Digital}. Available at: \goto{\hyphenatedurl{http://www.casadaculturadigital.com.br/}} [url(http://www.casadaculturadigital.com.br/)] [Accessed October 11, 2010].%
\nl%
FLM, 2010. Frente de Luta por Moradia. {\em Frente de Luta por Moradia}. Available at: \goto{\hyphenatedurl{http://www.portalflm.com.br/}} [url(http://www.portalflm.com.br/)] [Accessed October 10, 2010].%
\nl%
MNPR, FalaRua. {\em FalaRua}. Available at: \goto{\hyphenatedurl{http://www.falarua.org/}} [url(http://www.falarua.org/)] [Accessed August 23, 2010].%
\nl%
r3cl41m, 2010. Interview: Ocupação Avenida Ipiranga (en). {\em rtc}. Available at: \goto{\hyphenatedurl{http://rtc.noblogs.org/post/2010/10/11/interview-ocupacao-avenida-ipiranga-en/}} [url(http://rtc.noblogs.org/post/2010/10/11/interview-ocupacao-avenida-ipiranga-en/)] [Accessed October 18, 2010].%
\nl%
Rabatone, D., 2010. Transparência Hackday Moradia: 10 de outubro. {\em Esfera}. Available at: \goto{\hyphenatedurl{http://blog.esfera.mobi/transparencia-hackday-moradia-10-de-outubro/}} [url(http://blog.esfera.mobi/transparencia-hackday-moradia-10-de-outubro/)] [Accessed October 18, 2010].%
\nl%
Silva, D. \& Markun, P., 2010. Esfera. {\em Esfera}. Available at: \goto{\hyphenatedurl{http://blog.esfera.mobi/}} [url(http://blog.esfera.mobi/)] [Accessed October 18, 2010].%
\nl%
Ushahidi, 2010. Ushahidi :: Open Source Crowdsourcing Tools (FOSS). {\em Ushahidi}. Available at: \goto{\hyphenatedurl{http://ushahidi.com/}} [url(http://ushahidi.com/)] [Accessed October 18, 2010].%
\nl%
Wikipédia, 2010. {\em Mendigo}. In Wikipédia, a enciclopédia livre. Available at: \goto{\hyphenatedurl{http://pt.wikipedia.org/wiki/Mendigo}} [url(http://pt.wikipedia.org/wiki/Mendigo)] [Accessed October 18, 2010].%
\stopREF%
}

\define[]\myMedia%
{%
\startREF%
r3cl41m, 2010. {\em Transparência HackDay: Temático Moradia}, São Paulo: r3cl41m. Available at: \goto{\hyphenatedurl{http://www.archive.org/details/TransparnciaHackdayTemticoMoradia}} [url(http://www.archive.org/details/TransparnciaHackdayTemticoMoradia)] [Accessed October 18, 2010].%
\stopREF%
}

\define[]\myRefs%
{%
\startREF%
Feenberg, A., 2002. {\em 1. Introduction: The Parliament of Things}. In Critical Theory of Technology.  Oxford University Press. Available at: \goto{\hyphenatedurl{http://www-rohan.sdsu.edu/faculty/feenberg/CRITSAM2.HTM}} [url(http://www-rohan.sdsu.edu/faculty/feenberg/CRITSAM2.HTM)] [Accessed January 4, 2010].%
\stopREF%
}

\stopenvironment