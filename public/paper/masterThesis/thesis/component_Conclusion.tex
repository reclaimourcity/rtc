\startcomponent component_Conclusion
\product product_Thesis
\project project_MasterThesis

% definitions and macros
\environment envThesisAllEnvironments

\starttext

\startmode[tocLayout]
\chapter[conclusion]{Conclusion}

Contains the thesis' conclusion and further suggestions.
\stopmode

\startmode[draft]
\chapter[conclusion]{Conclusion}

\startARemark{ein erster versuch die Conclusion zu formulieren.}
\stopARemark
One practice of social struggle is is self-determination and autonomy. Struggle and actions are self-determinately organized in a participatory manner. Participation in the sense of taking part, as a way of creating new political spaces apart from the invited and de-politicized participatory spaces granted by institutional and other power agents. Those new spaces created through social struggle are the spaces for experimenting and learning, of organizing and practising different forms being together, of organizing life, production, work, differently from the current form which merely exclude and oppress those that are struggling. 

Thus organized and participatory struggle for a social transformation is not a coping mechanism of the marginalized in order to survive and to carry out what the society in general is supposed to carry out \bracket{or the welfare state for instance}. Participatory struggle is self-determined and aims to create new political space where none have been existing earlier, or the transformation of existing political spaces. 

The struggle for new political space is emancipatory because in those newly created spaces, participation is the principle for shaping it, to be heard, to respect the difference of the others and where ones own difference is accepted, where participation in the issues that affect ones own life is carried out in a just manner, not imposed through a power hierarchy and oppression. Minimum and Maximum Difference is what Henri Lefebvre imagines when he theorizes about the city, and the right to the city. 

Minimum Differences represents the current status quo, in society and the city. The city becomes homogenized, locally and globally if compared to other cites. Unwanted subjects that are considered incompatible are drawn out of neighbourhoods, central districts, even from the peripheries of the city. They cannot be assimilated in order to reach homogeneity.

Maximum Difference is what could be aimed by emancipatory struggle. Maximum Difference means respecting the others in their difference and being respected. Maximum Difference shall not be confused with individualism because Maximum Difference means constant social exchange and just negotiation with the others in order to determine the individual Maximum Difference.

Thus Maximum Difference is one of the bases the Right to the City is grounded on. Another base are the concept of spaces. Spaces have already been mentioned at several occasions \refMissingSrc{text mark mentioning spaces}, the city as lived space of the society, of the individual.


The city and thus the society it represents is supposed to be equalized in the sense that any difference between the single individuals is eliminated. In São Paulo and elsewhere worldwide those effects can be seen by uniform neighbourhoods, \bracket{new build} gentrification \bracket{at \refMissingSrc{Luz} for instance} where the old inhabitants are successively replaced by the city's ZEIS prorgam and the crack addicts expelled to more remote areas in order to improve the areas reputation and to construct a new upper class residential and leisure area.

\stopmode

\stoptext

\stopcomponent
