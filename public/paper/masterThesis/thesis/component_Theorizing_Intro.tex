\startcomponent component_Theorizing_Intro
\product product_Thesis
\project project_MasterThesis

% definitions and macros
\environment envThesisAllEnvironments
\environment envCfgThesisImages

\define[]\chapterTheorizing {\chapter[theorizing_intro]{Theoretical Themes}}

\starttext

\startmode[tocLayout]
\chapterTheorizing

Structure of concepts of the determined theoretical themes.

\stopmode

\startmode[draft]
\chapterTheorizing

\imgIParticipateFlyer

Structure of concepts of the determined theoretical themes.

\startARemark{Struktur und Inhalt des Theorizing Kapitels}

\Keyword{Citizenship}{theorizing+citizenship}

\aQuoteB{Arnstein}{1969}{216-224}\aQuoteInTextT{a Ladder of Citizenship Participation}

There is a critical difference between going through the empty ritual of participation and having the real power needed to affect the outcome of the process. 

The justification for using such simplistic abstractions is that in most cases the have-nots really do perceive the powerful as a monolithic "system," and powerholders actually do view the have-nots as a sea of "those people," with little comprehension of the class and caste differences among them.

citizenship exists only insofar as it is enacted, and its emerging figures have to do with empowerment strategies exercised in and from the cultural sphere. What the new social, ethnic, gender, gay and lesbian, religious or ecological movements demand is not only ideological representation but also socio-cultural recognition. They seek to become visible in their difference as citizens. This opens up a new mode for the political exercise of their rights, since this new visibility catalyzes the emergence of new political subjects. This was the subject visualized by feminism when it subverted the Left's profound machismo with its slogan: “the personal is political!” which came to embody both a sense of injury and victimization and a sense of recognition and empowerment.

\aQuoteQ{Barbero}{20}\aQuoteInTextT{Staging Citizenship: Performance, Politics, and Cultural Rights}

citizenship exists only insofar as it is enacted, and its emerging figures have to do with empowerment strategies exercised in and from the cultural sphere. What the new social, ethnic, gender, gay and lesbian, religious or ecological movements demand is not only ideological representation but also socio-cultural recognition. They seek to become visible in their difference as citizens. This opens up a new mode for the political exercise of their rights, since this new visibility catalyzes the emergence of new political subjects. This was the subject visualized by feminism when it subverted the Left's profound machismo with its slogan: “the personal is political!” which came to embody both a sense of injury and victimization and a sense of recognition and empowerment.

This is the only way we will escape the illusory quest for the reincorporation of alterity into some unified whole, be it nation, political party, or religion. Citizenship rights, those rights exercised today by the different cultural communities that constitute a nation, will then take center stage. This is the new value that attributes the human universality of rights to the specificity of its very diverse modes of perception and expression.

\Keyword{Participation}{theorizing+participation}

\aQuoteB{Arnstein}{1969}{216-224}\aQuoteInTextT{a Ladder of Citizenship Participation}

The poster highlights the fundamental point that participation without redistribution of power is an empty and frustrating process for the powerless. It allows the powerholders to claim that all sides were considered, but makes it possible for only some of those sides to benefit. It maintains the status quo.

\aKeyword{Max Difference}{theorizing+right to the city+max difference}

\aQuoteQ{Barbero}{20}\aQuoteInTextT{Staging Citizenship: Performance, Politics, and Cultural Rights}

The visibility of the Other—and every difference is an opportunity for dominance in a class-based society—together with the diversity of each contested identity today (contested not only in relation to other identities but in relation to itself) is a constitutive part of the recognition of rights. This is expressed in the phonetic similarity and semantic articulation of visibilidad (visibility) and veedurías (community oversight committees): those practices of investigation and intervention by citizens in the public sphere. 

\stopARemark{kommt noch ins methodology kapitel: experience - identity}

\aQuoteQ{Barbero}{20}\aQuoteInTextT{Staging Citizenship: Performance, Politics, and Cultural Rights}

Identity is not, then, what is attributed to someone by simply belonging to a group, but rather the narration of what gives meaning and value to the life and identities of individuals and groups.

\startARemark

\stopARemark{kommt noch ins methodology kapitel: action research - academia and research - listen to everyting that makes noise, speaks}

\aQuoteQ{Barbero}{20}\aQuoteInTextT{Staging Citizenship: Performance, Politics, and Cultural Rights}

What can academia and research do with all this? They can leave restrictive disciplines behind and begin to listen ethnographically to everything that speaks, screams, curses, makes noise, blasphemes, at the same time as it inaugurates, invents, energizes, liberates, emancipates, and creates. We are being compelled to think in a new way—one that, amid the frenetic globalization that threatens cultures, demands that we "reconstruct our local meanings," even those belonging to the most globalized practices and dimensions of social life. Every cultural interaction is always carried out by situated actors, and the meanings of enacted practices or reclaimed rights will ultimately lead us to social uses rooted in time and space. From this perspective, we discover that social ways of knowing do not exist simply to be accumulated and transmitted, but to be exercised as citizens, to be enacted performatively.

\startARemark

\startARemark{kommt noch in den open access teil}

\aQuoteW{Boyd}{2008}\aQuoteInTextT{open-access is the future: boycott locked-down academic journals}

\stopARemark

%---------------------------------------------------
% gets only displayed in unfinished mode

\showImperfection

%---------------------------------------------------

\stoptext

\stopcomponent
