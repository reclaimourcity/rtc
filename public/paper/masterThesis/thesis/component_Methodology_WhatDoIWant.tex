\startcomponent  component_Methodology_WhatDoIWant
\product product_Thesis
\project project_MasterThesis

% definitions and macros
\environment envThesisAllEnvironments

\define[]\sectionWhatDoIWant {\section[methodology_whatdoiwant]{What do I want?}}

\starttext

\startmode[tocLayout]

\sectionWhatDoIWant

The definition of concrete and general objectives.

\stopmode

\startmode[draft]

\sectionWhatDoIWant

Reflecting on the question \aKeyword[What do I want?]{thesis methodology + what do I want?} I would like to outline what I imagine the intention of this thesis could be, what its effect could be, how I think my research is supposed to be realized. Different \aKeyword{objectives} have been already mentioned when I tried to determine \aKeyword[Who am I?]{thesis methodology + who am I?} and \aKeyword[What should I do?]{thesis methodology + what should I do?}. Those objectives represent different scopes that I would like to gaze when asking \aKeyword[What do I want?]{thesis methodology + what do I want?}. 

\spaceHalf

\inright{guiding objectives}
On the one hand, certain objectives are not linked to the results of theorizing, thus they are not directly linked to the written result of this thesis. I perceive those objectives as immanent features of the process of research action(s). By realizing research action(s) based on the ground I prepared while asking \gotoTextMark[Who am I?]{methodology_whoami}, I realize those objectives, or not. I perceive those objectives are \aKeyword[guiding objectives]{objectives+guiding} for me and my \bracket{research} praxis, that I could apply in other contexts' as well because they are to a large extend not directly related to the situation in São Paulo, even though their relevance became apparent to me only when I stayed in São Paulo, together with my people. 

Looking back at the time in the city and as already mentioned at \gotoTextMark[other occasions]{methodology_whoami_experience}, these guiding objectives represent an kind of optimal praxis because reaching all of them or even just a fraction is difficult. They depend solely to a certain extend on my own praxis, will and attitude. Important factors such as time or the lack thereof for instance, are hard to influence because they may be imposed onto me by larger structural factors, the limited time frame this thesis has to be completed or the long lasting process of building relations based on friendship, solidarity and trust. 

Therefore I consider the \aKeyword[guiding objectives]{objectives+guiding} of my thesis as framework I try to act upon but I also know that I cannot and will not achieve all of them.

\spaceHalf

\inright{thesis objectives}
The other type of \aKeyword[objectives]{objectives+theorizing} I would like to mention are those that are directly linked to the thesis theorizing, its written form and the outlook beyond thesis completion. 

By writing this thesis from an \aKeyword[activist perspective]{perspective+activist}, by not being an academic observer but also by not being entirely engaged in struggle \gotoTextMark[as mentioned elsewhere]{methodology_whoami_actionresearch_engagement}, I feel a certain \aKeyword[dilemma]{thesis+dilemma}. I do not intend to propose what to do next from an academic perspective but I also cannot claim transformation as radical and profound as the people in São Paulo do, because I am still an alien in a way, even though not a complete stranger, but now disconnected from the city and its people, especially while writing this lines, back at home. I think I would feel better if I had formulated this thoughts after a couple of years of living and intense experiencing, as continuation of what I started to experience during my time in São Paulo, which is still an option to realize, but not at this very moment. 

I base this thesis theorizing on street experience, on street and movement theorizing, and I consider this thesis as part of the struggle of the people. In the midst of my dilemma I certainly believe that this thesis could contribute to a gain a different perspectives of the lived urban space and struggle we are all part of in our \bracket{everyday, academic, activist, marginalized, privileged, criminal} life, a different perspective on the processes that produce and shape our lived space, what we are doing to cope with that and what we are doing to transform it and possess the means of production of our lived space, the city, thus us.

\spaceHalf

\inright{the lack of a research question}
A final note though on the lack of a \aKeyword[research question]{thesis+research question}. In my current position I don't feel like formulating a research question. I could propose formulating a question about possible strategies for movement struggle but the current frame of my research does not offer the space for such a proposal nor does I feel that I am involved in any struggle in São Paulo to such an extend that it would make sense to even start about thinking about such a proposal. The \aKeyword[limiting factor time]{limiting factor+time} that \gotoTextMark[has been mentioned already]{methodology_whoami_experience_limitingfactortime} is one of the factors that needs to be eliminated in the first place in order to become actively involved in order to participate in struggle. At this moment, my thesis already pose a multitude of questions upon me, more than I could answer right now by realizing this thesis. I perceive my current research just as the first step that could be followed by another one, another day, that can draw on the experiences I made here. 

\showImperfection

\stopmode

\stoptext

\stopcomponent
