\startcomponent component_MethodologySelfConception
\product product_Thesis
\project project_MasterThesis

% definitions and macros
\environment envThesisAllEnvironments

\define[]\sectionSelfconception{\section[methodology::selfconception]{Putting it all together: Formation of a self-conception}}

\starttext

\startmode[tocLayout]

\sectionSelfconception

This sections combines all previously made statements in order to give a complete overview of the used and applied methodologies. It is going to construct the red lines through that guides through the thesis.

\stopmode

\startmode[draft]

\sectionSelfconception

A further conceptualization of the given themes may help to determine and construct the red line that is going to guide through this work. 

\spaceHalf
\startQuestionBack 
\startlines
{\bf Why should my research be relevant?}
\stoplines
\stopQuestionBack 
It should be relevant if it can be embedded in the struggle of the people. It is another tool which can be deployed in order to be part of the struggle it is talking about instead of being solely an academic analysis of the situation.

\startQuestionBack 
\startlines
{\bf How does my research becomes relevant?}
\stoplines
\stopQuestionBack 
its is open accessible - it is embedded and respects the local context - is or becomes part of the struggle of the people - not just academic work 

\startQuestionBack 
\startlines
{\bf What are the goals of my research?}
\stoplines
\stopQuestionBack 
no judgement of people, no analysis of people, no proposal for people, together with the people, neglecting of academic framework - in terms of time

\startQuestionBack 
\startlines
{\bf Who is the audience and who are the beneficiaries?}
\stoplines
\stopQuestionBack 
The audience can be 

\stopmode

\stoptext

\stopcomponent
