\startcomponent component_Methodology_WhoAmI
\product product_Thesis
\project project_MasterThesis

% definitions and macros
\environment envThesisAllEnvironments

\define[]\sectionMethodologyWhoAmI {\section[methodology::whoami]{Who am I?}}

\starttext

\startmode[tocLayout]

\sectionMethodologyWhoAmI

Here I define the thesis self-conception and the thesis approach to research.
\stopmode

\startmode[draft]
\sectionMethodologyWhoAmI

\inright{research is political}
I conceive my thesis and its research actions as inherently political. Political due to the fact that I consider this thesis as a medium that complements the struggle of the people I collaborated with. Political as well because I understand my approach to research actions as the intention to act in a non-hierarchical manner (even though this is not achievable to full extend). I understand non-hierarchical praxis as a critique of the current status quo of social interactions. Further depictions of these characteristics follow in the next sections of this chapter.

\spaceHalf

\inright{research based on the narrators standpoint}
Therefore, I personally consider important to make my personal attitude transparent in order to assess in how far this attitude affected and affects the course of this thesis. This thesis is not aimed to express objective narratives because objectivity is for me hard to achieve when expressing those experiences, narratives and observations that represent the foundation of this thesis. The concept of objectivity, which from my perspective means (absolute) neutrality, devoid of symbols and biased interpretations, thus the plain and \quote{real} nature of an object, is not realizable and probably not seminal for me either, because I cannot and don't want to disconnect myself from what I experienced, what people experienced, from my attitude that clearly influenced to large parts the realization of my research action(s) in São Paulo. 

The research actions that are assembled in this thesis are those that I volunteered to perform and experience or that just happened by incident, but never through external \quote{pressure} or other-directed \footnote{guided by external standards}. I could have chosen another frame, an existing academic or NGO project on the same topic, where I probably would have met the same people and visited the same places, but which perhaps would have resulted in a totally different outcome, based on other standpoints and attitudes. Is the reality I experienced then more valid than the others or vice versa? I think not, both have their legitimacy, they are probably motivated differently and by that narrate different stories, describe the same realities from different standpoints based on the narrators realities and context. In the words of Schrödinger I would then say

\spaceHalf

\inright{"The Fundamental Idea of Wave Mechanics", Nobel lecture, (12 December 1933)}
\startCitation
We cannot, however, manage to make do with such old, familiar, and seemingly indispensable terms as "real" or "only possible"; we are never in a position to say what really is or what really happens, but we can only say what will be observed in any concrete individual case. Will we have to be permanently satisfied with this...? On principle, yes. On principle, there is nothing new in the postulate that in the end exact science should aim at nothing more than the description of what can really be observed. The question is only whether from now on we shall have to refrain from tying description to a clear hypothesis about the real nature of the world. There are many who wish to pronounce such abdication even today. But I believe that this means making things a little too easy for oneself. \aQuote{Schrödinger}{1933}
\stopCitation

By not aiming to reproduce narratives in an objective way I do not mean to dismiss the idea of neutrality. The content of the thesis aims to reproduce those narratives as close as possible to the positions, ideas and words of those that shared them.

\stopmode

\stoptext

\stopcomponent
