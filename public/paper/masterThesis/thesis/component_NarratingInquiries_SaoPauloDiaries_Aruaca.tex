\startcomponent component_NarratingInquiries_SaoPauloDiaries_Aruaca
\product product_Thesis
\project project_MasterThesis

% definitions and macros
\environment envThesisAllEnvironments

\define[]\subsectionSaoPauloDiariesAruacca{\subsection[narratinginquiries_saopaulodiaries_aruacca]{Collective//aRUAcca}}

\define[]\subjectFirstEncounter{\subject{First//Encounter}}
\define[]\subjectStreetCinema{\subject{Cinéma//Rua}}
\define[]\subjectMeetings{\subject{Meetings//Actions}}
\define[]\subjectIpiranga{\subject{Interview//Ocupação Ipiranga}}

\starttext

\startmode[tocLayout]
\subsectionSaoPauloDiariesAruacca
This sections contains the experience made with the Aruaça Film Collective initiated by homeless people and indymedia brasil.
\subjectFirstEncounter
The first encounter with a autonomous collective in São Paulo.
\subjectStreetCinema
An invitation to street cinema.
\subjectMeetings
Some Experiences from meetings and movie shots.
\subjectIpiranga
An interview made with the Ocupação Ipiranga.
\stopmode

\startmode[draft]
\subsectionSaoPauloDiariesAruacca

\subjectFirstEncounter

On Saturday \bracket{29th of May 2010}, a massive \aKeyword[film-making workshop]{streets+actions+workshops+independent film making} took place at \Ay, close to \Se, at \locRuaCarmelias. A crowd of around 10-15 people have been attending the class, some of them regular participants, other just from time to time. 

The workshop consists of a series of classes in which \peopleInStreetSituation from the central areas shoot and cut their own \bracket{short} films. They use mobile devices for filming and free software for cutting and mastering. At this Saturday, three films have been already finalized and more are to come. There is supposed to be a public screening at \Se and other public spaces, on the streets.

The workshop is now in its second period and is going to be finished soon. The films are currently shot with a digital camera but at the beginning, mobile phones have been used for recording. The main problem of recording with mobile phones was their poor sound quality. Subtitles \bracket{in Portuguese} had to be added in order make interviews or conversations understandable. The current short films are produced with a digital camera and have a much better image and sound quality. During the class, several discussions among the people took place which emphasized on the advantages and disadvantages of the current workshop approach.

\spaceHalf

\inright{perceived drawbacks}
Aspects that still need improvement are:

The films are cut and mastered with \bracket{free} software. The people usually do not know how to use this software, thus, for cutting and mastering, special knowledge is necessary which excludes almost all participants from taking part in this process. 

Cutting and post production needs hardware with a better performance as accessible to the people. 

The mobile phones used previously could not produce video records in suitable quality, especially when recording speech, thus subtitles were necessary.

\spaceHalf

\inright{perceived advantages}
Some of the predominantly positive aspects have been formulated as:

The people recognize that they are able to produce own media if they get advice in order to learn how the technology part is working \bracket{camera, software, mastering}. 

The people recognize that the current approach has several drawbacks, in terms of \bracket{non} participation during post processing but also in terms of applied technology in general, which may lead in the rethinking of means of production \bracket{more simple to use}. 

The people have a huge amount of ideas to realize, in terms of film-making but also in terms of organizing themselves and spreading the information about their project to the public \bracket{screenings, contact to the press}. The people have a strong desire to show and talk about the situation they life in. Their films are a powerful medium to transport desires, problems and wishes to a broader audience.

\startPersonal
Actually, this workshop has been the start of putting tracings on my map of Säo Paulo. There I met many of the friends I spend time with later on. Interestingly, after this day, I did not see the people for one month or more. Only after meeting \Ju by chance again \gotoTextMark[one day]{text mark missing ju} we started to spend plenty of time together.
\stopPersonal

\subjectStreetCinema

The films that has been produced by the \aruassa collective are finally shown at public space. Here the invitation:

\startlines
{\bf When}: 26/06 – Saturday at 19h

{\bf Where}: projection at the outside walls of \aLocationNew{Pátio do Colégio}{http://osm.org/go/M@ziK2ABn--} (very close to \Se) Address – Praça Pátio do Colégio, 84 – São Paulo

{\bf Program}:

\em{Oração do Artista} (experimental – Valter Machado)
\em{Mané Taitana} (videoclip – Bob Neto)
\em{O homem que queria ser e (se) foi} (experimental – Sidney Cardoso)
\em{Rua do Carmo com Tabatinguera/esquinas} (entrevista – Valter Machado e Mateus)
\em{Atormentados} (videoclip – Bob Neto)
\em{Despretenso} (experimental – Felipe)
\em{Esconde-esconde} (experimental – kit e Valter Machado)
\em{Aspirina} (videoclip – Bob Neto)

All videos have been produced independently and autonomous with members of \abbrNew{Movimento Nacional de População de Rua}{MNPR} \footnote{\toTranslate[Movimento Nacional de População de Rua]{National Movement of People In Street Sitaution}}\aLinkNewF{http://falarua.org/} and with help of \toTranslate[Centro de Mídia Independente]{Indymedia}\aLinkNewF{http://midiaindependente.org/}, at \aLinkNew[Espaço Ay Carmela]{http://ay-carmela.birosca.org/}.

Spread the word!

Obs: the filmmakers will be present to give autographs!
\stoplines

\subjectMeetings

\reference[narratinginquiries_saopaulodiaries_aruacca_ideas]{}
\startDialog
\tell{\Ed}{Look, when you leave jail, you cannot return to your family or friends. They won't accept you any more, you are stigmatized. You are also stigmatized by the society because who wants to employ you or rent an apartment to you when you tell them that you come straight from jail. So you have no money and no perspective. You have just two options, either you enter the streets and you enter crime. The prison does not re-socialize you. You enter jail as a part time criminal but you leave it as a professional.}
\stopDialog

\subjectIpiranga

%---------------------------------------------------
% gets only displayed in unfinshed mode

\showImperfection

%---------------------------------------------------

\stopmode

\stoptext

\stopcomponent
