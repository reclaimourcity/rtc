\startcomponent component_MethodologyTheoryAndPractice
\product product_Thesis
\project project_MasterThesis

% definitions and macros
\environment envThesisAllEnvironments
\environment envCfgThesisImages

\define[]\sectionTheoryAndPractice{\section[methodology::motivation::theoryandpractice]{From theory to practice or vice versa?}}

\starttext

\startmode[tocLayout]

\sectionTheoryAndPractice

In this section, field-trip and theoretical approaches are lain out in detail.

\stopmode

\startmode[draft]

\sectionTheoryAndPractice

\startKeywords
deductive, inductive, empirical, theoretical, qualitative, quantitative, participatory, tyranny, hierarchies, authority, emancipation, self-determination
\stopKeywords

In order construct a methodological framework based on the questions and arguments formulated above, it may first be interesting to explore the direction this thesis is veering towards. Due to the fact this thesis is composed of theory and action, their extend could be be determined first.

\imgThesisTheroy

Several important aspects have been identified, mainly during the course of research actions in São Paulo, thus it can already be said that the shape of this thesis is strongly influenced by the experience made during those actions. The following factors have been identified and may therefore determine the concrete outcome and notion of the thesis: 

\startitemize
\item observation, empiricism and theory
\item induction and deduction
\item quality and quantity
\stopitemize

Those aspects can be aligned to the considerations that have been made about relevance, non-hierarchic/non-authoritarian behaviour and self-determination.

\stopmode

\stoptext

\stopcomponent
