\startcomponent component_Theorizing_Deconstruction
\product product_Thesis
\project project_MasterThesis

% definitions and macros
\environment envThesisAllEnvironments
\environment envCfgThesisImages

\define[]\sectionTheorizingDeconstruction{\section[theorizing_deconstruction]{De//Construct//Themes}}

\starttext

\startmode[tocLayout]
\sectionTheorizingDeconstruction

Deconstructing the meaning of theorizing themes.

\stopmode

\startmode[draft]
\sectionTheorizingDeconstruction

\imgIParticipateFlyer

\spaceHalf

\inright{deconstruction of participation}
The poster above is a wonderful opening for \deconstructing \participation. It says

\startCitation
I participate, You participate, he and she participates, We participate, You participate, They profit.
\stopCitation

and thereby precisely points to the deficiencies in the concept of \participation as it was meant back then and as it is meant today. This poster\footnote{A List of May 68 graffiti is documented at the \aLinkNewNoF[Bureau of Public Secrets]{http://www.bopsecrets.org/CF/graffiti.htm}} has been put on the streets in Mai 68, when students and workers gained momentum in 

\startCitation
the largest general strike that ever stopped the economy of an advanced industrial country, and the first wildcat general strike in history; revolutionary occupations and the beginnings of direct democracy; the increasingly complete collapse of state power for nearly two weeks; the resounding verification of the revolutionary theory of our time and even here and there the first steps toward putting it into practice; the most important experience of the modern proletarian movement that is in the process of constituting itself in its fully developed form in all countries, and the example it must now go beyond — this is what the French May 1968 movement was essentially, and this in itself already constitutes its essential victory. \aQuoteW{Bureau of Public Secrets}{1969}
\stopCitation

\quote{We all participate - We all benefit} is what one could excerpt out of this tiny insight. \Participation in putting revolutionary theory into praxis, overthrowing the capitalistic status quo for a while and genuinely participating in practising life with each other.

\quote{We participate - They profit} is what \participation is about in present \toMark{post-capitalistic?} times. We are invited to participate in many things. We are invited to crowdsource the net, the user is the producer, we can \refMissingSrc{Digg our most favourite blog post}, we participate in the construction of social communities\footnote{From Facebook, to Orkut and Google+}, we can conveniently participate in social campaigns with a \refMissingSrc{simple click}, the marginalized and oppressed are invited in participatory development projects that seems to end their poverty or at least reduce it, crisis management during times of catastrophes \refMissingSrc{incorporate citizen reports} from the ground in order to be more effective and precise in providing help, its seems that we can even participate in the planning of our direct environment, our neighbourhood, when local governments decide to give us a tiny share of \bracket{evaluation} power or at least allows us to select our most favourite proposal for a new shopping mall right around the corner, we are told to show civil courage to help the people on the streets around the corner, at work we can participate in unions to fight for better working conditions \bracket{at least where unions are established}, and last but not least it seems that we may even participate in \bracket{the illusion of} impacting public politics by putting a vote once in a few years. 

Even though this listing is relatively arbitrary and polemical, it may show what is currently understood by \participation. We participate in proposals of others, our participation becomes their profit. We are invited to participate by those that have the power to invite us, that already decided what we are supposed to do or that provides us with options for predetermined alternatives. This contemporary concept

\startCitation
highlights the fundamental point that participation without redistribution of power is an empty and frustrating process for the powerless. It allows the powerholders to claim that all sides were considered, but makes it possible for only some of those sides to benefit. It maintains the status quo. \aQuoteB{Arnstein}{1969}{216-224}
\stopCitation

This quote by \aQuoteInTextA{Sherry Arnstein} is excerpted from her \aQuoteInTextT{Ladder of Citizen Participation} that she has written from the standpoint of \participation of the excluded citizens in housing and urban planning decisions in the US. 

The \Ladder illustrates and denounces hierarchies within concrete projects, where the powerless and powerholders apparently \participate with the intention to genuinely propose ideas and articulate decisions about issues that bear direct constraints and necessities in the life of powerless.

\placefigure[force]{The \Ladder by \aQuote{Sherry Arnstein}{1969}}
{
\starttyping

	/BTEX \tfa\ss\color[green]{citizen control} /ETEX
			/BTEX \tfa\tt{citizen power} /ETEX			
	/BTEX \tfa\ss\color[green]{delegated power} /ETEX
			/BTEX \tfa\tt{citizen power} /ETEX			
	/BTEX \tfa\ss\color[green]{partnership} /ETEX			

	/BTEX \tfa\ss\color[green]{placation} /ETEX				
			/BTEX \tfa\tt{tokenism} /ETEX			
	/BTEX \tfa\ss\color[green]{consultation} /ETEX			
			/BTEX \tfa\tt{tokenism} /ETEX			
	/BTEX \tfa\ss\color[green]{informing} /ETEX			

	/BTEX \tfa\ss\color[green]{therapy} /ETEX
			/BTEX \tfa\tt{nonparticipation} /ETEX			
	/BTEX \tfa\ss\color[green]{manipulation} /ETEX
	
\stoptyping
}

The \Ladder serves as a basic starting point for \deconstructing \participation. Looking at \participationDe through the lenses of \powerStructures one could probably perceive more emerging issues caused by the \bracket{non}distribution of \powerDe. \Participation becomes \coopted and exploited by the powerholders to maintain the status quo, \participation becomes \depoliticized and does not aim to overcome structural inequalities beyond the local where \toMark{they take place}, \participation shrivels to a \technocraticMean to already predetermined ends.

\spaceHalf
\inright{1. non-distribution of \powerDe}
\ParticipationDe is a form of \refMissingSrc{tyranny} under current circumstances. It does not eliminate structural inequalities but rather reproduces the structural \powerHierarchies that are causing them. In this sense, the status quo is maintained or further enforced, the marginalized remain marginalized and the powerful remain in power. Maintaining \powerStructures is done as long as people are merely invited into predetermined participatory spaces set up and controlled by the power holders. 

\startCitation
[...] actions taken by the poor within the invited spaces of citizenship, however innovative, aim to cope with systems of hardship and are sanctioned by donors and
government interventions. \aQuoteB{Miraftab}{2006}{195}
\stopCitation

At \invitedSpaces, the powerless are manipulated and treated \aQuote{Sherry Arnstein}{1969}. Education and therapy are the means to provide the powerless with coping mechanisms for further survival in a persistent, structurally unequal, system. 

\toMark{At \invitedSpaces, the powerless are informed, consulted and placated \aQuote{Sherry Arnstein}{1969}.}

\spaceHalf

\inright{2. \coopting the idea of participation for single sided benefits}
\startCitation
[...]the ‘community’ has periodically been destroyed by underlying processes of development, only to be resurrected as the proper source of recovery through trustee-led intervention” \aQuoteB{Hickey and Mohan}{2004}{10}.
\stopCitation

As \aQuoteInTextA{Hickey and Mohan} put it, the very \powerStructures that produced and produces structural inequalities and exclusion are now proclaiming salvation by inviting the oppressed to take part in fixing their situation. \aQuoteInTextA{Hickey and Mohan} write from the standpoint of \bracket{international} development agency and address their failure of not having entailed structural transformation for the people, a transformation of \powerDe that would be finally directly exercised by the people.

A widening of their perception could conclude that those power structures that caused and causes exclusion and discrimination in society, no matter the form, are now even benefiting by apparently genuinely offering space for participation, by inviting the different groups and individuals that compose society, thus most of us, by pretending that all voices are being heard, that the grassroots are being involved, eventually in order to remain in control, to prevent dissent and self-determination and thereby further exercise power over the people in street situation, the inhabitants of favelas, the students at universities, the inhabitants of social housing, or the workers in precarious service industries for instance.

\startCitation
It is a kind of theft – to take away the valuable things of the people and to put them to work in a system that is against the people but in favor of the powerful and the rich. Not only the municipalities and the politicians but also many of the NGOs and ‘civil society’ structures and activists are guilty of playing a part in this ongoing theft against the people. It can make you feel like your struggle was useless. You fight for justice – for equality and for the world to be shared – and you end up with the promise of ‘service delivery’.“ \aQuoteW{Abahlali baseMjondolo}{2010}
\stopCitation

\spaceHalf
\inright{3. \depoliticization of participation}
This type of \participation is stripped off the political dimension, it is depoliticised because its aim is not learning and understanding how existing structures of power, oppression and injustice operate \aQuoteB{Hickey and Mohan}{2004}{11} in order overcome them. Once those political questions are excluded, how can \participation bear any idea of genuine social transformation that leads to a just society? It is then merely a method, a means to a different end, a \aQuoteInText{empty ritual}\aQuote{Arnstein}{1969} that does not discover new and uncontested spaces shaped by the people but that keeps them locked up in the current spaces of exclusion and powerlessness. \Participation becomes an obligation for every \quote{good citizen}, it becomes a must have for \quote{just} development projects, it is courteous for \quote{good governance practices} exercised by local, national and international institutions, but the structures of power never changes.

\startCitation
Western political thought has been that the central task of any political regime is education. The survival of the prevailing order depends upon depoliticizing young
people by making good citizens of them, by inviting, or even compelling them to participate. It is with good reason that we have long pathologized the figure of
the disengaged, apathetic youth, and groped frantically for therapeutic aids that might entice young people to participate (forgetting that time tends to make good
citizens of us all). And it is no wonder we have invested such hope in the potential of emerging media to engage young people and to encourage them to participate.
Participation, in the end, is truly much safer—and much easier to deal with—than politics. \aQuoteB{Barney}{2010}{145}
\stopCitation

\spaceHalf
\inright{4. participation as mere \technocraticMean to predetermined ends}
\Depoliticization reduces \participation to a mere method. Regimes of power \aQuoteInText{prove [...] participatory credentials} in their invitations to participate \aQuoteB{Hickey and Mohan}{2004}{16} but in turn just implement a \aQuoteInText{technical fix for complex problems of uneven development} \aQuoteB{Cleaver and Rahman in Hickey and Mohan}{2004}{59}. 

\Participation as such then renders an approach \toMark{that is rather technocratic than social}, where aim, methods and tool and principle decisions have already been determined beforehand and that does not intend to envision a broader view that contest superimposed structural inequalities nor does it intend to distribute power. \Participation merely sticks within the limited scope of a particular project \aQuoteB{Hickey and Mohan}{2004}{10}.

\startCitation
On the one hand, privatization, deregulation, unemployment and ‘precarization’ of labour (and ‘structural adjustment’ programmes at the periphery and semiperiphery of the world-system); on the other hand, attempts to bring people to ‘take part’ in the management of local-level state crisis (along with other measures like repression and ‘state of emergency’, as long as they are necessary and feasible). \aQuoteB{de Souza}{2006}{335}
\stopCitation

\spaceHalf
\inright{Summarizing}
Summarizing the previously said, the \deconstruction \participation reveals several deficiencies that do prevent the empowerment and self-determination of the people.

\placefigure[force]{Deconstructing participation.}
{
\starttyping

	/BTEX \tfa\ss\color[green]{does not distribute power} /ETEX

					/BTEX \tfa\ss\color[green]{co-opted for single sided benefits} /ETEX
	
						/BTEX \tfa\ss\color[green]{limited to the local} /ETEX			

/BTEX \tfa\ss\color[green]{carried out at the expense of the already powerless} /ETEX

		/BTEX \tfb\ss{PARTICIPATIONdeconstructed} /ETEX			

			/BTEX \tfa\ss\color[green]{a technocratic method} /ETEX			

		/BTEX \tfa\ss\color[green]{a mere invitation into predetermined spaces} /ETEX			

/BTEX \tfa\ss\color[green]{the maintenance of structural inequalities} /ETEX

						/BTEX \tfa\ss\color[green]{depoliticized} /ETEX
	
\stoptyping
}

Those deficiencies are not absolute nor do they occur all at once or have always equal shares. They shall primarily provide ground for critique of \participation as a mean for dis-empowerment and maintenance of the status quo. Deallocating these deficiencies from \participation could prepare the ground on which new meaning would be injected into its then empty shell. \aQuoteInTextA{Sherry Arnstein} remarks in her \aQuoteInTextT{Ladder of Citizen Participation}:

\startCitation
Obviously, the eight-rung ladder is a simplification, but it helps to illustrate the point that so many have missed - that there are significant gradations of citizen participation. Knowing these gradations makes it possible to cut through the hyperbole to understand the increasingly strident demands for participation from the have-nots as well as the gamut of confusing responses from the powerholders \aQuote{Arnstein}{1969}.
\stopCitation

Deficient \participationDe, \invitedSpaces, the powerless and power holders, are not just terms that represent a reality of development projects or other initiatives that explicitly addresses an excluded or discriminated group of people. \Participation pervades society and social relations independent from status or class, but it does not necessarily lead to a just society. 

\spaceHalf

\inright{invited participatory spaces are an expression of wider social power structures}
With \invitedSpaces I would not only refer to spaces determined by public and private institutions, the state, or \abbr{NGO}s. \InvitedSpaces could also be the spaces determined in struggle, by the grassroots or the civil society. They could be male dominated activist assemblies that structurally oppress female voices due to habits and apparently socially fixed norms, they could allow a vocal local elite to dominate discourse even in systems of horizontal decision making where everyone is supposed to have an equal voice, they could allow the majority society to oppress the needs and desires of minority groups. \PowerStructures starts to exists between people, determined by skills, different levels of education, their roles in predominant social discourses: about gender, \bracket{im}migration, race and colour, from where one is coming, in which area one is living. \PowerStructures exist thus not only between the poor and the rich, the have's and have not's.

\startCitation
In practice, because power relations between people are not addressed,
participation all too often involves only the voices of the vocal few and poor people
and women, in particular, tend to lose out, being marginalized and overlooked in
‘participatory’ processes. \aQuoteB{McEwan}{2005}{973}
\stopCitation

Deficient \participationDe means being detached from the processes that affects ones own life, being detached from the political space where those processes are shaped, being only invited to enter a predetermined participatory space that is not overlapping or touching the political one, thus remaining powerless with no means at hand for transformation. 

\startCitation
in the context of a hegemonic political and economic culture that not only accommodates participation but actually embraces, thrives, and insists upon it, and in light of proliferating technologies that effectively render routine participation obligatory, the ends that participation presently serves cannot be said to be unambiguously worthwhile. If we are looking for something to which we might attach our aspirations for a more just society, we might have to look for something other than mere participation. \aQuoteB{Barney}{2010}{144}
\stopCitation

\spaceHalf

\inright{deconstruction of the political space}
The answer to \quote{the other} seems to be located in the political space. The \politicalSpace denoted so often seems to be the space where we want to \participate in but where we are structurally excluded from. \Participation in the \politicalSpace seems to be necessary in order to shape the processes that affects our life's but apparently this space is not accessible, it is determined and owned by \powerStructures that prevents us from transforming those spaces or that intends to keep us under control and maintain the status quo.

So, how should we proceed? Even though I intended to dismiss categories where possible I made plenty of use them, mainly in form of binary expression, thus words that represent opposite roles and positions such as the powerless and the powerholders or just and injust. I partition the world as I perceive it into elements, elements that have a quality and a meaning: the majority society, a minority group, the oppressed, the streets, social movements, capitalism, the excluded, the invisible, the excluded, the neglected.

\startCitation
The 'poor,' however, does not designate an economically disadvantaged part of the population; it simply designates the category of peoples who do not count, those who have no qualifications to part-take in arche, no qualification for being taken into account. \aQuoteB{Rancière}{2000}{4}
\stopCitation

In the momentary state of \participatory praxis it seems that the desires and needs of powerless, the excluded, are elided, ignored. The powerholders keep the situation under control and prevent that the peoples 

10 thesen: 10

What thus characterizes a democracy is pure chance or the
complete absence of qualifications for governing. Democracy is that state of exception
where no oppositions can function, where there is no pre-determined principle of role
allocation.

10 thesen: 12

As we know, democracy is a term invented by its opponents, by all those who were
'qualified' to govern because of seniority, birth, wealth, virtue, and knowledge [savoir].
Using it as a term of derision, they articulated an unprecedented reversal of the order of
things: the 'power of the demos' means that those who rule are those who have no specificity in common, apart from their having no qualification for governing. Before being the name of a community, demos is the name of a part of the community: namely, the poor. The 'poor,' however, does not designate an economically disadvantaged part of the population; it simply designates the category of peoples who do not count, those who have no qualifications to part-take in arche, no qualification for being taken into account.

10 these: 13

The one who is 'unaccounted-for,'the one who has no speech to be heard, is the one of the demos.

10 thesen: 18

Politics cannot be deduced from the necessity of gathering people into communities. It is
an exception to the principles according to which this gathering operates. The 'normal' order of things is that human communities gather together under the rule of those qualified to rule -- whose qualifications are legitimated by the very fact that they are ruling. These governmental qualifications may be summed up according to two central principles: The first refers society to the order of filiation, both human and divine. This is the power of birth. The second refers society to the vital principle of its activities. This is the power of wealth. Thus, the 'normal' evolution of society comes to us in the progression from a government of birth to a government of wealth. Politics exists as a deviation from this normal order of things. It is this anomaly that is expressed in the nature of political subjects who are not social groups but rather forms of inscription of 'the (ac)count of the unaccounted-for.'

10 thesen: 19

There is politics as long as 'the people' is not identified with the race or a population,
inasmuch as the poor are not equated with a particular disadvantaged sector, and as long as the proletariat is not a group of industrial workers, etc... Rather, there is politics inasmuch as 'the people' refers to subjects inscribed as a supplement to the count of the parts of society, a specific figure of 'the part of those who have no-part.' 

Political struggle is not a conflict between well defined interest groups; it is an opposition of logics that count the parties and parts of the community in different ways. The clash between the 'rich' and the 'poor,' for instance, is the struggle over the very possibility of these words being coupled, of their being able to institute categories for another (ac)counting of the community. There are two ways of counting the parts of the community: The first only counts empirical parts -- actual groups defined by differences in birth, by different functions, locations, and interests that constitute the social body. The second counts 'in addition' a part of the no-part. We will call the first police and the second politics.

10 thesen: 20

The police is not a social function but a symbolic constitution of the social. The essence
of the police is neither repression nor even control over the living. Its essence is a certain
manner of partitioning the sensible. We will call 'partition of the sensible' a general law that defines the forms of part-taking by first defining the modes of perception in which they are inscribed. The partition of the sensible is the cutting-up of the world and of 'world;' it is the nemeïn upon which the nomoi of the community are founded. This partition should be understood in the double sense of the word: on the one hand, that which separates and excludes; on the other, that which allows participation (see Editor's note 2). A partition of the sensible refers to the manner in which a relation between a shared 'common' [un communpartagé] and the distribution of exclusive parts is determined through the sensible. This latter form of distribution, in turn, itself presupposes a partition between what is visible and what is not, of what can be heard from the inaudible.

The essence of the police is to be a partition of the sensible characterized by the
absence of a void or a supplement: society consists of groups dedicated to specific modes of action, in places where these occupations are exercised, in modes of being corresponding to these occupations and these places. In this fittingness of functions, places, and ways of being, there is no place for a void. It is this exclusion of what 'there is not' that is the police-principle at the heart of statist practices. The essence of politics, then, is to disturb this arrangement by supplementing it with a part of the no-part identified with the community as a whole. Political litigiousness/struggle is that which brings politics into being by separating it from the police that is, in turn, always attempting its disappearance either by crudely denying it, or by subsuming that logic to its own. Politics is first and foremost an intervention upon the visible and the sayable.

10 thesen: 23

 The only practical difficulty is in knowing which sign is required to recognize the sign; that is, how one can be sure that the human animal mouthing a noise in front of you is actually voicing an utterance rather than merely expressing a state of being? If there is someone you do not wish to recognize as a political being, you begin by not seeing them as the bearers of politicalness, by not understanding what they say, by not hearing that it is an utterance coming out of their mouths. And the same goes for the opposition so readily invoked between the obscurity of domestic and private life, and the radiant luminosity of the public life of equals. In order to refuse the title of political subjects
to a category -- workers, women, etc... -- it has traditionally been sufficient to assert that
they belong to a 'domestic' space, to a space separated from public life; one from which only groans or cries expressing suffering, hunger, or anger could emerge, but not actual speeches demonstrating a shared aisthesis. And the politics of these categories has always consisted in re-qualifying these places, in getting them to be seen as the spaces of a community, of getting themselves to be seen or heard as speaking subjects (if only in the form of litigation); in short, participants in a common aisthesis. It has consisted in making what was unseen visible; in getting what was only audible as noise to be heard as speech; in demonstrating to be a feeling of shared 'good' or 'evil' what had appeared merely as an expression of pleasure or pain

And the politics of these categories has always consisted in re-qualifying these places, in getting them to be seen as the spaces of a community, of getting themselves to be seen or heard as speaking subjects (if only in the form of litigation); in short, participants in a common aisthesis. It has consisted in making what was unseen visible; in getting what was only audible as noise to be heard as speech; in demonstrating to be a feeling of shared 'good' or 'evil' what had appeared merely as an expression of pleasure or pain.

10 thesen: 24

The essence of politics is dissensus. Dissensus is not the confrontation between interests
or opinions. It is the manifestation of a distance of the sensible from itself. Politics makes visible that which had no reason to be seen, it lodges one world into another (for instance, the world where the factory is a public space within the one where it is considered a private one, the world where workers speak out vis-à-vis the one where their voices are merely cries expressing pain). This is precisely why politics cannot be identified with the model of communicative action since this model presupposes the partners in communicative exchange to be pre-constituted, and that the discursive forms of exchange imply a speech community whose constraint is always explicable. In contrast, the particular feature of political dissensus is that the partners are no more constituted than is the object or the very scene of discussion. The ones making visible the fact that they belong to a shared world the other does not see -- cannot take advantage of -- the logic implicit to a pragmatics of communication. The worker who argues for the public nature of a 'domestic' matter (such as a salary dispute) must indicate the world in which his argument counts as an argument and must demonstrate it as such for those who do not possess a frame of reference to conceive of it as argument. Political argument is at one and the same time the demonstration of a possible world where the argument could count as argument, addressed by a subject qualified to argue, upon an identified object, to an addressee who is required to see the object and to hear the argument that he or she 'normally' has no reason to either see or hear. It is the construction of a paradoxical world that relates two separate worlds. 

10 thesen: 25

Politics thus has no 'proper' place nor does it possess any 'natural' subjects. A
demonstration is political not because it takes place in a specific locale and bears upon a
particular object but rather because its form is that of a clash between two partitions of the sensible. A political subject is not a group of interests or ideas: It is the operator of a
particular mode of subjectification and litigation through which politics has its existence.
Political demonstrations are thus always of the moment and their subjects are always
provisional. Political difference is always on the shore of its own disappearance: the people are close to sinking into the sea of the population or of race, the proletariat borders on being confused with workers defending their interests, the space of a people's public demonstration is always at risk of being confused with the merchant's agora, etc.

10 these: 26

The deduction of politics from a specific world of equals or free people, as opposed to
another world lived out of necessity, takes as its ground precisely the object of its litigation. It thus renders compulsory a blindness to those who 'do not see' and have no place from which to be seen.

10 these: 27

That the distinguishing feature of politics is the existence of a subject who 'rules' by the
very fact of having no qualifications to rule; that the principle of beginnings/ruling is
irremediably divided as a result of this, and that the political community is specifically a
litigious community

politics, along with philosophy's effort to resituate politics under the auspices of this law. The Gorgias, the Republic, the Politics, the Laws, all these texts reveal the same effort to efface the paradox or scandal of a 'seventh qualification'  make of democracy a simple case of the indeterminable principle of 'the government of the strongest,' against which one can only oppose a government of those who know [les savants]. These texts all reveal a similar strategy of placing the community under a unique law of partition and expelling the empty part of the demos from the communal body. 

10 thesen: 32

The essence of politics resides in the modes of dissensual subjectification that reveal the
difference of a society to itself. 

Isolated in this manner, this specific space can be nothing other than the place of the state and, in fact, the theorists of the 'return of politics' ultimately affirm that politics is out-dated. They identify it with the practices of state control which have, as their principal principle, the suppression of politics.

10 these: 33

The thesis thus amounts to asserting that the logical telos of capitalism makes it so that politics becomes, once again, out dated. And then it concludes with either the mourning of politics before the triumph of an immaterial Leviathan, or its transformation into forms that are broken up, segmented, cybernetic, ludic, etc... -- adapted to those forms of the social that correspond to the highest stage of capitalism. It thus fails to recognize that in actual fact, politics has no reason for being in any state of the social and that the contradiction of the two logics is an unchanging given that defines the contingency and precariousness proper to politics. Via a Marxist detour, the 'end of politics' thesis -- along with the consensualist thesis -- grounds politics in a particular mode of life that identifies the political community with the social body, subsequently identifying political practice with state practice. The debate between the philosophers of the 'return of politics' and the sociologists of the 'end of politics' is thus a straightforward debate regarding the order in which it is appropriate to take the presuppositions of 'political philosophy' so as to interpret the consensualist practice of annihilating politics.

10 thesen: thesis 6

If politics is the outline of a vanishing difference, with the distribution of social parts
and shares, then it follows that its existence is in no way necessary, but that it occurs
as a provisional accident in the history of the forms of domination. It also follows
from this that political litigiousness has as its essential object the very existence of
politics.

10 thesen: these 8

The principal function of politics is the configuration of its proper space. It is to disclose the world of its subjects and its operations. The essence of politics is the manifestation of dissensus, as the presence of two worlds in one


\startARemark{dann citizenship}
\stopARemark

\startARemark{dann lefebvre}
\stopARemark

\startARemark{who are the power holders}
muß noch klar gemacht werden und auch das die gesellschaft damit gemeint ist und partizipation nicht nur an projekten festgemacht werden muß. 

participation ist ein prozess, eine praxis und nicht nur ein tool.
\stopARemark

\addReference
{
Arnstein, S.R., 1969. A Ladder of Citizen Participation. {\em JAIP}, 35(4), p.216-224. Available at: \goto{\hyphenatedurl{http://lithgow-schmidt.dk/sherry-arnstein/ladder-of-citizen-participation.html}} [url(http://lithgow-schmidt.dk/sherry-arnstein/ladder-of-citizen-participation.html)] [Accessed May 31, 2011].
}

\addReference
{
Barney, D., 2010. “Excuse us if we don’t give a fuck”:
The (Anti-)Political Career of Participation. {\em Jeunesse: Young People, Texts, Cultures}, 2(2), p.138-146. Available at: \goto{\hyphenatedurl{http://jeunessejournal.ca/index.php/yptc/article/view/78/71}} [url(http://jeunessejournal.ca/index.php/yptc/article/view/78/71)] [Accessed September 2, 2011].
}

\addReference
{
de Souza, M.L., 2006. Social movements as “critical urban planning” agents. {\em City}, 10(3), p.327-342. Available at: \goto{\hyphenatedurl{http://abahlali.org/files/Paper_CITY-2006.pdf}} [url(http://abahlali.org/files/Paper_CITY-2006.pdf)] [Accessed May 17, 2011].
}

\addReference
{
Abahlali baseMjondolo, 2010. The high cost of the right to the city. {\em Pambazuka News}. Available at: \goto{\hyphenatedurl{http://www.pambazuka.org/en/category/features/63126/print}} [url(http://www.pambazuka.org/en/category/features/63126/print)] [Accessed July 19, 2010].
}
\addReference
{
Hickey, S. \& Mohan, G., 2004. {\em Towards participation as transformation: critical themes and challenges}. In Participation: from tyranny to transformation? Exploring new approaches to to participation in development. London; New York: ZED Books Ltd; Distributed exclusively in the U.S. by Palgrave Macmillan, pp. 3-24.
}
\addReference
{
Hickey, S. \& Mohan, G., 2004. {\em Relocating participation within radical politics of development: critical modernism and citizenship}. In Participation: from tyranny to transformation? Exploring new approaches to to participation in development. London; New York: ZED Books Ltd; Distributed exclusively in the U.S. by Palgrave Macmillan, pp. 59-74.
}

\addReference
{
Miraftab, F., 2006. Feminist praxis, citizenship and informal politics: Reflections on South Africa’s anti-eviction campaign. {\em International Feminist Journal of Politics}, 8(2), p.194-218. Available at: \goto{\hyphenatedurl{http://www.urban.uiuc.edu/faculty/miraftab/miraftab/IFJP.pdf}} [url(http://www.urban.uiuc.edu/faculty/miraftab/miraftab/IFJP.pdf)] [Accessed September 7, 2011].
}

\addReference
{
McEwan, C., 2005. New spaces of citizenship? Rethinking
gendered participation and empowerment
in South Africa. {\em Political Geography}, (24), p.969-991. Available at: \goto{\hyphenatedurl{http://dro.dur.ac.uk/1204/1/1204.pdf}} [url(http://dro.dur.ac.uk/1204/1/1204.pdf)] [Accessed September 1, 2011].
}

%------------------------------
\showImperfection

\aQuoteB{Barney}{2010}{144}

The basic logic of citizenship is inclusion and participation; the basic logic
of politics is exclusion and refusal. And this is why a culture of liberal democratic citizenship, a culture of the universalized invitation to participate, tends to produce politics only at and beyond its borders and margins. \aQuoteB{Barney}{2010}{144}

\aQuoteB{Barney}{2010}{139}

Participation, it would seem, is what citizenship is about. The prospect I would like to raise is that citizenship-as-participation is something altogether different from politics: that if participation, or taking part, is what citizenship is about, the possibility looms that neither citizenship nor participation necessarily conduces to politics. Indeed, as I suggest below, it may be the case that citizenship-as-participation is our best security against the possibility of politics.

\aQuoteB{Barney}{2010}{144}\aQuoteInTextT{“Excuse us if we don’t give a fuck”:
The (Anti-)Political Career of Participation}

at the very least they raise the question of the status of participation as both a political end and as a critical category under contemporary economic, social, and technological conditions. Participation is ambivalent, as open to stabilizing prevailing arrangements of power and injustice as it is to disrupting them. Thus, from a critical perspective, it would appear necessary to stipulate that participation is not an absolute, but rather only a contingent value: one whose worth is not intrinsic but rather derived from the ends it serves in any given context. Further, in the context of a hegemonic political and economic culture that not only accommodates participation but actually embraces, thrives, and insists upon it, and in light of proliferating technologies that effectively render routine participation obligatory, the ends that participation presently serves cannot be said to be unambiguously worthwhile. If we are looking for something to which we might attach our aspirations for a more just society, we might have to look for something other than mere participation. What we might actually need is politics, not just participation. It was suggested above that participation is what citizenship is about, but that citizenship-as-participation is something altogether different from politics. Much turns on what one thinks politics is, and on what one thinks politics is for.

\aQuoteB{Miraftab}{2004}{1}
While the former grassroots actions are geared mostly toward providing the
poor with coping mechanisms and propositions to support survival of their informal
membership, the grassroots activity of the latter challenges the status quo in the hope of larger societal change and resistance to the dominant power relations.


\stopmode

\stoptext

\stopcomponent
