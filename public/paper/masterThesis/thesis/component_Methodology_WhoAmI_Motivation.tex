\startcomponent component_Methodology_WhoAmI_Motivation
\product product_Thesis
\project project_MasterThesis

% definitions and macros
\environment envThesisAllEnvironments

\define[]\sectionMotivation{\subsection[methodology::whoami::motivation]{Motivations and Demands}}

\starttext

\startmode[tocLayout]

\sectionMotivation

This section describes the personal motivation for conducting research and writing the thesis and its implications.

\stopmode

\startmode[draft]

\sectionMotivation

\startKeywords
non-authoritative, non-hierarchic , genuine participation, open access,  practicability, emancipatory, field research methodology
\stopKeywords

Personal \aKeyword[motivations]{thesis+motivations} and \aKeyword[demands]{thesis+demands} on \bracket{my} \aKeyword[research actions]{thesis+research actions} hold a pivotal role, as of now more than \bracket{academic} \aKeyword[relevance]{research+relevance} which will be addressed in a \goto{later chapter}[methodology::whoami::relevance]. 

My motivations and demands are of subjective nature. \aKeyword[Subjectivity]{subjectivity} has already been briefly mentioned in the \goto{introductory paragraph}[methodology::intro] of the Methodology chapter and is a topic worth mentioning \bracket{briefly} again in order to make the demands on my research transparent. Later on, when I argue for \goto{Action Research}[methodology::whoami::actionresearch] as the thesis' research framework, further notions of \aKeyword[subjectivity]{subjectivity} and personal \aKeyword{standpoint} will be considered from the viewpoint of \aKeyword[knowledge production]{knowledge+production}.

\spaceHalf

\inright{revelation of subjective positions}
To begin with, I would like to reveal my subjective position in this research. In a sense I am a \aKeyword{subjective observer} and, as shown later on, to a certain extend I am also the \aKeyword{observed subject} due to the fact that I did not just passively observe my surroundings but also actively participated in it. For me it is important to reveal those facts in order to uncover the steps that are necessary to tie a red line through this thesis

\startCitation
As every investigating subject is different, her social position, and her political
values, should be explicitly clarified so to put a light on the question of subjectivity. Nonetheless this confession, necessary on one side, is not by itself sufficient to control the effects of the investigation, neither it is to clarify completely the author’s socio-political distortion. The way to Knowledge in precise science is usually filtered by a measuring tool, while in social sciences it is filter by a subjective observer. On one side it is proper to describe the measuring tool, on the other it is correct to reveal subjective positions. But none of these epistemological path will get to the understand of pure Reality.  \aQuoteB{Cattaneo}{2006}{20}
\stopCitation

\spaceHalf

\inright{subjective positions already uncovered}
Some of my subjective positions and impressions have already been formulated in the notes about my \goto{São Paulo experience}[methodology::whoami::experience]. These notes represent a first \aKeyword[reflection]{who am I?+reflection} about this thesis and its research action(s) and bundle statements that formulate a basic \aKeyword[self-conception]{self-conception+basic}as...

\spaceHalf

\textBoxedRoundMaxDef[0.4]{...an attempt for genuine participation and acceptance of differences,}
\textBoxedRoundMaxDef[0.4]{thus, an attempt for realizing research action(s) in a non-authoritarian and non-hierarchical manner.}
\textBoxedRoundMaxDef[0.4]{... an attempt to shape a complementary component of the people's struggles on the streets.}
\textBoxedRoundMaxDef[0.4]{...an attempt to find a non-elitist but common language, where academic and non-academic spaces may converge.}
\textBoxedRoundMaxDef[0.4]{...an attempt to shape a space for exchange and raising of awareness about the people's struggles.}
\textBoxedRoundMaxDef[0.4]{...an attempt to reflect on my personal practice and the contention of constraints and existing contradictions between different spheres of reality.}
\textBoxedRoundMaxDef[0.4]{thus, an attempt to dissolve the separation of academic, political, social and private space.}
\textBoxedRoundMaxDef[0.4]{... an attempt to avoid to represent or speak for anybody. I want to be together with the people and experience myself what they are experiencing.}
\textBoxedRoundMaxDef[0.4]{thus, I am aware of my twofold role and its contradictions as scholar and activist and my alignment with the latter.}
\textBoxedRoundMaxDef[0.4]{...an attempt to adapt the frame this thesis is embedded in on institutional level according to the principles I formulate here.}

\spaceHalf

\inright{research is political}
I conceive my thesis and its research actions as inherently \aKeyword[political]{research+is political}. Political due to the fact that I consider it as a medium that complements the struggle of the people I collaborated with. Political as well because I understand my approach to research actions as the intention to act in a non-hierarchical and emancipatory manner (even though I can not achieve this to full extend). I understand \aKeyword[non-hierarchical]{research+non-hierarchical} and \aKeyword{emancipatory}praxis as a critique of the status quo of current social praxis. 

Emancipation and genuine participation could create spaces at all levels of the city that would not function according to the excluding logic of the currently existing ones, according to the logic of those spaces that are representing and \bracket{re}producing social discrimination. Actions performed in those space could have the potential to either transform them or create new spaces where discrimination and its roots are not existing any more or are at least contested and progressively dismantled. 

Hence, \aKeyword{space is political}. It is political because the space the city represents \bracket{here, the space of the the street population in São Paulo}, is the space of resistance and struggle for transformation, the space where a multitude of realities unfold, the space where my personal practice unfolds as well.

\spaceHalf

\inright{strident communication as a building block towards social transformation is \aKeyword[relevant]{research+relevance}}
My position that this thesis is a complementary component of existing struggles of street people in São Paulo, does not mean that I expect concrete social change as its direct result and outcome. For me, provoking social transformation goes far beyond the scope of this work, far beyond the scope of the frame it is embedded in and constraint by. However, I think that every step towards an emancipatory and self-determined transformation of society is worthwhile to undertake. Therefore I hope this thesis may contribute at bit to undertake further steps into that direction and at least function as a \aKeyword[strident mean of communication and distribution]{objectives+general} for that purpose. 

\spaceHalf

\inright{strident communication means open access to information}
In order to be strident \footnote{here in the sense of the German meaning \quote{plakativ}} and probably provoke ideas or reactions, this thesis and its content have to be freely accessible. The knowledge that accumulates here and which probably could be used elsewhere, has to be freely accessible. 

\startARemarkSuch 
muß eigentlich in ein anderes kapitel

I could even go as far as to say that this thesis could also be considered as another type of self-determined action, because I considered it as a component of a struggle and it was made possible by genuine participation between \quote{us}. I actually don't know in how far this kind of closed loop positively contributes to the thesis' form and content or in how far it would cancel out the narratives of other actions which are supposed to be included here in the first place. propositions can also be found in
\stopARemark

\spaceHalf

\startARemark
hier könnte/sollte ich nochmal kurz auf xxx eingehen und dessen konzept von politischen räumen
\stopARemark

\spaceHalf

\inright{research becomes relevant if it can contribute to an emancipatory social transformation}
A political self-conception of my research action(s) could become \aKeyword{relevant} from my point of view, if those action(s) contribute, even though just to a small extent, to an emancipatory social transformation. Hence, I would like to denote those additional positions as ...

\spaceHalf

\textBoxedRoundMaxDef[0.4]{... relevant to consider my research action(s) as political. }
\textBoxedRoundMaxDef[0.4]{... relevant to consider my research as small building block towards a a social transformation which should be strident in order to distribute information.}
\textBoxedRoundMaxDef[0.4]{... an attempt to make the information this thesis provides freely accessible and usable and not just locked-in the academic space.}
\textBoxedRoundMaxDef[0.4]{  thus, I prefer open access to all information, narratives and thoughts this thesis is composed of.}
\textBoxedRoundMaxDef[0.4]{\toMark{... an attempt to turn something present but invisible, visible but not to \quote{invent} something completely new,}}
\textBoxedRoundMaxDef[0.4]{  thus, I prefer to evoke a reaction and not just a sole analytical and systematically sound reproduction of \bracket{a} \quote{reality}.}
\textBoxedRoundMaxDef[0.4]{... an attempt to decouple the question of relevance of research action(s) from the \bracket{western} scientific norm of being innovative, objective and systematic, in order to examine situations solely on an atomic level, which masks out the context those situations have been embedded in,}
\textBoxedRoundMaxDef[0.4]{  thus, I prefer to work qualitative and event driven instead of systematic and quantitative, in order to make as many of contingent experiences as possible}
\spaceHalf

The hereby presented \bracket{subjective} motivations and demands, compose the basic layer this thesis is build upon. The very first position of non-authoritarian and non-hierarchical behaviour pervades in a sense all other positions. Therefore I would like to draw attention on this position in order to clarify its fundamental importance for me.

\stopmode

%---------------------------------------------------
% gets only displayed in unfinshed mode

\doifmode{unfinished}
{

\subject{keywords}
\placeRegKeyword[criterium=section]

\subject{translations}
\placeRegTransLa[criterium=section]

\subject{references}

\startREF

\nl%
Cattaneo, C., 2006. Investigating neorurals and squatters’ lifestyles: personal and epistemological insights on participant observation and on the logic of ethnographic investigation. {\em Athenea Digital}, 10, p.16-40. Available at: \goto{\hyphenatedurl{http://redalyc.uaemex.mx/pdf/537/53701002.pdf}} [url(http://redalyc.uaemex.mx/pdf/537/53701002.pdf)] [Accessed May 21, 2011]. 
\nl%

\stopREF

}

%---------------------------------------------------

\stoptext

\stopcomponent
