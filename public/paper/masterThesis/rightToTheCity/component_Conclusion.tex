\startcomponent component_Conclusion
\product product_TheRightToTheCity
\project project_MasterThesis

% definitions and macros
\environment envCfgPaperLanguageEn
\environment envRttcUrl
\environment envRttcText

\starttext

\chapter[conclusion]{Conclusion}

The concept of the \rttc according to \hlef and its further development clearly shows that this {\em right} is supposed to be an undetachable attribute of every person and represents a momentarily {\em utopian} model of a society.  The \rttc is continuously (re)negotiated in socially interwoven dialogues between every single individual and will therefore lead to new modes of organisation, the shaping of lived space or forms of interpersonal communication and attitudes, by overcoming the contemporary, accepted as \quote{naturally} given or \quote{discursively fixed}, social norms.

If we consider the present world and its societies, the \rttc won't be accomplished by mere waiting or the acceptance of institutionalised appropriation. By its core, the \rttc might enable a self-organized and self-determined society where every individual is expressing and producing itself in as diverse and comprehensive ways as possible, not separated from but embedded in its environment, its lived space. This excludes a normative granting of rights with its intrinsic power structures and hierarchies, as it has been laid out previously, by default.

It can already been seen in contemporary discourse, mainly on institutional level such as the \quote{\em Right to the city} discourse of UNESCO/UN-Habitat, that the used concept of the \rttc is still condensed to a normative right that is supposed to be granted to the citizen by a however defined city administration:

\quote{\em The \inright{management of social transformations} right to the city signifies societal ethics cultivated through living together and sharing urban space. It concerns public participation, where urban dwellers possess rights and cities—city governments and administrations—possess obligations or responsibilities. Civil and political rights are fundamental, protecting the ability of people to participate in politics and decision-making by expressing views, protesting and voting. The exercise of substantive urban citizenship thus requires an urban government and administration that respects and promotes societal ethics. It also demands responsibilities of citizens to use and access the participatory and democratic processes offered.} (Brown, 2009, p.17)

Another \inright{the brazilian city statute} often cited example is the Brazil City Statute from 2001. Even though not as comprehensive as the \rttc, it grants, among others, (homeless) city dwellers the right to make use of (occupy) empty buildings that actually do not serve any social function if abandoned since years, and which force the city to invest in the city to improve the living conditions of its citizens (da Silva, 2009). In the light of massive repression against urban social movements, occupations and favelas in Brazilian cities alone in 2010, its is obvious that those rights merely exist on paper and have no impact in daily reality (two of 4 big squatted buildings that have been occupied in October 2010 by around 3800 people in the centre of São Paulo, are already evicted and its people expulsed back to the peripheries of the city; the intended \revitalization of the centre of Rio de Janeiro for the comming mega events and its already beginning repression has already been lain out briefly). 

The analysis of \worldcityness, the city's function in the global \spaceofflows that determines globalization, and the resulting processes visible in the city's urban and social structure may have indicated (to a certain extend) how the contemporary city is shaped and that the \rttc cannot be achieved under contemporary (capitalistic) conditions.

How the \rttc can be accomplished then, by the citizens on the streets, remains to be seen. The social struggle that \hlef reckoned in order to achieve the \rttc to its full extend is slightly visible in the struggle of urban social movements, campaigns and grassroots groups. Even though they are already focusing on the concept of the \rttc in various flavours, only a few go currently beyond the claim (often still addressed to the cities governments) for decent housing or the right to work in the centre. 

\quote{\em the question of what kind of city we want cannot be divorced from that of what kind of social ties, relationship to nature, lifestyles, technologies and aesthetic values we desire. The right to the city is far more than the individual liberty to access urban resources: it is a right to change ourselves by changing the city}.(Harvey, 2008)

\stoptext

\stopcomponent
