\startcomponent component_NarratingInquiries_SaoPauloDiaries_DayAndNight
\product product_Thesis
\project project_MasterThesis

% definitions and macros
\environment envThesisAllEnvironments

\define[]\subsectionSaoPauloDiariesDayAndNight{\subsection[narratinginquiries_saopaulodiaries_dayandnight]{Day//Night//Streets}}

\define[]\subjectDayOne{\subject{Day One}}
\define[]\subjectNightOne{\subject{The Night}}
\define[]\subjectDayTwo{\subject{Day Two}}

\starttext

\startmode[tocLayout]
\subsectionSaoPauloDiariesDayAndNight
The section is a narrative of two days and a night experience on the streets of São Paulo.
\subjectDayOne
Introduction to several spots in the centre.
\subjectNightOne
Looking for a place to sleep.
\subjectDayTwo
Visiting more places, a emergency case and the final visit of CONDEPE.
\stopmode

\startmode[draft]
\subsectionSaoPauloDiariesDayAndNight
Today, \Ma and I are going to realize I our two days journey on the streets. We meet at \Ay, its early in the afternoon. \Ma intention is to introduce me to the lived space of the people in street situation, that I experience how to organize daily life on the streets, how people support each other and how people are adversely affected by public policies, institutions and agents.

\spaceHalf

\inright{heavy rain and its effect on the streets}
This week São Paulo is going to be \aLinkMediaNew[hit by a cold weather front]{http://bit.ly/mQxbEJ}\toMark{13/07/2010}. It is already cold today, probably 15 to 20 degrees during the day but at night, temperature will drop to 10 to 12 degrees. It will start to rain as well.

Some days later heavy rain is basically stopping the movement in the city for one week. \aLinkMediaNew[Newspapers report impassable roads]{http://bit.ly/nrKPY7} \toMark{14/07/2010} caused by floods throughout the city.

\startPersonal
The cold weather has severe effects on the movement of the people on the streets. Those that are in streets situation will leave the lower places of the centre in order ascend the steep hills up to \aLocationNew{Avenida Paulista}{http://osm.org/go/M@y3Wjuw}. Looking at the city topographically, \aKeyword[Sé]{streets+places+streets+sé} or \aKeyword[República]{streets+places+streets+república} are located in the valleys of the surrounding hills. Once heavy rain hits the city, those lower places are much more prone to floods then the upper areas on the hills, where \aKeyword[Avenida Paulista]{streets+places+streets+avenida paulista} is located for instance. Thus heavy rains forces people to leave the lower areas and ascend upwards in order to avoid being flooded away and in order to find a more or less secure and dry place to sleep. 

Even though those urban floods are not comparable to river floods, during the period of heavy rain small rivers pop up everywhere on the side walks and the streets because the city is sealed by constructions and water is searching its way wherever possible, accumulating in streams that make it impossible to sleep on the ground. 
\stopPersonal

\subjectDayOne

But for now, we start our journey downwards, descending \aLocationNew{Rua  Carmelias}{http://osm.org/go/M@ziKy3jL--} until it hits \aLocationNew{Rua Frederico Alvarenga}{http://osm.org/go/M@ziK8Ucu--}. There, at the corner we meet one of \Mas friends. He is sitting there at the corner most of the day, almost everyday. Right now he is sleepy and does not talk much. We do not stay long, \Ma is just asking how his friend is doing and then we continue. \Ma says that he knows him since he hits the streets, years back, and since then the guy always stayed at that corner every day. 

We are heading towards the \Tendas. They are not far away, just two streets, located below \aLocationNew{Viaduto 25 de Marco}{http://osm.org/go/M@ziLozVF--} where \aLocationNew{Avenida Rangel Pestana}{http://osm.org/go/M@ziLozVF--} traverse a branch of \aLocationNew{Rio Tietê}{http://osm.org/go/M@zie@K}\aLinkNewF[Rio Tietê at Wikipedia]{http://bit.ly/pvlcQg} that crosses through the northern part of the city's central area.

\spaceHalf

\inright{loosing means to generate income}
We arrive at the junction where \aLocationNew{Rua Dom Pedro II and Avenida Rangel Pestana}{http://osm.org/go/M@ziK3nYF--} are crossing and meet another friend of \Ma. He tells us that the \abbr{GCM} has just taken all his possessions, his bag and all the goods he was selling on the streets because he could not show a permission as street vendor when \abbr{GCM} has been checking him. Thus he has just lost all means to generate income.

\spaceHalf

\inright{entering the Ţendas}
We cross the street and entering the \toTranslate[Tendas]{tents}. \Tendas is the name of an area converted to a public service that receives people in street situation. The \toTranslate[Tenda de convivência]{Tenth of gathering} at Parque Dom Pedro is a service provided by the Secretaria Municipal de Assistência Social of the Prefeitura de São Paulo \aLinkNewF{http://bit.ly/qImvjC}.

The \Tendas at Parge Dom Pedro consists basically of a huge tent, open to one side, packed with people sitting on banks and chairs, in front of a TV. The backmost part is occupied for showers and toilets. Outside the big tent, groups of people are sitting everywhere, some of them talking, some just quite. The atmosphere is rather depressed. We talked to a social worker who told us that they are also offering workshops from time to time, such as \toMark{nochmal in meine aufzeichnungen schauen}. The \Tendas are open during the day from 8h in the morning until 9h in the evening and closed for the night.  

\startARemark{woher kommen die leute nochmal? alberque oder strasse oder beides?}\stopARemark
\startARemark{mehr über die tendas in meinen aufzeichnungen}\stopARemark

\startPersonal
The \Tendas are located at a spot of the centre that is usually not much frequented, except by cars that rush over the three lane avenue that is passing above them in order to traverse the river. It is noisy and the air is polluted by massive traffic that is circulating all day long from the centre to the eastern regions of São Paulo and vice versa. The traversals at \aLocationNew{Parque Dom Pedro}{http://osm.org/go/M@ziLozVF--}, thus above the \Tendas, eventually connect the centre with Brás and the \aLocationNew{Radial Leste}{http://osm.org/go/M@zoEwXb-} highway that head towards the eastern margins of the city. 

Especially the notion of artesian workshops becomes contradictory when we thought about the friend we met just before, who is actually living from vending artesian goods on the streets and who has been expelled by the police for doing so. 
\stopPersonal

According to \Ma, the \aKeyword[tendas]{streets+places+institutional+tendas} are just another way of keeping people in their miserable situation because they do not provide a  single proposal to sustainably improve the situation of the people. For \Ma they are solely a justification for public institutions, civil and police agents, to banish people from the central commercial areas and send them here because the city wants clean the commercial centre of all unwanted subjects. What awaits them at the \Tendas is the just a TV, some workshops, food and sanitation. \Ma said that the chemical sanitation is dirty like hell and that he would not take a shower here anyway.

\startDialog
\tell{\Ma}{Look what people can do here. Nothing. They just sit in front of the TV all day long and wait until the place closes its doors at night. Tomorrow they will be here again but how can they improve their situation then?}
\stopDialog

\spaceHalf

\inright{heading towards Sé}
We leave the the \Tendas after a while and head towards \Se. 

\subjectNightOne
\reference[narratinginquiries_saopaulodiaries_dayandnight_lookingforaplacetosleep]{}Looking for a place to sleep.
\subjectDayTwo
Visiting more places, a emergency case and the final visit of CONDEPE.

%---------------------------------------------------
% gets only displayed in unfinshed mode

\showImperfection

%---------------------------------------------------
\stopmode

\stoptext

\stopcomponent