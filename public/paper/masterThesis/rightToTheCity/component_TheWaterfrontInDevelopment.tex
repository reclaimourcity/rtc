\startcomponent component_TheWaterfrontInDevelopment
\product product_TheRightToTheCity
\project project_MasterThesis

% definitions and macros
\environment envCfgPaperLanguageEn
\environment envRttcUrl
\environment envRttcText

\starttext
\section[waterfrontExample]{The waterfront in development}

{\em Contemporary \waterfront revitalization projects are realized for example in Hamburg, Cologne, Rio de Janeiro or Istanbul and are discussed very briefly in the following paragraphs.} 

\from[wikiEnHamburgs] \from[hafencity] \inright{Hamburg's \hafencity} represents a large scale neoliberal inner city urbanisation project, realized in an once fallow \from[hafencityPortArea]. Until the 1990's, all port activities had been shifted from the inner city edge of the river Elbe, the southern edge, to the northern edge and left large areas of fallow land (Bauriedl, 2007). 

The \hafencity is designed to push the \quote{label Hamburg} on the international market of competing (european and world) cities (Bischoff et al, 2009; Hafencity, 2010, p.4). The \hafencity shall not only strengthen the city's competitiveness but strives to gain recognition as the inner city urban development blueprint for the 21th century European city (Hafencity Masterplan, 2006, p.8). It is supposed to attract and profile the creative and digital, thus information based, industries (Hafencity Masterplan, 2006, p.14). The focus on advanced producer services is expressed with the intended construction of 1 million square meters of office space, which could serve for 35 to 50 thousand working places and could attract further public and private investment (Hafencity Masterplan, 2006, p.6). 

Mixed (here: demand driven) forms of living, tailored for different needs, are promoted, with a special focus on up-scale housing (Hafencity Masterplan, 2006, p.14). Besides promoting (exclusive) living on the waterfront, the \hafencity tries to gain recognition as tourist attraction with an \from[hafencitySeaport] and as the site of high-end leisure amenities and landmark architecture, such as the gigantic Opera building, the so called \from[hafencityElbphilharmonie], one of the modules of the waterfront development toolkit that has been previously applied for example at Sydney's waterfront, which is well-known worldwide for (or represented by) its \from[wikiEnSydneyOpera] building. 

The real impact of the \hafencity on Hamburg's social and urban structure remains to be seen. The \hafencity is still work in progress, with an intended construction period of 25 years. Already visible today is a beginning segregation of local inhabitant structures, where \from[wikiEnCondos] are the pre-dominant form of housing while social and affordable housing is almost not existent (Statistikamt Nord, 2007, p.3-4, Statistikamt Nord, 2008, p.231-232).

Another \inright{Colognes's \rheinauhafen} example of \waterfront \revitalization in Germany can be witnessed in \from[wikiEnColognes] \from[rheinauhafen]. The \rheinauhafen, once an inland transportation, stock turnover and trading hub (Dietmar, Rackoczy, 2002), well located in the old city centre, is now splurging with gigantic landmark architecture (like Hamburg's \from[hafencityElbphilharmonie]) in form of three \from[wikiDeKranhaus] buildings. The harbour experienced its decline in the 70's of the 20th century. In 1976, the city of Cologne finally decided to suspend its service until 2000 in order to revitalize the area (Blenck et al, 2001).  Until then, all port  activities had to be shifted to newly constructed harbours outside the city, to its northern and the southern extensions (Blenck et al, 2001). Despite the decline of the port, the area has never been abandoned completely. Several single cultural projects had been realized in the 1990's such as the {\em Immhof Schokoladenmusem} in 1993 or the {\em German Sport and Olympic Museum} in 1999 (Dietmar, Rackoczy, 2002). The initial contest for \revitalization proposals was hosted in \from[rheinauhafen1992] (Modernes Köln, 2010).

The \rheinauhafen is designed to \from[rheinauhafenIntegrate] with a mix of housing, office space, amenities and other consumption infrastructure (RVG, 2010). Hence, the \rheinauhafen is supposed to revaluate and promote the inner city and transform the area into a lively district. 

The waterfront is promoted almost completely as exclusive living space: the \from[kranhausPandion] shall finally be composed of 135 luxury condos while the former warehouse \from[rheinauhafenSiebengebirge] and the newly constructed  \from[rheinauhafenWohnwerft] have already sold their complete stock of apartments (RVG, 2010, RVG, 2010). Even though rented apartments are offered as well, their potential clientèle is limited to those who can afford high rents and living space beyond 100 square meters (Immonet, 2010).

Comfortable living shall be guaranteed by Cologne's largest underground parking deck (Stredich, 2010) or the privately owned \from[rheinauhafenSporthafen] which is operated by Cologne's Marina Club (Pesch, 2008). Security and protection against unwanted subjects (IVV, 2010, p.18) is enforced by private security agencies, surveillance infrastructure and the living in gated communities (here: entrance permitted by a janitor) (Immonet, 2010). 

Besides housing, the \rheinauhafen is offering large amount of office space for advanced producer services. Its main aim is to attract the creative and digital industries: a hub of hightech, information based, companies is already emerging, composed of branch offices of international companies such as {\em Electronic Arts} or {\em Microsoft} (RVG, 2010).

Similar to the \hafencity, the \rheinauhafen is work in progress, although in its last state. A segregated inhabitant structure can already be observed: social and affordable housing is absent while expensive and large apartments for a correspondingly potent clientèle are the norm (Höhmann, 2010). Potential effects such as \newbuildgentrification in neighbouring quarters like the \from[wikiDeSeverinsviertel], due to the clustering of hightech, creative, cultural and tourism industries and upscale housing, remains to be seen. Even though the \rheinauhafen occupies a huge area of public space, its realization mainly focus on the construction of a new elitist neighbourhood that hosts hightech industries, housing for the hightech working force and correspondingly exclusive leisure amenities.

Several \inright{Istanbuls \waterfront: the Galata Port Project} \waterfront development projects have been realized in \from[wikiEnIstanbul] since the 1980's. In contrast to port cities in post industrialized countries, where inner city ports turned into abandoned \brownfields, ports in Turkish cities are still active (Butuner, 2006, p.04). One of those ports that is intended to be \revitalized is the \from[galataPort] located in the historic \from[wikiEnGalata] quarter.

\from[galataPort] witnessed its decline as trading hub in the 1980's when it became unsuitable for trucks and vessels due to the increase of inner city traffic. Its mode of operation was therefore reduced to passenger transport only. Since then, the port couldn't fulfil this function (Butuner, 2006, p.08). 

An attempt to turn it fully operational is the \from[galataPortProject]. The intended \revitalization of the port shall convert the area into a leisure zone that represent a prestigious tourist entry to the city (Butuner, 2006, p.08). A modern cruise terminal shall emerge where, according to official estimations, around 12 million tourists could arrive through the water ways over the course of a year (Butuner, 2006, p.08). The necessary consumption infrastructure is expected to boost commercial and tourist industries. Hence, the general aim of \from[galataPortProject] is the (re)definition of Istanbul's image on national and international level in order to develop a new identity known for its cultural, tourist and commercial strength (Shafik \& Steele, 2010, p.07).

Due to the fact that the project has not been realized so far, it remains to be seen how it would affect the \from[wikiEnGalata] area. The project follows the logic of \waterfront \revitalization as catalyst for the construction of Istanbul's global image and the development of a strong tourist and commercial industry. From a contemporary real estate point of view, \quote{the long-awaited Galataport project is the most important factor driving prices exponentially up in the neighbourhood} (Kalkavan, 2010). In contrast to the institutional point of view, residents and NGO's are criticizing "\quote{the content and scope of the project} (Butuner, 2006, p.08) as destructive. An urban transformation and renewal policy, as it has been set-up in 2005, grants full authorization to municipalities in order to initiate and realize urban development projects (Tan, 2007, p.08). This power has already been used in other districts, such as the Sululuke district for example, to evict and displace the local Gypsy community and demolish their houses under the umbrella of urban development (Tan, 2009).

The \inright{the old port in Rio de Janeiro} last example is going to take a look into the near future. \from[wikiEnRio] is going to be one of the main sites that host the \from[rio2014] and \from[rio2016] in 2014 and 2016 respectively. In the course of the preparation of those spectaculars, one large scale urban development project will be the total transformation of the old \from[rioPortArea]. The construction of the port has been part of the urban reform that took place in Rio between 1903 and 1906. Pollutive industries settled in the port area but until the 1940's it turned into a non-place. Industries shifted to neighbouring quarters such as São Cristovão, along the rails, and finally to the peripheries of the city. (Ferreira Santos, 2005) The main port activities can nowadays be found in the \from[rioPortoSepetiba] (da Silva, 2009). With the construction of several main roads, the port area has finally been completely separated from central area (Ferreira Santos, 2005).

The deteriorated port areas are nowadays the site of the struggle of city dwellers, urban social movements and the speculative real estate industries. Since the 1950's, the relocation of port activities, the loose of economic and industrial importance to São Paulo, the loose of status as capital and the subsequent legal prohibition to settle in the central area enforced by \from[wikiEnFavela] cleaning and demolition of \from[corticios], left vast abandoned spaces and buildings in the central area (da Silva, 2009).

\definitionframed
{
\inright{cortiços and favelas} \from[corticios] are the predecessors of \from[wikiEnFavela]. They are the places and agglomeration of houses where the excluded live, the oppressed, all those who don't mix with the bourgeoisie (Azevedo, 1890).
}

With the announcement of two mega events, the port area is intended to be reintegrated into the cities structure. The revitalization of the area is concentrated around several cores: Transport; Technology, Communication; Habitation; Environment; Tourism; Culture; Leisure.

These urban interventions follow the \from[barcelonaModel] (Barcelona Bulletin, 2009), thus the logic of \worldcityness. The city is supposed to be transformed into a city of spectacular that offers exclusive living, world class tourist amenities such as an international museum (Ferreira Santos, 2005) and a shift from the informal, tertiary, sector to advanced producer sector (Prefeitura Rio, 2010). In order to accomplish these goals, among others, city dwellers in risk areas are supposed to be resettled (Rio Negócios, 2010b) while urbanization projects shall take place in various favelas (Rio Negócios, 2010a). 

A change in the cities social and urban structure has already been addressed by \from[rioSocialMovementMoradia], \from[occupationsRio] and favela inhabitants (NPC, 2010). A constant conflict with the agents of real estate speculation and the police is emerging (Pela Moradia, 2010) due to intimidations and intended displacements from the central areas (Marques, 2010)

\quote{\em We see: 130 favelas that are scheduled to be removed until the Olympic Games. Millions of evictions and removals are necessary for the construction of three big highways (Transcarioca, TransWest e TransOlympic). All 73 plots of Metro land, all located in regions with infrastructure, will be sold in order to make way for the new Metro lines instead of using them for public housing. The port zone, where around 70\% of the land is public, has been targeted by the Olympic plans as well, in order to enforce the \gentrification of the region. Security policies, including UPP's (unidades da polícia pacificadora - peace making police units), are prioritising the creation of pacified zones (and walls) around Olympic infrastructure and the tourist routes to and at revitalized areas.} (Marques, 2010)

\boxframed
{\tfx 
original quote: \quote{\em Vejamos: estão previstas remoções de 130 favelas até as Olimpíadas. Para a construção de 3 grandes vias rodoviárias (Transcarioca, Transoeste e Transolímpica) serão necessários milhares de despejos e remoções. Os 73 terrenos do Metrô, todos em áreas com infraestrutura, ao invés de usados para habitação popular, serão vendidos para fazer caixa para o metrô prometido ao COI. A Zona portuária carioca, onde cerca de 70\% do solo é público, também entrou nos planos Olímpicos, para reforçar o projeto de aburguesamento da região. A política de segurança, o que inclui as UPPs, tem como prioridade criar zonas de paz (e de muros) nos entornos dos equipamentos esportivos, nas vias de acesso dos turistas a esses equipamentos e nas áreas valorizadas ou em vias de valorização.} (Marques, 2010)
}
\stoptext

\stopcomponent
