\startcomponent component_Methodology_WhoAmI_Experience
\product product_Thesis
\project project_MasterThesis

% definitions and macros
\environment envThesisAllEnvironments
\environment envCfgThesisImages

\define[]\sectionExperience {\subsection[methodology_whoami_experience]{Notes//São Paulo//City//Extremes}}

\starttext

\startmode[tocLayout]

\sectionExperience

This chapter gives a very brief overview about the time in São Paulo

\stopmode

\startmode[draft]

\sectionExperience

This thesis finally developed out of the experience made in São Paulo. I arrived in the city in May 2010 and aimed to stay until October. In June, I decided to stay until November. The impressions and experiences gained during that period are the subject of the following \aKeyword[synopsis]{thesis+synopsis} which is aimed to transparently reflect the circumstances I dived into and had to deal with and which affected the way I acted during that time.

My time in the city was only determined in terms of \quote{where to stay}, \quote{what would I like to do} and \quote{how much time do I have}. I had no real contacts to people nor groups, even though I had email contact in advance, mainly to grassroots and political groups, collectives and spaces such as \aKeyword[Indymedia São Paulo]{collectives+indymedia são paulo} \aLinkNewF{http://www.midiaindependente.org} or the self-organized space called \aKeyword[Espaço Ay Carmela]{spaces+espaço ay carmela}\aLinkNewF{http://ay-carmela.birosca.org}. In a sense I preferred that situation which meant for me the maximum possible freedom in order to decide how to proceed, to define the course of my research actions, which in turn also meant that I could first take as much time as possible to assimilate the city and let the city assimilate me. 

Finally, I got in touch with a loose group of people from the streets which whom I spend about three month and during which I became \bracket{partly} involved in their realities, struggles and actions. This thesis is a narration about this time and actions.

\spaceHalf

\inright{a personal remark}
Even though I intended maximum freedom, I already had an concept for a research action in mind and on paper when I came to the city. This idea was related to the usage of mobile communication for grassroots organization but finally rendered impossible to realize due to those constraints that I lay out in the following sections. Thus, the final topic and direction of this thesis is differs almost completely from the one I had in mind when I decided to go to São Paulo. The detailed process of this transformation is documented on the \aKeyword[thesis' blog]{tools+blog} \aLinkNewF{http://rtc.noblogs.org}, which has been set up for the documentation of the research process, transparency purposes and in order to guarantee free access to the assembled information. I will not lay out the transformation in detail at this point but would like to refer everyone interested to the documentation available online. 

\spaceHalf

\inright{mobility and its versatile dimensions}
Getting acquainted with the city first of all meant for me, before anything else, practising how to use the city which is like nothing else I have seen and experienced before. I had to adopt basically everything that I knew about the flow of a city, the motion within a city. Things that are inherent in daily practice in German cities, had to be reconsidered. Transportation and the question of how to reach one particular place and how to return became suddenly a must when being on the run for longer trips through the city. The dense bus network had been a challenge from the beginning on, with its myriads of lines, stops, paths, its enormous city coverage and its range. Later, after loosing the fear of getting lost, its nodes became inherent to the daily adventure of travelling through the city.

Complementary to public transport, that also includes rings of trains and metros, is the apparently most uncommon transportation vessel for this environment, the bike. Even though São Paulo's steep topography and its scale, the massive and aggressive traffic, the daily traffic jams, heavy air pollution especially on hot days, and the non-awareness and recklessness of car, bus and truck drivers which often seemed to just ignore and overlook cyclist, doesn't seem to be the favourite environment for using a bike. However, the bike actually became my favourite means of transportation because it gave me a lot of flexibility and freedom. It also enabled me to arrive at places that would have been much more difficult to reach solely by public transport. I also shared the bike from time to time with some of the people I stayed with, thus from my point of view it was not just a means of transportation but also a means of communication and a shared resource among us.

\spaceHalf

\inright{space and scale}
\reference[methodology_whoami_experience_locations]{}Leaving the street level and zooming out to the metropolitan scale, São Paulo's dimension is just too extensive for me to grasp completely. My sphere of action was therefore mainly delimited by several districts starting from \locPompeia and \locBarraFunda in the western zone of central São Paulo to \locSe and \locRepublica in the centre and further on to \locBras and \locMooca in the east. 

\imgOsmSaoPaulo

\spaceHalf

\inright{language and its versatile dimensions}
Concurrently, access to and contact with the city's spaces has been possible through language. Language became even more crucial for getting in contact with people, in order to understand their narrations and explanations and without language to communicate, São Paulo would have remained locked for me because I could not even ask for the route or the destination of a bus line, let alone communicate with people beyond small talk. Thus, my knowledge of Brazilian Portuguese facilitated my arrival and the further assimilation of the city. Even though this sounds convenient, my Portuguese has been a bit rugged at these days, thus improvement was necessary. This necessity represents another reality of my initial time in the city where it has been important to examine my language skills and practice as much as possible. 

Afterwards, on the streets, my understanding of Portuguese was contested again due to the plurality of accents the people spoke. This plurality exists on the one hand because the people I met on the streets came from all over Brazil, a reproduction of the image of São Paulo as immigration city \aQuoteB{Bogus and Pasternak}{2004}{2}. For me, accents from the south of Brazil has been much easier to grasp and understand than accents from the north and north-east. I had always difficulties to fully grasp the meaning of conversations when people came from Pernambuco for instance. Their \translate[giria]{slang} \bracket{which means slang or parlance in Portuguese} has often been too fast and fuzzy for me, thus I missed a lot of words and therefore the sense of the spoken during such occasions.

On the other hand, if one perceives the streets as one of the spaces the city is composed of, shaped by a particular but very heterogeneous group of people, a particular \aKeyword{giria} has been developed in that space and is used by those that indwell it, just as it is the case for São Paulos's massive hip hop community or any other group that is constructed around a particular identity and/or which constructs that identity. 

\startCitation
Identity is not, then, what is attributed to someone by simply belonging to a group, but rather the narration of what gives meaning and value to the life and identities of individuals and groups. \aQuoteB{Barbero}{2009}{20}
\stopCitation

In this sense, \aKeyword{giria} is another aspect that impedes approaching people from that group because it is difficult to understand and contains unknown habits, symbols, and expressions and therefore a particular local knowledge is necessary for its decoding. \aKeyword{Giria} also determines who belongs to the \toTranslate[família]{family} \bracket{of people in street situation, for instance} and who does not belong to it, who is an outsider. 

\spaceHalf

\inright{time and temporal constraints}
\reference[methodology_whoami_experience_limitingfactortime]{}Putting those aforementioned aspects together, one factor that pervades them all is \aKeyword[time]{limiting factors+time}. Time is necessary for gaining the \aKeyword[situated knowledge]{knowledge+sitauted} I previously determined as personally lacking but that I consider necessary in order to start realizing (research) actions(s) based on reasonable ground. As it probably can be seen from those personal descriptions above, plenty of time was already necessary just to cope with the numerous overwhelming and unfamiliar situations.

The concept of time plays a crucial role because sufficient time \bracket{or the lack thereof} was one prevailing factor in order to even start accepting this thesis as something reasonable for me. Without the option to stay at least for 5 to 6 month plus the same amount of time to assemble everything, a stay abroad would not have been an option to me and a plain theoretical work would have been the most reasonable alternative. What would have been the result then if I had restricted myself to the official period of 5 month for conducting research actions and writing the thesis? This very limited time frame would have made it very difficult for me to accept  the city as that space that formed my new temporal reality, which represents my life for the time to be and not just a space to rush through. Perhaps it would have been necessary to be just part of another existing \bracket{research} project while reproducing the dominant \bracket{social} top-down hierarchies and power relations, that foreign western academic agents and their intended \quote{research objects} often represents. 

\spaceHalf

\inright{foreign expert and local adept}
A contradiction produced by those hierarchies is the inverted concept of knowledge, where the one that lacks local knowledge but is embedded in the academic world has more power or status than the one that is the local adept, who knows everything about his or her surroundings, but is discriminated and lives at the outer margins of society. How can I then consider me some kind of \quote{expert} that is able to judge, analyse and propose if I know nothing about the local situations, realities and struggles. Even during 6 month in São Paulo I possessed just a tiny fraction of the plurality of realities that this \aKeyword[city of extremes]{city of extremes+são paulo}\footnote{The expression \quote{city of Extremes} has been lent from \aQuoteInTextT{A cidade dos extremos} by \aQuoteInTextA{Lucia Bogus and Suzana Pasternak} \aQuoteY{2004} } produces. 

\spaceHalf

\inright{knowledge and power hierarchies}
Another question remains: would I like to act as such an \quote{expert} anyway, even with the proper knowledge. I personally would not exploit my expertise and experience to gain or exercise power \bracket{in order to produce content for the benefit of the thesis} nor do I identify myself with the role of an \bracket{academic} scholar because this role is already loaded with power hierarchies and symbols that conflict with my \inright{personal conviction} personal conviction. In their work \aQuoteInTextT{What have the Romans ever done for us?}, \aQuoteInTextA{Barker and Cox} describe the role of the scholar \bracket{here meant as scholar of social movement} as follows:

\spaceHalf

\startCitation
The scholar acts as a traditional intellectual, carrying out directive and theoretical activity on behalf of already-existing, and already-powerful, social classes and groups. Their directive activity is entailed in the administration and development of an education system which is a central mechanism in reproducing class inequality and in legitimating the social order. \aQuote{Barker and Cox}{2001}
\stopCitation

\spaceHalf

\inright{contradictions and tensions in different roles}
If I then define my \aKeyword[role in this research]{streets+roles+personal}, I clearly sympathize with the people I have been together, I feel myself much more belonging to their struggles, as to what the contemporary academic world symbolizes \bracket{even though I do not deny the importance of academic work and analysis, eventually I make use of it in this thesis as well}. This fact certainly affects the way I act and decide because I am socialized much more by the activist than the academic space and certainly perceive their opposed positions, especially when trying to practice according to my own personal attitudes and convictions in those spaces but also with respect to the formation of knowledge, which is produced according to different concepts and motivations. 

Quoting \aQuoteInTextA{Barker and Cox} once more, the contradictions thus also emerge due to diverging role concepts, where

\spaceHalf

\startCitation
[...] those who are drawn to this field of academic study are themselves former or continuing activists and participants in actual movements and movement organizations. [...] Those with feet in both camps are often aware of contradictions and tensions in their different roles \aQuote{Barker and Cox}{2001}.
\stopCitation

\spaceHalf

For me, non-hierarchical/non-authoritative and genuine participation is an attitude applicable in all areas of practice, may they be political motivated, related to academic research or just belong to daily life. I consider the \bracket{from my perspective} discursively defined areas of private life, research or struggle as at least overlapping, if not the same sometimes. This also means that I attempt not to reproduce them as separate spheres of my life.

\reference[methodology_whoami_experience_notionsofrelevance]{}Hence my intention is to do research based on those and other attitudes (which will be exposed as \gotoTextMark[list of demands]{methodology_whoami_motivation} later on) and write this thesis because \inright{personal motivation} I consider it \aKeyword[relevant]{research+relevance+personal} for me and the reflection on my personal practice, \aKeyword[relevant]{research+relevance+struggle} as a complementary component of the struggle of the people, \aKeyword[relevant]{research+relevance+struggle} for the interconnection of academic space, marginalized space, political space and social space \bracket{social space here as a synonym for society, thus the city}, their interchange and for \aKeyword[raising consciousness]{objectives+raising consciousness}.

\aQuoteInTextA{Anna Tsing} \aLinkNewF{http://anthro.ucsc.edu/directory/details.php?id=35} asks in \aQuoteInTextT{Friction} what other possibilities are there for \aKeyword[knowledge production]{knowledge+production} and wonders why other modes of knowledge production, \aKeyword[narrations]{narration} for instance, cannot be justified in academic terms, even though they would complement and support the spaces of struggle and academic theorizing.

\spaceHalf

\startCitation
How has it happened that in order to stay true for hopes for a more liveable earth, one must turn away from scholarly theory? [...] Might it be possible to use other scholarly skills, including the ability to tell a story that both acknowledges imperial power and leaves room for possibility?
\aQuoteB{Tsing}{2005}{267}
\stopCitation

\spaceHalf

\inright{spaces not seen as atomic units but interdependently connected}
I don't intend to distinguish those spaces as separate from each other, the \aKeyword[research space]{space+research} separated from the \aKeyword[social space]{space+social} which represents or is represented by the city, separated from \aKeyword[private space]{space+personal} of my life in São Paulo.  I didn't define how many hours per day I enter the research space, nor hours to enter the daily life or social space.  

\reference[methodology_whoami_experience_genuineparticipation]{}Certainly, those spaces existed and exist but for me, I perceive them as organically converging, diverging, overlapping and sometimes matching, depending on the context all those different situations have been embedded in. When I was on the streets, I often met people whose daily reality I participated in, when I went to \aKeyword{Ay Carmela} or simply roamed the streets in order to absorb the city. In those cases we either spent time together, which could be time considered as research action, as socializing, leisure or political action, or all together at the same time, or we just continued on our sparate paths. 

Clearly, the separation of those spaces existed and exist, because eventually, I didn't life together with the people on the streets, we \bracket{just} shared plenty of time together.  This also meant that my time in São Paulo was a time where I personally didn't need to take care about organizing my life because I had a definite place to life, a determined number of month to stay and I could freely organize my time without hassle for work or earning money. This is one hierarchical aspect which I could not resolve and which implies that I was in the luxury position to freely organize my time and research action(s) and be together with people whose situation was exactly contrary, which struggle every day. 

\spaceHalf

\inright{the purpose of those narratives}
The purpose to narrate those impression is simply the fact that it took time for me to arrive in São Paulo, especially if the reality I came from and the one I arrived in are so diverging. In my case it took about two month, which had been important and necessary for me but resulted in no concrete or visible outcome for this thesis at first glance. The establishment and deepening of tied contacts on the base of amity and solidarity took another one or two month and suddenly the remaining time in the city had been drastically reduced. For me, the \aKeyword{whole process}[objectives] was important and contributes to the thesis as much as the concretely realized research action(s). I consider the whole period as enriching for me and my personal practice and definitely not as a mere obligation in order to gain a degree. This \aKeyword[synopsis]{synopsis+experience} also serves as a summary in order to reflect on my role and my status and the circumstances that affected my time in São Paulo.

Having said this, perhaps some of the factors that mainly impacted the course of this thesis are clearer now, thus let's see how the red line through it can be tied.

\addReference
{
Barbero, J.M., 2009. Staging Citizenship: Performance, Politics, and Cultural Rights. {\em e-Misférica}, (62). Available at: \goto{\hyphenatedurl{http://hemisphericinstitute.org/hemi/en/e-misferica-62/martin-barbero}} [url(http://hemisphericinstitute.org/hemi/en/e-misferica-62/martin-barbero)] [Accessed September 3, 2011].
}

\addReference
{
Barker, C. \& Cox, L., 2001. “What have the Romans ever done for us?” Academic and activist forms of movement theorizing. Available at: \goto{\hyphenatedurl{http://www.iol.ie/\~mazzoldi/toolsforchange/afpp/afpp8.html}} [url(http://www.iol.ie/\~mazzoldi/toolsforchange/afpp/afpp8.html)] [Accessed July 18, 2011]. 
}

\addReference
{
Bogus, L.M.M. \& Pasternak, S., 2004. The City of Extremes: Socio-Spatial Inequalities in São Paulo. Available at: \goto{\hyphenatedurl{http://www.vrm.ca/documents/City_Extremes.pdf}} [url(http://www.vrm.ca/documents/City_Extremes.pdf)] [Accessed May 20, 2011].
}

\addReference
{
Tsing, A.L., 2004. {\em Friction: An Ethnography of Global Connection}, Princeton University Press.
}


%---------------------------------------------------
% gets only displayed in unfinished mode

\showImperfection

%---------------------------------------------------

\stopmode

\stoptext

\stopcomponent
