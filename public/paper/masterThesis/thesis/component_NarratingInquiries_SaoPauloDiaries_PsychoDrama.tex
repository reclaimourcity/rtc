\startcomponent component_NarratingInquiries_SaoPauloDiaries_PsychoDrama
\product product_Thesis
\project project_MasterThesis

% definitions and macros
\environment envThesisAllEnvironments
\environment envCfgThesisImages

\define[]\subsectionSaoPauloDiariesPsychodrama{\subsection[narratinginquiries_saopaulodiaries_psychodrama]{Psycho//Drama}}

\starttext

\startmode[tocLayout]
\subsectionSaoPauloDiariesPsychodrama
What is Psychdrama?
\stopmode

\startmode[draft]
\subsectionSaoPauloDiariesPsychodrama
What is \Psicodrama?

\startCitation
\Psicodrama is a form of research or therapy base on three roots: theatre, psychology and sociology. And, in a way it is a form to come closer to each other, to do something with what one encounters, to liberate oneself, to leave the stereotypes. [...] The first idea is citizenship, to discover through encounter that there exist the other, people with which one can close a pact, like the friends of the neighbourhood that meet in order to take care of that abandoned space there, or to demand more public transport. The second is to show the people to leave the \quote{non-place}. You enter the Metrô, you sit down there at a place next to another person, in front of each another and once you leave you do not even know if the other was a teen, a man, a woman. It is like as if the public space that is full of people would be a \quote{non-place}, where the people remain inside a bubble, isolated from one another. What we are trying to show is that this \quote{non-place} mainly present in the big cities can be transformed and that discovering the other is always interesting, because life is really evolving in the encounter, in the \quote{intermediate}. \aQuote{Cesarino}{2009}
\stopCitation

This citation is taken off the 2009 October issue of the street journal Ocas. \Va gave me a copy of it. One day \Va invited me to join him at a Saturday morning. He said that he is going since years, since seven years exactly, to the \Psicodrama. I have never heard of \Psicodrama but \Va said I will find out if i join him.

So here we are, its Saturday, we met around 11 o'Clock at the \locCCSP at \locLiberdade. The \abbrNew{Centro Cultural de São Paulo}{CCSP}\toTranslateT{Centro Cultural de São Paulo}{Cultural Centre of São Paulo} is a \aLinkNew[public cultural institution]{http://www.centrocultural.sp.gov.br/index.asp} open to everybody. It is a huge complex offering spaces for concerts, cinemas, discussions, reading, books and breakdance in its wide hallways, exhibitions or theatre. 

Here we are in one of those spaces, me and several others sitting on the ground level, on a kind of open gallery, watching down upon the floor below us. There, \Va is already active, talking with his friends while more people are arriving, some of them heading immediately to the seats, others are entering the main floor, young and old, all genders, \Va the only from the streets.

The \Psicodrama has already a long history at the \CCSP. It started in 2003 and takes place since then almost every Saturday late morning. Its open for everyone to take part. 

\imgPsychoDramaOcasPrint

After some time, the \Psicodrama starts. This time as well as at the following Saturday's when I join \Va, the people suggest the topic of the day. Today's topic will be the soon coming elections. People build groups of interest to discuss what they would like to say about the coming elections, about the politicians and how they are going to present their thoughts and feelings. Finally, each group came up with a plan, ready to present it in an improvised play. \Va plays today the master of puppets behind Dilma, the soon to be elected new president of Brasil. 

\startPersonal
I do not exactly recall now what the different groups have done. I had to think about what I just saw, about this wonderful way of approaching each other, taking part in formulating decisions, entering in discussion. \Va said that it took a huge portion of effort to accept the approaches the \Psicodrama proposes, not judging and not feeling judged by others. 
\stopPersonal

After the play of each group, people, on the scene but also the audience is being asked about what they perceived, how they perceived, what they think about the just see, how it fits with their opinion, where they disagree. From there the journey continued, always incorporating what has been done just before, what has been felt, the different opinions.

\startPersonal
I personally consider \Psicodrama wonderful and emancipatory because all those that take part do in a sense rule. There exist no discrimination, no matter where the people are coming from, what their conditions are, with all their many differences. The \Psicodrama is in a sense a free agreement between one another. It is moderated though, but it appeared that there has been always plenty of space for everyone. I also found interesting that it was not male dominated and that men have also not dominating the space by their behaviours. \Psicodrama is a proposal for action, to start practising the togetherness in one space, be it the city or theatre, not out of necessity but by the genuinely taking-part in the space and its production.
\stopPersonal

\imgPsychoDramaTwo

\startPersonal
I perceive \Psicodrama emancipatory as well in the sense that it aims to overcome the concept of imprisonment as treatment for those that do not fit into the \bracket{mainstream} society. This concept can be seen in the \gotoTextMark[Tendas]{narratinginquiries_saopaulodiaries_dayandnight_tendas} as the place of treating the  symptoms of being in street situation and trying to hide what is not supposed to exist, it can be seen in the prison and system as human deposits and the institutions of mental health treatment. \toTranslate{Delírios Urbanos}{Urban Delirium} says that it 
\stopPersonal

\startDialog
[...] mergulha no universo da saúde mental, [...], mostrando que é possivel construir uma sociedade livre dos antigos hospitais psiquiátricos, verdadeiros depósitos de gente que ainda restistem no Brasil, 21 anos depois do início do movimento de luta antimanicomial. \aQuoteB{Delírios Urbanos}{2009}{13}\footnote{[...] dives into the universe of mental health, [...], showing that it is possible to build a society free of psychiatric hospitals, true human deposits that still exist in Brasil, 21 years after the the begin of the struggle of the Asylum movement.}
\stopDialog

Another day, just before I had to depart, \Va tells me that he thinks about proposing \Psicodrama as a method of awareness rising for the police. He and some people of \RedeRua are having meetings with officials of the \abbrNew{Academia de Polícia "Dr. Coriolano Nogueira Cobra"}{ACADEPOL} at the Campus of the \abbrNew{Universidade de São Paulo}{USP}. He says that he could imagine \Psicodrama to switch roles temporarily, let the police play a person in street situation and show them how its is feeling being hit and kicked when one is laying already on the ground. He invited me to this meetings but finally we did not manage to go there together. One day stayed both at \locUSP but we missed each other because I have been there by bike but \Va had to manage somehow to arrive at the campus from the centre which is a long and expensive journey.

\addReference
{
Delírios Urbanos, 2009. Psicodrama Para Todos. {\em OCAS saindo das ruas}, (67), p.10-13.
}

\addReference
{
Cesarino, A.C., 2009. Psicodrama para todos.
}

\addReference
{
Delírios Urbanos, 2009. Psicodrama para todos.
}

\addReference
{
OCAS, 2009. OCAS. {\em OCAS saindo das ruas}, (67).
}


%---------------------------------------------------
% gets only displayed in unfinished mode

\showImperfection

%---------------------------------------------------
\stopmode

\stoptext

\stopcomponent
