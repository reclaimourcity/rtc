\startcomponent component_HowToRead
\product product_Thesis
\project project_MasterThesis

% definitions and macros
\environment envThesisAllEnvironments


% setup for table all cells
\setupTABLE[row][each]
[
	background=color,
	foreground=color,
	frame=off,
	rulethickness=1pt,
	corner=00,
	offset=2pt,
	backgroundcolor=lightgray,
	foregroundcolor=black,
	framecolor=yellow,
	style=\tt\small,
]

%\setupTABLE [column][first]
%[	
%	background=color,
%	frame=off,
%	rulethickness=2pt,
%	backgroundcolor=yellow,
%	foregroundcolor=black,
%]


\define[]\sectionHowToRead{\section[howtoread]{Read//This//Text}}

\starttext

\startmode[tocLayout]
\sectionHowToRead

describes how to read this thesis
\stopmode

\startmode[draft]
\sectionHowToRead

This thesis utilizes its own text formatting and colouring style in order to visually determine the meaning of certain words and text passages.

The following list describes the applied formats and their corresponding meanings.

\spaceHalf

\placetable[left]{How to read: thesis layout and format}
{
\bTABLE
\bTR
\bTD translations \eTD 
\bTD \eTD 
\eTR
\bTR
\bTD abbreviations \eTD 
\bTD\eTD
\eTR
\bTR
\bTD citations \eTD 
\bTD\eTD
\eTR
\bTR
\bTD text sources \eTD 
\bTD\eTD
\eTR
\bTR
\bTD narrations and inquiry \eTD 
\bTD\eTD
\eTR
\bTR
\bTD  narrations and theorizing \eTD 
\bTD\eTD
\eTR
\bTR
\bTD dialog \eTD 
\bTD\eTD
\eTR
\bTR
\bTD definitions \eTD 
\bTD\eTD
\eTR
\bTR
\bTD objectives \eTD 
\bTD\eTD
\eTR
\bTR
\bTD keywords \eTD 
\bTD\eTD
\eTR
\bTR
\bTD links to text passages \eTD 
\bTD\eTD
\eTR
\bTR
\bTD links to online sources \eTD 
\bTD\eTD
\eTR
\bTR
\bTD links to locations \eTD 
\bTD\eTD
\eTR
\bTR
\bTD footnotes \eTD 
\bTD\eTD
\eTR
\bTR
\bTD layout \eTD 
\bTD\eTD
\eTR
\eTABLE
}

%---------------------------------------------------
% gets only displayed in unfinished mode

\showImperfection

%---------------------------------------------------
\stopmode

\stoptext

\stopcomponent
