\startcomponent component_MethodologyMotivation
\product product_Thesis
\project project_MasterThesis

% definitions and macros
\environment envThesisAllEnvironments

\define[]\sectionMotivation{\section[methodology::motivation]{Motivations and Demands}}

\starttext

\startmode[tocLayout]

\sectionMotivation

This section describes the personal motivation for conduction research and writing the thesis and its implications.

\stopmode

\startmode[draft]

\sectionMotivation

\startKeywords
non-authoritative, non-hierarchic , genuine participation, open access,  practicability, emancipatory, field research methodology
\stopKeywords

Supplementary to those proposed questions above, personal motivations and demands on (my) research can be formulated as follows in order to further derive basic principles this thesis can be based on.

\spaceHalf
\startQuestionBack 
\startitemize[3]
\item I do not want to write a thesis just for a project or a degree.
\item I do not want to keep information, that has been given to me, just in an academic circle.
\item I prefer open access to information, stories and thoughts this thesis is composed of.
\item I do not want to represent anybody, I want to be together with the people and experience myself what they are experiencing.
\item I do not want to constrain people just for the cause of the thesis realization.
\item I see the thesis as a complementary tool in the struggle of the people on the streets.
\item I would like to attempt to find a non-elitist but common language, where academic and non-academic language converge.
\item I would like to approach the realization of the thesis in a non-authoritarian and non-hierarchical manner.
\item I do not want to \quote{invent} something \quote{new} but rather like to turn something present but invisible visible.
\item How and should I adapt the thesis to the period of time granted by the institutional academic framework?
\item I would like to learn something applicable for my personal practice.
\stopitemize
\stopQuestionBack 
\spaceHalf

\startAReminder
deshalb ist es gut alles zu erklären, damit das level der subjektivität eingeschätz werden kann :D

As every investigating subject is different, her social position, and her political
values, should be explicitly clarified so to put a light on the question of subjectivity. Nonetheless this confession, necessary on one side, is not by itself sufficient to control the effects of the investigation, neither it is to clarify completely the author’s socio-political distortion. The way to Knowledge in precise science is usually filtered by a measuring tool, while in social sciences it is filter by a subjective observer. On one side it is proper to describe the measuring tool, on the other it is correct to reveal subjective positions. \aQuote{Athenea Digital}{20}

Their aim is to generate a creating function and not anymore a
simple reproduction of the existing world. This tension between creation and reproduction reminds me of art, specifically to the difference between some modern age and contemporary artists. While the first used to reproduce a “natura morta” on a paint exactly as it would look like in reality, contemporary artists are willing to generate reactions in the observers of any type. \aQuote{Athenea Digital}{21}
\stopAReminder

\stopmode

\stoptext

\stopcomponent
