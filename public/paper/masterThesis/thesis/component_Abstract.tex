\startcomponent component_Abstract
\product product_Thesis
\project project_MasterThesis

% definitions and macros
\environment envThesisAllEnvironments

\starttext

\startmode[tocLayout]
\chapter[abstract]{Abstract}

Contains the english summary of thesis' content.

\stopmode

\startmode[draft]
\chapter[abstract]{Abstract}

This work emerged out of a six month empirical stay in São Paulo, Brasil. In a sense it is a rhizomatic map , consisting of many points and tracings, without fixed start and end, without a hierarchical order. This work aims to provide an insight, even though just as a scratch on the surface, to the life and struggle on the streets of São Paulo.

At the same time This work is part >of< and not an analysis >about< social movements. She is not intending to report about coping strategies of marginalised groups but about the discourse, the dreams and the difficulties that apparently seems to be inherent to a social praxis beyond discrimination and oppression. She will do this by narrating their insights and experiences, as diaries, from the standpoint of the streets and not science.

This work itself is also subject to its own research. She is asking about the meaning and the production of knowledge, how she is perceiving knowledge, how knowledge can be made accessible and reclaimable by all of us, how it can be produced without exploiting others, together and not for the profit and benefit of just a few. Action Research is her form of action, her knowledge is partial, Open Access is her principle of knowledge distribution.

Eventually this work is curious to learn more, from the streets and from theory. She would like to know more about the Right to the City,  a term she often run across in São Paulo and elsewhere. She would like to know more about genuine participation, about the idea of politic and the police, the partition of the sensible, about the idea of maximal difference and the city as the space of all those differences, of genuinely taking part in its social production, the aim of social movements.

{\bf Key Words:} action research, aRUAssa, Ay Carmela!, chaos, collectives, difference, knowledge, have part, Ocas, occupations, open access, partial knowledge, participation, politic, police, right to the city, rhizome, São Paulo, self-determination, social movements, social struggle, social praxis, space, standpoint, streets, rambles, take part, independent media
\stopmode

\stoptext

\stopcomponent
