\startcomponent  component_Methodology_WhatShouldIDo_Theorizing

\product product_Thesis
\project project_MasterThesis

% definitions and macros
\environment envThesisAllEnvironments

\define[]\subsectionMethodologyActionsTheorizing {\subsection[methodology::actions::theorizing]{Methods//Theorizing}}

\starttext

\startmode[tocLayout]

\subsectionMethodologyActionsTheorizing

Describes the framework for theorizing.

\stopmode

\startmode[draft]

\subsectionMethodologyActionsTheorizing

\startKeywords
citizenship, right to the city, spatial justice, participation, open access, media, theorizing, journals, resources
\stopKeywords

In contrast to research actions in São Paulo, \aKeyword[theorizing]{thesis+theorizing} for this thesis remains to a large extend my individual work as it has \goto{already been stated}[actionresearch::nocotheorizing]. This thesis theorizing draws on other theorizing manifold in form and expression. This thesis theorizing draws also on those \goto{denotations }[actionresearch::notionsoftheorizing] made when introducing \abbrFull{AR} as overall research framework. The \goto{already defined}[methodology::whatdoiwant::general] \aKeyword[objective]{thesis+objectives} of theorizing is the convergence if not even the overcoming of separateness of academic space and the social space, thus the mutual nurturing of both spaces in order to facilitate struggle for social transformation.

In order to reach this objective, the partial knowledge from the streets must find its way into this thesis as well as the partial academic knowledge that may be related to the issues and conflicts on the streets.

I further stick to the demand of open accessibility of produced content but also of used sources of information. This has several consequences, mainly with respect to access of academic papers but also with respect to access to information from sources in São Paulo, such as newspapers or movement content. I would like to separately handle the types of access to information in those two spaces.

\startAReminder[is theorizing knowledege production or content production] is theorizing also die from wie wissen generiert wird, und content production ist der ausdruck den dieses wissen annimmt?
\stopAReminder

\placefigure[location=here]{Types of sources of knowledge and content}
{
\starttyping

				open access

			/BTEX \tfa\ss\color[green]{oa journals} /ETEX

				/BTEX \tfa\ss\color[green]{own content} /ETEX
							/BTEX \tfa\ss\color[green]{movement content} /ETEX

academia			/BTEX \tfa\ss\color[green]{newspapers	} /ETEX			society

					/BTEX \tfa\ss\color[green]{books} /ETEX

	/BTEX \tfa\ss\color[green]{traditional journals} /ETEX
	
				restricted access

\stoptyping
}

\subject{Access to academic knowledge}

\inright{open access journals, blogs and other open sources}
As sources of academic knowledge I will mainly use open access journals \footnote{\toMark{what is a journal definition missing in footnote}} and papers that are freely available on the internet. All sources will be listed in the thesis \refMissingSrc{reference chapter} with the link to their download address. I will also make use of articles and essays available on blogs if I find them useful for this thesis. In certain cases, such as books and the like, no online access may be possible. I will try to minimize this kind of sources wherever possible as long as I think that their exclusion can be compensated with an equivalent that is open accessible.


This thesis' theorizing refers to two dimensions of research, the theoretical dimension and the activist dimension. In both cases knowledge, information and data are produced and utilized in order to theorizing for the peoples struggles, in order to produce \aKeyword{alternative content}.

Following the thesis' demand of \aKeyword{open access} to produced knowledge and information, sources of information useful for this thesis are supposed to be open accessible as well, in order to give everyone the possibility to easily accl ess utilized sources and allow an individual or collective reflection on them.

The main string of 

Open Access Journals (online)
- Movement Theorizing, Social Science, Philosophy

Other Content (online and offline)
- Newspaper, Scientific Journals with free papers, Books, Other References, Blogs

Movement Content (online and offline)
- Street Journals, Dossiers, Movement, Photos, Videos, Own Content

%--------------------------------------
% only active in "unfinished" mode

\showImperfection{irgendwelche referenzen}

\stopmode

\stoptext

\stopcomponent
