\startcomponent component_component_TheDecipheringOfGlobalCities
\product product_TheRightToTheCity
\project project_MasterThesis

% definitions and macros
\environment envCfgPaperLanguageEn
\environment envRttcUrl
\environment envRttcText

\starttext
\section[globalcity]{The Deciphering of Global Cities}

The \inright{urbanization necessary for the survival of capitalism} deciphering of (urban) spaces under contemporary, thus capitalistic, conditions, reveals the processes, roles, hierarchies and power structures that \hlef also perceived in the urban development of Paris in the 60's of the last century (Harvey, 2008 and Holm, 2010). A Paris that strongly segregates citizens according to a social-economic logic that leads to their expulsion to remote ghettos, the {\em Banlieus} (Holm, 2010). \lef also noted the extinction in distinction between urban and rural spaces (or the town and the city) due to the creation of {\ integrated and transnational spaces} (Harvey, 2008), a concept which is further developed in the context of \globalcities (or \worldcities) and their prominent role as backbone of globalisation. According to Harvey, \hlef predicted that the city and its development is central for the survival of capitalism and the future (but also contemporary) space of political (and) class struggle (Harvey, 2008).

The \inright{the global city} \globalcity is a further abstraction of spatiality and represents the city that is already detached from notions such as {\em urban} or {\em rural}. The \globalcity may occupy a geographical space but its function can only be understood in the context of a network of interconnected,  thus {\em integrated} (global or world) cities. These interconnections are of {\em national and transnational} scope and need no longer be of physically nature, in form of streets, highways or railways, instead they are purely virtual. Virtual networks which allow instant information and data exchange, with the (global) cities as its nodes and hubs. \from[wikiEnCastells] describes this setup as the\spaceofdataflows rather then the \spaceofplaces even though the city itself is the interface between \spaceofflows and \spaceofplaces and won't dissolve into the virtual network (Castells, 2004, p.85)

% space of flows and space of places
\setupDescrTable
\bTABLE[split=no]
\bTABLEbody
\bTR
  \bTC  space of places \inright{{\em spaces of places and flows according to Manuel Castells (2004, p.83)}}
 \eTC
  \bTC  space of flows \eTC
\eTR
\bTR
  \bTC  space of places are those places where people live in, where the daily life is experienced, which are bound to a locality, with physical and symbolic meanings, with form and function. Places are the symbols of the dominant power structures such as media, economy, technology or institutions, that exist and act in the \quote{space of flows}. The \quote{space of places} seems to have a similar notion as Lefebvre's \quote{lived spaces}. \eTC
  \bTC  spaces of flows can exist everywhere, they have no special characteristics, they are {\em non-places}. Its flows can be perceived as flows of information, capital, symbols, technology, images, social interactions, etc. The city is the interface between the local \quote{space of places} and the global \quote{space of flows}, thus is interlinks people or activities, in distinct geographical regions, worldwide, simultaneously. On the other hand, the city is one of the nodes that builds the global, electronically interconnected, network of spaces of flows, thus network of (global) cities.  \eTC
\eTR
\eTABLEbody
\eTABLE

\from[wikiEnSassens] \inright{Saskia Sassen's global city} developed the concept of \globalcities in the her book \quote{\em The global city: New York, London, Tokyo (1991)}. She assessed that \globalcities are the strategic command and control sites of global economy which particularly agglomerate {\em key services} of the so called F.I.R.E. sectors: finance, insurance, real estate. Those services serve an international customer base and are oriented to the world markets. They are also sites of production (of innovations) and offer (generate) the necessary market in order to trade them. Her primary focus on advanced services and innovations characterizes the internationalisation of production, the organisation of the division of labour on a global scale (Fröbel et al, 1980, in Beaverstock et al, 2009), i.e. production processes (often depending on cheap labour) are spatially (often globally) separated from design, innovation and management processes (Sassen, 1994 in Taylor et al, 2009), thus \globalcities are the prime sites of the knowledge elite that is accumulating there. 

\definitionframed{{\em Advanced producer services} \inright{advanced producer services} are those services that are produced, invented, traded and exchanged in and for the global economy. They are injected to and extracted from the global \spaceofflows and by that rely solely on the network of digital communication infrastructure, thus, their core can only consist of information and data. Those services operate on instruction, advice, planning, interpretation, strategy, knowledge, creativity, culture (Taylor et al, 2009), finance, real estate, or insurance (Sassen, 1991). }

\Sassens work also represents a shift from the existing concepts of the \worldcity at that time. \from[wikiEnFriedmann] and \from[wikiEnWolff] (Friedmann and Wolff, 1982) observed their \worldcity as command and \quote{control centres of capital flows in world economy} (Beaverstock et al, 2009). Their \worldcity has been mainly described as intense concentration of formal powers represented by major (international) corporate headquarters and institutions. \sassens \globalcity shifted the focus from the concentration of solely formal power that control (international) capital flows, to the agglomeration of services and innovations which enforce an internationalisation of production (Beaverstock et al, 2009). 

Similar \inright{Peter  HAll's World City} observations (of {\em institutional} power structures) have been made 3 decades earlier as well. \from[wikiEnPeterHall] described the \worldcity of his decade as

\quote{\em "[...] centres of political power, both national and international, and of the
organizations related to government; centres of national and international trade
and all kinds of economic activity, acting as entrepôts for their countries and
sometimes for neighbouring countries also." }(Hall, 1966)

The \inright{GaWC's World City Rooster} initial concept of the \globalcity has been further developed by the \from[gawc], an academic think tank on cities in globalization at \from[loughborough]. The early focus on concentration of advanced producer services as the indicator for {\em global cityness} has been substituted by measurement of \quote{network connectivity} as key indicator. According to Taylor (2009), \quote{network connectivity} defines how well a city is connected to the global network of cities, thus the global economy. It is not just the number of advanced producer service firms or offices a city possesses but the level of {\em connectivity} these sites generate by being interconnected with other sites in other cities.  The level of connectivity is translated into a rooster of world cities whose highest level defines pure world cities which have the highest level of advanced producer service connections to other cities, on a global scale, thus, they are the most important {\em command and control sites} of global economy, those nodes with the highest capacity in the global {\em space of flows}. 

The \inright{urban development and mega events} accomplishment of \worldcity status leads to a phenomena emblematic for contemporary globalization: the competition among cities on a global scale (Taylor et al, 2009). On the one hand, cities try to construct their global image in order to attract more services, more investment, more highly skilled information workers; on the other hand, cities try to attract for example mega-events such as Olympic Games or the Football World Championship, conventions and other spectaculars. Those events are often the trigger for large scale (gigantic) urban development projects which might completely change the face of the city in a relatively short time frame, i.e. {\em revitalization} of neighbourhoods and central city areas that result in the implicit {\em displacement} of old established neighbourhoods by emerging gentrification and the accompanied increase of living costs or direct {\em displacement} of neighbourhoods and slums by \quote{slum clearance} through \quote{beautification} projects (Greene, 2003, p.163). Further on, the implemented urban development schemes are then exported and replicated worldwide. One of the prominent examples is the so called \from[barcelonaModel] which has been realized as {\em flagship urban development project} for the Olympic Games of Barcelona in 1992. It attracted massive private investments and caused among others \quote{the revival of the real estate market [which] was rapid and voracious, from the Olympic nomination in October 1986 to the middle of 1990 [...] The market price of new and previously-built housing between 1986 and 1992 grew, respectively, 240\% and 287\%} (Brunet, 1995, p.17). The \quote{Barcelona Model} serves as a blueprint for contemporary urban development in the context of mega events, lately for the preparations of the Olympic Games that will be held in \from[rioOlympicGames] in 2016 (Fox, 2010).

\stoptext

\stopcomponent
