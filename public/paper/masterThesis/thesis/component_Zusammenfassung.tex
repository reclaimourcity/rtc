\startcomponent  component_Zusammenfassung
\product product_Thesis
\project project_MasterThesis

% definitions and macros
\environment envThesisAllEnvironments

\starttext

\startmode[tocLayout]
\chapter[zusammenfassung]{Zusammenfassung}

Contains the german version of the thesis' summary.
\stopmode

\startmode[draft]
\chapter[zusammenfassung]{Zusammenfassung}

\language[de]
{
Diese Arbeit entwickelte sich aus einem halbjährigen Aufenthalt in São Paulo, Brasilien. Diese Arbeit ist in gewisser Weise eine rhizomatische Karte, bestehenden aus vielen Punkten und Spuren, ohne festen Beginn und Ende, ohne hierarchische Ordnung, die einen Einblick, wenn auch oberflächlich, in das Leben und die sozialen Kämpfe auf den Strassen São Paulo's bietet. 

Zugleich ist diese Arbeit ein Teil und keine Analyse von sozialen Bewegungen. Sie wird nicht über Bewältigungsstrategien von unterdrückten Gruppen berichten sondern über den Diskurs, die Träume und die Schwierigkeiten, die eine andere Praxis jenseits von Diskriminierung und Unterdrückung mit sich bringt. Sie tut das indem sie aus São Paulo erzählt, aus ihren Tagebüchern und Erlebnissen, aus der Sichtweise der Strasse, nicht der Wissenschaft.

Diese Arbeit hat auch sich selbst zum Thema, sie fragt sich nach der Bedeutung und der Erzeugung von Wissen, und wie sie selbst Wissen betrachtet, wie Wissen für alle nutzbar und zugänglich gemacht werden kann, wie Wissen erzeugt werden kann ohne andere auszubeuten, miteinander und nicht zum Profit und Nutzen weniger. Action Research ist ihre Aktionsform, ihr Wissen ist partiell, Open Access ist ihr Verteilungsprinzip.

Letztendlich ist diese Arbeit neugierig Anderes zu erfahren, von der Strasse und aus der Theorie. Sie möchte mehr Wissen über das Recht auf Stadt, ein Ausdruck der ihr schon öfter begegnete, in São Paulo und anderswo. Sie möchte mehr Wissen, über das Sein und Nicht-Sein von Partizipation, über die Idee von Politik, die Unterteilung des Fühlbaren, über die Idee von maximaler Differenz und der Stadt als Ort des Zusammenseins aller Unterschiede und des freien Zusammenlebens, das Ziel sozialer Bewegungen.

{\bf Schlagworte:} Action Research, aRUAssa, Ay Carmela!, Besetzungen, Chaos, Differenz, Kollektive, Ocas, Open Access, partielles Wissen, Partizipation, Politik, Polizei, Raum, Recht auf Stadt, Rhizom, São Paulo, Selbstbestimmung, soziale Bewegung, sozialer Kampf, soziale Praxis, Standpunkt, Strasse, Streifzüge, Teilhaben, Teilnehmen, unabhängige Medien, Wissen
}
\stopmode
\stoptext

\stopcomponent
