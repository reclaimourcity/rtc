\startcomponent component_Methodology_WhoAmI_NonAuthoritarian
\product product_Thesis
\project project_MasterThesis

% definitions and macros
\environment envThesisAllEnvironments

\define[]\subsectionNonAuthoritarian{\subsection[methodology_whoami_nonauthoritarian]{Non-[Authoritarian//Hierarchical] Attitude}}

\define[]\subjectNonHierarchical{\subject{Non-Hierarchical Praxis}}

\define[]\subjectNonAuthoritarian{\subject{Non-Authoritarian Praxis}}

\starttext

\startmode[tocLayout]

\subsectionNonAuthoritarian

This section shall explain why non-authoritarian and non-hierarchical attitude is a prerequisite when conducting  research.

\subjectNonHierarchical

My justification for non-hierarchical praxis

\subjectNonAuthoritarian

My justification for non-authoritarian praxis

\stopmode

\startmode[draft]

\subsectionNonAuthoritarian

\startKeywords
non-authoritative, non-hierarchic , motivation, emancipatory, self-determination
\stopKeywords

\inright{non-authoritarian and non-hierarchical pervasion}
\aKeyword[Non-authoritarian]{attitude+non-authoritarian} and \aKeyword[non-hierarchical]{attitude+non-hierarchical} attitude shall pervade the ground on which this research shall be elaborated. This aspect is fundamentally important for me due to the fact that it represent certain attitudes I try to follow in my personal practice but also due to the fact that this thesis is partly based on information provided freely by people that are struggling for a social transformation of their marginalized reality. I consider my research action(s) to be part of this social and political struggle and therefore argue that my research praxis is supposed to follow the attitudes of my personal praxis as well. This praxis is also reflected in my approach to research that is mainly determined by \abbrNew{Action Research}{AR}\aKeywordInVi{action research} as I will explain later in more detail \gotoTextMark[later on]{methodology_whaoami_actionresearch}.

From my point of view, non-authoritarian and non-hierarchical attitude is strongly interdependent. 

\subjectNonHierarchical

The fact that research is often embedded in an existing academic framework already represents an implicit \aKeyword[hierarchy]{research+hierarchies} which could, and often lead(s) to situations where research agents \bracket{scholars and research projects for instance} primarily follow their own agenda and logic, in their terms of participation, constraints and benefits.\reference[methodology_whoami_nonauthoritarian_benefits]{}

One example is the question that \gotoTextMark[has been asked me]{narratinginquiries_saopaulodiaries_sometalks-hecticguy} in São Paulo, in how far academic research with marginalized people really supports the struggle of the \quote{participants}, whose role is basically limited to the provision of information utilized by the scholar.  The Scholar writes his or her thesis and through its completion, he or she gains a degree that offers better possibilities on the \bracket{academic} job market and career outlooks while the participants still do not see any improvement of their situation. 

One could argue that through the scholars' then more powerful position, he or she can direct more \bracket{institutional} power to provoke those transformation that research was lacking but for me personally this is no argument. Gaining a better position for instance has a more or less immediate effect on ones own life while trying to realize social change through institutions takes a long time with unpredictable outcome, if there is an outcome that leads to proficient transformation at all. One can question the role of institutions as \quote{change-makers} and agents of transformation in general if one looks at decades of aid and development projects that did not lead to any large scale and sustainable transformation of social inequalities existing all over the world. 

Therefore \inright{participation exerted as tyranny or emancipatory self-determination} I would like to question the terms under which \aKeyword[participation]{research+participation} in \bracket{research} actions is defined and realized. Is it exerted as a simple justification for the realization of inherently unjust research or \bracket{development} projects or is it exerted as an non-hierarchical and emancipatory approach to exercise self-determination in research but also in struggle? The notion of \participation is relevant for me in the context of for social \aKeyword[struggle]{participation+struggle} but also for \aKeyword[research]{participation+research} in general. I would like to leave the academic space in order to enter the streets and join the people, to realize this thesis research from a different \standpoint, from the \standpoint of the streets and its people.

Coming back to the notion of non-hierarchical praxis I perceive another implicit power hierarchy inherently embedded in the academic framework of my research because I can go abroad, made possible through scholarship, a situation barely realisable by those that shall participate in or which are addressed by particular research action, hence the status as foreign research agent automatically implies a difference in status of the research agent itself and those that shall participate in the agents actions \bracket{if intended at all}. 

This situation can be described plastically with a quote of a guy from the streets \gotoTextMark[I met and talked with]{narratinginquiries_saopaulodiaries_sometalks_hecticguy} in a small and shady street in the centre of São Paulo:

\startCitation
Tell me, what does a guy from the first world do here in the third world? Why are you here? Don't you have problems to solve and analyse in your country? \aQuoteP{2010}
\stopCitation

\subjectNonAuthoritarian

\inright{non-authoritarian actions in order to conduct research, neither oppressive nor seductive}
Non-authoritarian attitude is the practice I am affiliated with. Here, a contradiction could arise because I would like to experience and get insight into those situations that would finally be incorporated into research. I would like to get in touch with the people and become active in their struggle, because I feel solidary with their struggle, because I would like to realize research actions as part of this struggle. In order to do this, to get in touch, to experience, I would never impose acts of authoritative actions upon those that share their information, that share their trust in me. Neither through implicit or explicit actions, nor in oppressive or seductive ways.

Apart from the question of access, thus access to the people, their reality, their struggle, the question of access to the research' outcome is related to a non-hierarchically attitude as well. 

\spaceHalf

\inright{open access instead of exclusion from knowledge}
The \rhizomaticMap that my thesis represents is composed of \gotoTextMark[personal narrations]{narratinginquiries_intro} and \gotoTextMark[theorizing]{theorizing_intro} drawn on theory conceptualized in books, journals or available on the internet. I would like to question the way this sets of information are \bracket{or have to be} made accessible by academia. 

Little is openly \bracket{thus freely accessible} published in academic circles due to an elitist attitude and the \aKeyword[commodification]{knowledge+commodification} of knowledge and information, where knowledge, even though elaborated and produced in public institutions or based on peoples knowledge, remains behind impermeable walls, remains solely accessible to those that have selected and appropriated that knowledge or which have the necessary \bracket{monetary} resources or the necessary status in order to do so. 

This situation describes another facet of \aKeyword[purpose]{research+purpose} and \aKeyword[demand]{research+demand} on my research action(s), here as a demand of \aKeyword[free distribution]{objectives+free distribution} and \aKeyword[open accessibility]{objectives+open access} to the thesis outcome.

\spaceHalf

\inright{research actions conducted in existing institutional frameworks or self-determinately organized}
Another notion of rather practical nature is the time frame reserved for thesis writing and research action(s). The initially contemplated and official period for research actions and theoretical examination had to be 5 month, 2 to 3 reserved for empirical research abroad, the remaining time reserved for theorizing and writing of the final thesis.

Now, I exceeded those specifications due to the fact that research action(s) in São Paulo already lasted 6 month. Writing this thesis took another 6 month \bracket{not full time}. Thus again I return to the question of \aKeyword[constraints]{research+constraints} and \aKeyword[benefits]{research+benefits} the academic research agent is \gotoTextMark[accepting and seeking]{methodology_whoami_nonauthoritarian_benefits}.

Looking back at the time that has been passed since I arrived in São Paulo, I have realized that if I had followed the \aKeyword[strict time setting]{restrictions+time} imposed on me, I wouldn't have had the time to reflect on and adjust to the situation I entered. I would not have to time for self-organisation of my research action(s), to get in touch with the people nor to build relations among us. This would then probably have led to a work that just followed the logic of acquiring an academic title or developing a technical fix while leaving context, approach, praxis and effect of conducted research actions rather insignificant, just as necessary means to the anticipated end.

\spaceHalf

The concept of \participation as \aKeyword[non-authoritarian]{attitude+non-authoritarian} and \aKeyword[non-hierarchical]{attitude+non-hierarchical} praxis composes the thesis title thus represents the intended approach to \gotoTextMark[research actions]{methodology_whoami_actionresearch} in São Paulo. \Participation is not an uncontested concept as shown later.  By \deconstructing and \reconstructing it I try to get a grasp on its various meanings. \Participation in \researchActions and as subject of \theorizing is therefore another \aKeyword[objective]{objectives+participation} of this thesis.

%\startARemark{hier könnte nochmal eine übersicht hin} , welche themen von der antiathoritären frage durchzogen sind, als letzter überblick bevor wissenschaftliche herangehensweisen betrachtet werden.
%\stopARemark

%\startARemark{noch wichtig?}
%\inright{do fundamental questions correlate with motivation and demands?}
%This has consequences for my research actions and rises more fundamental questions: 

%\spaceHalf

%\placefigure{}
%{
%\textBoxedRoundMax[0.3]{ What is the purpose of research? How do I conduct research? What is my role? How is my role perceived? How do I approach people? What do I want from them? What do they want from me? Can we find common ground to collaborate coequally?}
%}

%\inright{open access}
%\textBoxedRoundMax{For whom? To whom? By whom?}
%\stopARemark


%\inright{overview of mentioned anti-authoritarian / non-hierarchical themes.}
%\startARemark{hier kann noch ne word map rein}
%\stopARemark

\showImperfection

\stopmode

\stoptext

\stopcomponent
