\startcomponent component_Theorizing_Intro
\product product_Thesis
\project project_MasterThesis

% definitions and macros
\environment envThesisAllEnvironments
\environment envCfgThesisImages

\define[]\chapterTheorizing {\chapter[theorizing_intro]{Theoretical Themes}}	

\starttext

\startmode[tocLayout]
\chapterTheorizing

Structure of concepts of the determined theoretical themes.

\stopmode

\startmode[draft]

\imgCidadaniaGraffitiMohino

\chapterTheorizing

Structure of concepts and determined theoretical themes.

%----------------------------------------

The final part is first of all an invitation for further elaboration and criticizing.

It is formulated as a proposal from a standpoint that aims for social transformation, visible, practised and demanded in the social struggles in São Paulo. Correspondingly, the proposal includes the themes of \Citizenship, \Participation, the \RightToTheCity and \SelfDetermination and intends to \toMark{relate them to} emancipatory praxis. Therefore it contest the very same themes loaded with meaning from the standpoint of public institutions, the state, \bracket{inter}national agencies, thus the actual power holders.

The proposal aims to give those themes a different sense, to \reconstruct them with emancipatory content and show how practice could and does already elaborate around it. By reconstructing I mean to inject other meaning into those words that are \bracket{most likely} familiar to us and which \bracket{most likely} provoke a particular meaning, often determined by the public discourse led by the power holders or the discourse that is hold in society.

In a sense this theorizing proposal could be perceived as a critique of the current \toMark{capitalistic} status quo \bracket{the society, thus the city, represents}, where those are discriminated and oppressed that are differing from or dropping out the mainstream public discourse, that do not want or cannot act as it is required in terms of social norms, duties and obligations. But rather then merely criticising by producing just another set of words and phrases that say what has already been said many times, the aim of theorizing could be extended in order to draw on what people are self-determinately organizing and practising already, to draw on the self-conceptions and attitudes that are already formulated, practised, reflected upon and overthrown if considered inappropriate or not sufficient. 

\spaceHalf

\inright{theorizing as proposal \bracket{and critique}}
Thus my aim would be then to make a proposal and to see where it would lead to, how it could contemplate other proposals that are aiming for emancipatory social transformation as well but that are using a different language, different sets of concepts, formulated from different standpoints.

\spaceHalf

\inright{stripping off co-opted and depoliticized meaning: deconstruction of themes}
I think the first step necessary would be the \deconstruction of the commonly accepted meaning of words in order to perceive which power hierarchies are embedded in them. Once dismantling is done, a different sense can be injected. Deconstruction is critique but its formulation is just the first step.

I would also not like to invent new expressions or themes because in my opinion what is written shall remain understandable and should not be hidden behind a complex web of language and words that only specialists familiar with the topic can understand. \toMark{Using familiar sets of words and expressions could also help to argue about social struggle in terms that are understandable by those outside of it}.

\spaceHalf

\inright{injecting critical and emancipatory content: reconstruction of themes}
Therefore I would like to use themes and words like \citizenship or \participation even though they are heavily contested as we will \refMissingSrc{see soon}. This would then be the second step of theorizing, gaining back the meaning of words that are often co-opted and tamed by public, political and institutional discourse, stripped off their political and critical attitude. \Reconstruction of themes means injection of critical and emancipatory content.

The themes for theorizing \gotoTextMark[have already been determined]{ref missing to text mark}. They are culled out the social struggles of the streets of São Paulo, from \gotoTextMark[campaigns]{tex mark missing to MNCR} and \gotoTextMark[movement theorizing]{text mark missing Dossie Forum Centro Vivo}, from \gotoTextMark[individual discussions]{text mark missing to discussion}, from \gotoTextMark[particular actions]{text mark missing to FLM, aRUAssa, Psycho Drama}. 

I would like to complement them by a few more ideas: the idea of \politics as imagined by \refMissingSrc{Jaques Rancier} which may impact the way participation is de- and re-constructed - the idea of \sspace as imagined by \refMissingSrc{Henri Lefebvre} in order facilitate the imagination of the spaces we struggle in, the spaces we life in, the spaces the city represents and creates, how those spaces affect us and how we affect them, how the imagination of space could probably broaden the ability to  formulate critiques, demands, proposals and benefit actions and practices aimed for social transformation - the idea of \minDifference and \maxDifference as imagined by \refMissingSrc{Henri Lefebvre} as well, which I would like to regard as a basic concept for a just city, for genuine \participation in society, for the \rightToTheCity. \refMissingSrc{footnote with a list of references for further reading.}

I do not intend to theorize about the state or concepts of democracy in particular, about available categories one can relate themes like \citizenship to. I will make use of available categories if I they help me to argue and to explain but in principle I would like to remain abstract.

%---------------------------------------------------
% gets only displayed in unfinished mode

\showImperfection

%---------------------------------------------------

\stoptext

\stopcomponent
