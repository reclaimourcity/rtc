\startcomponent component_MethodologyTheoryAndPracticeQualitative
\product product_Thesis
\project project_MasterThesis

% definitions and macros
\environment envThesisAllEnvironments

\define[]\subsectionQualitative{\subsubsection[methodology::motivation::qualitative]{Qualitative versus Quantitative?}}

\starttext

\startmode[tocLayout]

\subsectionQualitative

This sub-section gives a brief excursion about qualitative and quantitative data and why this thesis is considered as qualitative.

\stopmode

\startmode[draft]

\subsectionQualitative

\startKeywords
qualitative, quantitative
\stopKeywords

While being in São Paulo, it became clear over the course of the first two month that moving through the city means learning how the city functions. This meant to learn how to use public transport, how to use and impropriate the city in order to loose the fear of getting lost. Further on it meant to impropriate the slang of the streets and finally to get in touch with the people, in this case with the people that live and work on the streets. 

What became evident was the fact that new dimensions of the city emerged after social relations had been established. Those dimensions are partly invisible in daily life (more on this later on) and can often just be perceived if one spends hours together on the streets, hanging around or walking from one spot to another and by that circling through the city, hearing many stories, perceiving many situations, such as this emblematic one:

\startCitation
ein gutes beispiel
\stopCitation

The experience of being on the streets, embedded in and being a part of a network of people, lead to a more comprehensive understanding of the city, its dimensions and realities. This comprehension also means that any research based on quantitative measures, would have failed or at least led to wrong assumptions because any questions and situations imagined before hand would have been non-relevant (in the sense as stated in this chapter), due to the lack of understanding of the city, the situation of the people and a biased European perspective. 

\startADefinition
{\bf quantitative} \inright{data sets, normalisation, indicators, statistics, calculus, theory generation, theory prove}
 {\em methods generate quantities, thus relatively large amounts of normalized and standardized data sets. These sets can for example be generated by conducting surveys, a classical method where copies of a set of questions are distributed to many people, probably several hundreds or more, and where an analysis of the received answers is performed afterwards, when all surveys had been re-collected. Analysis can be done statistically for example, because surveys are usually standardized to a set of questions (every participant receives the same questions so to speak) that are related to one or more indicators. An indicator is a \toMark{formal} representation of a specific domain of interest: if a survey asks about your housing condition (rent, squatted, property, homeless, etc), a related indicator could represent the domain of \quote{housing-types} or something else. Thus quantitative approaches are about (relatively) large amounts of data and information, normalisation, indicators, statistics and calculus. Quantitative approaches inherently generalize the made observations because collected data sets usually contain large numbers of entries and are often considered representative (though not all-encompassing), thus generalizable.
}

{\bf qualitative} {\em }  \inright{non standardized, interpretative, comprehensive, theory extension, theory adoption}
{\em methods generate comprehensive data, thus sets of information that could represent a particular context in order to understand its meaning in a comprehensive, holistic or qualitative manner. This data is not meant for statistical analysis even though it can be related to indicators as well, but due to the small size of data-sets equipped with a large amount of non standardized context information, statistical analysis renders not feasible. Therefore, analysis is often done in interpretative ways. Referring to the quantitative example, qualitative data would not only ask about the housing type, but probably also about the materials it is constructed of, the neighbourhood, the age  and once an answer doesn't fit  (i.e. no house but the streets) the process of information gathering can be adjusted to the new situation. In addition, those information may be gathered during an interview or a talk, thus involves inter-personal communication and personal opinions. Qualitative approaches do not generalize because the number of data sets, though comprehensive, are not sufficient to draw a general picture from it, maybe just a small picture from a small spot. Therefore, qualitative information and analysis can be used to extend or adjust existing knowledge by new facets.
}
\stopADefinition

\spaceHalf

\inright{quantitative shortcomings}
Quantitative approaches require access to a relatively large group of people and other resources of knowledge. Several drawbacks emerge from that necessity: 

\spaceHalf

\inright{enforced power hierarchies due to distance and impersonality: definition shapers still rule information providers}
The relative distance and impersonality to people, which is immanent due to the fact that a large group is addressed. This impersonality and distance can be seen as a hierarchy between the research agent and the research participants. The participants are solely perceived as information providers and not as participants (in the sense discussed in the \toLink{participation chapter}). This form of participation just means outsourcing the information generation process while keeping the powerful position of defining (probably wrongly) which type of information is necessary or relevant. Here we enter the question of power again which is supposed to be dissolved in the course of this research.

\spaceHalf

\inright{dissolution of power hierarchies requires genuine participation}
Dissolution of power hierarchies means that people are involved in shaping the whole process of research. This would mean that decisions about the relevance of research, the type of research objectives, the provided information and the usage thereafter are decided together with and by the people. In order to accomplish this situation one would require access and involvement in the peoples networks and communities. This accomplishment would require much more time then granted by the institutional framework this thesis is embedded in and is therefore impossible to achieve.

\spaceHalf

\inright{generalization leads to elision of the margins}
The motivation of gathering large sets of data is usually to be able to generalize analysis, by proving previously formulated hypotheses or by creating new theories. Due to the fact that collected data sets seldom represents the whole spectrum of variants and possibilities, generalization do not cover all aspects of the area of interest, which could be no problem in certain areas, such as the measurement of rainfall, but should be taken in mind when actions are socially motivated and generalization may lead to the exclusion of the margins of society. Thus, generalization is problematic for this research because no one shall be elided.

\spaceHalf

\inright{lack of local knowledge leads to wrong perceptions}
A quantitative approach applied for this thesis research, such as a survey, that had been designed before hand, during preparation of the research, without the concrete knowledge of the city, would have been composed of assumptions that led astray. 

\spaceHalf

\inright{qualitative shortcomings and benefits}
The previously mentioned drawbacks are party dismantled by going qualitative rather than quantitative. Qualitative approaches can have similar shortcomings than quantitative ones, depending on the type of approach and the motivation of its application. 

The question of information gathering in the qualitative context can lead to similar, wrongly made assumptions than already explained above, due to the fact that local knowledge must be gathered somehow. Thus, a qualitative interview based on wrong perceptions and lack of local knowledge can lead to the same misleading assumptions than a wrongly designed quantitative survey. 

On the other hand, qualitative approaches do not aim to be generalizable, hence inherent to the access to local knowledge is the process of becoming part of the peoples networks and communities, on a much smaller scale, probably just on interpersonal or group level. Thus decisions and discussions about research, its relevance, its objectives and so forth can be mediated directly by and with the people. This process also requires time and trust between each other but could open other viewpoints on research for all. 

If we return to the notions of hierarchy and emancipation, one could argue that the processes of becoming acquainted with people, building relations and being together, already provides plenty of space for acting in a non-hierarchical, non-authoritarian and emancipatory manner. This possibilities are of course not absolute nor are they vested due to the fact that power hierarchies pervade all aspects of social life of individuals and groups, thus are always visible or perceivable \toMark{ref missing?}.

However, having the possibility to observe and take part in a genuine participatory manner opens up spaces that would not have been reached just through plain non-genuinely-participatory approaches. A note from São Paulo that certainly impacted the way I see the relevance and significance of my research, may illustrate this claim.

\startCitation
After I met \toMark{Matheus} one day at \toMark{Ay Carmela (a self-organized centre)}, we fixed a date when he was going to show me the streets where he lived for 15 years, in the centre of São Paulo. First we planned to spend one week on the streets but later reduced that time to two days. What I have seen and experienced in those days is barely comparable to informations gathered through semi-personal surveys or interviews which would have failed to transport the intensity and manifoldness of situations. This trip on the streets gave me answers to questions that I wouldn't have been able to formulate in advance and once it passed it wasn't necessary to ask any more because everything had already been unfold \aQuoteP{2010}.
\stopCitation

\spaceHalf

\inright{the thesis foundation in terms of empirical and qualitative categories}
Lets summarize the initial attributes of this thesis research actions:

\startitemize[3]
\item The concrete experience and the gained local \toMark{or endogenous?} knowledge finally proposed the research action approach and theme for this thesis. {\em This thesis research focus and approaches are determined by local knowledge, observations and experience.}
\item The process of gathering local knowledge imply that this thesis is drawn on observed and experienced, thus, qualitative information which are further theoretically contemplated. 

\startAReminder
der nächste satz ist falsch und muß neu formuliert werden
\stopAReminder

{\em This thesis is inherently empirical and inductive.}

\stopitemize

\stopmode

\stoptext

\stopcomponent
