% general environment files
\environment envRTC
\environment envCmdRTC
\environment envTextRTC
\environment envUrlRTC

% specific urls vor this text
\environment envUrlOcupacaoIpiranga
\environment envLinksOcupacaoIpiranga

\starttext

% archive.org
[en]

The interview was recorded on Saturday, 09th of October 2010, at the occupation of FLM (Frente de Luta por Moradia) at Avenida Ipiranga in the center of São Paulo. One of the women that coordinates the occupation explains their organizational structure, the aim of their struggle, from where the people are coming....

The occupation is one of four simultaneous occupations that were realized on the first October weekend in 2010 by various movements that fight for housing rights for the poor and low income citizens in the city.

The recording was realized by occupation Avenida Ipiranga, arRUAssa (a video collective of activists from indymedia Brazil and the national movement of the street population, MNPR) and r3cl41m.
\nl
[pt]

A entrevista foi gravado no Sábado, nove de outubro 2010, na ocupação do FLM (frente de luta por moradia) na Avenida Ipiranga, no centro de São Paulo. A coordenadora da ocupação explica a estrutura organizatória da ocupação, as propostas da gente, de onde as famílias vem...

Essa ocupação é uma de quatro ocupações que foram realizados no primeiro fim de semana no centro de São Paulo por movimentos de moradia.

A gravação foi realizada junto com a ocupação na Avenida Ipiranga, o coletivo arRUAssa (com ativist@s do cmi brasil e do movimento nacional da população de rua - MNPR) e r3cl41m.

% blogpost english
\nl
\nl
Yesterday, on Saturday the 9th of October, we visited one of the \from[flmocupa] in the center of the city to make interviews and record videos with the people. We are r3cl41m and the video collective \from[arruassa] with people from \from[cmi] and the Movimento de População de Rua (\from[mnpr]). The interview took place at the occupation of the Frente de Luta por Moradia (\from[flmsite]) at Avenida Ipiranga in the center of São Paulo, close to República. 

Several housing movements occupied \from[flmmapa] in center of the city on the 4th of October in order to \from[flmproposta] for the ongoing housing speculation in the city and the non existence of affordable housing for poor and low income citizens in the center. 

More information from our interview will follow soon. In the meantime, one can get daily updates on the site of \from[flmsite] and on \from[flmtwitter]. 

There was announced as well a digital mapping initiative that should be elaborated during the \from[hackdayz] that took place today at the \from[casa] in Barra Funda/São Paulo. The initiative aims to map empty and abandonded buildings in the city, as well as the street population, or lets say the attributes of street population. Some impressions of the hackdayz will be given in the next post, as well as some audio recordings.

The video version done by \from[arruassa] collective can be seen at \from[bliptv].

%video goes here

The complete interview is freely available at \from[archive] and dedicated to the \from[commons].

{\bold\em Update 16.10.2010}

{\em The interview is now available in single parts as well.}
{\em The video of the arRUAssa collective has been embedded as well}
\nl

\myStartItemizeArchive

\head Interview  with the coordination of the occupation at Avenida Ipiranga \myFromArchive{0}

complete recording, original audio

\head Why did the people come to the center? \myFromArchive{1}

\head Where did the people life before? \myFromArchive{2}

\head There are many empty buildings in the center \myFromArchive{3}

\head The risk of eviction \myFromArchive{4}

\head Street population \myFromArchive{5}

\head The poor are exploited \myFromArchive{6}

\head What are the demands of the movement? \myFromArchive{7}

\head How is the occupation organized? \myFromArchive{8}

\head How many buildings have been occupied? \myFromArchive{9}

\head Negotiations with the city \myFromArchive{10}

\head The occupation could become permanent housing \myFromArchive{11}

\head Homeless hostels are no alternative \myFromArchive{12}

\head The working groups of the occupation \myFromArchive{13}

\head The social function of abandoned buildings \myFromArchive{14}

\head Sewage \myFromArchive{15}

\head The number of people that life here \myFromArchive{16}

\head Gender discourse \myFromArchive{17}
\myStopItemize
\nl
{\bf Media}
\myMedia

{\bf Links}
\myLinks

% blogpost portuguese
\nl
\nl
Ontem, Sábado 9 de Outubro, a gente visitavam uma das  \from[flmocupapt] no centro da cidade para entrevistar as pessoas e para gravar vídeos. Nós somos r3cl41m e o coletivo \from[arruassa] com ativist@s da \from[cmipt] e do Movimento de População de Rua (\from[mnpr]). A entrevista foi feito na ocupação da Frente de Luta por Moradia (\from[flmsite]) na Avenida Ipiranga, perto á República, no centro.

Vários movimentos de moradia ocuparam junto \from[flmmapapt] no centro no dia 04 de Outubro para \from[flmpropostapt] a especulação imobiliária persistente e a non-existência de moradia de baixa renda no centro da cidade.

Mais informações da entrevista em breve. Entretanto, notícias atuais podem ser achado no site do \from[mnpr] e no \from[flmtwitter].

Uma outra coisa que parece interessante foi anunciado para os \from[hackdayz] que aconteceram hoje na \from[casa] em Barra Funda/São Paulo. A proposta é a fazer um mapeamento digital dos imoveis vazios na cidade e também um mapeamento da população em situação de rua, ou seja os atributos que definem as (isso parece estranho para mim). Alguns impressões desse dia será relatar no próximo post, junto com gravações de áudios.   

O vídeo do coletivo \from[arruassa] está disponível na \from[bliptv] agora também.

%video goes here

A entrevista inteira está disponível no site \from[archive] e dedicado aos \from[commons].

{\bold\em Update 16.10.2010}

{\em A entrevista está disponível em pedaços individuais agora também.}
\nl
% generate a numbered list
\setupitemize[headstyle=bold]

\startitemize[N][start=0]
\head Entrevista com a coordenadora da ocupação Avenida Ipiranga \myFromArchive{0}

gravação inteira, áudio original

\head Como a gente chegavam no centro? \myFromArchive{1}

\head Onde a gente moraram antes? \myFromArchive{2}

\head Muitos prédios abandonados no centro \myFromArchive{3}

\head Risco da reintegração \myFromArchive{4}

\head Moradores de rua \myFromArchive{5}

\head Os pobres são usadas \myFromArchive{6}

\head Qual é a proposta do movimento? \myFromArchive{7}

\head Como funciona a organização da ocupação \myFromArchive{8}

\head Quantos prédios foram ocupados? \myFromArchive{9}

\head Negociação com a prefeitura \myFromArchive{10}

\head A ocupação pode virar para moradia permanente \myFromArchive{11}

\head Albergue não é alternativa \myFromArchive{12}

\head Comissões da ocupação \myFromArchive{13}

\head Função social dos prédios abandonados \myFromArchive{14}

\head Esgoto \myFromArchive{15}

\head Número das pessoas que vivem na ocupação \myFromArchive{16}

\head Discussão sobre gênero \myFromArchive{17}
\stopitemize
\nl
{\bf Mídia}
\myMedia

{\bf Links}
\myLinks
\stoptext