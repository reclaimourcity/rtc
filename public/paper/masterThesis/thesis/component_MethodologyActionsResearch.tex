\startcomponent component_MethodologyActionsResearch
\product product_Thesis
\project project_MasterThesis

% definitions and macros
\environment envThesisAllEnvironments

\define[]\subsectionActionsResearch {\subsection[methododology::actions::research]{Grounded//Participant//Observ[er][ation]//Action}}

\enablemode[draft]
\disablemode[tocLayout]

\starttext

\startmode[tocLayout]

\subsectionActionsResearch

\stopmode

\startmode[draft]

\subsectionActionsResearch

\startKeywords 
participant observation, participating observer, grounded theory, action research 
\stopKeywords

After all those reflections on attitudes, a thesis self conception and determinations of \quote{relevance} aspects, I want to describe those research action principles that has been followed in São Paulo.

As already mentioned in the introductory chapter \refMissingSrc{Methodology Introduction}, I came to São Paulo without prior knowledge of the city and without concrete idea about the purpose of my stay. So to say, the city drove me into the direction which is finally illustrated in this thesis. Most experiences and information approached me in a non-systematic manner because I was not looking for particular events nor situations or people, but I tried to absorb and keep hold of everything.

\spaceHalf

\inright{participant observation (PO) }
For me, this premise seems optimal for conducting \aKeyword{Participant Observation }(PO). Even though PO serves as the basic framework for realizing research action(s), it does not appear to completely grasp the research actions intention and realization. As explained in more detail later on, PO can be seen as a method to passively observe scenes and situations. Therefore I would like to extend the concept of Participant Observation with the concept of \inright{participating observer} the \aKeyword{Participating Observer}, who actively participates in concrete action(s) and events as well, and by doing so, has an active impact on the form of those observation(s) that he or she wants to keep note of. 

\startARemark
I don't know if the next parts about Grounded Theory and Action Research will make it into the final version
\stopARemark

\inright{grounded theory (GT)}
Additionally, a portion of \aKeyword{Grounded Theory} (GT) is apparent here as well, because GT does not pre-suppose anything when starting research and incorporates all available types of information in the formulation of a research objective, thus in a sense, it is driven by (or grounded on) every imaginable type of observation, from the streets to the newspapers.

I would fuse those three approaches, \aKeyword{Participant Observation}, \aKeyword{Participating Observer} and \aKeyword{Grounded Theory}, under the umbrella of \aKeyword{Action Research} (AR). AR is the chosen concept because the inherent intentions and objects expressed by AR are to a certain extend the intentions and objectives of this thesis. More on this in the corresponding section on Action Research \refMissingSrc{Action Research chapter}.

\stopmode

\stoptext

\stopcomponent
