\startcomponent component_NarratingInquiries_SaoPauloDiaries_Grouping
\product product_Thesis
\project project_MasterThesis

% definitions and macros
\environment envThesisAllEnvironments

\define[]\sectionSaoPauloDiariesGrouping {\section[narratinginquiries_soapaulo_grouping]{Structural//Complexity}}


% setup for table all cells
\setupTABLE[row][each]
[
	background=color,
	foreground=color,
	frame=on,
	rulethickness=1pt,
	corner=00,
	offset=2pt,
	backgroundcolor=black,
	foregroundcolor=yellow,
	framecolor=yellow,
	style=\tt\small,
]

\setupTABLE [row][first]
[	
	background=color,
	backgroundcolor=black,
	foreground=color,
	foregroundcolor=yellow,
	frame=on,
	rulethickness=1pt,
]

\setupTABLE [column][first]
[	
	background=color,
	frame=off,
	rulethickness=2pt,
	backgroundcolor=yellow,
	foregroundcolor=black,
]

\starttext

\startmode[draft]
\sectionSaoPauloDiariesGrouping 

I would like to shortly draw upon the complexity of the world I have perceived. While narrating, I will not try to determine a certain order, assign too much functions and roles to people, places or actions. That short snapshot taken by research actions is my extended memory, the memory of that particular time that I spend with people and that that they spend with me. Therefore I would like to denote the subjects of that memory, even though in a simplistic manner, in order given notions of the sensible embedded in the \gotoTextMark[already mentioned questions]{methodology_intro} \WhoAreWe \WhatDoWeWant \WhatShouldWeDo. Those notions are sensible in the spaces created by social struggle, and in the spaces that enforce social struggle due to its structural inequalities and could hopefully catch a glimpse on what social transformations actually  are and could further be possible through social struggle and actions in urban spaces and beyond.

The notions in the following map emerged organically, they are representing mainly what I have been in touch with more frequently, what probably has been denoted by my narrations more frequently as well.

\placetable[force, split]{The diary's notions of the sensible of social struggle}
{
\bTABLE
\bTR
	\bTD[nr=2, align={middle,lohi}, width=3cm] \bracket{political} subjects\eTD 
	\bTD collectives \eTD 
	\bTD movements \eTD 	
	\bTD individuals \eTD 
	\bTD groups \eTD
\eTR
\bTR
\eTR
\bTR
	\bTD[nr=2, align={middle,lohi}] places \eTD 
	\bTD public spaces \eTD 
	\bTD cultural \eTD 
	\bTD self determined \eTD 
	\bTD institutional \eTD
\eTR
\bTR
	\bTD abandoned \eTD
	\bTD settlements \eTD
\eTR 
\bTR
	\bTD[nr=2, align={middle,lohi}] standpoint \eTD 
	\bTD jail \eTD 
	\bTD crime \eTD 
	\bTD addict \eTD 
	\bTD individual \eTD
\eTR 
\bTR
	\bTD movement \eTD
	\bTD in street situation \eTD
\eTR 
\bTR
	\bTD[nr=3, align={middle,lohi}] aims and goals \eTD 
	\bTD visility \eTD 
	\bTD dignity \eTD 
	\bTD right to the city\eTD 
	\bTD participation \eTD
\eTR 
\bTR
	\bTD work \eTD 
	\bTD housing \eTD 
	\bTD health care \eTD 
	\bTD eduction \eTD
\eTR 
\bTR
	\bTD access to the city \eTD
\eTR 
\bTR
	\bTD[nr=2, align={middle,lohi}] roles \eTD 
	\bTD passive observer \eTD 
	\bTD observing participant \eTD 
	\bTD participating observer \eTD
\eTR 
\bTR
\eTR
\bTR
	\bTD[nr=4, align={middle,lohi}] actions \eTD 
	\bTD street cinema \eTD 
	\bTD workshops \eTD 
	\bTD psycho drama \eTD 
	\bTD housing occupation \eTD
\eTR 
\bTR
	\bTD daily solidarity \eTD 
	\bTD film making \eTD 
	\bTD manifestations \eTD 
	\bTD recycling \eTD
\eTR
\bTR
	\bTD mini-feira \eTD 
	\bTD open university \eTD
	\bTD collective lunch \eTD
	\bTD festivals \eTD
\eTR 
\bTR
	\bTD street journals \eTD 
\eTR
\bTR
	\bTD[nr=2, align={middle,lohi}] organizing \eTD 
	\bTD rules of conduct \eTD 
	\bTD assembly \eTD 
	\bTD horizontal \eTD 
	\bTD vertical \eTD
\eTR 
\bTR
\eTR
\eTABLE
}

\stopmode

\stoptext

\stopcomponent
