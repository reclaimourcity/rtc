\startcomponent component_MethodologyExperience
\product product_Thesis
\project project_MasterThesis

% definitions and macros
\environment envThesisAllEnvironments

\define[]\sectionExperience {\section[methodology::experience]{Notes about the São Paulo experience}}

\enablemode[draft]
\disablemode[tocLayout]

\starttext

\startmode[tocLayout]

\sectionExperience

This chapter gives a very brief overview about the time in São Paulo

\stopmode

\startmode[draft]

\sectionExperience

This thesis finally developed out of the experience made in São Paulo. I arrived in the city in May 2010 and aimed to stay until October in the same year. In June I decided to stay until November. 

My stay in the city was only determined in terms of "where to stay", "what would I like to do" and "how much time do I have". I had no real contacts to people nor groups, even though I had email contact in advance, mainly to grassroots and political groups and spaces such as \toMark{Indymedia São Paulo  (link fehlt)} or \toMark{Espaço Ay Carmela}. My intention was exactly that situation which meant for me the maximum possible freedom in order to decide how to proceed, to define the course of my research actions, which in turn also meant that I could first take as much time as possible to assimilate the city and let the city assimilate me. 

\spaceHalf

\inright{mobility and its versatile dimensions}
This situation meant for me first of all, before anything else, to practice how to use the city which is like nothing else I have seen and experienced before. I had to adopt basically everything that I knew about the flow of a city, the motion within a city. Things that are inherent in daily practice in German cities, had to be reconsidered. Transportation and the question of how to reach one particular place and how to return became suddenly a must when being on the run for longer trips through the city. The dense bus network had been a challenge from the beginning on, with its myriads of lines, stops, paths, its enormous city coverage and its range. Later, after loosing the fear of getting lost, its nodes became inherent to the daily adventure of travelling through the city.

Complementary to public transport, which also includes rings of trains and metros, is the probably most uncommon transportation vessel, the bike. Even though São Paulo's steep topography and its scale, the massive and aggressive  traffic, the daily traffic jams, heavy air pollution, especially on hot days, and the non-awareness and recklessness of car, bus and truck drivers which often seemed to just ignore and overlook cyclist, doesn't seem to be in favour of using a bike, but the bike was actually my main mean of transportation because it gave me a lot of flexibility and freedom \startAReminder hier fehlt noch etwas mehr zum fahrrad\stopAReminder

\spaceHalf

\inright{language and its versatile dimensions}
Concurrently, access to and contact with the city's spaces and its people primarily has been possible through language. Language became even more crucial for getting in contact with people in order to understand their narratives and explanations and without language to communicate, São Paulo would have remained locked for me since my arrival because I could not even ask for the way or the destination of a bus line, let alone communicate with people beyond small talk. Thus, my knowledge of Brazilian Portuguese simplified my arrival and the further assimilation of the city. Even though this sounds fine for now, my Portuguese has been rather \toMark{rugged} at the beginning, thus improvement was necessary. This necessity represents another reality of the initial period of assimilation where it has been important to examine my language skills and practice as much as possible. Afterwards, on the streets, my understanding of Portuguese was contested again due to the plurality of accents the people spoke. This plurality exists on the one hand because the people on the streets come from all over Brazil, as it is the case of the whole city \refMissing if one perceives the city as city of immigrants. For me, accents from the south has been much easier to grasp and understand than accents from the north and north-east. I had always difficulties to fully grasp the meaning of the spoken when it came from someone from the north, from \toMark{Pernambuco or Salvador}. On the other hand, the streets perceived as one space that forms the city, with a particular but very heterogeneous \toMark{Gemeinschaft}, uses and develops their own \toTranslate{giria (slang)}, just as the city's massive hip hop \toMark{Gemeinschaft} does or the any other \toMark{Gemeinschaft} that has constructs a particular identity. Thus slang is another aspects that made access to people difficult because slang is on the one hand difficult to understand and has to be decoded and assimilated and on the other hand determines who belongs to the \quote{family} and who is a foreigner.

\spaceHalf

A final aspect is the extend and scale of the city
\startAReminder hier fehlen noch die ausmasse der stadt\stopAReminder

The purpose to relate those impression is simply the fact that sometime it takes time to arrive somewhere, especially if the reality I came from and the one I arrived in are so diverging. In my case it took about two month, which had been important and necessary for me but resulted in no concrete or visible outcome for this thesis at first glance. The establishment and deepening of tied contacts on the base of amity and solidarity took another one or two month and suddenly the remaining time in the city had been drastically reduced. For me, the whole process was important and contribute to the thesis as much as the concretely conducted research action(s). I consider the whole period as enriching for me and my personal practice and definitely not as a mere obligation in order to gain a degree. 

\spaceHalf

\inright{time and temporal constraints}
Here, the concept of time plays a crucial role because sufficient time was the prevailing factor in order to even start accepting this thesis as something reasonable for me. Without the option to stay at least for 5 to 6 month and the same time to assemble everything, a stay abroad would not have been an option to me. What would have been the result then if I had restricted myself to the official period of 5 month for conducting research actions and writing the thesis? This very limited time frame would have made it very difficult for me to accept  the city as that space that forms my new temporal reality, which represents my life for the time to be and not just a space to rush through. Perhaps it would have been necessary to be just part of another existing (research) project while reproducing the dominant top-down hierarchies that western academic agents and their intended \quote{research objects} often represents. 

\spaceHalf

\inright{foreign expert and local adept}
Another contradiction produced by those hierarchies is the \startAReminder reversed concept of knowledge, where the one that lacks local knowledge but is embedded in an academic frame has more power or status than the one that is the local adept, who knows everything about his or her surroundings, but is maybe marginalized and lives at the outer margins of society. \stopAReminder. How can I then consider me as some kind of \quote{expert} that is able to judge, analyse and propose if I know nothing about the local situations, realities and struggles. Event 6 month in São Paulo gave me just a hint to the plurality of realities that this city of extremes produces. 

\spaceHalf

\inright{knowledge and power hierarchies}
The other questions is if I would like to act as such an \quote{expert} anyway, even with the proper knowledge. I personally would not exploit my adept knowledge \toMark{richtiger ausdruck?} in order to gain power nor would I accept the role of a scholar because this role is already loaded with hierarchies and symbols that conflict with my personal conviction that non-hierarchical/non-authoritative and genuine participation is an attitude applicable in all areas of practice, may they be political motivated or related to academic research or just belong to daily life. Thus for me, no invisible border exist between the discursively defined areas of private life, research or struggle. I consider them as at least overlapping, if not the same.

Therefore, my intention was and is to do research based on those and other attitudes (which will be exposed in a later section) and write this thesis because I consider it relevant \toMark{hier fehlt noch was}, \toMark{but not in separate spaces}, the research space separated from the private life space in São Paulo. I didn't define hours per day to enter the research space, nor hours to enter the daily life space. They certainly existed and exist but organically converge, diverge, overlap and are the same sometimes, dependent on the situation. When I was on the streets, I often met people whose daily reality I participated in, even if I went to \toMark{AyCarmela} or simply roamed the streets in order to absorb the city. Clearly, the separation of spaces exist because at the end, I didn't life together with the people on the streets, we \quote{just} shared plenty of time together. \toMark{hier fehlt noch was} 

Having said this, maybe some of the factors that mainly impacted the course of this thesis are clearer now, thus let's see how the red line through it can be tied.

\stopmode

\stoptext

\stopcomponent
