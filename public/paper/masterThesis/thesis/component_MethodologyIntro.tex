\startcomponent component_MethodologyIntro
\product product_Thesis
\project project_MasterThesis

% definitions and macros
\environment envThesisAllEnvironments

\define[]\chapterMethodology {\chapter[methodology]{Methodology}}

\enablemode[draft]
\disablemode[tocLayout]

\starttext

\startmode[tocLayout]
\chapterMethodology 

The methodology chapter. Describes the thesis methodology in terms of theoretical and practical work.

\stopmode

\startmode[draft]
\chapterMethodology

\startAReminder 
die referenz auf empirie raus weil ich keine empirische arbeit schreibe
\stopAReminder

\inright{aims of this chapter}
The aim of a Methodology chapter should be to transparently describe the context this thesis is embedded in and the corresponding procedures that have been chosen in order to articulate and fulfil the thesis' aims. Transparency in order to expose the reasons why (research) approaches, methods and tools have been chosen. Transparency in order to permit the appraisal of my own  subjectivity in narrations, observations and actions. 

In addition, or better complementary, from my point of view, is the question on which base the argumentation for selected approaches and the thesis aims are grounded, how the underlying motivation and self-conception can be articulated and how this articulation leads to approaches and the definition of possible objectives and research actions. 

\inright{subjectivity and the influence of my personal conviction}
The {\em Methodology} chapter shall therefore not only determine theoretical and research approaches, methods and tools, objectives or research actions, but shall primarily address my general and particular interest in the specific topic, the context in which this thesis \toMark{has been and is going to be} embedded in, i.e. driven by theoretical consideration and/or narrations and observations, and the definition of the thesis' self-conception, all of which are the fractions that I would like to take into account in order argue for concrete procedures and means of action. Thus, for me, a conceptualization will help to construct the frame in which this thesis can be embedded in, based on acceptable ground and a well defined self conception, which can partly be perceived as subjective product of my personal conviction which is \toMark{rather libertarian and autonomous than academic, neo-liberal or institutional}. I also perceive this thesis as political because I define this research as inherently political. Political due to the fact that I consider this thesis as one mean that complements the struggle of the people that try to open up new political spaces to navigate in. \toMark{Political as well because the thesis research actions aim to be non-hierarchical (even though this is not achievable under the current circumstances), which can also be understood as a critique of the current status quo of any social relation.} Further depictions of these characteristics follow in the next sections of this chapter.

\inright{describe different realities based on the narrators realities}
Therefore, I personally consider important to make my personal attitude transparent in order assess in how far this attitude affected and affects the course of this thesis. This thesis is not aimed to express objective narratives because objectivity is for me hard to achieve when reproducing social narratives and observations, \toMark{which lay at the core of this thesis}. The concept of objectivity, which from my perspective means (absolute) neutrality, devoid of symbols and interpretations, thus the plain and \quote{real} nature of the object, is not realizable and probably not seminal for me either, because I cannot and don't want to disconnect myself from my attitude which clearly influenced at least the realization of my research action(s) in São Paulo. 

The observations and actions that form this thesis are those that I volunteered to perform and experience or that just happened by incident, but never through external \quote{pressure} or heteronomously determined. I could have chosen another frame, an existing academic or NGO project on the same topic, where I probably would have met the same people and visited the same places, but which perhaps would have resulted in a totally different outcome, based on other points of views and attitudes. Is my reality then more valid than the others or vice versa? I think not, both have their legitimacy, they are probably motivated differently and by that narrate different stories, describe different realities based on the narrators realities, even if their purpose is the same. In the words of Schrödinger I would then say

\inright{"The Fundamental Idea of Wave Mechanics", Nobel lecture, (12 December 1933)}
\startCitation
We cannot, however, manage to make do with such old, familiar, and seemingly indispensable terms as "real" or "only possible"; we are never in a position to say what really is or what really happens, but we can only say what will be observed in any concrete individual case. Will we have to be permanently satisfied with this...? On principle, yes. On principle, there is nothing new in the postulate that in the end exact science should aim at nothing more than the description of what can really be observed. The question is only whether from now on we shall have to refrain from tying description to a clear hypothesis about the real nature of the world. There are many who wish to pronounce such abdication even today. But I believe that this means making things a little too easy for oneself. \aQuote{Schrödinger}{1933}
\stopCitation

By not aiming to reproduce narratives in an objective way I don't mean to dismiss the idea of neutrality. The content of the thesis aims to reproduce those narratives as close as possible to the positions, ideas and words of those that shared them.

\stopmode

\stoptext

\stopcomponent
