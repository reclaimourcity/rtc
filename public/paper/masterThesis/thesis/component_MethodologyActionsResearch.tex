\startcomponent component_MethodologyActionsResearch
\product product_Thesis
\project project_MasterThesis

% definitions and macros
\environment envThesisAllEnvironments

\define[]\subsectionActionsResearch {\subsection[methododology::actions::research]{Action//Activist//Research}}

%\enablemode[draft]
%\disablemode[tocLayout]

\starttext

\startmode[tocLayout]

\subsectionActionsResearch

This section describes the concept of Action Research applied as theoretical and practical framework of this thesis.

\stopmode

\startmode[draft]

\subsectionActionsResearch

\startKeywords
action, activist, action research, activist research, framework, participatory, transformation
\stopKeywords

I determined \abbrNew{Action Research}{AR} as the overall methodological framework for this thesis. 

In a sense, the word \abbrFull{AR} by itself seems already matching the way I intend to do research. This may be a relatively weak justification but nonetheless it nearly hits the mark. \abbr{AR} is research that emerges from within the movement, from within a situation, having an active role in the movement, in the situation and in its intended transformation \aQuoteB{}{}{}. 

\aKeyword{Workers Inquiriy}, the subjective workers view on and analysis of the situation of being exploited and the knowledge about their needs and the necessary transformation according to those needs, is one instance of \abbrFull{AR} that emerged from within a movement, conducted by the activists themselves. In Italy for example, visible in the movement of the \toTranslate{Operaismo} that originated from the debates in the journal \quote{Quaderni Rossi} in the early sixties of the 20th century \aQuoteW{Wildcat}{1964} \aLinkNewF{http://www.wildcat-www.de/wildcat/64/w64opera.htm} and which lead to the workers struggle at the end of this decade.

\spaceHalf

\startCitation
[...] sie waren durch ihre Untersuchungen auf kommende Kämpfe vorbereitet, hatten die Probleme innerhalb der Fabrik analysiert, hatten die Arbeiterdiskussion verfolgt, um die Arbeiterforderungen auf die Flugblätter schreiben zu können und auf Versammlungen als politische Linie durchzusetzen. Sie hatten gelernt, »daß es schon Kämpfe gibt, bevor sie offen ausbrechen \aQuoteW{Wildcat}{1964}
\stopCitation

\spaceHalf

But not just \aKeyword{Workers Inquiriy} can , if one looks at social movement and collectives in general, one can see that all of them articulate their own knowledge and analysis of their reality in order to take the next step to overcome it. 

Lets take 




One justification of AR is that one must be there on the local level, one must see and experience the local space in order to understand the situation (see Tsing, Friction).

Even though \abbr{AR} may seem 

\abbr{AR} is said to be favourable for research that intends to work for a social transformation of an inequitable situation \aQuoteB{}{}{}. 

This may not be the only area of application because \abbr{AR} is not bounded to a specific attitude but rather represent a set of tools that can be used for basically any purpose, thus \abbr{AR} is applied in corporations \aQuoteB{}{}{}, as well as in social movements research \aQuoteB{}{}{} or by activists themselves \aQuoteW{Periferies Urbanes}{2010}\aLinkNewF{http://periferiesurbanes.org/?p=165&lang=en}.

Hence, \abbrFull{AR} is a organic framework consisting of various concrete methodologies, that depend on the context it is applied in. 

One attribute of \abbr{AR} is its participative research approach \aQuoteB{}{}{}. 



\stopmode

\stoptext

\stopcomponent
