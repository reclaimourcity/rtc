\startcomponent component_Methodology_WhoAmI_NonAuthoritarian
\product product_Thesis
\project project_MasterThesis

% definitions and macros
\environment envThesisAllEnvironments

\define[]\subsectionNonAuthoritarian{\subsection[methodology_whoami_nonauthoritarian]{Non-[Authoritarian//Hierarchical] Attitude}}

\define[]\subjectNonHierarchical{\subject{Non-Hierarchical Praxis}}

\define[]\subjectNonAuthoritarian{\subject{Non-Authoritarian Praxis}}

\starttext

\startmode[tocLayout]

\subsectionNonAuthoritarian

This section shall explain why non-authoritarian and non-hierarchical attitude is a prerequisite when conducting  research.

\subjectNonHierarchical

My justification for non-hierarchical praxis

\subjectNonAuthoritarian

My justification for non-authoritarian praxis

\stopmode

\startmode[draft]

\subsectionNonAuthoritarian

\startKeywords
non-authoritative, non-hierarchic , motivation, emancipatory, self-determination
\stopKeywords

\inright{non-authoritarian and non-hierarchical pervasion}
\aKeyword[Non-authoritarian]{attitude+non-authoritarian} and \aKeyword[non-hierarchical]{attitude+non-hierarchical} attitude shall pervade the ground on which this research shall be elaborated. This aspect is fundamentally important for me due to the fact that this concepts represent certain attitudes that I try to follow in my personal practice but also due to the fact that this thesis is partly based on information provided freely by people that are struggling for a social transformation of their marginalized reality. I consider my research action(s) to be part of this social and political struggle and therefore argue that my research praxis is supposed to follow the attitudes of my personal praxis. This praxis is also reflected in my approach to research that is mainly determined by \abbrNew{Action Research}{AR}\aKeywordInVi{action research} as I will explain later in more detail \gotoTextMark[later on]{methodology_whaoami_actionresearch}.

From my point of view, non-authoritarian and non-hierarchical attitude is strongly interdependent. 

\subjectNonHierarchical

The fact that research is often embedded in an existing academic framework already represents an implicit \aKeyword[hierarchy]{research+hierarchies} which could, and often lead(s) to situations where research agents \bracket{scholars and research projects for instance} primarily follow their own agenda and logic, \toMark{in terms of participation, constraints and benefits, for them, the project or the academic circle.} \reference[methodology_whoami_nonauthoritarian_benefits]{}

One example is the \refMissingSrc{often heard question}, that has also been directed to me in São Paulo, in how far academic research with marginalized people really supports the struggle of the \quote{participants} whose role is basically limited to the provision of information utilized by the scholar to write his or her thesis and through whose completion he or she gains a degree that offers better possibilities on the \bracket{academic} job market and career outlooks while the participants still do not see any improvement of their situation. 

One could argue that through a the scholars' more powerful position, he or she can direct more \bracket{institutional} power to provoke those transformation that research was lacking but for me personally this is no argument. Gaining a better position for instance has a more or less immediate effect on ones own life while trying to realize social change through institutions takes a long time with unpredictable outcome, if there is an outcome that leads to proficient transformation at all. One can question the role of institutions as \quote{change-makers} and agents of transformation in general if one looks at decades of aid and development projects that did not lead to any large scale and sustainable transformation of social inequalities existing all over the world \refMissing. 

Therefore \inright{participation exerted as tyranny or emancipatory self-determination} I want to question the terms under which \aKeyword[participation]{research+participation} in \bracket{research} actions is defined and exercised. Is it exerted as a simple justification for the realization of inherently unjust research actions or \bracket{development} projects \refMissingSrc{tyranny to transformation} or is it exerted as an non-hierarchical and emancipatory approach to exercise self-determination \refMissingSrc{tyranny to transformation} in research but also in struggle? The question of \aKeyword[participation]{participation} is relevant for me in the context of for social \aKeyword[struggle]{participation+struggle} but also for \aKeyword[research]{participation+research} in general because I would like leave the academic space to enter the streets and join the people to realize this thesis research from a different \aKeyword{standpoint}, from the \aKeyword[standpoint of the streets ]{standpoint+streets} and its people.

Coming back to the notion of non-hierarchical praxis I perceive another implicit power hierarchy inherently embedded in the academic framework of my research due to the fact that I can go abroad, possible through scholarships, a situation barely realisable by those that shall participate in or which are addressed by particular research actions, hence, here, the status as foreign research agent automatically implies a difference in status between the research agent itself and those that shall participate in the agents actions \bracket{if this is supposed to happen at all}. 

This situation can be described plastically with a quote of a street dweller \gotoTextMark[I met and talked with]{narratinginquiries_saopaulodiaries_sometalks::hecticguy} in a small and shady street in the centre of São Paulo:

\startCitation
Tell me, what does a guy from the first world do here in the third world? Why are you here? Don't you have problems to solve and analyse in your country? \aQuoteP{2010}
\stopCitation

\subjectNonAuthoritarian

\inright{non-authoritarian actions in order to conduct research, neither oppressive nor seductive}
Non-authoritarian attitude is the practice I am affiliated with. Here, a contradiction could arise because I would like to experience those situations and conditions that would be incorporated for a however defined research purpose. Therefore I would like to get in touch with the people and become active in their struggle, because I feel solidary with their struggle, because I would like to realize research actions as part of this struggle. In order to do this, to get in touch, to experience, I would never impose acts of authoritative actions upon those that provide information, that share their trust with me, neither through implicit or explicit actions, nor in oppressive or seductive ways.

Apart from the question of access, thus access to the people, their reality, their struggle, the question of access to the research' outcome is related to a non-hierarchically attitude as well. 

\spaceHalf

\inright{open access instead of exclusion from knowledge}
Due to the fact that my thesis is composed of \gotoTextMark[personal narrations]{narratinginquiries_intro} that is pass over into \gotoTextMark[theorizing]{theorizing_intro} drawn on theory conceptualized in books, journals or available through the internet, I would also like to question the way this sets of information are \bracket{or have to be} made accessible by academia. 

Little is openly \bracket{thus freely accessible} published in academic circles due to an elitist attitude and the \aKeyword[commodification]{knowledge+commodification} of knowledge and information, where knowledge, even though elaborated and produced in public institutions or based on peoples knowledge, remains behind impermeable walls, remains solely accessible to those that have selected and appropriated that knowledge or which have the necessary \bracket{monetary} resources or the necessary status in order to do so. 

This situation describes another facet of \aKeyword[purpose]{research+purpose} and \aKeyword[demand]{research+demand} on my research action(s), here as demands of \aKeyword[free distribution]{objectives+free distribution} and \aKeyword[open accessibility]{objectives+open access} to the thesis outcome.

\spaceHalf

\inright{research actions conducted in existing institutional frameworks or self-determinately organized}
Another notion of rather practical nature is the time frame reserved for thesis writing and research action(s). The initially contemplated and official period for research actions and theoretical examination had to be 5 month, 2 to 3 reserved for empirical research abroad, the remaining time reserved for theorizing and writing of the final thesis.

Now, I exceeded those specifications due to the fact that research action(s) in São Paulo already lasted 6 month, while writing this thesis took another 6 month. Thus again I return to the question of \aKeyword[constraints]{research+constraints} and \aKeyword[benefits]{research+benefits} the academic research agent is \gotoTextMark[accepting and seeking]{methodology_whoami_nonauthoritarian_benefits}.

Looking back at the time that has been passed since I arrived in São Paulo, I have realized that if I had followed the \aKeyword[strict time setting]{restrictions+time} imposed on me, I wouldn't have had the time to reflect on and adjust to the situation I entered. I would not have to time for self-organisation of my research action(s), to get in touch with the people nor to build relations among us. This would then probably have led to a work that just followed the logic of acquiring an academic title or developing a technical fix \refMissing{tyranny - participation} while leaving context, approach, praxis and effect of conducted research actions rather insignificant, just as necessary means to the anticipated end.

\startARemark{weiß noch nicht ob ich den teil so beibehalte}

The notion of lack or abundance of time is directly related to conditions under which research actions are realized and how research is organized: either embedded in existing institutional frameworks, following their inherent constraints and logic of imposing predetermined objectives from above upon others, or self-determined and emancipatory, adopting the particular standpoint from below of those that shape and participate. 
\stopARemark

\spaceHalf

The concept of \aKeyword{participation} is visible in this thesis title and theorizing but also represents the \refMissingSrc{basic approach} of research realization in São Paulo. Therefore, in depth theorizing about participation on an \refMissingSrc{abstract} and \refMissingSrc{concrete} level is one \aKeyword[objective]{objectives+participation} of this thesis.

\startARemark{hier könnte nochmal eine übersicht hin} , welche themen von der antiathoritären frage durchzogen sind, als letzter überblick bevor wissenschaftliche herangehensweisen betrachtet werden.
\stopARemark

\startARemark{noch wichtig?}
\inright{do fundamental questions correlate with motivation and demands?}
This has consequences for my research actions and rises more fundamental questions: 

\spaceHalf

\placefigure{}
{
\textBoxedRoundMax[0.3]{ What is the purpose of research? How do I conduct research? What is my role? How is my role perceived? How do I approach people? What do I want from them? What do they want from me? Can we find common ground to collaborate coequally?}
}

\inright{open access}
\textBoxedRoundMax{For whom? To whom? By whom?}
\stopARemark


\inright{overview of mentioned anti-authoritarian / non-hierarchical themes.}
\startARemark{hier kann noch ne word map rein}
\stopARemark

\showImperfection

\stopmode

\stoptext

\stopcomponent
