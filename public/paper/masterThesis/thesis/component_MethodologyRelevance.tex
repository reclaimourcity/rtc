\startcomponent component_MethodologyRelevance
\product product_Thesis
\project project_MasterThesis

% definitions and macros
\environment envThesisAllEnvironments

\define[]\sectionRelevance{\section[methodology::relevance]{What does relevance mean?}}

\starttext

\startmode[tocLayout]

\sectionRelevance

This section argues about the relevance of \quote{relevance in research} and contests its usual meaning arguing that research can be relevant even if it is not a breaking new topic but that it is relevant if it is relevant for the people that collectively conduct research. Here, four questions regarding relevance are formulated which are going to be tackled in the following chapters.

\stopmode

\startmode[draft]

\sectionRelevance

\inright{ relevance of research and social struggle}
In order to begin, the question of \quote{relevance} shall be contested as well as the question of \quote{feasibility}, the form of publication and access of knowledge or the frame in which this thesis has been written. From my point of view, those points are all interrelated and interdependent. 

In order to construct viable methodology approaches and argumentation lines, I would like to take a set of questions into account, each of them related to the notion of \quote{relevance}. Those questions haven been posed by Don Mitchell in {\em What Makes Justice Spatial? What Makes Spaces Just?} \aQuoteB{Brown et al}{2007}{8} as follows:

\startCitation
\inright
{ 
\startitemize[packed, fit]
\sym{} how becomes?
\sym{} why should?
\sym{} what goals?
\sym{} for whom? 
\stopitemize
}

Mitchell’s work goes beyond calls for social or political \quote{relevance} in research and practice by reminding us that determinations of relevance always unfold in a historical and professional matrix. With Lynn Staeheli, he has written that calls for relevance in professional practice \quote{cannot be separated from questions about why research should be relevant, how research becomes relevant, the goals of research (including political goals), and the intended audiences and beneficiaries of research} (Staeheli and Mitchell 2005: 357). Those questions of why, how, what and for whom also lie at the heart of any movement for justice.\stopCitation

Those questions may serve as a solid base for further explorations in order to determine the motivation(s) and realization of this thesis. The answer to them may affect the way how individual elements of this thesis are going to be shaped, but may also determine the thesis' abstract layers in terms of applied theories, objectives or the utilized language.

\spaceHalf
\startQuestionBack 
\startitemize[3]
\item Why should research be relevant?
\item How does research becomes relevant?
\item What are the goals of research?
\item Who is the audience and who are the beneficiaries?
\stopitemize
\stopQuestionBack 

\stopmode

\stoptext

\stopcomponent
