\startcomponent  component_Methodology_WhatDoIWant
\product product_Thesis
\project project_MasterThesis

% definitions and macros
\environment envThesisAllEnvironments

\define[]\sectionWhatDoIWant {\section[methodology::whatdoiwant]{What do I want?}}

\starttext

\startmode[tocLayout]

\sectionWhatDoIWant

The definition of concrete and general objectives.

\stopmode

\startmode[draft]

\sectionWhatDoIWant

By giving \toMark{hints} to answers to the question \aKeyword[What do I want?]{thesis methodology + what do I want?} I would like to outline what I imagine the intention of this thesis could be, what its effect could be, what options it could illustrate. Different \aKeyword{objectives} have been already mentioned when I tried to determine \aKeyword[Who am I?]{thesis methodology + who am I?} and \aKeyword[What should I do?]{thesis methodology + what should I do?}. Those objectives represent different scopes that I would like to gaze when asking \aKeyword[What do I want?]{thesis methodology + what do I want?}. 

\spaceHalf

\inright{guiding objectives}
On the one hand, certain objectives are not linked to the results of theorizing, thus they are not directly linked to the written result of this thesis. I perceive those objectives as immanent features of the process of research action(s). By realizing research action(s) based on the ground I prepared while asking \goto{Who am I?}[methodology::whoami], I realize those objectives, or not. I perceive those objectives are \aKeyword[guiding objectives]{objectives+guiding} for me and my \bracket[research] praxis, that I could apply in other contexts' as well because they are to a large extend not directly related to the situation in São Paulo, even though their relevance became apparent to me only when I stayed in São Paulo, together with my people. 

Looking back at the time in the city and as already mentioned at \refMissingSrc{several occasions}, these guiding objectives represent an kind of optimal achievement. To reach all of them or even just a fraction is difficult because they depend just to a certain extend on my own praxis, will and attitude. Important factors such as time or the lack thereof for instance, are hard to influence because they may be imposed onto me by larger structural factors, the limited time frame this thesis has to be completed or the long lasting process of building relations based on friendship, solidarity and trust. 

Therefore I consider the \aKeyword[guiding objectives]{objectives+guiding} of my thesis as framework I try to act upon accordingly but I also know that I cannot and will not achieve all of them.

\spaceHalf

\inright{thesis objectives}
The other type of objectives I would like to mention are those that are directly linked to the thesis theorizing, its written form and the outlook beyond thesis completion. 

By writing this thesis from an activist perspective, by not being an academic observer but also by not being fully involved \missingRefSrc{as mentioned elsewhere}, I feel a certain dilemma. I do not intend to propose what to do next from an academic perspective but I also cannot claim transformation as radical and profound as the people do, because I am still an alien in a way, even though not a complete stranger, but now disconnected from the city and its people, especially while writing this lines, back in Germany. I think I would feel better if I had formulated this thoughts after some years of living and intense experiencing, as continuation of what I started to experience, which is an option, but not the case at this very moment. 

I base this thesis theorizing on street experience, on street and movement theorizing, and I consider this thesis as part of the struggle of the people. In the midst of my dilemma I still think that this thesis could contribute to a gain a different perspective of the lived urban space we are all part of in our \bracket{everyday, academic, activist, marginalized, privileged, criminal} life, a different perspective on the processes that produce and shape our lived space, what we are doing to cope with them and what we are doing to transform and possess those means of production of our lived space, the city.

\stopmode

\stoptext

\stopcomponent
