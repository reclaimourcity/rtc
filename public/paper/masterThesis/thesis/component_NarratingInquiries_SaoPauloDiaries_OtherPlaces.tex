\startcomponent component_NarratingInquiries_SaoPauloDiaries_OtherPlaces
\product product_Thesis
\project project_MasterThesis

% definitions and macros
\environment envThesisAllEnvironments
\environment envCfgThesisImages

\define[]\subsectionSaoPauloDiariesOtherPlaces{\subsection[narratinginquiries_saopaulodiaries_otherplaces]{Other//Places}}

\define[]\subjectSaoPauloDiariesMohino{\subject{Vida//Mohino}}
\define[]\subjectSaoPauloDiariesGrajau{\subject{Zona//Sul}}
\starttext

\startmode[tocLayout]
\subsectionSaoPauloDiariesOtherPlaces
About other places
\subjectSaoPauloDiariesMohino
About a cultural manifestation at Mohino.
\stopmode

\startmode[draft]
\subsectionSaoPauloDiariesOtherPlaces
At the following pages I would like to trace a place I primarily visited out of curiosity and solidarity, where I did not know anybody but that is a manifestation of urban struggles as well. Tracings are left this time at \locMohino.

\subjectSaoPauloDiariesMohino

\aQuoteInText{A vida é um moinho} \aQuoteInTextA{Farias} took place yesterday in solidarity with the inhabitants of \aKeyword[Favela do Moinho]{streets+places+settlements+moinho}, right in the centre of São Paulo, in Bom Retiro between the stations Barra Funda and \Luz.

\imgFlyerMoinho

The day was organized by various hip hop and graffiti crews in solidarity with the \locMohino community in order to collect food and clothes for the inhabitants and to raise awareness about serious issues \Mohinho has to face, but also to show the beautiful sides of the place. The chosen scenario was impressing, on the roof of the old (and now squated) industrial building, right in the middle of the favela. The buildings has been a compound of the previously existing Fábrica Moinho Matarazzo. More information about the history of the Matarazzo family and its industrial complex in Brazil can be found in the dossier \aQuoteInTextT{O Grupo Matarazzo nas terras do município}\aQuote{Raízes}{2002}.


\spaceHalf

\inright{entering \Mohinho}
I entered \Mohinho below the bridge, by crossing the rails, where an old guy was sitting and observing the train movement, to guarantee a safe traversal. Already inside, still below the bridge, foot, rice, water, vegetables have been piled up on several tables, where I put another sack of rice that I brought. From there, another young guy brought us to the  abandoned but still inhabited industrial complex right in the middle of \Mohinho, massively build of concrete, probably 5 storey high, on its top floor the solidarity hip-hop event taking place.

During the whole day a massive crowd of kids of all ages have been around. Especially attractive to them: everyone who threw them in the air and played with them; everyone with a kind of digital camera: really nice that kids in the age of 5 to 6 years (estimated) already know the general functioning of a digital camera, or learn it hyper fast – which button to press in order to trigger the camera – which symbol indicates that no space is left for taking pictures – how to skip through the photos – and so on; everyone that painted and made graffiti; everyone who made hip hop: one young boy from \Mohinho showed his breakdance skills – a youg girl MC from \Mohinho sung together with two of the invited MC’s.

Several people showed their poetical skills and work by spreading their ready and improvised poems over the roof and by singing acapellas. From the hip hop point of view there have been several artists around, mostly male but also one female mc with her DJane (sadly don’t remember the name).

She made a good statement about the precarious water supply conditions in the favela and was more reflective than the often heard (also this day) abstract calls for resistance. Her general statement was that it is a scandal that there is no water supply in \Mohinho but that it is an even greater scandal that the supply that existed before was cut off due to orders by the cities politics who deliberately accepted the worsening of an already precarious housing and living condition, especially if one considers that several hundred families with kids life here without access to common services such as water, sanitation and electricity.

A collection of super interesting impressions and interview fragments with people from \Mohinho (and from Ocupacão Prestes Maia) can be found in a text called \aQuoteInTextT{Periferia é Periferia em qualquer lugar}\aQuoteB{Sampaio}{2007}{59-74}.

Besides the lack of sufficient water supply (which consisted of only illegally connected lines before their cut off in 2009) there have been (still exist?) many restrictions that are imposed on the inhabitants of Moinho. 

Two examples: the inhabitants didn’t have a legal postal address thus neighbours receive post for some of them \aQuoteB{Sampaio}{2007}{64}; electricity supply exist only through illegally connected lines due to the fact that cities electricity supply Eletropaulo did not install a electricity net. It didn’t consider the inhabitants of \Mohinho as the legal owners of that land where the favela is located \aQuoteB{Sampaio}{2007}{64}.

The text also draws a more nuanced picture of the different communities and individuals within the community of \Mohinho, like the people that life in the fabric building, the people that life in self-constructed buildings, the ones close to the train tracks or the ones under the bridge at the entrance.

What remains is the impression that culture (here: Hip Hop culture) provides one way to educate people (those that life outside) about a non acceptable situation (here: lack of water supply and electricity) by inviting them to the place they would not have access to otherwise and by doing so, learn a bit about the local situation. In fact, I would have been great if people from \Mohinho also talked about good and bad things they perceive in their environment, or that flyers or handouts had been distributed with more information...

The intentional denial of water (and electricity) supply also shows that precarious living conditions are often imposed onto the people by simply denying them the right for a decent life and access to the city.

The day has been covered by two hip hop magazines as well, \aLinkNew[Central Hip-Hop /BF]{http://bit.ly/ashLTI} and \aLinkNew[Portal Rap Nacional]{http://www.rapnacional.com.br/2010/index.php/noticias/rap-marca-presenca-na-favela-do-moinho/}

\addReference
{
Central Hip-Hop /BF, 2010. “A Vida é Um Moinho” mobiliza o Hip-Hop social. {\em Central Hip-Hop - 2011 - Bocada Forte - Desde 1999}. Available at: \goto{\hyphenatedurl{http://centralhiphop.uol.com.br/site/?url=materias_detalhes.php\&id=1065}} [url(http://bit.ly/ashLTI)] [Accessed September 13, 2011].
}

\addReference
{
Farias, P., 2010. Hip-Hop marca presença na Favela do Moinho. {\em Portal Rap Nacional 2011}. Available at: \goto{\hyphenatedurl{http://www.rapnacional.com.br/2010/index.php/noticias/rap-marca-presenca-na-favela-do-moinho/}} [url(http://www.rapnacional.com.br/2010/index.php/noticias/rap-marca-presenca-na-favela-do-moinho/)] [Accessed September 13, 2011].
}

\addReference
{
Raízes, 2002. {\em O Grupo Matarazzo nas terras do município}, Available at: \goto{\hyphenatedurl{http://www.fpm.org.br/raizes/edicao25/pag5a22.pdf}} [url(http://www.fpm.org.br/raizes/edicao25/pag5a22.pdf)] [Accessed August 2, 2010].
}

\addReference
{
Sampaio, R., 2007. {\em Periferia é Periferia em Qualquer Lugar}, Universidade de São Paulo. Available at: \goto{\hyphenatedurl{http://www.nossasaopaulo.org.br/portal/files/PeriferiaEPeriferiaEmQualquerLugar.pdf}} [url(http://www.nossasaopaulo.org.br/portal/files/PeriferiaEPeriferiaEmQualquerLugar.pdf)] [Accessed August 2, 2010].
}


%---------------------------------------------------
% gets only displayed in unfinished mode

\showImperfection

%---------------------------------------------------
\stopmode

\stoptext

\stopcomponent
