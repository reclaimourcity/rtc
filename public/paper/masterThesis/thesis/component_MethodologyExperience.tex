\startcomponent component_MethodologyExperience
\product product_Thesis
\project project_MasterThesis

% definitions and macros
\environment envThesisAllEnvironments

\define[]\sectionExperience {\section[methodology::experience]{Notes about the São Paulo experience}}

\starttext

\startmode[tocLayout]

\sectionExperience

This chapter gives a very brief overview about the time in São Paulo

\stopmode

\startmode[draft]

\sectionExperience

This thesis finally developed out of the experience made in São Paulo. I arrived in the city in May 2010 and aimed to stay until October in the same year. In June, I decided to stay until November. The impressions and experiences gained during that period are the subject of the following \aKeyword[synopsis]{methodology+synopsis} which is aimed to transparently reflect the circumstances I dived into and had to deal with and which affected the way I acted during my stay.

My time in the city was only determined in terms of \quote{where to stay}, \quote{what would I like to do} and \quote{how much time do I have}. I had no real contacts to people nor groups, even though I had email contact in advance, mainly to grassroots and political groups, collectives and spaces such as \aKeyword[Indymedia São Paulo]{collectives+indymedia são paulo} \aLinkNewF{http://www.midiaindependente.org} or the self-organized space called \aKeyword[Espaço Ay Carmela]{spaces+espaço ay carmela}\aLinkNewF{http://ay-carmela.birosca.org}. I intended exactly that situation which meant for me the maximum possible freedom in order to decide how to proceed, to define the course of my research actions, which in turn also meant that I could first take as much time as possible to assimilate the city and let the city assimilate me. 

Finally, I got in touch with a loose group of people from the streets which whom I spend two to three month during which I got \bracket{partly} involved in their realities, struggles and actions. This thesis is thus a narration about this time and actions.

\startARemark
Even though I intended maximum freedom, I already had an concept for a research action in mind and on paper when I came to the city. This idea was related to the usage of mobile communication for grassroots organization but finally rendered impossible to realize due to those constraints that I lay out in the following sections. Thus, the final topic and direction of this thesis is differs almost completely from the one I had in mind when I decided to go to São Paulo. The detailed process of this transformation is documented on the thesis' blog \aLinkNewF{http://rtc.noblogs.org}, which has been set up for the documentation of the research process, transparency purposes and in order to guarantee free access to the assembled information. I will not lay out the transformation in detail at this point but would like to refer everyone interested to the documentation available online. 
\stopARemark

\inright{mobility and its versatile dimensions}
Getting acquainted with the city first of all meant for me, before anything else, practising how to use the city which is like nothing else I have seen and experienced before. I had to adopt basically everything that I knew about the flow of a city, the motion within a city. Things that are inherent in daily practice in German cities, had to be reconsidered. Transportation and the question of how to reach one particular place and how to return became suddenly a must when being on the run for longer trips through the city. The dense bus network had been a challenge from the beginning on, with its myriads of lines, stops, paths, its enormous city coverage and its range. Later, after loosing the fear of getting lost, its nodes became inherent to the daily adventure of travelling through the city.

Complementary to public transport, which also includes rings of trains and metros \refMissingSrc{some map}, is the apparently most uncommon transportation vessel for this environment, the bike. Even though São Paulo's steep topography and its scale, the massive and aggressive traffic, the daily traffic jams, heavy air pollution especially on hot days, and the non-awareness and recklessness of car, bus and truck drivers which often seemed to just ignore and overlook cyclist, doesn't seem to be the favourite environment for using a bike. However, the bike actually became my favourite means of transportation because it gave me a lot of flexibility and freedom. It also enabled me to arrive at places that would have been much more difficult to reach solely by public transport. I also shared the bike from time to time with some of the people I stayed with, thus from my point of view it was not just a means of transportation but also a means of communication and a shared resource among us.

\spaceHalf

\inright{space and scale}
Leaving the concrete street level and zooming out to the metropolitan scale, São Paulo's dimension is just too extensive for me to grasp completely. My sphere of action was therefore mainly delimited by several districts \refMissing starting from Pompeia \refMissing and Barra Funda \refMissing in the western zone of central São Paulo to Sé \refMissing and República  \refMissing in the center and further on to Bŕas \refMissing  and Mooca \refMissing in the east \refMissing. 

\startAReminder
hier können noch karten eingefügt werden
\stopAReminder

\spaceHalf

\inright{language and its versatile dimensions}
Concurrently, access to and contact with the city's spaces has been possible through language. Language became even more crucial for getting in contact with people, in order to understand their narratives and explanations and without language to communicate, São Paulo would have remained locked for me since my arrival because I could not even ask for the way or the destination of a bus line, let alone communicate with people beyond small talk. Thus, my knowledge of Brazilian Portuguese facilitated my arrival and the further assimilation of the city. Even though this sounds convenient, my Portuguese has been rather \toMark{rugged} at the beginning, thus improvement was necessary. This necessity represents another reality of the initial period of assimilation where it has been important to examine my language skills and practice as much as possible. 

Afterwards, on the streets, my understanding of Portuguese was contested again due to the plurality of accents the people spoke. This plurality exists on the one hand because the people I met on the streets came from all over Brazil, as it can be generalized for the whole city \refMissingSrc{são paulo immigration city} if one perceives the city as city of immigrants. For me, accents from the south of Brazil has been much easier to grasp and understand than accents from the north and north-east. I had always difficulties to fully grasp the meaning of the spoken word when people came from \toMark{Pernambuco or Salvador}, for example. Their \toTranslate{giria} \bracket{which means slang or parlance in Portuguese} has often been too fast and fuzzy for me, thus I missed a lot of words and therefore the sense of the spoken during such occasions.

On the other hand, if one perceives the streets as one of the spaces that forms the city, inhabited and shaped by a particular but very heterogeneous \aKeyword{gemeinschaft}, a particular \aKeyword{giria} has been developed in that space and is used by those that shape and indwell it, just as it is the case for São Paulos's massive hip hop \aKeyword{gemeinschaft} or any other \aKeyword{gemeinschaft} that is constructed around a particular identity and/or which constructs that identity. In this sense, \quote{slang} is another aspect that impedes approaching people from that \aKeyword{gemeinschaft} because it is difficult to understand and contains unknown habits, symbols, and expressions and therefore a particular local knowledge is necessary for its decoding and assimilation. Slang also determines who belongs to the \quote{family} \bracket{of street people, for example} and who does not belong to, thus is an \quote{outsider}. 

\spaceHalf

\inright{time and temporal constraints}
Putting those aforementioned aspects together, one factor that pervades them all is time. Time is necessary for gaining the local knowledge I previously described as personally lacking and which I consider necessary in order to start realizing (research) actions(s) based on reasonable ground. As it probably can be seen from those situative descriptions above, plenty of time was aleady necessary just to cope with the numerous overwhelming and unfamiliar situations.

Thus, the concept of time plays a crucial role because sufficient \bracket{or still the lack of?} time was the prevailing factor in order to even start accepting this thesis as something reasonable for me. Without the option to stay at least for 5 to 6 month plus the same amount of time to assemble everything, a stay abroad would not have been an option to me and a plain theoretical work would have been the most reasonable alternative. What would have been the result then if I had restricted myself to the official period of 5 month for conducting research actions and writing the thesis? This very limited time frame would have made it very difficult for me to accept  the city as that space that forms my new temporal reality, which represents my life for the time to be and not just a space to rush through. Perhaps it would have been necessary to be just part of another existing (research) project while reproducing the dominant \bracket{social} top-down hierarchies and power relations, that western academic agents and their intended \quote{research objects} often represents. 

\spaceHalf

\inright{foreign expert and local adept}
A contradiction produced by those hierarchies is the \startAMark reversed concept of knowledge, where the one that lacks local knowledge but is embedded in an academic frame has more power or status than the one that is the local adept, who knows everything about his or her surroundings, but is maybe marginalized and lives at the outer margins of society. \stopAMark. How can I then consider me as some kind of \quote{expert} that is able to judge, analyse and propose if I know nothing about the local situations, realities and struggles. Event 6 month in São Paulo gave me just a hint to the plurality of realities that this city of extremes produces. 

\spaceHalf

\inright{knowledge and power hierarchies}
Another question remains: do I would like to act as such an \quote{expert} anyway, even with the proper knowledge. I personally would not exploit my adept knowledge \toMark{richtiger ausdruck?} in order to gain power \bracket{in order to produce content for the finalization of the thesis} nor would I accept the role of a scholar because this role is already loaded with power hierarchies and symbols that conflict with my \inright{personal conviction} personal conviction. In their work \aQuoteInTextT{What have the Romans ever done for us?} \aQuoteY{2001}, \aQuoteInTextA{Barker and Cox} describe the role of the scholar \bracket{here meant as scholar of social movement} as follows:

\spaceHalf

\startCitation
The scholar acts as a traditional intellectual, carrying out directive and theoretical activity on behalf of already-existing, and already-powerful, social classes and groups. Their directive activity is entailed in the administration and development of an education system which is a central mechanism in reproducing class inequality and in legitimating the social order. \aQuote{Barker and Cox}{2001}
\stopCitation

\spaceHalf

\inright{contradictions and tensions in different roles}
If I then define my role in this research, I clearly sympathize with the people I have been together, I feel myself more belonging to their struggles, as to what the contemporary academic world symbolizes \bracket{even though I do not deny the importance of academic work and analysis, eventually I make use of it in this thesis as well}. This fact certainly affects the way I act and decide because I am socialized much more by the activist than the academic space and certainly perceive their diametrical opposed poles, especially when trying to practice my own personal attitudes and convictions in those spaces but also with respect to the formation of knowledge, which is produced according to different concepts and motivations. 

Quoting \aQuoteInTextA{Barker and Cox} once more, the contradictions thus also emerge due to the diverging role concepts where

\spaceHalf

\startCitation
[...] those who are drawn to this field of academic study are themselves former or continuing activists and participants in actual movements and movement organizations. [...] Those with feet in both camps are often aware of contradictions and tensions in their different roles \aQuote{Barker and Cox}{2001}.
\stopCitation

\spaceHalf

Thus for me, non-hierarchical/non-authoritative and genuine participation is an attitude applicable in all areas of practice, may they be political motivated, related to academic research or just belong to daily life. I consider the \bracket{from my point of view} discursively defined areas of private life, research or struggle as at least overlapping, if not the same sometimes. This also means that I attempt not to reproduce them.

Hence my intention was and is to do research based on those and other attitudes (which will be exposed in a later section) and write this thesis because \inright{personal motivation} I consider it \toMark{relevant for me and the reflection on my personal practice, relevant as a complementary component of the struggle of the people, relevant for the interconnection of academic space, marginalized space, political space and social space \bracket{social space here meant as a synonym for society and the city}, their interchange and awareness raising}.

\aQuoteInTextA{Anna Tsing} \refMissingSrc{http://anthro.ucsc.edu/directory/details.php?id=35} asks in \aQuoteInTextT{Friction} what other possibilities exist for knowledge production aside the academic norms but also wonders why different approaches cannot be justified even though they would complement and enrich each other

\spaceHalf

\startCitation
How has it happened that in order to stay true for hopes for a more liveable earth, one must turn away from scholarly theory? [...] Might it be possible to use other scholarly skills, including the ability to tell a story that both acknowledges imperial power and leaves room for possibility?
\aQuoteB{Tsing}{2005}{267}
\stopCitation

\spaceHalf

\inright{spaces not seen as atomic units but interdependently connected}
I don't intend to distinguish those spaces as separate from each other, the research space separated from the social space which represents or is represented by the city, separated from private space of my life in São Paulo.  I didn't define hours per day to enter the research space, nor hours to enter the daily life or social space.  

Certainly, those spaces existed and exist but for me, I perceive them as organically converging, diverging, overlapping and sometimes matching, depending on the context all those different situations have been embedded in. When I was on the streets, I often met people whose daily reality I participated in, when I went to \aKeyword{Ay Carmela} or simply roamed the streets in order to absorb the city. In those cases we either spent time together, which could be time considered as research action, as socializing, leisure or political action, or all together at the same time, or we just continued on our sparate paths. 

Clearly, the separation of those spaces existed and exist, because eventually, I didn't life together with the people on the streets, we \quote{just} shared plenty of time together.  This also meant that my time in São Paulo was a time where I personally didn't need to take care about organizing my life because I had a definite place to life, a determined number of month to stay and I could freely organize my time without hassle for work or earning money. This is one hierarchical aspect which I could not resolve and which implies that I was in the luxury position to freely organize my time and research action(s) and be together with people whose situation was exactly contrary, which struggle every day. 

\spaceHalf

\inright{the purpose of those narratives}
The purpose to relate those impression is simply the fact that it took time for me to arrive in São Paulo, especially if the reality I came from and the one I arrived in are so diverging. In my case it took about two month, which had been important and necessary for me but resulted in no concrete or visible outcome for this thesis at first glance. The establishment and deepening of tied contacts on the base of amity and solidarity took another one or two month and suddenly the remaining time in the city had been drastically reduced. For me, the whole process was important and contribute to the thesis as much as the concretely conducted research action(s). I consider the whole period as enriching for me and my personal practice and definitely not as a mere obligation in order to gain a degree. This \aKeyword{synopsis} also serves as a summary for me in order to reflect on my role and my status and the circumstances that affected my stay in São Paulo.

Having said this, perhaps some of the factors that mainly impacted the course of this thesis are clearer now, thus let's see how the red line through it can be tied.

\stopmode

\stoptext

\stopcomponent
