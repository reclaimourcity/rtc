\startcomponent component_MethodologyIntro
\product product_Thesis
\project project_MasterThesis

% definitions and macros
\environment envThesisAllEnvironments
\environment envCfgThesisCommands
\environment envCfgThesisDescriptions

\define[]\chapterMethodology {\chapter[methodology_intro]{Methodology//Self/Conception}}

\starttext

\startmode[tocLayout]
\chapterMethodology 

The methodology chapter. Describes the thesis methodology in terms of theoretical and practical work.

\stopmode

\startmode[draft]
\chapterMethodology

\startKeywords
\methodology, \aKeyword[introduction]{methodology+introduction}, \guidingQuestions, \WhoAmI, \WhatShouldIDo, \WhatDoIWant, \WhoAreWe, \WhatShouldWeDo, \WhatDoWeWant,\selfConception
\stopKeywords

\spaceHalf

\inright{aims of this map}
Starting this written work I would like to transparently map a self-conception of my thesis that affects the selection of particular research actions, that enables me to discover  and articulate possible aims of this thesis.

First of all some thoughts about the structure of my map. In a sense the mapping of a self-conception, a methodology after all, is the result of research actions in São Paulo where I mainly participated in the realities of the people from the streets in the city's centre. I caught a glimpse of the peoples struggle, the forms of organizing, the theorizing about the situation in the city and its effects on the people. A  \aKeyword[transparent]{transparency} mapping shall expose those experiences and the arguments for choosing \bracket{research} \aKeyword[actions]{research+actions}, \aKeyword[methods]{research+methods} and \aKeyword[tools]{research+tools}. Transparency also in order to permit the appraisal of my own subjectivity in narrations, observations and actions. 

%Supplementary, or better complementary, from my perspective, is the question on which base the argumentation for selected approaches and the thesis objectives are grounded. In particular I would like to ask how the thesis' underlying \aKeyword[motivation]{thesis+motivation} and \selfConception can be formulated and how this articulation impacts the selection to \bracket{research} approaches and the definition of possible \aKeyword[objectives]{objectives} and \bracket{research actions}. 

\spaceHalf

\inright{subjectivity and the influence of my personal attitude}
This mapping shall thus not only determine theoretical and research approaches, methods and tools, objectives and actions, but shall also address my personal conviction and \standpoint that is shaping this work to a certain extend. Theoretical considerations, experiences, narrations and observations made on the streets, the definition of a self-conception, those are the fractions that I would like to consider in order to argue for concrete means of action. This mapping will help me to learn constructing the frame this thesis can be embedded in. 

Departing from such a rhizomatic mapping can certainly be associated with my personal attitude that is autonomous rather than academically determined. In São Paulo I perceived my role as the role of an activist rather than an academic scholar. 

I would therefore like to arrange the tracings on my map around questions that are basically asked by \socialMovements in order to \theorize about their situation, their struggle, their resistance, their proposals.

\startCitation
More formally, movement theorizing consists of the processes of unofficial thought that movement activists constantly work with - geared primarily towards the practical question 'what should we do?', but including all sorts of related questions, such as 'who are we?', 'what do we want?', 'who is on our side?', 'who are they and what are they doing?', 'what can we do?' \aQuoteW{Barker and Cox}{2001}
\stopCitation

The following \guidingQuestions help me to finally determine a \methodology this work can be elaborated on. Each question exhibits two particular perspectives, from a personal \standpoint and from a more general \standpoint.

\spaceHalf

\textBoxedRoundMaxQuestion[0.4]{{\ss\WhoAmI} Outlines the basic thesis' self-conception and attitude which is strongly interrelated with my personal conviction. It also outlines the general thesis perspective on research actions which is mainly driven by the question of knowledge production.}

\textBoxedRoundMaxQuestion[0.4]{{\ss\WhatDoIWant} Determines the thesis objectives drawn of the proposals formulated by the first question. Those objectives are formulated as general and specific proposals. Specific for the time in São Paulo for instance or as general aim of research.}

\textBoxedRoundMaxQuestion[0.4]{{\ss\WhatShouldIDo} Outlines concrete methods, tools and research actions. What shall be done is asked for practical/empirical and theoretical research actions.}

\stopmode

\addReference
{
Barker, C. \& Cox, L., 2001. “What have the Romans ever done for us?” Academic and activist forms of movement theorizing. Available at: \goto{\hyphenatedurl{http://www.iol.ie/\~mazzoldi/toolsforchange/afpp/afpp8.html}} [url(http://www.iol.ie/\~mazzoldi/toolsforchange/afpp/afpp8.html)] [Accessed July 18, 2011].
}

%---------------------------------------------------
% gets only displayed in unfinshed mode

\showImperfection

%---------------------------------------------------

\stoptext

\stopcomponent
