\startcomponent component_MethodologyIntro
\product product_Thesis
\project project_MasterThesis

% definitions and macros
\environment envThesisAllEnvironments
\environment envCfgThesisCommands
\environment envCfgThesisDescriptions

\define[]\chapterMethodology {\chapter[methodology::intro]{Methodology}}

\starttext

\startmode[tocLayout]
\chapterMethodology 

The methodology chapter. Describes the thesis methodology in terms of theoretical and practical work.

\stopmode

\startmode[draft]
\chapterMethodology

\startKeywords
methodology, introduction, guiding questions, who am i?, what should i do?, what do I want?
\stopKeywords

\spaceHalf

\inright{aims of this chapter}
The aim of this \aKeyword[Methodology chapter]{methodology+introduction} should be to transparently describe the context this thesis is embedded in and the appropriate actions that have been chosen in order to articulate and fulfil the thesis' aims.\aKeyword[Transparency]{transparency} in order to expose the reasons why \bracket{research} \aKeyword[actions]{research+actions}, \aKeyword[methods]{research+methods} and \aKeyword[tools]{research+tools} have been chosen. Transparency in order to permit the appraisal of my own subjectivity in narrations, observations and actions. 

Supplementary, or better complementary, from my point of view, is the question on which base the argumentation for selected approaches and the thesis objectives are grounded. This means in particular to ask how the thesis underlying \aKeyword[motivation]{thesis+motivation} and \aKeyword[self-conception]{thesis+self-conception} can be articulated and how this articulation impacts the selection to \bracket{research} approaches and the definition of possible \aKeyword[objectives]{thesis+objectives} and \bracket{research} actions. 

\spaceHalf

\inright{subjectivity and the influence of my personal conviction}
The \aKeyword[Methodology]{methodology} chapter shall therefore not only determine theoretical and research approaches, methods and tools, objectives and actions, but shall also address my personal conviction and standpoint that is also reflected by this thesis. Theoretical considerations, experiences, narrations and observations made on the streets, the definition of the thesis' self-conception, those are the fractions that I would like to consider in order to argue for concrete procedures and means of action. Such an argumentation will help me to construct the frame this thesis can be embedded in, which is then immanently comprised of a well defined self conception and emancipatory terms research actions. Departing from such an approach can certainly be associated with my subjective and personal conviction which is \toMark{rather libertarian and autonomous than academic, neo-liberal or institutional}.

Before I carry on I would like to briefly describe my thoughts about the structure of this chapter. In a sense the determination of methodology has been a result of research actions in São Paulo where I mainly participated in the realities of the people from the streets in the city's centre. I caught a glimpse of the peoples struggle, the organizational forms, the theorizing about the situation in the city and its effects on the people. I perceived my role as the role of an activist rather than an academic scholar. Therefore I will arrange this chapter according to questions that are basically asked by \aKeyword[social movements]{social movement} in order to \aKeyword[theorize]{theorizing+movement} about their struggle or resistance.

\startCitation
More formally, movement theorizing consists of the processes of unofficial thought that movement activists constantly work with - geared primarily towards the practical question 'what should we do?', but including all sorts of related questions, such as 'who are we?', 'what do we want?', 'who is on our side?', 'who are they and what are they doing?', 'what can we do?' \aQuoteW{Barker and Cox}{2001}
\stopCitation

The following three \aKeyword[questions]{methodology+guiding questions} will frame the sections of my \aKeyword[Methodology]{methodology} which hopefully provides answers to all of them. Each section addresses two perspectives on the given question.

\spaceHalf

\textBoxedRoundMaxQuestion[0.4]{\aKeywordInVi{who am I?}{\ss\toTranslateO[Wer bin ich?]{Who am I?}} Outlines the basic thesis' self-conception and attitude which is strongly interrelated with my personal conviction; Outlines the general thesis perspective on research actions which is mainly driven by the question of knowledge production and my role in research.}

\textBoxedRoundMaxQuestion[0.4]{\aKeywordInVi{what do I want?}{\ss\toTranslateO[Was will ich?]{What do I want?}} Determines the thesis objectives determined off the first question. Those objectives are of general and specific nature.}

\textBoxedRoundMaxQuestion[0.4]{\aKeywordInVi{what should I do?}{\ss\toTranslateO[Was soll ich tun?]{What should I do?}} Outlines concrete methods, tools and research actions off the two previous questions. What shall be done is asked for practical/empirical and theoretical research actions.}

\stopmode

%---------------------------------------------------
% gets only displayed in unfinshed mode

\doifmode{unfinished}
{

\subject{keywords}
\placeRegKeyword[criterium=section]

\subject{translations}
\placeTranslationsLocal

\subject{still to translate}
\placeUndefinedTranslationsLocal

\subject{references missing}
\placeMissingReferencesLocal

\subject{text marks}
\placeTextMarksLocal

\subject{remarks}
\placeRemarksLocal

\subject{used references in text}
\placeQuotesLocal

\subject{references}

\startREF

\nl%
Barker, C. \& Cox, L., 2001. “What have the Romans ever done for us?” Academic and activist forms of movement theorizing. Available at: \goto{\hyphenatedurl{http://www.iol.ie/\~mazzoldi/toolsforchange/afpp/afpp8.html}} [url(http://www.iol.ie/\~mazzoldi/toolsforchange/afpp/afpp8.html)] [Accessed July 18, 2011]. \nl%

\stopREF

}

%---------------------------------------------------

\stoptext

\stopcomponent
