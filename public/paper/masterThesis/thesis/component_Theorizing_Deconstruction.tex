\startcomponent component_Theorizing_Deconstruction
\product product_Thesis
\project project_MasterThesis

% definitions and macros
\environment envThesisAllEnvironments
\environment envCfgThesisImages

\define[]\sectionTheorizingDeconstruction{\section[theorizing_deconstruction]{De//Construct//Themes}}

\starttext

\startmode[tocLayout]
\sectionTheorizingDeconstruction

Deconstructing the meaning of theorizing themes.

\stopmode

\startmode[draft]
\sectionTheorizingDeconstruction

\imgIParticipateFlyer

\spaceHalf

\inright{deconstruction of participation}
The poster above is a wonderful opening for \deconstructing \participation. It says

\startCitation
I participate, You participate, he and she participates, We participate, You participate, They profit.
\stopCitation

and thereby precisely points to the defect in the concept of \participation as it was meant then and as it is meant today. This poster\footnote{A List of May 68 graffiti is documented at the \aLinkNewNoF[Bureau of Public Secrets]{http://www.bopsecrets.org/CF/graffiti.htm}} has been published in Mai 68, when students and workers gained momentum in 

\startCitation
the largest general strike that ever stopped the economy of an advanced industrial country, and the first wildcat general strike in history; revolutionary occupations and the beginnings of direct democracy; the increasingly complete collapse of state power for nearly two weeks; the resounding verification of the revolutionary theory of our time and even here and there the first steps toward putting it into practice; the most important experience of the modern proletarian movement that is in the process of constituting itself in its fully developed form in all countries, and the example it must now go beyond — this is what the French May 1968 movement was essentially, and this in itself already constitutes its essential victory. \aQuoteW{Bureau of Public Secrets}{1969}
\stopCitation

\quote{We participate - They profit}. This is what \participation is about in present capitalistic times. We are invited to participate in many things. We are invited to crowdsource the net, the user is the producer, we can \refMissingSrc{Digg our most favourite blog post}, we participate in the construction of social communities\footnote{From Facebook, to Orkut and Google+}, we can conveniently participate in social campaigns with a \refMissingSrc{simple click}, the marginalized and oppressed are invited in participatory development projects that seems to end their poverty or at least reduce it, crisis management during times of catastrophes \refMissingSrc{incorporate citizen reports} from the ground in order to be more effective and precise in providing help, its seems that we can even participate in the planning of our direct environment, our neighbourhood, when local governments decide to give us a tiny share of \bracket{assessment} power or at least allows us to select our most favourite proposal for a new shopping mall right around the corner, we are told to show civil courage to help the homeless around the corner, at work we can participate in unions to fight for better working conditions \bracket{at least there where unions are established}, and last but not least it seems that we can even have an impact on public politics by putting our vote once a few years. 

Even though this listing is relatively arbitrary and polemical, it may show what is currently understood by participation. We participate in proposals of others, our participation becomes their profit. We are invited to participate by those that have the power to invite us, that already decided what we are supposed to do or that provides with options for predetermined alternatives. This contemporary concept

\startCitation
highlights the fundamental point that participation without redistribution of power is an empty and frustrating process for the powerless. It allows the powerholders to claim that all sides were considered, but makes it possible for only some of those sides to benefit. It maintains the status quo. \aQuoteB{Arnstein}{1969}{216-224}
\stopCitation

This quote by \aQuoteInText{Sherry Arnstein} is taken from the \aQuoteInTextT{Ladder of Citizen Participation}. It is written from the standpoint of participation of the powerless in housing and urban planning decisions. I raises general issues that are specific for this type of unequal \participation.

\spaceHalf
\inright{1. no distribution of power}
\Participation becomes a new form of \refMissingSrc{tyranny} under such circumstances. Here it does not eliminate structural inequalities but rather reproduces the structural power relations that causes them. In this sense, the status quo is maintained or further enforced, the marginalized remain marginalized and the power full remain in power. \aQuoteInTextA{Hickey and Mohan} note that:

\startCitation
[...]the ‘community’ has periodically been destroyed by underlying processes of development, only to be resurrected as the proper source of recovery through trustee-led intervention” \aQuoteB{Hickey and Mohan}{2004}{10}.
\stopCitation

\spaceHalf
\inright{2. co-opting the idea of participation for single sided benefits}
They write from the standpoint of \bracket[international] development projects and their failure to provoke emancipatory transformation for the powerless but their analysis could also also generally conclude that the power structures that caused and causes inequality and discrimination, no matter of what form, are now even benefiting by \quote{genuinely} offering space for participation, by further \toMark{oppressing} the marginalized.

\spaceHalf
\inright{3. depoliticization of participation}
Such a \participation is stripped off the political dimension, it is depoliticised because its intention is not an understanding how existing structures of power, oppression and injustice operate \aQuoteB{Hickey and Mohan}{2004}{11} and therefore does not intent to transform them. \Participation is merely a mean to a different end. 

\spaceHalf

\inright{4. different standpoints, similar effects}
From the standpoint of a \toMark{development} project \participation is often used as a methodology in order to \aQuoteInText{prove its participatory credentials} \aQuoteB{Hickey and Mohan}{2004}{16} but which just implements a \aQuoteInText{technical fix for complex problems of uneven development} \aQuoteB{Cleaver and Rahman in Hickey and Mohan}{2004}{59}. Those technocratic approaches that only focus on a particular local situation have led to a depoliticization because participation takes only place within the limited scope of a project \aQuoteB{Hickey and Mohan}{2004}{10}, does not go beyond it.

From the standpoint of the state, a different situation unfolds:

\startCitation
On the one hand, privatization, deregulation, unemployment and ‘precarization’ of labour (and ‘structural adjustment’ programmes at the periphery and semiperiphery of the world-system); on the other hand, attempts to bring people to ‘take part’ in the management of local-level state crisis (along with other measures like repression and ‘state of emergency’, as long as they are necessary and feasible). \aQuoteB{de Souza}{2006}{335}
\stopCitation

\aQuoteInTextA{David Barney} writes in \aQuoteInTextT{“Excuse us if we don’t give a fuck”: The (Anti-)Political Career of Participation}

\startCitation
The basic logic of citizenship is inclusion and participation; the basic logic
of politics is exclusion and refusal. And this is why a culture of liberal democratic citizenship, a culture of the universalized invitation to participate, tends to produce politics only at and beyond its borders and margins. \aQuoteB{Barney}{2010}{144}
\stopCitation

\addReference
{
Arnstein, S.R., 1969. A Ladder of Citizen Participation. {\em JAIP}, 35(4), p.216-224. Available at: \goto{\hyphenatedurl{http://lithgow-schmidt.dk/sherry-arnstein/ladder-of-citizen-participation.html}} [url(http://lithgow-schmidt.dk/sherry-arnstein/ladder-of-citizen-participation.html)] [Accessed May 31, 2011].
}

\aQuoteB{Barney}{2010}{144}\aQuoteInTextT{“Excuse us if we don’t give a fuck”:
The (Anti-)Political Career of Participation}

at the very least they raise the question of the status of participation as both a political end and as a critical category under contemporary economic, social, and technological conditions. Participation is ambivalent, as open to stabilizing prevailing arrangements of power and injustice as it is to disrupting them. Thus, from a critical perspective, it would appear necessary to stipulate that participation is not an absolute, but rather only a contingent value: one whose worth is not intrinsic but rather derived from the ends it serves in any given context. Further, in the context of a hegemonic political and economic culture that not only accommodates participation but actually embraces, thrives, and insists upon it, and in light of proliferating technologies that effectively render routine participation obligatory, the ends that participation presently serves cannot be said to be unambiguously worthwhile. If we are looking for something to which we might attach our aspirations for a more just society, we might have to look for something other than mere participation. What we might actually need is politics, not just participation. It was suggested above that participation is what citizenship is about, but that citizenship-as-participation is something altogether different from politics. Much turns on what one thinks politics is, and on what one thinks politics is for.

\aQuoteB{Barney}{2010}{145}\aQuoteInTextT{“Excuse us if we don’t give a fuck”:
The (Anti-)Political Career of Participation}
Western political thought has been that the central task of any political regime is education. The survival of the prevailing order depends upon depoliticizing young
people by making good citizens of them, by inviting, or even compelling them to participate. It is with good reason that we have long pathologized the figure of
the disengaged, apathetic youth, and groped frantically for therapeutic aids that might entice young people to participate (forgetting that time tends to make good
citizens of us all). And it is no wonder we have invested such hope in the potential of emerging media to engage young people and to encourage them to participate.
Participation, in the end, is truly much safer—and much easier to deal with—than politics.

\stopmode

\stoptext

\stopcomponent
