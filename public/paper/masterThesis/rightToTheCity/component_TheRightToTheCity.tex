\startcomponent component_TheRightToTheCity
\product product_TheRightToTheCity
\project project_MasterThesis

% definitions and macros
\environment envCfgPaperLanguageEn
\environment envRttcUrl
\environment envRttcText


\starttext
\chapter[rttc]{The Right to the city}

The \inright{\hlef and his writings about the \rttc} \rttc can be referred back to the French sociologist and philosopher \from[wikiEnLefebfre]. Several of his writings are dedicated to and develop a concept of the \rttc, among them {\em Le droit à la ville (The right to the city)} (Lefebvre, 1968), {\em Le droit à la ville suivi de Espace et politique} (Levebfre, 1974), {\em The production of space} (Lefebvre, 1991) or {\em Writings on cities} (Lefebvre, 1996).

\hlefs \inright{an abstract framework} concept of the \rttc can be reckoned as a highly abstract framework that aims to conceptualize a critique of the development of the (industrialized, western) city that neglects (the majority of) its inhabitants. With the \rttc he provides a conceptualization of a {\em right} that each citizen is supposed to possess. 

The abstractness of this conceptualization makes it actually difficult to fully grasp its meanings and corresponding real world consequences (Purcell, 2002, p.99) and leaves much space for theorization and interpretation, which is clearly visible in the different discourses about the \rttc, between the points of views of institutional agents and agencies and those of urban (social) movements which began to incorporate the \rttc as a fundamental claim of their struggles. These differences will be briefly explored and laid out later on. For now, let us see how the \rttc is formulated.

According \inright{interdependent formation of space and human} to \hlef, the city consists of (social) spaces and (social) practices that are shaped, accessed and carried out by its inhabitants, which in turn are shaped by those spaces and practices, thus, the city is

\quote{\em [...] man’s most successful attempt to remake the world he lives in more after his heart’s desire. But, if the city is the world which man created, it is the world in which he is henceforth condemned to live. Thus, indirectly, and without any clear sense of the nature of his task, in making the city man has remade himself.}(Park, 1967 in Harvey, 2008)

The \inright{multidimensional spatiality} concept of space in the context of \lefs work refers to various forms of perceptions of space. According to him, the spaces that shape the city are not only of physical or geographical nature but can be considered as {\em perceived}, {\em conceived} and {\em lived spaces} (Purcell, 2002, p.102).

This multidimensional spatiality is important in the context of the \rttc because urban space cannot just be divided and reduced to its individual components in order to be analysed separately. This would lead to \quote{the analytic destruction of space} (Prigge, 2008, p.47).

% hier fehlt eine referenz bei foucalt
Foucault \inright{space is not only geographically determined} noted that {\em space} had always been just referred to a {\em geography}: the living space, the city, the state, something given, natural, land, surface. The reduction of space to something physical neglects the fact that, here particularly in the urban context, spatial structures symbolize the invisible social relations, roles, hierarchies and powers that are distributed in the physical or geographical (urban) space (Prigge, 2008, p.47)

\setupDescrTable
% head on every page, stretch columns
\bTABLE[split=repeat]
\bTABLEbody
\bTR
  \bTC  perceived space \inright{perceived, conceived and lived spaces according to Henri Lefebvre}
 \eTC
  \bTC  conceived space \eTC
  \bTC  lived space \eTC
\eTR
\bTR
  \bTC  perceived space is the physical space one encounters when moving through the city. the urban reality that interlinks the daily reality, thus, the urban network of routes that links daily routine, the private life, leisure, work  \eTC
  \bTC  conceived space is space loaded with meanings and concepts \eTC
  \bTC  lived space is perceived and conceived space, the personal experience of space and social life, the physical space charged with symbols and imagination of its {\em inhabitants}, its {\em users}, its {\em producers} \eTC
\eTR
\eTABLEbody
\eTABLE

\quote{\em the urban is [...] pure form; a place of encounter, assembly, simultaneity. This form has no specific content, but is a center of attraction and life. It is an abstraction, but unlike a metaphysical entity, the urban is a concrete abstraction, associated with practice [. . .] What does the city create? Nothing. It centralizes creation. Any yet it creates everything. Nothing exists without exchange, without union, without proximity, that is, without relationships. The city creates a situation, where different things occur one after another and do not exist separately but according to their differences. The urban, which is indifferent to each difference it contains, . . . itself unites them. In this sense, the city constructs, identifies, and sets free the essence of social relationships [...] }(Lefebvre, 2003, in Prigge, 2009, p.49)

If \inright{a society's spatial practice} then the city is shaped by its inhabitants and vice versa, the deciphering of (urban) spaces can reveal the {\em spatial practices} of a society while \quote{\em spatial practice of a society secretes that society's space} (Lefebvre, 1991, in Prigge, 2008, p51).

Due to this interdependency of space, spatial practice and citizen, the production of (urban) space reflects the (social) conditions a society lives in. At the same time, the (social) conditions of a society reflect the way how (urban) space is produced. 

\stoptext

\stopcomponent
