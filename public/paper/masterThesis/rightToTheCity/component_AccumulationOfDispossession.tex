\startcomponent component_AccumulationOfDispossession
\product product_TheRightToTheCity
\project project_MasterThesis

% definitions and macros
\environment envCfgPaperLanguageEn
\environment envRttcUrl
\environment envRttcText

\starttext

\section[accumulation]{Accumulation of Dispossession}

Returning \inright{accumulation of dispossession} to the \rttc, David Harvey describes the coherence of contemporary urban development processes as \quote{accumulation of dispossession}. In \from[newLeftReviewRttc] (2008) he argues that

\quote{\em the global urbanization boom has depended, as did all the others before it, on the construction of new financial institutions and arrangements to organize the credit required to sustain it. [...] urbanization, we may conclude, has played a crucial role in the absorption of capital surpluses, at ever increasing geographical scales, but at the price of burgeoning processes of creative destruction that have dispossessed the masses of any right to the city whatsoever.} (Harvey, 2008)

A small part of the city's inhabitants benefit from the (planned) changes that reshape the city's structure. For the remaining majority, the social conditions are worsening, because they are dispossessed from the freedom to freely live in, move around, use the city. Dispossessed from their homes, quarters and neighbourhoods, their social networks, culture, work, from the right to shape the city, thus dispossessed from any \rttc. 

This total transformation of the city has engendered a new lifestyle, a lifestyle of consumerism in all possible variants, which turned quality of life and the city itself into a commodity while the defence of property became of major political interest. These \quote{new} norms can be identified in the massive spatial fragmentation of the city, in the {\em segregation} of the society and the displacement of its members, may it be due to {\em gentrification}, repression, state led {\em requalification and revitalization} or complete renewal of city quarters, which all feed into the \quote{\em accumulation by dispossession [which] lie at the core of urbanization under capitalism} (Harvey, 2008). {\em Displacement and dispossession} take thus place in {\em perceived spaces} but also in {\em conceived and lived spaces} where a dispossession of any sense of affiliation to a (however defined) society takes place.

\stoptext

\stopcomponent
