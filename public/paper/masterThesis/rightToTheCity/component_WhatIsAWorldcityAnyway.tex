\startcomponent component_WhatIsAWorldcityAnyway
\product product_TheRightToTheCity
\project project_MasterThesis

% definitions and macros
\environment envCfgPaperLanguageEn
\environment envRttcUrl
\environment envRttcText

\starttext

\section[worldcityHowsHow]{What is a worldcity anyway}

Even though the abstract manifestations of the \worldcity are difficult to grasp, in daily reality one can nonetheless translate the abstract to the real. The examples of \waterfront development have shown that not only so called \worldcities such as New York or London are affected by or enforce (mega) urban development processes and models, but that smaller cities and cities in the global south are replicating them as well, as it has been seen in the case of Cologne and Rio de Janeiro.

In another oversimplified \worldcity sketch one could draw connections from the concentration of advanced producer services that reside within a city, to the necessity of skilled and ultra mobile knowledge workers that organize these services and produce innovations. The services and workers to attract, demand a safe city and high class cultural amenities, probably their own districts or quarters, perhaps newly constructed as part of (mega) {\em urbanisation projects} that \revitalize \brownfields or displace long established quarters, perhaps {\em gentrified} or {\em cleaned} from \quote{unwanted} subjects. Due to the effects of the city on its hinterland (the integration of urban and rural spaces but also the transnational effects caused in other world regions), a rising stream of migrants on national and international level can be observed, which are caused by devastating changes in rural areas provoked by i.e. large scale \from[wikiEnAgribusiness], commanded and controlled from and within the network of (world) cities. Arriving in the city, there exist often no access to the city's {\em lived space}, i.e. due to the lack of necessary education, different languages or valid papers and therefore the denial of access to learning facilities or work which would help to guarantee a decent living. The result is further {\em precarization} of living conditions i.e. in the peripheries of the cities or informal settlements; the necessity to work informally with corresponding institutional repression; a rising fraction of population dumped on the streets. 

This \inright{world city or worldcityness} sketch is by far not complete nor does it claim to be perfectly accurate. It is obvious that different cities in different geographical regions are sometimes shaped by similar, sometimes by different processes. \Castells notes that each city is always connected, up to a certain extend, to the global \spaceofflows or the global network of (world) cities, thus each city possesses a smaller or larger fraction of \worldcityness and by that experiences often similar effects and transformations of their \livedspaces (Castells, 2004).

\stoptext

\stopcomponent
