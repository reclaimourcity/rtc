\startcomponent component_MethodologyTheoryAndPracticeInductive
\product product_Thesis
\project project_MasterThesis

% definitions and macros
\environment envThesisAllEnvironments

\define[]\subsectionInductive{\subsubsection[methodology::motivation::inductive]{Inductive versus Deductive}}

\starttext

\startmode[tocLayout]

\subsectionInductive

This sub-section argues whether inductive or deductive approaches have been used or not and what the alternatives are.

\stopmode

\startmode[draft]

\subsectionInductive

\startKeywords
deductive, inductive, reflective, narrative
\stopKeywords

\startAReminder 
in diesem abschnitt stellt sich die frage was neben induktiv und deduktiv noch benannt werden sollte, da ja keine strukturierten daten erhoben wurden. vielleicht kann das kapitel auch wech?
\stopAReminder

\startADefinition
{\bf inductive} procedures originate from particular situations, the local, which is supposed to be verified against a theory. Usually, a corresponding theory already exists and no new theory is supposed to be formulated. The local is always a snapshot of a particular situation which cannot be easily generalized by default. The conclusions drawn from the local can be used to adjust, support or object existing theories. Therefore, the local is described and represented by empirical data which is supposed to be representative and reliable enough to be considered valid for the contention or adjustment of the corresponding theory.
\stopADefinition

\startADefinition
{\bf deductive} procedures originate from theory, the general, which is supposed to be translated to the local, to particular situations, where it can be validated and proved. On the local it shall be verified whether the proposed theory is valid or not or if it needs adjustments. On the local this is done with empirical data which is supposed to be representative and reliable enough to evaluate the correspondingly formulated theory. 
\stopADefinition

\stopmode

\stoptext

\stopcomponent
