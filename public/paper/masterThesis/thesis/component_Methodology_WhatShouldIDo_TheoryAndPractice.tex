\startcomponent component_Methodology_WhatShouldIDo_TheoryAndPractice
\product product_Thesis
\project project_MasterThesis

% definitions and macros
\environment envThesisAllEnvironments
\environment envCfgThesisImages

\define[]\sectionTheoryAndPractice{\subsection[methodology::whatshouldido::theoryandpractice]{From theory to practice or vice versa?}}

\starttext

\startmode[tocLayout]

\sectionTheoryAndPractice

In this section, field-trip and theoretical approaches are lain out in detail.

\stopmode

\startmode[draft]

\sectionTheoryAndPractice

\startKeywords
deductive, inductive, empirical, theoretical, qualitative, quantitative, participatory, tyranny, hierarchies, authority, emancipation, self-determination
\stopKeywords

In order construct a methodological framework based on the \bracket{subjective} motivations and demands \refMissingSrc{motivation-demand chapter} formulated above, I first want to explore available schemes of scientific research approaches and balance reasons for and against them. Knowledge about their content and form will permit me to

\spaceHalf

\textBoxedRoundMax[0.4]{determine the methodological approach(es) embodied in the thesis research action(s)}

\textBoxedRoundMax[0.4]{determine the direction this thesis is further veering towards}

\textBoxedRoundMax[0.4]{argue for the chosen research action(s) and by doing so induce transparency in order to comprehend the decisions made to realize this thesis}

As principle scientific scheme I have chosen the \quote{inductive-deductive} model of \refMissingSrc{inductive-deductive model}

Due to the fact this thesis is finally composed of theory and practice (action), the extend and relevance of those aspects could be be determined first.

\imgThesisTheroy

Several important aspects have been identified, mainly during the course of research actions in São Paulo, thus it can already be said that the shape of this thesis is strongly influenced by the experience made during those actions. The following factors have been identified and may therefore determine the concrete outcome and notion of the thesis: 

\startitemize
\item observation, empiricism and theory
\item induction and deduction
\item quality and quantity
\stopitemize

Those aspects will be aligned to the considerations that have been made earlier about \quote{Motivation and Demands} of this thesis.

\stopmode

\stoptext

\stopcomponent
