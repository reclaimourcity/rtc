\startcomponent component_Methodology_WhatDoIWant_ObjectivesGeneral
\product product_Thesis
\project project_MasterThesis

% definitions and macros
\environment envThesisAllEnvironments

\define[]\subsectionWhatDoIWantGeneral {\subsection[methodology::whatdoiwant::general]{General Objectives}}

\starttext

\startmode[tocLayout]

\subsectionWhatDoIWantGeneral

Defines the general objectives of this thesis.

\stopmode

\startmode[draft]

\subsectionWhatDoIWantGeneral

(p11) Therefore I hope this thesis could contribute to undertake further steps into that direction and to function as least as a strid 

(p25) It shall benefit movement theorizing in the sense that it provides access to academic knowledge that is normally not accessible to non-academics. Access to academic theorizing can help to further strengthen own positions as argued REF MISSING(elsewhere) by understanding from which standpoint discourses and discussions are actually mediated by those that are opposed by the movements, on which arguments this discourses are based on. 

(p25) By perceiving this thesis as part of the struggle of the people, a further intention is to inject movement content in academic space. It has been already argued REF MISSING(elsewhere) that movement theorizing is not less relevant than [academic theorizing and that the produced content could help to overcome or converge the borderlines between those two very different spaces. 

(p29) Opening knowledge , is a main perspective and general objective , as much as participation and collaboration in research,

(p29) The production of alternative content is a minor tendency in the sense that the thesis still has its academic shape to a certain extend but accompanying alternative content can be created mainly in form of documentation of the thesis’ process, published on open platforms, freely accessible.

(p30) general objectives
make academic knowledge openly accessible
building a network of solidarity
spread awareness about context of social struggles
inject movement knowledge into academia

(p39) One of the already defined objectives of this thesis is the convergence and eventual overcoming of separateness of theorizing in academic and social space, thus the mutual nurturing of both spaces in order to facilitate the struggle for social and emancipatory transformation.

(p39) In order to reach this objective , the partial knowledge from the streets must enter this thesis as well as the related partial academic knowledge . It will be proposed in the next paragraph which knowledge, theoretical discussions and considerations this could eventually be.

(p40) I would also like to note again that this thesis is written from a particular standpoint , from the standpoint of the people on the streets because I  considered me affiliated with them when I was in São Paulo and due to the fact that I experienced the city to a certain extend through them and through their experience. 

(p41) ..I would propose to take the following themes into account, as core of this thesis theorizing : the right to the city , self-determination , participation , citizenship , spatial justice(probably)

(p42) ...I would like to examine these themes on an abstract level as my contribution to movement theorizing and struggle and in order to provide access to related content and knowledge that could be applied in further movement theorizing.

(p42) ... I would like to examine the mentioned themes from the standpoint of self-determined and emancipatory praxis because those standpoints are inherent to this thesis but often also part of the praxis of social movements and collectives in São Paulo.

(p42) ... I would like to consider the new [social, political, lived] spaces that could be
constructed while examining the selected themes.


\stopmode

\stoptext

\stopcomponent
