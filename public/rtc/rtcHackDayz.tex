% global environments
\environment envRTC
\environment envUrlRTC
\environment envCmdRTC
\environment envTextRTC

% local environment
\environment envUrlHackDays
\environment envLinksHackDays
\starttext

% archive.org

\archiveDescrEn

\archiveDescrPt

\archiveNotes

% rtc english

This recording documents the \from[hackday] which took place on Sunday, the 10th of October at the \from[casadigital] in Barra Funda/São Paulo.  The initial discussion with all participants has been recorded almost completely, only the last speeches are missing, which is pretty sad because this part is worth to be analyzed in terms of critical theory of technology, the implementation of technology based {\em participatory} projects (for the sake of technology?) and for what reasons projects are pushed forward. Thus, this last part will be reconstructed from memory as good as possible :)

The \from[hackday] was articulated by \from[esfera] with the objective to make use of web and mobile tools to realize a project in order to enforce an institutional discourse around the issue of {\em lack of housing} in the city. The proposal is looking for a cooperation of all citizens, here in guise of {\em hackers, journalists, web and real life citizens, students, etc} and social movements (which are of course also composed by citizens), with representatives of \myMNPR  and \myFLM.

One of the proposal of the \from[hackday] was to discuss and elaborate two kinds of digital mappings related to urbanization processes in the center of São Paulo. As possible tools, digital mappings platforms such as \from[ushahidi] could be used, in conjunction with mobile phones and forms of online messaging, websites, microblogging, etc, in order to gather and visualize data and to allow for greater participation.

One mapping could be conducted in order to create a database of empty buildings in the city. Those information could then be used further in order to put pressure on public institutions and to continuously blame the level of housing speculation (also with respect to the \from[occupyEN] that are taking place right now in the center, with more than 3800 people occupying 4 large buildings that are empty since years).

A second mapping could be conducted in order to categorize(!) the street population. The justification for this proposal is the currently missing definition of the {\em street population}, for example in the \from[wikiptEN].  Especially this second proposal can (and should) be contested due to several reasons which will be laid out later in this post. 

After the first part of the meeting, which consisted of discussions and exchange of ideas, a split in working groups was suggested in order to {\em produce} \quote{a result}. The need to {\em produce} \quote{a result} was justified/enforced by the eventual loss of financial support for this project (proposal) in case that no \quote{result} would have been available. Another issue was the concrete-casted belief that technology constraints can be easily solved, because technological tools can always be adjusted to the actual needs, and people that possess the necessary knowledge are willing to do so. This may be completely right, but one could ask, how a horizontal organization could be realized, how knowledge could be distributed to unskilled people, thus how it could be multiplied, how the division of labor between skilled people that have the responsibility/power over infrastructural issues and the {\em workers} that solely produce information can be eliminated or at least diminished, how one can eliminate the dependency of third party financing and by that also project based thinking in order to build a sustainable process which survives by the willingness of the people to push it forward and could be maintained as longs as necessary instead of relying on financial support which is usually finite. 

{\em Finally, from my personal point of view, any proposal of a mapping of already vulnerable persons and communities should be questioned from an ethical point of view in the first place and at least discussed in great detail with the people itself. One could also ask why \quote{we} always like to analyze those that life in (from our point of view) precarious situations, the marginalized and excluded, and what would be the benefit for them to do so, instead of analyzing the processes and power structures that led and lead to those situations, which would then eventually lead to a massive critique of the system/society we live in. Those points have diminished my expectations of an otherwise good initiative which shows that (a part of) the better situated citizens are also caring about the worsening situation in the city, with respect to the large housing gap for low income and poor citizens in central areas and elsewhere and the denial of the right/access to the city for a growing number of people, if we take for example the street population which is growing every year.} 

Some of those questions and caveats may have been addressed and answered in the working groups and their results may be available online at some point of time. Nonetheless, its worth to go a bit further and briefly analyze the argumentation for a technology based project from the view of the \quote{critical theory of technology}. 

\quote{\em The Critical Theory of Technology [...] argues that the real issue is not technology or progress per se but the variety of possible technologies and paths of progress among which we must choose. Modern technology is no more neutral than medieval cathedrals or The Great Wall of China; it embodies the values of a particular industrial civilization and especially of its elites,which rest their claims to hegemony on technical mastery. We must articulate and judge these values in a cultural critique of technology. By so doing, we can begin to grasp the outlines of another possible industrial civilization based on other values. This project requires a different sort of thinking from the dominant technological rationality, a critical rationality capable of reflecting on the larger context of technology. (Feenberg, 2002)}  

Thus, the mentioned \quote{\em larger context of technology} will be the issue of the next post. In the meantime, all recordings are made available for listening on \from[archive], dedicated to the \from[commons]. 

\myStartItemizeArchive

\head Recording of the meeting during Transpârencia HackDay \myFromArchive{0}

complete recording, original audio

\myStopItemize
\nl
{\bf Media}
\myMedia
\nl
{\bf Links}
\myLinks
\nl
{\bf References}
\myRefs
\nl

% rtc portuguese

\stoptext