\startcomponent component_HowToRead
\product product_Thesis
\project project_MasterThesis

% definitions and macros
\environment envThesisAllEnvironments

\define[]\chapterHowToRead{\chapter[howtoread]{Read//This//Text}}

\starttext

\startmode[tocLayout]
\chapterHowToRead

describes how to read this thesis
\stopmode

\startmode[draft]
\chapterHowToRead

Before I begin I would like to lay out the structure of this work. I did not use a common document or thesis structure. This thesis rather reshapes the different flow of my research. It mainly consist of three main maps that follow the short \gotoTextMark[Introduction]{intro} to what you, the reader, could expect when continuing.

\spaceHalf
\inright{guiding questions}
The flows of this work are influenced by three \guidingQuestions that lay at the heart of my research. Those \guidingQuestions are: \WhoAmI \WhatDoIWant \WhatShouldIDo

I used those questions because this thesis is a narration of social movement's struggle. Therefore I am attempting to narrated my thesis from the \standpoint of social movements. Social movements use similar questions to position themselves in the world they are struggling in. Their main \guidingQuestions are: \WhoAreWe \WhatDoWeWant \WhatShouldWeDo

I am asking \gotoTextMark[Who am I?]{methodology_whoami} in order to determined the ground on which my research is based. The answer to this question is to one extend a subjective determination of a self conception, a subjective \gotoTextMark[list of demands and motivations]{methodology_whoami_motivation} about my research. To the another extend its the \gotoTextMark[determination of my approach to research actions]{methodology_whoami_actionresearch} in São Paulo. 

The second question asked, \WhatDoIWant, draws on the answers of the first and proposes concrete \gotoTextMark[aims, intentions and objectives]{methodology_whatdoiwant} of my research actions. 

The answers to third question, \WhatShouldIDo, \gotoTextMark[determines approaches]{methodology_whatshouldido} that could help to realize the proposed aims, intentions and objectives.

\spaceHalf
\inright{map one}
Those three \guidingQuestions compose the \gotoTextMark[first map]{methodology_intro} that could be conceived as a map of \methodology tracings. In short it determines why I would like to realized research actions, what it means for me, how I approach it in order to be finally able to experience and obtain insights from the social struggle of streets in São Paulo, as part of the streets, the space this thesis is mainly lingering in.

\spaceHalf
\inright{map two}
My insights and experiences are then narrated in the second map, in form of \gotoTextMark{diaries from the streets}. Those narrations and diaries could be conceived as the empirical work of this thesis. My \narrations are mainly the product of those approaches proposed in the first section. The tracings in this second map do not follow a certain order, chronologically for instance, but are distributed randomly in space and time.

\spaceHalf
\inright{map three}
The tracings on the third map represent themes for \gotoTextMark[theorizing ]{theorizing_intro}, from what I have learned on the streets, what is transported in my narrations and diaries. \Theorizing is also the product of the approaches proposed on the first map. The third map could be conceived as a theoretic map. I conceive it as an extension of my narrations rather than an introduction or framing of the \quote{topic}. Its tracings are driven by the experiences and insights from the streets, therefore it is the last map to be constructed in this work.

\inright{layout}
After having briefly presented the thesis structure I would like to further draw on its layout. This thesis utilizes a particular text formatting and colouring style in order to visually determine the meaning of certain words and text passages.

\spaceHalf
\inright{the margin for indicating text passages}
On each page one can perceive a larger margin that should provide space for taking notes on a printed version and that also serves to indicate text passages by short descriptions

\spaceHalf
\inright{the footer links back to the table of content}
Each page contains a header and footer. The footer is composed of the current page number at its right edge and the current section name at its left edge. At the begin of the section name there is a link that leads back to the \gotoTextMark[table of content]{tocComplete}, named \goto{Start}[tocComplete].

\spaceHalf
\inright{formats and meanings}
Apart from structural page elements, words and text passages may have distinct meanings that are expressed by various styles of text formatting.

\spaceHalf
\inright{translations}
Words that need to be translated are indicated as \toTranslate[to translate]{zum übersetzten}. All translations are listed under the section \gotoTextMark[Translations]{translations}.

\spaceHalf
\inright{abbreviations}
Words that are abbreviated are indicated as \abbrNew{An Abbreviation}{AA} when they occur for the first time. Later on they are either used in their abbreviated form: \abbr{AA} or in the full form: \abbrFull{AA}. A list of all abbreviations can be found under  \gotoTextMark[Abbreviations]{abbreviations}

\spaceHalf
\inright{keywords}
Keywords are indicated as \aKeyword[How To Read]{how to read}. A complete list of keywords can be found in the \gotoTextMark[Index]{index}

\spaceHalf
\inright{footnotes}
Footnotes to certain text passages or words are simply indicated as \footnote{an example of a foodnote}

\spaceHalf
\inright{document links}
A link to another location within this document is indicated as \gotoTextMark[Introduction]{intro}. 

\spaceHalf
\inright{web links}
A link to an online sources, a website for instance, is indicated as \aLinkNew[the thesis blog ]{https://rtc.noblogs.org}. All utilized links to online resources are listed under \gotoTextMark[Links//General]{refLinks}. Web links produce a footnote with the link address.

\spaceHalf
\inright{locations links}
A link to a position at \aLinkNew[openstreetmap]{http://openstreetmap.org} is indicated as \locSe. Locations are mainly used throughout my narrations. All utilized location links are listed under \gotoTextMark[Links//Locations]{refLinksLocations}. Location links produce a footnote with the location address.

\spaceHalf
\inright{objectives}
Throughout the text I formulated proposals for objectives of my research actions. Those proposals are indicated as 

\spaceHalf

\textBoxedRoundMaxObj{proposal of an objectives}

\spaceHalf

\inright{definitions}
Throughout the text I determine what is important for my research actions. Those definitions are indicated as

\spaceHalf

\textBoxedRoundMaxDef{a definition or determination of something relevant}

\spaceHalf

\inright{questions}
Throughout the text I asked questions that should help me to structure what I am doing. Questions are indicated as 

\spaceHalf

\textBoxedRoundMaxQuestion{asking of a question}

\spaceHalf

\inright{important sections}
Sections that I would like to highlight as a reminder for instance are indicated as

\spaceHalf

\textBoxedRoundMax{something to highlight}

\spaceHalf

\inright{narrating subjective thoughts}
My subjective thoughts and feelings are indicated throughout my narrations as

\startPersonal
A personal thought or a feeling.
\stopPersonal

\spaceHalf

\inright{narrating dialogs}
Dialogues between people are indicated throughout my narrations as 

\startDialog
\tell{He}{says something}
\tell{I}{say something}
\tell{You}{say something}
\stopDialog

\spaceHalf

\inright{citations}
Citations occur throughout the whole text. Complete text passages taken from a particular source are indicated as

\startCitation
Citations occur throughout the whole text. Complete text passages taken from a particular source are indicated as this.
\stopCitation

\spaceHalf
\inright{references}
References to complete text passages are indicated as \aQuoteB{author}{year}{X} for a text reference, \aQuoteW{author}{year} for an online reference.

References to authors, titles or short text passages within paragraphs are indicated as \aQuoteInText{a short text passage}, \aQuoteInTextT{a title}, \aQuoteInTextA{an author}.

References are structured according their meaning. All bibliographic references such as books, texts, and the like are listed under \gotoTextMark[Reference//Bibliography]{refRefBib}. All references to media content such as journal and newspaper articles are listed under \gotoTextMark[Reference//Media Coverage]{refRefMedia}. All sources that I utilize to find bibliographic content are listed under \gotoTextMark[Reference//Content//Sources]{refRefBibSources}. All sources I utilize to find journal and newspaper content are listed under \gotoTextMark[Reference//Media//Sources]{refRefMediaSource}

%---------------------------------------------------
% gets only displayed in unfinished mode

\showImperfection

%---------------------------------------------------
\stopmode

\stoptext

\stopcomponent
