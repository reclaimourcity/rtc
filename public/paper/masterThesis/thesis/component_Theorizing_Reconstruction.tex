\startcomponent component_Theorizing_Reconstruction
\product product_Thesis
\project project_MasterThesis

% definitions and macros
\environment envThesisAllEnvironments
\environment envCfgThesisImages

\define[]\sectionTheorizingReconstruction{\section[theorizing_reconstruction]{Re//Construct//Themes}}

\starttext

\startmode[tocLayout]
\sectionTheorizingReconstruction

Reconstruct theorizing themes.

\stopmode

\startmode[draft]
\sectionTheorizingReconstruction

\startCitation
Political struggle is not a conflict between well defined interest groups; it is an opposition of logics that count the parties and parts of the community in different ways. 
\stopCitation






10 thesen: 10

What thus characterizes a democracy is pure chance or the
complete absence of qualifications for governing. Democracy is that state of exception
where no oppositions can function, where there is no pre-determined principle of role
allocation.

10 thesen: 12

 the 'power of the demos' means that those who rule are those who have no specificity in common, apart from their having no qualification for governing. Before being the name of a community, demos is the name of a part of the community: namely, the poor. The 'poor,' however, does not designate an economically disadvantaged part of the population; it simply designates the category of peoples who do not count, those who have no qualifications to part-take in arche, no qualification for being taken into account.

10 these: 13

The one who is 'unaccounted-for,'the one who has no speech to be heard, is the one of the demos.

10 thesen: 18

Politics cannot be deduced from the necessity of gathering people into communities. It is
an exception to the principles according to which this gathering operates. The 'normal' order of things is that human communities gather together under the rule of those qualified to rule -- whose qualifications are legitimated by the very fact that they are ruling. These governmental qualifications may be summed up according to two central principles: The first refers society to the order of filiation, both human and divine. This is the power of birth. The second refers society to the vital principle of its activities. This is the power of wealth. Thus, the 'normal' evolution of society comes to us in the progression from a government of birth to a government of wealth. Politics exists as a deviation from this normal order of things. It is this anomaly that is expressed in the nature of political subjects who are not social groups but rather forms of inscription of 'the (ac)count of the unaccounted-for.'

10 thesen: 19

There is politics as long as 'the people' is not identified with the race or a population,
inasmuch as the poor are not equated with a particular disadvantaged sector, and as long as the proletariat is not a group of industrial workers, etc... Rather, there is politics inasmuch as 'the people' refers to subjects inscribed as a supplement to the count of the parts of society, a specific figure of 'the part of those who have no-part.' 

And the politics of these categories has always consisted in re-qualifying these places, in getting them to be seen as the spaces of a community, of getting themselves to be seen or heard as speaking subjects (if only in the form of litigation); in short, participants in a common aisthesis. It has consisted in making what was unseen visible; in getting what was only audible as noise to be heard as speech; in demonstrating to be a feeling of shared 'good' or 'evil' what had appeared merely as an expression of pleasure or pain

10 thesen: 24

The essence of politics is dissensus. Dissensus is not the confrontation between interests
or opinions. It is the manifestation of a distance of the sensible from itself. Politics makes visible that which had no reason to be seen, it lodges one world into another (for instance, the world where the factory is a public space within the one where it is considered a private one, the world where workers speak out vis-à-vis the one where their voices are merely cries expressing pain). 

In contrast, the particular feature of political dissensus is that the partners are no more constituted than is the object or the very scene of discussion. The ones making visible the fact that they belong to a shared world the other does not see -- cannot take advantage of -- the logic implicit to a pragmatics of communication. The worker who argues for the public nature of a 'domestic' matter (such as a salary dispute) must indicate the world in which his argument counts as an argument and must demonstrate it as such for those who do not possess a frame of reference to conceive of it as argument. 

Political argument is at one and the same time the demonstration of a possible world where the argument could count as argument, addressed by a subject qualified to argue, upon an identified object, to an addressee who is required to see the object and to hear the argument that he or she 'normally' has no reason to either see or hear. It is the construction of a paradoxical world that relates two separate worlds. 

10 thesen: 25

Politics thus has no 'proper' place nor does it possess any 'natural' subjects. A
demonstration is political not because it takes place in a specific locale and bears upon a
particular object but rather because its form is that of a clash between two partitions of the sensible. 

A political subject is not a group of interests or ideas: It is the operator of a
particular mode of subjectification and litigation through which politics has its existence.
Political demonstrations are thus always of the moment and their subjects are always
provisional. Political difference is always on the shore of its own disappearance: the people are close to sinking into the sea of the population or of race, the proletariat borders on being confused with workers defending their interests, the space of a people's public demonstration is always at risk of being confused with the merchant's agora, etc.

10 these: 27

That the distinguishing feature of politics is the existence of a subject who 'rules' by the
very fact of having no qualifications to rule; that the principle of beginnings/ruling is
irremediably divided as a result of this, and that the political community is specifically a
litigious community

politics, along with philosophy's effort to resituate politics under the auspices of this law. The Gorgias, the Republic, the Politics, the Laws, all these texts reveal the same effort to efface the paradox or scandal of a 'seventh qualification'  make of democracy a simple case of the indeterminable principle of 'the government of the strongest,' against which one can only oppose a government of those who know [les savants]. These texts all reveal a similar strategy of placing the community under a unique law of partition and expelling the empty part of the demos from the communal body. 

10 these: 33

The thesis thus amounts to asserting that the logical telos of capitalism makes it so that politics becomes, once again, out dated. And then it concludes with either the mourning of politics before the triumph of an immaterial Leviathan, or its transformation into forms that are broken up, segmented, cybernetic, ludic, etc... -- adapted to those forms of the social that correspond to the highest stage of capitalism. It thus fails to recognize that in actual fact, politics has no reason for being in any state of the social and that the contradiction of the two logics is an unchanging given that defines the contingency and precariousness proper to politics. Via a Marxist detour, the 'end of politics' thesis -- along with the consensualist thesis -- grounds politics in a particular mode of life that identifies the political community with the social body, subsequently identifying political practice with state practice. The debate between the philosophers of the 'return of politics' and the sociologists of the 'end of politics' is thus a straightforward debate regarding the order in which it is appropriate to take the presuppositions of 'political philosophy' so as to interpret the consensualist practice of annihilating politics.

10 thesen: thesis 6

If politics is the outline of a vanishing difference, with the distribution of social parts
and shares, then it follows that its existence is in no way necessary, but that it occurs
as a provisional accident in the history of the forms of domination. It also follows
from this that political litigiousness has as its essential object the very existence of
politics.

10 thesen: these 8

The principal function of politics is the configuration of its proper space. It is to disclose the world of its subjects and its operations. The essence of politics is the manifestation of dissensus, as the presence of two worlds in one

\showImperfection

\stopmode

\stoptext

\stopcomponent
