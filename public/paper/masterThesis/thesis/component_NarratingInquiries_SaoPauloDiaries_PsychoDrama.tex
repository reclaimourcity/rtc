\startcomponent component_NarratingInquiries_SaoPauloDiaries_PsychoDrama
\product product_Thesis
\project project_MasterThesis

% definitions and macros
\environment envThesisAllEnvironments
\environment envCfgThesisImages

\define[]\subsectionSaoPauloDiariesPsychodrama{\subsection[narratinginquiries_saopaulodiaries_psychodrama]{Psycho//Drama}}

\starttext

\startmode[tocLayout]
\subsectionSaoPauloDiariesPsychodrama
What is Psychdrama?
\stopmode

\startmode[draft]
\subsectionSaoPauloDiariesPsychodrama
What is \Psicodrama?

\startCitation
\Psicodrama is a form of research or therapy base on three roots: theatre, psychology and sociology. And, in a way it is a form to come closer to each other, to do something with what one encounters, to liberate oneself, to leave the stereotypes. [...] The first idea is citizenship, to discover through encounter that there exist the other, people with which one can close a pact, like the friends of the neighbourhood that meet in order to take care of that abandoned space there, or to demand more public transport. The second is to show the people to leave the \quote{non-place}. You enter the Metrô, you sit down there at a place next to another person, in front of each another and once you leave you do not even know if the other was a teen, a man, a woman. It is like as if the public space that is full of people would be a \quote{non-place}, where the people remain inside a bubble, isolated from one another. What we are trying to show is that this \quote{non-place} mainly present in the big cities can be transformed and that discovering the other is always interesting, because life is really evolving in the encounter, in the \quote{intermediate}. \aQuote{Cesarino}{2009}
\stopCitation

This citation is taken off the 2009 October issue of the street journal Ocas. \Va gave me a copy of it. One day \Va invited me to join him at a Saturday morning. He said that he is going since years, since seven years exactly, to the \Psychodrama. I have never heard of \Psychodrama but \Va said I will find out if i join him.

So here we are, its Saturday, we met 11 o'Clock at the \locCCSP at \locLiberdade. the \abbrNew{Centro Cultural de São Paulo}{CCSP}\toTranslateT{Centro Cultural de São Paulo}{Cultural Centre of São Paulo} is a huge complex offering spaces for concerts, cinemas, discussions, reading, books and breakdance on its wide hallways. Here we are in one of those spaces, me and several others sitting on the ground level, on a kind of open gallery, watching down upon the lower floor below us. There, \Va is already active, talking with his friends while more people are arriving, some of them heading immediately to the seats, others are entering the main floor, young and old, all genders, \Va the only from the streets.

The \Psychodrama has already a long history at the \abbr{CCSP}. It started in 2003 and takes place since then almost every Saturday late morning. Its free for everybody to take part. For me, the \Psychodrama is emancipatory because all those that take part rule. There exist no discrimination, no matter from where the people are coming, what their conditions are. 

\imgPsychoDramaOne

After some time, the \Psychodrama starts. This time as well as at the following Saturday's when I join \Va, the people suggest the topic of the day. Today's topic will be the soon coming elections. People build groups of interest to discuss what they would like to say about the coming elections, about the politicians and how they are going to present their thoughts and feelings. Finally, each group came up with a plan, ready to present it as an improvised piece. \Va plays today the master of puppets behind Dilma, the soon to be elected new president of Brasil. This whole scene was so much fun. In a sense it hit exactly the mark. 

After the play of each group, people, on the scene but also the audience is being asked about what they perceived, how they perceived, what they think about the just see, how it fits with their opinion, where they do not agree. The \Psychodrama offered much space for experimentation. 

\startPersonal
I do not recall exactly what the others have done because I had to think about what I just saw, about this wonderful way of talking to each other, taking part in formulating decisions, something that has also been criticized as not existent by one of the groups. 
\stopPersonal

\imgPsychoDramaTwo

\addReference
{
Cesarino, A.C., 2009. Psicodrama para todos.
}

%---------------------------------------------------
% gets only displayed in unfinished mode

\showImperfection

%---------------------------------------------------
\stopmode

\stoptext

\stopcomponent
