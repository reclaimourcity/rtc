\startcomponent component_WaterfrontsEverywhere
\product product_TheRightToTheCity
\project project_MasterThesis

% definitions and macros
\environment envCfgPaperLanguageEn
\environment envRttcUrl
\environment envRttcText

\starttext

\section[waterfront]{Waterfronts everywhere}

Other \inright{\waterfront \revitalization can be observed globally in Amsterdam, Barcelona, London, Venice, Sidney, Toronto, Bangkog, Honh Kong, Shanghai, Tokyo, Cape Town, Buenos Aires (Bruttomessi, 1993), Havana or Bilbao (Marshall, 2001)} large scale urban development schemes have been adopted worldwide as well (for European projects, see for example {\em Les Cahiers no 147(2007)} ), such as the so called \waterfrontdevelopment in port cities (Butuner, 2006, p.3), the \revitalization of degenerated port areas (or \brownfields, the general term for degenerated industrial areas). \Waterfront \revitalization is used here as emblematic example for a \quote{ genuine urban revolution} (Bruttomesso, 1993, p.10). Many of the world's contemporary large and big cities are port cities due to the former importance of a settlement's proximity to the water and the port as trigger for the emergence of transportation, trading, manufacturing or military activities and by that also for physical/geographical expansion and residential growth. The contemporary transformation of the waterfront, which represents the interface between the city and the water, is therefore a new  \quote{transformation in [the city's] physical layout, function, use and social pattern} (Butuner, 2006, p.3).

\Revitalization of those often run down or abandoned industrial \waterfronts and port areas that are the last evidence of the downfall of (heavy) industries and manufacturing in inner city regions, serves several purposes: from a pragmatically point of view, degenerated inner city \brownfields (from ports to fabrics) are often the last vast and unused spaces available in inner city regions. \inright{waterfront revitalization} The \waterfront and port areas are of special interest due to their prime location on to the water (rivers or the sea) and their proximity to the inner city. A revitalized and spectacular \waterfront has therefore the potential to boost the local tourism industry, to control the resident and living structure by attracting a more exclusive class of citizens, businesses and services but also to improve the \quote{image} of the city on national and global scale and to position the city's \quote{label} in the {\em globalcityness} competition (Butuner, 2006, p.3). Hence, a \revitalized \waterfront is evidence of the  transformation of an industrial city to a post-industrial city, whose new mode of production is based on services, culture or tourism. 

Since its early realizations in (post industrial) cities such as Boston or San Francisco (Marshall, 2001, p.118, 119) in the late 60's, early 70's of the 20th century, \waterfront \revitalization has been used as a generic and standardized tool (Butuner, 2006, p4 and Bruttomessa, 1993) that has been and still is applied in many cities in the global north and south.

Even \inright{effects of \waterfrontdevelopment} though the local context and times always differ, similar transformations in a city's physical and social structure can be observed as a result of \waterfront \revitalization:

{\bf social \segregation}

due to the fact that the revitalized waterfront may become an exclusive life, work, leisure and tourism zone, that relies on a correspondingly exclusive real estate, consumption and service infrastructure, accessible only for those who can afford it (Höhmann, 2010; Niedt, 2006, p.110) 

{\bf social \exclusion and \surveillance} 

in order to guarantee that unwanted subjects, from homeless people to skaters, are not frequenting the area and disturb its shiny and exclusive image (IVV, 2010, p19).

{\bf \gentrification} 

of the revitalized waterfront, the displacement of current residents (if existent) by a wealthier class due to increased living costs (Bischof, 2007, p.64; Hogan, 2006, p.31, Niedt, 2006, p.110) 

{\bf \newbuildgentrification}

that affects neighbouring quarters. No direct displacement of (old) residents takes place on the previously not inhabited \waterfront but the new lifestyle that emerges after \revitalization may leak to neighbouring quarters and cause \gentrification there (Davidson and Less, 2005 p.1168, 1169).

\stoptext

\stopcomponent
