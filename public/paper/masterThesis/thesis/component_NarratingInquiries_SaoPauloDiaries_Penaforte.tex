\startcomponent component_NarratingInquiries_SaoPauloDiaries_Penaforte
\product product_Thesis
\project project_MasterThesis

% definitions and macros
\environment envThesisAllEnvironments

% text specific definitions and macros
\environment envPenaforteUrl
\environment envPenaforteLinks
\environment envPenaforteText
\environment envCfgThesisImages

\define[]\subsectionSaoPauloDiariesPenaforte{\subsection[narratinginquiries_saopaulodiaries_penaforte]{Penaforte}}

\starttext

\startmode[tocLayout]
\subsectionSaoPauloDiariesPenaforte

Diary of visiting Penafort on a Sunday in São Paulo.

\stopmode

\startmode[draft]
\subsectionSaoPauloDiariesPenaforte

\Gibson is a friend that I got to know on a Sunday in July, by coincidence, at the entry of \locAy{Ay Carmela}. At this day I met with \Ma there and \Gi suddenly passed by with a friend of him. They were heading towards \locParqueDomPedro, but \Ma and \Gi are friends and so we started talking and \gotoTextMark[headed through the neighbouring streets]{narratinginquiries_saopaulodiaries_sundaystreetscentregibson} later on that afternoon.

\Gibson doesn't live on the streets (he said that he stayed only one month on the streets, back in the 80ties) but normally spends his time with the people at \locSe, \locLargoSaoFrancisco and other central areas. He said that he is an \quote{\em informal streetworker}. He said that he likes to talks and discuss with the \aKeyword{people in street situation} and tries to (re)establish their self confidence, tries to give them back the feeling that they are humans and just listens to their stories because not many others are doing so.

He said that he is doing what he is doing because one of the realities of the streets is that the realities of the \aKeyword{people in street situation} are nearly \aKeyword[invisible]{streets+invisibility} for the eyes of the always running citizens, although their presence is massive, especially in the \locCentre, but throughout São Paulo as well.

After I met him the first time we didn't see for some weeks but ran into each other again at \locSe one day at late afternoon, when \Ju and I were just coming from \locMissingSrc{OCAS}, our destination \locRepublica. 

He was sitting with two friends on a bench. One young boy, below twenty, and an older guy with a long grey beard, couldn't speak fluently anymore. Their appearance, hands, faces, foots, shoes, clothes, marked by the streets. We embraced us all like family and friends. \Gi introduced us and said that he was spending this afternoon there, at \locSe, together with his friends, talking and listening. 

He then invited me to join him for lunch on the following Sunday at 
\toTranslate[Refeitorio Communitario]{community refectory}\aLinkContentNewF[Refeitorio Communitario]{http://bit.ly/mSREpy} at \aNewLocation[Consolocão]{http://osm.org/go/M@y3X2vUO--}, commonly called \Penaforte. 

\Penaforte is one of those places that exists throughout the city where receive, among others, free breakfast or lunch, sometimes opened every day, sometimes at particular days only. I several times with different people to different places, scattered over the centre. This time, with \Gi we planned to go there for lunch on Sunday. Our Sunday was special because a \toTranslate[Grupo de Sopa]{soup kitchen} from the \toMark{welche?} church was supposed to cook then. Today's group is coming once a month, in order to prepare food for the people. 

So, we met on the next Sunday at around 11:30 a.m in front of\Penaforte, at the margins of \locMissingSrc{Bela Vista} and \locMissingSrc{Consolacao}. \locMissingSrc{Penaforte} is not far from the centre, from \locRepublica or \locMissingSrc{Anhangabú}, maybe a 15 minutes walk. On the map it is even closer to \locMissingSrc{Avenida Paulista}, the avenue that symbolizes the high-speed of city, its wealth and informality. I guess I cannot properly describe\AvPaulista in just a few words, but the reality that is visible just starting a few meters from here, climbing upwards through a steep and wealthy belt of high-risers is quite different than ours here on the streets, waiting for lunch.

Its already hot, like 30 degrees or more, and even though the doors are still closed and lunch is going to be served that day around 01:00 p.m, a huge crowd of people, probably 200+, has already gathered and more are arriving continuously, waiting in front of the building, occupying the side-walk and parts of the street. The crowd consists mainly of men, a few women, I saw perhaps 5, and some homosexual men.

Right at the entrance to the street, a 3 or 4 floors high \aKeyword[abandoned and barricaded building]{streets+places+institutional+abandoned+building} starts to occupy the space till the house that hosts\Penaforte. Such a long queue of people without house are waiting for food in front of an abandoned building.

%\startARemark{bild vom dem leeren haus bei penaforte}
%\stopARemark

When \Gi came he embraced some people that he knew, we talked a little in the shadows of the buildings on the other side of the street, waiting. The general atmosphere is rather depressed, people are waiting, alone, or with their own small crowd, sometime talking, often just silent. The people that are occupying the margins of the streets are laying on \toTranslate[Papelão]{cardboard}, the others sitting or standing. I see many ordinary looking people, normal clothes rather then run down, one guy wearing a suit, appearing more business like. \Gi told me that most of the people are coming from \Albergues and that those in situation of the streets \bracket{ day and night} are only the minority here. I did not ask why... On \refMissingSrc{other occasions} I learned what the it means to live in \Albergues. 

After a while, maybe 20 minutes after we met, two cars of the \toTranslate[Guarda Civil Municipal]{municipal civil police}. The \abbrNew{Guarda Civil Municipal}{GCM} appears, stops, and 4 police men and women get out the cars. The two cars pull in at the other side of the street, while the 4 officers straightly approach those people laying on the cardboard, ordering them to leave the street immediately to return on the side-walk, which all of them do after some discussion with those officers.  This street is little frequented by cars on that Sunday and the side-walk is already crowded with people waiting.

\imgPenafortePoliceOne

After that action, the officers return to the other side of the street where one of their cars is still parking, waiting in the shadow of the adjacent building while observing the scene. The other police car is passing by once in a while. I asked \Gi why they are at all because for me it appeared solely as a demonstration of power sending the guys away from the street. \Gi said that they always stop once they pass by because they know that during that time, many people are here and they looking for \quote{troublemakers}. 

\startARemark{The \abbr{GCM} is one of actors in the São Paulo that are actively engaged in oppressing} people in street situation \refMissingSrc{referenzen zu berichten über die polizei, condepe, park, policia militar, etc}
\stopARemark

\imgPenafortePoliceTwo

Then suddenly, two men from the crowd near the entrance, start to argue, first shouting loud, then starting to fight, mainly pushing each other around. The reason is not obvious because \Gi and I also stay on the other side of the street, talking. The waiting police is intervening immediately, trying to separate them with batons in their hands, shouting on them and pushing them away from each other. People from inside the house are joining the scene as well, trying to calm down the two guys as well as the police. 

\Gi said that its sad to see that the people are still fighting against each other although they are all in similar situations. In this case the fight was probably for a better position in the line. 

After a while, the scene settled down again, the police returned to other side of the street, continuing observing the scene and finally the doors are opened and the waiting people are entering the building. Lunch is ready. 

We wait a bit and follow once everybody is inside. We enter a huge hall, mainly equipped with rows of tables and chairs, crowded immediately and no space for everybody to sit. We are standing and waiting as well. Even though many people are inside, the atmosphere is still very dense and depressing, people are not talking much, just waiting for their food. 

At \Penaforte one must be registered in order to receive a meal. Everyone is supposed to possess a piece of paper that proves that he or she is registered. This paper must be shown at a counter at the entrance, additionally, everybody puts his or her name on a list of attendance. Among food, the place offers medical support once a month, has a small library, a small basketball/football field in the backyard, and offers several empowerment \aKeyword[workshops]{streets+actions+workshop+artesian craft} such as creation of artesian crafts.\Penaforte is existing since 2000 and is organized and maintained by \gotoTextMark[RedeRua]{mark missing} in cooperation with the \refMissingSrc{Prefeitura de São Paulo} \aQuoteWA{RedeRua}. 

On this Sunday, the people formed groups in order to receive their lunch. When I was there with \Ju \gotoTextMark[some days before]{mark missing}, the lunch was distributed directly in the huge hall to everybody, this time, each group receives their lunch in the basement floor, accessible for all over a long ramp. Each group has to wait until the previous one has finished. 4 or 5 groups have been formed, which altogether rush through their lunch in little more than one hour. Once a group is finished they almost immediately leave the place.

Once we finished, everybody leaves soon, just a few stay a bit longer, talking to the assistants and volunteer workers. \Gi and I also stay longer, talking a young guy that lives in an \Albergue close by and looking at the newly published \gotoTextMark[Guia da Rua]{narratinginquiries_alternativecontent_guiadarua} whose aim is to raise awareness about the legal rights of the people in street situation. 

After some time we leave as well, together with the guy, in order to take a walk to the centre, chatting about the coming elections \bracket[which has been held last Sunday]. For the guy, none of the candidates are an option because the situation on the street never improved during the last years, instead, it became more and more difficult, with more and more people entering the streets, thus he don't believe in any of the politicians and prefers not to vote at all. He is also upset by not having access to information concerning the streets. He says that he stays in an \Albergue in the central area and do not know when a new  \Albergue gets opened or closed, this kind of information never arrives at him. He said that he heard about a new \Albergue at \locMissingSrc{Lapa} weeks after its opening and if he had known he would have tried to get a place there because he cannot stand the people at his place, to many of them are aggressive. From there we entered the question of mobile phones and the ridiculous high rates for calls imposed by any operator in Brazil, which simply doesn't allow him to afford and use one \footnote{for statistics about phone operators and prices take a look at \aQuoteW{Horst}{2009}\aLinkContentNewF[New Media Practices in Brazil]{http://bit.ly/olYxid}}. The young guy left us at some corner and \Gi and I were wondering how it came that he had to enter an\Albergue. During our walk \Gi was telling him that he should keep his spirit up and that his situation is not forever. What else could he do?

Finally we arrived at \locMissingSrc{Anhangabú} and split up to go our own ways for that day. \Gi was always busy and organizing things and was heading towards some appointment that day. Before we split we made an appointment to \gotoTextMark[meet for lunch]{narratinginquiries_saopaulodiaries_padredecha} at the following Tuesday at \locMissingSrc{Padre de Chá}, a place run by the Franciscan church, close to \locSe.

\addReference
{
Horst, H., 2009. New Media Practices in Brazil, Part V: Mobile Phones. {\em Futures of Learning}. Available at: \goto{\hyphenatedurl{http://futuresoflearning.org/index.php/Firda\_08/comments/new\_media\_practices\_in\_brazil\_part\_v\_mobile\_phones/}} [url(http://bit.ly/olYxid)] [Accessed November 2, 2010].
}
\addReference
{
Rede Rua, 2010. Refeitório Comunitário do Povo da Rua. {\em Associação Rede Rua}. Available at: \goto{\hyphenatedurl{http://www.rederua.org.br/index.php?option=com\_content\&task=view\&id=4\&Itemid=7}} [url(http://bit.ly/mSREpy)] [Accessed November 2, 2010].
}

%---------------------------------------------------
% gets only displayed in unfinshed mode

\showImperfection

%---------------------------------------------------

\stopmode

\stoptext

\stopcomponent