\startcomponent component_Objectives
\product product_TheRightToTheCity
\project project_MasterThesis

% definitions and macros
\environment envCfgPaperLanguageEn
\environment envRttcUrl
\environment envRttcText

\starttext

\chapter[objectives]{Objectives}

The aim of this paper is to give an overview about the \rttc. The \rttc is a concept that is part  of the discourse about contemporary urban development with its accompanied and resulting social inequalities. It is referred to in institutional discourse on international level by UN-Habitat/UNESCO, on city level in form of statutes that intends to grant more rights to city dwellers (such as in Brazil and Mexico) but emerges also in various forms in the struggle of (urban) social movements and grassroots groups all over the world. Therefore it seems to be interesting to take a closer look to its roots that has been initially planted by the French sociologist and philosopher \from[wikiEnLefebfre].

Due to its complexity, it is not the aim of this paper to comprehensively lay out the \rttc to its full extend and in all its possible forms and facets. The paper focus on the basic concepts and takes a further look at contemporary interpretations of scholars such as \from[wikiEnHarvey] and \from[wikiEnCastells]. Hence, the \rttc is used here primarily to describe the development of the \quote{capitalist city}, its negative effects for its inhabitants and the potential utopias that could be made possible, in contrast to other approaches such as the \quote{creative city} developed by urban theorists like \from[wikiEnFlorida]. 

In order to relate the \rttc to concrete urban development processes, another concept, the \globalcity, is going to be examined. The \globalcity serves as the frame to take a look at four concrete urban development projects in various cities (two German, one Turkish and one Brazilian city), that fit all into the category of \waterfront \revitalization. The selected examples shall conceptualize the impacts of large scale urban development projects on the urban and social structure of a city. Similar to the \rttc, the \globalcity can examined from different points of views. Here, it is intended to focus on the global city as driver of globalization which will be related directly in the notion of the \quote{capitalist city}.

The paper will finally return to an analysis of the \rttc related to the \worldcity with its given examples and its potential meaning in terms of citizenship, the shaping of the city and the production of urban space, driven from below.
\stoptext

\stopcomponent

