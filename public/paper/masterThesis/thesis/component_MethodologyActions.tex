\startcomponent component_MethodologyActions
\product product_Thesis
\project project_MasterThesis

% definitions and macros
\environment envThesisAllEnvironments

\define[]\sectionActions{\section[methodology::actions]{Grounded//Participant//Observ[er][ation]//Action}}

\starttext

\startmode[tocLayout]

\sectionActions

This chapter finally describes the applied practical and theoretical approaches.

\stopmode

\startmode[draft]

\sectionActions

\startKeywords 
participant observation, participating observer, grounded theory, action research 
\stopKeywords

After all those reflections on attitudes, a thesis self conception and determinations of \quote{relevance} aspects, I would like to describe those research action principles that has been followed in São Paulo and guided me through the writing of this thesis.

As already mentioned in the introductory chapter \refMissingSrc{Methodology Introduction}, I came to São Paulo without prior knowledge of the city and without concrete but diffuse ideas about the purpose of my stay. So to say, the city drove me into the direction which is finally illustrated by this thesis. Most experiences and information approached me in a non-systematic manner because I was not looking for particular events nor situations or people, but tried to absorb and keep hold of everything.

\spaceHalf

\inright{participant observation (PO) }
For me, this premise seems optimal for conducting \aKeyword{Participant Observation }(PO). Even though PO serves as the basic framework for realizing research action(s), it does not appear to completely grasp the actions intention and their realization. As explained in more detail later on, PO can be seen as a method to \bracket{passively} observe scenes and situations. Therefore I would like to extend the concept of Participant Observation with the concept of \inright{participating observer} the \aKeyword{Participating Observer}, who actively participates in concrete action(s) and events as well, and by doing so, has an active impact on the form of those observation(s) that he or she wants to keep note of. 

\startARemark
bin noch nicht sicher ob Grounded Theory teil der arbeit ist
\stopARemark

\inright{grounded theory (GT)}
A portion of \aKeyword{Grounded Theory} (GT) is apparent here as well, because GT does not pre-suppose anything when starting research and incorporates all available types of information in the formulation of a research objective, thus in a sense, it is driven by (or grounded on) every imaginable type of action(s) and observations, on the streets, to the newspapers and the internet.

Finally, I would like to fuse those three approaches, \aKeyword{Participant Observation}, \aKeyword{Participating Observer} and \aKeyword{Grounded Theory}, under the umbrella of \aKeyword{Action Research} (AR). The intentions and objectives inherent to AR are to a certain extend the intentions and objectives of this thesis. Especially the contention of the notion of \quote{knowledge} and its \quote{production}, that is articulated through AR, match the defined thesis' attitude to upgrade \quote{universally accepted} knowledge production mechanism driven by scholarly and academic discourse with the knowledge produced by the people, on a genuine participative and emancipatory manner. Hence for me, AR is  the hook for practical and theoretical research action(s).

The next section \refMissingSrc{Action Research chapter} is supposed to shed light on the concept of Action Research.

\stopmode

\stoptext

\stopcomponent
