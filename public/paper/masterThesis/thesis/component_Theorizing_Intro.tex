\startcomponent component_Theorizing_Intro
\product product_Thesis
\project project_MasterThesis

% definitions and macros
\environment envThesisAllEnvironments
\environment envCfgThesisImages

\define[]\chapterTheorizing {\chapter[theorizing_intro]{Theoretical Themes}}

\starttext

\startmode[tocLayout]
\chapterTheorizing

Structure of concepts of the determined theoretical themes.

\stopmode

\startmode[draft]

\imgCidadaniaGraffitiMohino

\imgMncrDireitoACidadeOne

\imgMncrDireitoACidadeTwo

\imgMncrDireitoACidadeThree

\chapterTheorizing

Structure of concepts and determined theoretical themes.

%----------------------------------------

This final part is first of all an invitation for further elaboration and criticizing. 

It is formulated as proposal about citizenship, participation, the right to the city, self-determination, autonomy, from a standpoint that intends social transformation. 

It contest the very same themes loaded with meaning from the standpoint of public institutions, the state, or international agencies, thus the actual power holders, and tries to reload them with emancipatory content.

By reloading I mean to inject other meaning into those words that are \bracket{probably} familiar to us and which already evoke a particular meaning, often determined by the discourse introduced by the power holders or the discourse that is hold by the society.

In a sense it could be seen as a critique of the current status quo \bracket{the economic system, the society, the state, the city} that discriminates those that are differing from or dropping out the public discourse, that do not want or cannot act as it is expected in terms of social norms, duties and obligations. 

But rather then merely criticising by producing just another set of words and phrases that say what has already been said many times, its aim could be to draw on what people are self-determinately organizing and practising already, to draw on the self-conceptions and attitudes that are already formulated, practised, reflected upon and overthrown if considered inappropriate. 

Thus the aim could be to make a suggestion and to see where it could lead to, how it could contemplate other proposals that are aiming for emancipatory social transformation as well but are using a different language, different sets of concepts, that are formulated from different standpoints.

I think a necessary step is the deconstruction of the commonly accepted meaning of words to see those power hierarchies that produced those commonly accepted meanings and once this is done, a different sense can be injected. I would not like to invent new words or themes because in my opinion everything shall remain understandable and should not be hidden behind a complex web of language and words that only specialists familiar with the topic can understand.

Therefore I am intending to use several concepts that I perceive are interrelated and that are elements of the demands and practice of social struggles in São Paulo. 

One practice of social struggle is is self-determination and autonomy. Struggle and actions are self-determinately organized in a participatory manner. Participation in the sense of taking part, as a way of creating new political spaces apart from the invited and de-politicized participatory spaces granted by institutional and other power agents. Those new spaces created through social struggle are the spaces for experimenting and learning, of organizing and practising different forms being together, of organizing life, production, work, differently from the current form which merely exclude and oppress those that are struggling. 

Thus organized and participatory struggle for a social transformation is not a coping mechanism of the marginalized in order to survive and to carry out what the society in general is supposed to carry out \bracket{or the welfare state for instance}. Participatory struggle is self-determined and aims to create new political space where none have been existing earlier, or the transformation of existing political spaces. 

The struggle for new political space is emancipatory because in those newly created spaces, participation is the principle for shaping it, to be heard, to respect the difference of the others and where ones own difference is accepted, where participation in the issues that affect ones own life is carried out in a just manner, not imposed through a power hierarchy and oppression. Minimum and Maximum Difference is what Henri Lefebvre imagines when he theorizes about the city, and the right to the city. 

Minimum Differences represents the current status quo, in society and the city. The city becomes homogenized, locally and globally if compared to other cites. Unwanted subjects that are considered incompatible are drawn out of neighbourhoods, central districts, even from the peripheries of the city. They cannot be assimilated in order to reach homogeneity.

Maximum Difference is what could be aimed by emancipatory struggle. Maximum Difference means respecting the others in their difference and being respected. Maximum Difference shall not be confused with individualism because Maximum Difference means constant social exchange and just negotiation with the others in order to determine the individual Maximum Difference.

Thus Maximum Difference is one of the bases the Right to the City is grounded on. Another base are the concept of spaces. Spaces have already been mentioned at several occasions \refMissingSrc{text mark mentioning spaces}, the city as lived space of the society, of the individual.


The city and thus the society it represents is supposed to be equalized in the sense that any difference between the single individuals is eliminated. In São Paulo and elsewhere worldwide those effects can be seen by uniform neighbourhoods, \bracket{new build} gentrification \bracket{at \refMissingSrc{Luz} for instance} where the old inhabitants are successively replaced by the city's ZEIS prorgam and the crack addicts expelled to more remote areas in order to improve the areas reputation and to construct a new upper class residential and leisure area.


De//Construction

Re//Construction

\startARemark{Struktur und Inhalt des Theorizing Kapitels}
\stopARemark

\inright{\aKeyword[Citizenship]{theorizing+citizenship}}
\aQuoteB{Arnstein}{1969}{216-224}\aQuoteInTextT{a Ladder of Citizenship Participation}

There is a critical difference between going through the empty ritual of participation and having the real power needed to affect the outcome of the process. 

The justification for using such simplistic abstractions is that in most cases the have-nots really do perceive the powerful as a monolithic "system," and powerholders actually do view the have-nots as a sea of "those people," with little comprehension of the class and caste differences among them.

citizenship exists only insofar as it is enacted, and its emerging figures have to do with empowerment strategies exercised in and from the cultural sphere. What the new social, ethnic, gender, gay and lesbian, religious or ecological movements demand is not only ideological representation but also socio-cultural recognition. They seek to become visible in their difference as citizens. This opens up a new mode for the political exercise of their rights, since this new visibility catalyzes the emergence of new political subjects. This was the subject visualized by feminism when it subverted the Left's profound machismo with its slogan: “the personal is political!” which came to embody both a sense of injury and victimization and a sense of recognition and empowerment.

\aQuoteW{Barbero}{20}\aQuoteInTextT{Staging Citizenship: Performance, Politics, and Cultural Rights}

citizenship exists only insofar as it is enacted, and its emerging figures have to do with empowerment strategies exercised in and from the cultural sphere. What the new social, ethnic, gender, gay and lesbian, religious or ecological movements demand is not only ideological representation but also socio-cultural recognition. They seek to become visible in their difference as citizens. This opens up a new mode for the political exercise of their rights, since this new visibility catalyzes the emergence of new political subjects. This was the subject visualized by feminism when it subverted the Left's profound machismo with its slogan: “the personal is political!” which came to embody both a sense of injury and victimization and a sense of recognition and empowerment.

This is the only way we will escape the illusory quest for the reincorporation of alterity into some unified whole, be it nation, political party, or religion. Citizenship rights, those rights exercised today by the different cultural communities that constitute a nation, will then take center stage. This is the new value that attributes the human universality of rights to the specificity of its very diverse modes of perception and expression.

\spaceHalf

\inright{\aKeyword[Participation]{theorizing+participation}}
\aQuoteB{Arnstein}{1969}{216-224}\aQuoteInTextT{a Ladder of Citizenship Participation}

The poster highlights the fundamental point that participation without redistribution of power is an empty and frustrating process for the powerless. It allows the powerholders to claim that all sides were considered, but makes it possible for only some of those sides to benefit. It maintains the status quo.

\aQuoteB{Ackerly}{2000}{43}\aQuoteInTextT{Political Theory and Feminist Social Criticism}

According to Estlund, deliberative democracy requires only competent voters (1997: 185) and certain ``social and structural circumstances'' (1997: 190) which include allowing participation by all, a focus on common not individual or group interests, a ``shared conception of justice,'' fair evaluation of arguments, knowledge of one another's views, pluralism in life experience and culture, and the welfare of all participants suf®cient to enable people to participate in public deliberations (1997: 190±191).

\aQuoteB{Ackerly}{2000}{66, 67}\aQuoteInTextT{Political Theory and Feminist Social Criticism}

Like much of the tradition of feminism, its function is more broad than criticism about women's social standing. Third World feminist social criticism is social criticism. Moreover, though practically grounded and thus realizable, it is also theoretically useful. In fact, it offers a solution to the practical and theoretical problems deliberative democratic theorists had in describing principles and procedures for deliberation that were informed and inclusive, with broad participation without the tyranny of the majority,
the tyranny of the method, or the tyranny of the meeting.

Finally, in the real world, people don't want to and don't have time to participate in deliberations to the extent deliberative theorists would require.49 Thus, the theory that relies on participation (2) needs to explain either why people will participate willingly or why their lack of participation doesn't matter for the theory.

\aQuoteB{Ackerly}{2000}{115, 135}\aQuoteInTextT{Political Theory and Feminist Social Criticism}

Participation ± to participate freely, fully, and equally in the full range of community life, including its familial, social, economic and political decision making

For some people, effective participation in deliberation itself is ful®lling (regardless of particular successes or failures). But as discussed in chapter 2 for some people deliberative fora create obligations not opportunities. Children, family illness, and economic demands are possible competing obligations. Spending time in remote places and working uninterrupted are possible competing desires. If people are not participating because competing obligations are inhibiting their participation, critics need to consider the competing obligations as sources of inequality.

\spaceHalf

\inright{\aKeyword[Max Difference]{theorizing+right to the city+max difference}}
\aQuoteW{Barbero}{2009}\aQuoteInTextT{Staging Citizenship: Performance, Politics, and Cultural Rights}

The visibility of the Other—and every difference is an opportunity for dominance in a class-based society—together with the diversity of each contested identity today (contested not only in relation to other identities but in relation to itself) is a constitutive part of the recognition of rights. This is expressed in the phonetic similarity and semantic articulation of visibilidad (visibility) and veedurías (community oversight committees): those practices of investigation and intervention by citizens in the public sphere. 

\aQuoteB{Ackerly}{2000}{192}\aQuoteInTextT{Political Theory and Feminist Social Criticism}

Mouffe's alternative model of citizenship recognizes citizenship as neither the source of one's identity nor an aspect of one's identity. Instead identities are determined through interaction with others and may vary according to context and those present (1992: 326). Thus, questions of identity are deliberatively determined, not given and static. Finally, Mouffe's view of citizenship has implications for community and the boundaries between public and private activity. In the context of interaction with others, citizens negotiate the appropriate boundaries of public and private, ``because every interaction is private while never immune from the public conditions prescribed by the principles of citizenship'' (1992: 325). 

\spaceHalf

\inright{\aKeyword[Autonomy and Self-Determination]{theorizing+self determination}}
\aQuoteB{Castoriadis}{1991}{163-63 and 316}\aQuoteInTextT{Autonomy}

As a germ, autonomy emerges when explicit and unlimited interrogation explodes on the scene – an interrogation that has bearing not on the “facts” but on the social imaginary significations and their possible grounding. This is a moment of creation, and it ushers in a new type of society and a new type of individuals. I am speaking intentionally of germ, for autonomy, social as well as individual, is a project. 

The rise of unlimited interrogation creates a new social-historical eidos: reflectiveness in the full sense, self-reflectiveness, as well as the individual and the institutions which embody it. The questions raised are, on the social level: Are our laws good? Are they just? Which laws ought we to make? And, on the individual level: Is what I think true? Can I know if it is true – and if so, how?... Autonomy comes from autos-nomos: [to give to] oneself one’s laws... Autonomy does not consist in acting according to a law discovered in an immutable reason and discovered once and for all. It is the unlimited self-questioning about the law and its foundations and the capacity, in light of this interrogation, to make, to do, and to institute [and therefore also, to say]. 

Autonomy is the reflective capacity of a reason creating itself in an endless movement, both as individual and social reason... If the autonomous society is that society which self institutes itself explicitly and lucidly, the one that knows that it itself posits its institutions and its significations, this means that it knows as well that they have no source other than its own instituting and signification-giving activity, no extrasocial guarantee. We thereby encounter once again the radical problem of democracy. 

Democracy, when it is true democracy, is the regime that explicitly renounces all ultimate guarantees and knows no limitations other than its self-limitation... This amounts to saying that democracy is the only tragic political regime – it is the only regime that takes risks, that faces openly the possibility of its self-destruction.

Cornelius Castoriadis, “Power, Politics,
Autonomy” and “The Logic of Magmas
and the Question of Autonomy,” both
in The Castoriadis Reader [Oxford UP,
1991], pp. 163-63 and 316.



%----------------------------------------

\startARemark{kommt noch ins methodology kapitel: experience - identity}
\stopARemark

\spaceHalf

\inright{\aKeyword[Identity]{identity}}
\aQuoteW{Barbero}{20}\aQuoteInTextT{Staging Citizenship: Performance, Politics, and Cultural Rights}

Identity is not, then, what is attributed to someone by simply belonging to a group, but rather the narration of what gives meaning and value to the life and identities of individuals and groups.

%----------------------------------------

\startARemark{kommt noch ins methodology kapitel: action research}
\stopARemark

\spaceHalf

\inright{\aKeyword[Knowledge]{knowledge+academia}}
\aQuoteW{Barbero}{20}\aQuoteInTextT{Staging Citizenship: Performance, Politics, andCultural Rights}

What can academia and research do with all this? They can leave restrictive disciplines behind and begin to listen ethnographically to everything that speaks, screams, curses, makes noise, blasphemes, at the same time as it inaugurates, invents, energizes, liberates, emancipates, and creates. We are being compelled to think in a new way—one that, amid the frenetic globalization that threatens cultures, demands that we "reconstruct our local meanings," even those belonging to the most globalized practices and dimensions of social life. Every cultural interaction is always carried out by situated actors, and the meanings of enacted practices or reclaimed rights will ultimately lead us to social uses rooted in time and space. From this perspective, we discover that social ways of knowing do not exist simply to be accumulated and transmitted, but to be exercised as citizens, to be enacted performatively.

\spaceHalf

\inright{\aKeyword[Knowledge]{knowledge+proposing}}
\aQuoteB{Hall}{2009}{68-69}\aQuoteInTextT{A River of Life: Learning and Environmental Social Movement}

New generations of social movements are not merely oriented to “critiquing”
dominant society but they are simultaneously engaged in regenerative activities
and offering alternatives to reshape the very grammar of life. In short, we see the
transition from a phase of “protest” to a phase of “proposal”. I am interested in
giving close attention to what these proposals represent and signify, what is their
intent and content in rebuilding both human and non-human collectivities. These
new generations of social movements are not single issue based or enveloped within the larger narratives of nationalist struggles or 'equal opportunity' within the state and/or the market but they have opened the Pandora's box of multiissues and multi-actors. Most important for us is the fact that each actor is embodied with his/her own pragmatic and symbolic productivities. Finally, whether social movement leadership is aware or not, there is a creation of knowledge arising and taking shape as well as the appearance of a wide range of pedagogical and social learning strategies. Thus we value social movement space as much for its process as for its results.

\spaceHalf

\inright{\aKeyword[Identity]{knowledge: how to know/what to know}}
\aQuoteB{Cox and Fominaya}{2009}{11}\aQuoteInTextT{Movement knowledge}
Thus we want to argue that the attempt to “own” knowledge production from
above is fundamentally mistaken. If social movements are knowledge producers,
and generate ways of knowing grounded in particular experiences and for locally
practical purposes, then (as activists and as researchers) we cannot know a priori
“how to know”, still less how other people should know. What we learn in our
own movements, as we work on particular projects, campaigns and strategies, is
new to us, and what we learn from our allies doubly so – since it is grounded not
in a remaking of our own worlds but in their remaking of theirs.

%----------------------------------------
\startARemark{kommt noch zu action research}
\stopARemark


%----------------------------------------

\startARemark{kommt noch zu den social movements}
\stopARemark

%----------------------------------------

\startARemark{kommt noch in den open access teil}
\stopARemark

\spaceHalf

\inright{\aKeyword[Open Access]{knowledge+open access}}
\aQuoteW{Boyd}{2008}\aQuoteInTextT{open-access is the future}

%----------------------------------------

\imgIParticipateFlyer

%---------------------------------------------------
% gets only displayed in unfinished mode

\showImperfection

%---------------------------------------------------

\stoptext

\stopcomponent
