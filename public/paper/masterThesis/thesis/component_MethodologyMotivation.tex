\startcomponent component_MethodologyMotivation
\product product_Thesis
\project project_MasterThesis

% definitions and macros
\environment envThesisAllEnvironments

\define[]\sectionMotivation{\section[methodology::motivation]{Motivations and Demands: being concrete}}

\starttext

\startmode[tocLayout]

\sectionMotivation

This section describes the personal motivation for conducting research and writing the thesis and its implications.

\stopmode

\startmode[draft]

\sectionMotivation

\startKeywords
non-authoritative, non-hierarchic , genuine participation, open access,  practicability, emancipatory, field research methodology
\stopKeywords

Personal motivations and demands on \bracket{my} research actions hold a pivotal role, as of now more than \bracket{academic} relevance \refMissing which will be addressed and lain out in a later chapter \refMissingSrc{Relevance Chapter}. My motivations and demands are of subjective nature. Subjectivity has already been briefly stressed in the introductory paragraph of the Methodology chapter \refMissingSrc{Methodology Introduction} and is a topic worth mentioning \bracket{briefly} again in order to make the demands on my research transparent.

\startCitation
As every investigating subject is different, her social position, and her political
values, should be explicitly clarified so to put a light on the question of subjectivity. Nonetheless this confession, necessary on one side, is not by itself sufficient to control the effects of the investigation, neither it is to clarify completely the author’s socio-political distortion. The way to Knowledge in precise science is usually filtered by a measuring tool, while in social sciences it is filter by a subjective observer. On one side it is proper to describe the measuring tool, on the other it is correct to reveal subjective positions. But none of these epistemological path will get to the understand of pure Reality.  \aQuote{Athenea Digital, 2006}{20}
\stopCitation

Thus, my intention here is to reveal my subjective positions because I am the subjective observer and, as shown later on, I am also to a certain extend the subjective due to the fact that I did not just passively observe but also actively participated as observer. For me it is important to reveal those facts in order to make clear \toMark{why I decided to tie the current red line through this thesis }. 

Some of my subjective positions have already been formulated in the last section \refMissingSrc{Experience}. They basically summarize a basic reflection on this thesis and the corresponding research action(s) as ...

\spaceHalf
\startQuestionBack 
\startitemize[3]
\item ... an attempt to shape a complementary component of the people's struggles on the streets.
\item ...an attempt to find a non-elitist but common language, where academic and non-academic spaces may converge.
\item ...an attempt to shape a space for exchange and raising of awareness about the people's struggles.
\item ...an attempt to reflect on my personal practice and the contention of constraints and existing contradictions between different spheres of reality.
\item ...an attempt to dissolve the discursively produced and \bracket{socially} defined roles of academic agent and research \quote{subject}
\item thus, an attempt to dissolve the separation of academic, political, social and private space.
\item ...an attempt for genuine participation and acceptance of differences,
\item thus, an attempt for realizing research action(s) in a non-authoritarian and non-hierarchical manner.
\item ...an attempt to adapt the frame this thesis is embedded in on institutional level according to the principles I formulate here.
\stopitemize
\stopQuestionBack 
\spaceHalf

My basic consideration of this thesis as a complementary component of existing struggles of street people in São Paulo does not mean that I think that social change will result as an direct outcome. For me, provoking social change goes far beyond the scope of this work, far beyond the the frame it is embedded in and constraint by. However, I think that every step towards a transformation of society towards emancipation and self-determination is worthwhile to undertake, thus I hope this thesis may contribute to undertake further steps into that direction.

\startARemark
I could even go so far to say that this thesis shall be considered as another type of self-determined action, because it is considered as a component of a struggle and it was made possible by genuine participation between \quote{us}. I actually don't know in how far this kind of closed loop positively contributes to the thesis' form and content or in how far it would cancel out the narratives of other actions which are supposed to be included here in the first place. 
\stopARemark

Several other yet not mentioned positions are equally important to me. I would like to denote them as ...

\spaceHalf
\startQuestionBack 
\startitemize[3]
\item ... an attempt to avoid to represent anybody thus avoid hierarchies. I want to be together with the people and experience myself what they are experiencing.
\item ... an attempt to avoid to keep information that has been given to me just in the closed academic circle.
\item thus, I prefer open access to all information, narratives and thoughts this thesis is composed of.
\item \toMark{... an attempt to turn something present but invisible, visible but not to \quote{invent} something new,}
\item thus, I prefer to evoke an reaction and not just a sole analytical and systematically sound reproduction of \bracket{the} \quote{reality}.
\item ... an attempt to decouple the question of relevance of the research action(s) from the \bracket{western} scientific norm of being innovative, subjective and systematic, 
\item thus, I prefer to work qualitative and event driven instead of systematic and quantitative
\stopitemize
\stopQuestionBack 

\spaceHalf

\startAReminder
hier fehlt noch was
\stopAReminder

\stopmode

\stoptext

\stopcomponent
