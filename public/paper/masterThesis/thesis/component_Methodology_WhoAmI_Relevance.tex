\startcomponent component_Methodology_WhoAmI_Relevance
\product product_Thesis
\project project_MasterThesis

% definitions and macros
\environment envThesisAllEnvironments

\define[]\sectionRelevance{\subsection[methodology_whoami_relevance]{Relevancy//for Whom//for What?}}

\starttext

\startmode[tocLayout]

\sectionRelevance

This section argues about the relevance of \quote{relevance in research} and contests its usual meaning arguing that research can be relevant even if it is not a breaking new topic but that it is relevant if it is relevant for the people that collectively conduct research. Here, four questions regarding relevance are formulated which are going to be tackled in the following chapters.

\stopmode

\startmode[draft]

\sectionRelevance

When I think about \WhoAmI and \WhatDoIWant I tried to propose some answers through the selection of \abbrFull{AR} as \gotoTextMark[research framework]{methodology_whoami_actionresearch} and through the definition of my \gotoTextMark[self-conception]{methodology_whoami_motivation}.  Those proposals include several notions of \aKeyword{relevance} \gotoTextMark[here]{methodology_whoami_actionresearch_notionsoftheorizing} and \gotoTextMark[there]{methodology_whoami_experience_notionsofrelevance}, mostly in terms of \quote{relevant for me and the way I think my research has to be organized and realized}. 

Even though I think that those notions are relevant I still have the feeling that I did not yet draw emphasis on the question of relevance of my research related to the people and their struggle, the struggle this thesis is supposed to be embedded in as well in order. Thus, I probably have to ask why this thesis is relevant in the first place and for whom? 

\spaceHalf

\inright{ relevance of research and social struggle}
In order to begin with, I would like to contest the notion of \aKeyword[relevance in academic terms]{relevance+academic} because I do not think that my research must merely produce new knowledge and content used to be solely injected into the academic space.

In order to start contesting I would like to taking a set of questions related to relevance into account. Those questions haven been posed by \aQuoteInTextA{Don Mitchell} in \aQuoteInTextT{What Makes Justice Spatial? What Makes Spaces Just?}:

\spaceHalf

\startCitation
Mitchell’s work goes beyond calls for social or political \quote{relevance} in research and practice by reminding us that determinations of relevance always unfold in a historical and professional matrix. With Lynn Staeheli, he has written that calls for relevance in professional practice \quote{cannot be separated from questions about why research should be relevant, how research becomes relevant, the goals of research (including political goals), and the intended audiences and beneficiaries of research} (Staeheli and Mitchell 2005: 357). Those questions of why, how, what and for whom also lie at the heart of any movement for justice \aQuoteB{Brown et al}{2007}{8}.
\stopCitation

\spaceHalf

Thus, if I focus my perspective to the \aKeyword[space of struggle]{space+struggle} and if I align my research to the \aKeyword[standpoint]{standpoint} of the people and movements, I am probably able to define \aKeyword[research objectives]{objectives} that provide orientation to determine how my research becomes relevant outside academia and how I prevent to focus on the already alluded \gotoTextMark[means to an end]{methodology_whoami_nonauthoritarian_benefits} that merely result in personal or academic benefits.

Therefore I would like to determine my \aKeyword[research objectives]{research+objectives} according to the questions of relevance from the standpoint of the people and movements, interlinked with my previously defined \gotoTextMark[self-conception]{methodology_whoami_motivation} and \gotoTextMark[framework]{methodology_whoami_actionresearch} of research. 

\spaceHalf

\placefigure[force]{Questions that could determine research objectives from the perspective of relevancy.}
{

\textBoxedRoundMaxQuestion[0.4] 
{
Why should my research be relevant?
}

\textBoxedRoundMaxQuestion[0.4] 
{
How does my research becomes relevant?
}

\textBoxedRoundMaxQuestion[0.4] 
{
What are the goals of my research?
}

\textBoxedRoundMaxQuestion[0.4] 
{
Who is the audience and who are the beneficiaries?
}
}

I will not answer those questions right now because they will hopefully unfold when I determine the research objectives of this thesis in \gotoTextMark[the next section]{methodology_whatdoiwant}.

\addReference
{
Brown, N. et al., 2007. What Makes Justice Spatial? What Makes Spaces Just?
Three Interviews on the Concept of
Spatial Justice. Available at: \goto{\hyphenatedurl{http://www.justspaces.org/pdf/Brown_Crit_Plan_v14.pdf}} [url(http://www.justspaces.org/pdf/Brown_Crit_Plan_v14.pdf)] [Accessed May 9, 2011].
}

\showImperfection

\stopmode

\stoptext

\stopcomponent
