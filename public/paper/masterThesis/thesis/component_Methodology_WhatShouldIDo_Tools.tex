\startcomponent component_Methodology_WhatShouldIDo_Tools
\product product_Thesis
\project project_MasterThesis

% definitions and macros
\environment envThesisAllEnvironments

\define[]\subsectionTools{\subsection[methodology::whatshouldido::tools]{Tools//Online//Offline}}

\starttext

\startmode[tocLayout]

\subsectionTools

Describes all used tools for research and writing.

\stopmode

\startmode[draft]

\subsectionTools

\startKeywords
tools, writing, research, thesis, research, bibliography management, open maps, media archives, blog
\stopKeywords

The tools that I used during may research actions and thesis completion are mainly tools for documentation purposes and information structuring. I intend to raise \aKeyword{transparency} about what I am doing here and how I reached the various stations from the begin to the end of my research. Transparency can only be reached if the information, the theorized knowledge, the produced content this thesis is composed of, is freely accessible. This also means for me that other should be able to reproduce \bracket{technical wise} what I have done. Thus, \aKeyword{open access} to this thesis is mandatory but it also means that free access to the used tools is mandatory. 

The tools that I mainly applied are made for the virtual space, made to manage the flow of information in various fashions. They are mainly composed of \aKeyword[free software]{tools+free software}\footnote{\refMissingSrc{what is free software}} and \aKeyword[non-commercial]{tools+non-commercial} \aKeyword[social web services]{tools+social web}.

Free software tools help me above all to produce content while non-commercial web services help me to document, freely publish and establish transparency.

\placefigure[location=here]{Scope for tools applicable during thesis realization.}
{
\starttyping

				/BTEX {Writing Process} /ETEX

	/BTEX \tfa\ss\color[green]{Bibliography} /ETEX 
	/BTEX \tfa\ss\color[green]{Management} /ETEX 

		/BTEX \tfa\ss\color[green]{Writing} /ETEX 
		/BTEX \tfa\ss\color[green]{Environment} /ETEX 			  
/BTEX {Written} /ETEX					/BTEX \tfa\ss\color[green]{Blog} /ETEX		/BTEX \tfa\ss\color[green]{Open} /ETEX		/BTEX {Audio\/Visual} /ETEX
/BTEX {Content} /ETEX							/BTEX \tfa\ss\color[green]{Maps} /ETEX		/BTEX {Content}/ETEX
								
								/BTEX \tfa\ss\color[green]{Media} /ETEX 
								/BTEX \tfa\ss\color[green]{Archives} /ETEX			 
/BTEX \nl /ETEX
				/BTEX {Research Actions} /ETEX

\stoptyping
}

The thesis blog is the main platform that interconnects all content produced with other tools and services, on other platforms. From there, all available content can be freely accessed, the thesis and research history can be traced and reproduced. In detail, the applied tools and services are

\textBoxedRoundMaxDef[0.4]{blog: rtc.noblogs.org}

\textBoxedRoundMaxDef[0.4]{media archives: archive.org}

\textBoxedRoundMaxDef[0.4]{media archive: videobin.org}

\textBoxedRoundMaxDef[0.4]{writing tools: github}

\textBoxedRoundMaxDef[0.4]{writing tools: context}

\textBoxedRoundMaxDef[0.4]{open maps: openstreetmap.org}

\textBoxedRoundMaxDef[0.4]{bibliography management: zotero}


\placefigure[location=here]{Applied tools for thesis realization.}
{
\starttyping

				/BTEX {Writing Process} /ETEX

	/BTEX \tfa\ss\color[green]{zotero.org} /ETEX 

		/BTEX \tfa\ss\color[green]{github.com} /ETEX 
		/BTEX \tfa\ss\color[green]{tex} /ETEX 			  
/BTEX {Written} /ETEX				/BTEX \tfa\ss\color[green]{rtc.noblogs.org} /ETEX			/BTEX {Audio\/Visual} /ETEX
/BTEX {Content} /ETEX										/BTEX {Content}/ETEX
							/BTEX \tfa\ss\color[green]{openstreetmap.org} /ETEX

								/BTEX \tfa\ss\color[green]{archive.org} /ETEX 
								/BTEX \tfa\ss\color[green]{videobin.org} /ETEX			 
/BTEX \nl /ETEX
				/BTEX {Research Actions} /ETEX

\stoptyping
}

%--------------------------------------
% only active in "unfinished" mode

\showImperfection
{
irgendwelche referenzen
}

\stopmode

\stoptext

\stopcomponent
