\startcomponent component_NarratingInquiries_SaoPauloDiaries_Aruaca
\product product_Thesis
\project project_MasterThesis

% definitions and macros
\environment envThesisAllEnvironments

\define[]\subsectionSaoPauloDiariesAruacca{\subsection[narratinginquiries_saopaulodiaries_aruacca]{Collective//aRUAcca}}

\define[]\subjectMeetings{\subject{Meetings and Actions}}
\define[]\subjectOpiranga{\subject{Interview with Ocupação Ipiranga}}

\starttext

\startmode[tocLayout]
\subsectionSaoPauloDiariesAruacca
This sections contains the experience made with the Aruaça Film Collective initiated by homeless people and indymedia brasil.
\subjectMeetings
Some Experiences from meetings and movie shots.
\subjectOpiranga
An interview made with the Ocupação Ipiranga.
\stopmode

\startmode[draft]
\subsectionSaoPauloDiariesAruacca
\subjectMeetings
\reference[narratinginquiries_saopaulodiaries_aruacca_ideas]{}

\startDialog
\tell{\Ed}{Look, when you leave jail, you cannot return to your family or friends. They won't accept you any more, you are stigmatized. You are also stigmatized by the society because who wants to employ you or rent an apartment to you when you tell them that you come straight from jail. So you have no money and no perspective. You have just two options, either you enter the streets and you enter crime. The prison does not re-socialize you. You enter jail as a part time criminal but you leave it as a professional.}
\stopDialog

\subjectOpiranga

%---------------------------------------------------
% gets only displayed in unfinshed mode

\showImperfection

%---------------------------------------------------

\stopmode

\stoptext

\stopcomponent
