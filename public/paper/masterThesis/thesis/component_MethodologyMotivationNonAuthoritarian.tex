\startcomponent component_MethodologyMotivationNonAuthoritarian
\product product_Thesis
\project project_MasterThesis

% definitions and macros
\environment envThesisAllEnvironments

\define[]\subsectionNonAuthoritarian{\subsection[methodology::motivation::nonauthoritarian]{Attitude: Non-Authoritarian and Non-Hierarchical}}

\starttext

\startmode[tocLayout]

\subsectionNonAuthoritarian

This section shall explain why non-authoritarian and non-hierarchical attitude is a prerequisite when conducting  research.

\stopmode

\startmode[draft]

\subsectionNonAuthoritarian

\startKeywords
non-authoritative, non-hierarchic , motivation, emancipatory, self-determination
\stopKeywords

\inright{pervasion of research: non-authoritarian and non-hierarchical attitude}
Non-authoritarian and non-hierarchical attitude shall pervade the the ground on which this research shall be elaborated. This aspect is fundamentally important for me due to the fact that this thesis is partly based on information whose wells are people and due to the fact that it defines itself as social and political research.

From my point of view, non-authoritarian and non-hierarchical attitude is strongly interdependent. The fact that academic research is often already embedded in an existing academic framework already represents an implicit hierarchy which could, and often lead(s) to situations where research agents (scholars and projects) primarily follow their own agenda and logic, in terms of participation, constraints and benefits, for them, the project or the academic circle. 

Therefore \inright{participation exerted as tyranny or emancipatory self-determination} I want to question the terms under which \quote{participation} in research actions is defined and exercised in order to allow cooperation. Is it exerted as a simple justification for intrinsically unjust research actions \refMissingSrc{ref aus from tyranny to transformation} or as an non-hierarchical and emancipatory approach to exercise self-determination \refMissingSrc{ref aus from tyranny to transformation}?

Another hierarchy implicitly exists due to the fact that research actions can be conducted abroad, often made possible through scholarships or other ways of funding, a situation barely realisable by those that shall participate in or which are addressed by the particular research, hence, here, the status as foreign research agent automatically implies a difference in status between the research agent itself and those that are addressed to participate in the agents actions \bracket{if this is supposed to happen at all}. 

This situation can be described plastically with a quote of a street dweller from a conversion we had on a small and shady street in the centre of São Paulo:

\startCitation
Tell me, what does a guy from the first world here in the third world? Why are you here? Don't you have problems to solve and analyse in your country? \aQuoteP{2010}
\stopCitation

\spaceHalf

\inright{non-authoritative actions in order to conduct research, neither oppressive nor seductive}
Non-authoritarian attitude is the practice I am affiliated with. Here, a contradiction arise because one would like to get insight into situations and conditions that could be used for a however defined research purpose, but this shall never lead to acts of authoritative actions against those that provide information, neither implicit or explicit, neither oppressive nor seductive.

\spaceHalf

\inright{do fundamental questions correlate with motivation and demands?}
This has consequences for my research actions and rises more fundamental questions: 

\spaceHalf

\textBoxedRoundMax[0.3]{ What is the purpose of research? How do I conduct research? What is my role? How is my role perceived? How do I approach people? What do I want from them? What do they want from me? Can we find common ground to collaborate coequally?}

Apart from the issue of access, thus access to the people and their reality, the question of access to the research' outcome is related to a non-hierarchically attitude as well. Due to the fact that my research is based on \inright{co-authored  information and personal experiences} co-authored  information and personal experiences on the one hand , and on theories and theoretical frameworks conceptualized in books, journals or available through the internet on the other hand, one shall question the way this compound of information is (or has to be) made accessible. 

Little is openly \bracket{thus freely accessible} published in academic circles due to an elitist attitude and the economization of knowledge, where knowledge, even though formulated and elaborated in public institutions or based on peoples indigenous experience, remains behind impermeable walls, thus remains solely accessible to those that have \quote{created} this knowledge or which have the necessary monetary resources or the appropriate status in order to do so. 

This situation describes another aspect of the question of \quote{purpose} of and \quote{demand} on my research action(s), here in terms of outcome and its distribution and accessibility.

\spaceHalf

\inright{open access}
\textBoxedRoundMax{For whom? To whom? By whom?}

A final aspect of rather practical nature is the time frame of the thesis and its research actions. The initially contemplated and official period for research actions and theoretical examination had to be 5 month, 2 to 3 reserved for research abroad. 
\inright{research actions conducted in existing institutional frameworks or organized self-determinately }
Now, this thesis exceeds those specifications due to the fact that the research sojourn lasted already 6 month. Thus, once more, one enters the question of constraints and benefits for the research agent, which probably wouldn't allow further reflection and adjustments of the thesis context due to its strict time setting and would have probably led to a work that just follows the logic of acquiring a title while leaving approach, outcome and effect of conducted research actions rather insignificant, just as means to an end. It also questions the conditions under which research actions could be conducted and how research is organized, embedded in existing institutional frameworks, i.e. university programs or NGO \refMissingSrc{what is an NGO link missing} projects, or self-determined, transparent, emancipatory and questionable over the course of actions.

\stopmode

\stoptext

\stopcomponent
