\startcomponent component_NarratingInquiries_SaoPauloDiaries_SomeTalks
\product product_Thesis
\project project_MasterThesis

% definitions and macros
\environment envThesisAllEnvironments
\environment envCfgThesisImages

\define[]\subsectionSaoPauloDiariesSomeTalks{\subsection[narratinginquiries_saopaulodiaries_sometalks]{Encounters//Talks//Streets}}

\define[]\subjectAy{\subject{Around Ay Carmela}}
\define[]\subjectCrack{\subject{Meeting again}}
\define[]\subjectPCC{\subject{PCC}}
\define[]\subjectVeganLunch{\subject{Collective Vegan Lunch}}

\starttext

\startmode[tocLayout]
\subsectionSaoPauloDiariesSomeTalks

\subjectAy

\subjectCrack
Meeting the young guy again.

\subjectPCC
PCC in the centre.

\subjectVeganLunch
Socializing at Ay Camela

\stopmode

\startmode[draft]

\subsectionSaoPauloDiariesSomeTalks

\subjectAy

\reference[narratinginquiries_saopaulodiaries_sundaystreetscentregibson]{} Today is the last Sunday of the month, its end of \toMark{June?}. I went to \Ay for \aKeyword[collective vegan lunch]{actions+collective lunch} that is organized \aLinkNew[every last sunday]{http://ay-carmela.birosca.org/node/334} of the month and met \Ma there. 

We met the first time at the free mapping festival \aKeyword[você está aqui, mas por quê - um festival de mappeamento livre]{actions+festival de mappeamento livre}\aLinkNewF[você está aqui, mas por quê]{http://ay-carmela.birosca.org/node/433}\aLinkNewF[i am here but why]{https://rtc.noblogs.org/post/2010/07/25/i-am-here-but-why/} in June. \Ma knows the streets of city inside out because he was in street situation more than a decade and just recently, some month ago, managed to leave the streets and moved into an apartment. 

He is super engaged in the struggle of the people because as he said, he constantly fears that people will continue to die on the streets because they lack everything and are prone to the violence on the streets. He refers to the massacre of 15 people in street situation in August 2004, killed by death squats of the military police and private security agents \aQuoteB{Dossiê}{2009}{63-64}\footnote{The mentioned Dossier is collaborative work of various civil society organisation and non governmental institutions and meticulously maps the illegal execution of people by institutional and state agents, by the police on the streets or in prisons for instance}\aQuoteB{o Trecheio}{2009}{1}\footnote{\aKeyword[O Trecheiro]{streets+journals+o trecheiro} is a street journal published by \aKeyword[Rede Rua]{streets+groups+rede rua} which includes reports written by \aKeyword[people in street situation]{streets+people+in street situation}} but also to the current \toTranslate[cleansing]{higienista} policy of the city which aims to \refMissingSrc{clean the the central areas} \bracket{Sé, República} off the people in street situation by putting them in social institutions such as the \refMissingSrc{tendas} at \locMissingSrc{Parque Dom Pedro}. More on this \gotoTextMark[later on]{text mark missing}.

\imgOTrecheiroMassacre

\Ma offered me a trip through the city, to show me around, literally, in order to get in touch with people in street situation but also to experience what is meant by {\bf being} in street situation, how we then have to organize our day and night, our food and shelter. While making this plan, here at the entrance of \Ay, a friend of \Ma is passing by, together with a teenager. It's \Gi, with whom I will spend some time \gotoTextMark[here]{narratinginquiries_saopaulodiaries_penaforte} and \gotoTextMark[there]{narratinginquiries_saopaulodiaries_padredecha}. \Gi says that he is accompanying the young guy because he cannot take care of himself. They are heading towards \locMissingSrc{Parque Dom Pedro} and \Gis stop and talk with us makes him quickly anxious continue their way. 

\Ma, like he is always doing, is introducing me, telling \Gi what I am doing in São Paulo, that we are making plans to stay on the streets. \Gi like he is always doing is super kind and says that he thinks that it is important that people outside Brazil get to know what is going on in São Paulo. \Gi does not stay long because his friend urges him to continue. \Gi says that he is dedicated to him today because he is still a very young guy, just recently entered the streets and he needs someone who shows him around, shows him the places that he should know. Before \Gi and his friend leaves we agree on meeting again later this afternoon, at the place where \Ma is living now.

Shortly afterwards we are heading towards \Mas place. After arrival we first of all finalized our plan. \Ma said that we should stay at least one week on the streets, moving through different areas of the city, staying in the centre but also moving to the east, Brás, Mooca, Belem. 

{\em I felt a bit uncomfortable because I just got acquainted to the city and did not feel ready for such a long trip.}

Our compromise is finally just an introduction, \gotoTextMark[two days on the streets]{narratinginquiries_saopaulodiaries_dayandnight}. 

Then \Ma gave me a some material that he has been collecting about the struggle on the streets. 

\startARemark{hier können fotos von den sachen die \Ma mir gegeben hat rein} 
plus erklärungen dazu

\startitemize
\item ausweis zum kongress der catadores
\item handschriftliche aufzeichnungen
\item buch "vida na rua"
\item ...
\stopitemize

\stopARemark

He says that he is also a militant of the \abbrNew{Movimento Nacional da Populacão de Rua}{MNPR} and that he is taking now some time off to organize his own life and because he is fearing repression by police, that according to him is observing active members of \abbr{MNPR}. 

I asked him what he thinks about the \aKeyword{right to the city} but for him another question is much more important. The question of \aKeyword{access to the city}. Being formerly in street situation, this means access to educational facilities, access to decent work, access to a decent housing, access to participation in the cities decisions. For him, being on the streets means not having access at all, to nothing, not even food, being totally excluded. 

Once \Gi arrived after a while we decided to take a walk back to \Ay and talk with people in \locMissingSrc{that area}. \Ma says that it is late enough in that afternoon to meet people there because they are arriving around that time in order to secure their space. 

We enter \aNewLocation[Rua do Carmo]{http://osm.org/go/M@ziKw9eb--} and pass by an \aKeyword[abandoned construction site]{streets+abandoned+construction site}, an unfinished high-riser, maybe 20 floors high, its red bricks exposed, no glass in its windows, on the second floor clotheslines packed with laundry to dry and in front of it, on the small court leading to the street, some children playing while people are leaving and entering the improvised ramp into the buildings interiors.

On the other side of the street some \aKeyword[blocked up and abandoned houses]{streets+abandoned+house} as well, colonial style.  At the corner to a back road there is a church. There we are heading towards, to the back road because there, people are declaring their space for the night to come, on the stairs to side entrances of that church.

\startARemark{hab ich noch ein bild davon oder von der straße?}
\stopARemark

There we meet a group of four guys. Neither \Gi nor \Ma know them but we are getting in touch with them right away. Two of them are already massively drunken, another one, probably the oldest is hectically talking and standing still, the fourth one, the youngest, is sitting with the other two on the stairs. \Ma introduces us and asks them to tell me a bit about their situation, telling them that I am in the city to get to know about the struggle on the streets. 

\startPersonal
For me this kind of situations have always been uncomfortable. I always feel somehow exposed as a stranger, somebody different, not belonging to the place or the people \bracket{which is how it is anyway}, even though my clothes are in similar bad conditions then of most of those I meet, even though I speak Portuguese and even though I its my demand for transparency that the people know directly why I am there at that particular moment. Here my role is that of a \aKeyword[passive observer]{role+passive observer} and this sometimes leads to arguing and reluctance.
\stopPersonal

\reference[narratinginquiries_saopaulodiaries_sometalks::hecticguy]{}The hectically guy is asking immediately:

\startDialog
\tell{He}{Does he understand Portuguese.}
\tell{Me}{Sure I do.}
\tell{He}{Then tell me, what does a guy from the first world do here in the third world? Why are you here? Don't you have problems to solve and analyse in your country?}
\stopDialog

\startAReminder{ein teil fehlt noch}
\stopAReminder

The young guys interrupts and wants to know if I speak English as well.

\startAReminder{wie war nochmal der name von ihm?}
\stopAReminder

After affirming he is asking me questions about Germany in rough and broken English slang. He says that he has been to South Germany for a month on a trip with his religious youth group. He says that he is not from São Paulo, just arrived a month ago and went immediately to the streets. The hectic guy is interrupting us, asking:

\startDialog
\tell{He}{Do you believe in god?}
\tell{Me}{No, I don't.}
\tell{He}{Whooo, Irmãos, did you hear that, he is not believing in God. You are not a good person if you don't believe in god.}
\tell{Me}{So what, I believe in something else...}
\tell{He}{Whooo, did you guys hear that, he is not believing in God.}
\stopDialog


Luckily, \Ma came to safe me but now he has to start arguing with the hectic guy:

\startDialog
\tell{\Ma}{Irmão, I lived on the streets for many years and I have never seen you in this area. You don't wanna tell me what I have to believe in or he.}
\stopDialog

\Gi finally calmed everything down with his conciliatory way of arguing, telling the hectic guy that it does not matter in what we believe and that one religion is not better then another. What counts that we show solidarity for each other. Point.

The hectic guy, now calming down, but ready to start another ritual. This time he offers everybody to drink from a bottle they are sharing. In the meantime a car stops right at our side, the driver asking through the window

\startDialog
\tell{\Ma}{E aí irmãos, what's up? How are things going? Anything's going on around?}
\stopDialog

The guys tell him that everything is relaxed, nothing special is going on. This seems sufficient for him and he drives away. Once the car is out of sight the hectic guy asks:

\startDialog
\tell{\Ma}{Irmãos, I wanna buy a new bottle. I need five Reais. Alemão \bracket{that is me}, can you spare some Reais?}
\stopDialog

I gave him the five Reais note I was carrying with me, the others said they do not have money. The hectic guys takes them and disappears. We did not see him that afternoon again. Then \Ma has to leave as well, \Gi and me stay a bit longer. \Gi is talking to the two drunken guys and the young guy starts to talk with me again. After a while he says

\startDialog
\tell{He}{Alemão, I just wanna ask you if you can give me five Reais. Look, its nothing personal. I just wanna tell you that we need Crack now and it's better for you to leave because it will become urgent soon and you probably don't want to be here then.  It's really  important now because we had our last stone already some hours ago. So I ask you, do you have five Reais now?}
\tell{Me}{Sorry, but I gave my money the other guy....}
\tell{He}{Look, its really important for us now. How long would it take for you to go home get the money?}
\tell{Me}{I life too far away. It would take 2 hours or so to come back...}
\tell{He}{Ok, then better you get off soon and if you could get any money it would really help us much...}
\stopDialog

\Gi is approaching me and we decide to leave because the situation won't become favourable for us, so its better to say good by. The young guy then called out to us:

\startDialog
\tell{He}{Pass by the other day, alemão, we are always here around this time.}
\stopDialog

While we are slowly walking towards \locMissingSrc{Sé} and then further on to \locMissingSrc{República} \Gi said:

\startDialog
\tell{\Gi}{It was good that we left, it was not safe any more, they turn crazy when they need crack. And the next time you better don't give them money. The one guy never came back and probably also went to get some crack. One stone is just five Reais.}
\stopDialog

This was the last time I saw \Gi for a while, until I run into him some weeks later at \gotoTextMark[Praca da Sé]{narratinginquiries_saopaulodiaries_penaforte}.

\subjectCrack

\subjectPCC

On another day I meet \Ma again. We are talking about the four guys we met some days ago. \Ma asks if I have an idea who that guy in the car has been, who was stopping and asking the crowd there. I say that he is probably police or something. \Ma says that this guy was no police but that he was patrolling for the \abbrNew[Primeiro Comando do Capital]{PCC}. He was patrolling and asked the people on the street what they have seen, if something happened, if police was around. \Ma asks me if I every heard that  loud fireworks somewhere, not during football games, but just when I was in the centre. He said that those fireworks are the signals for new deliveries arriving at the \toMark{bocas de fumo} \toMark{and those patrols belonging to them}.

We continued to talk about \abbr{PCC}. Just some weeks ago, \abbr{PCC} lunched an assault on the captain of the \abbrNew{Roteiro}{ROTA}, a special unit of the \abbrFull{MP} of São Paulo. On the \toMark{welcher tag} they tried to kill the captain of the \abbr{ROTA} in front of his house when he left it in the morning \refMissing. As an answer, the \abbr{ROTA} was seeking revenge and killed \toMark{how many} people, \toMark{mainly youth from the peripheries} within 3 days \refMissing. This provoked another assault of the \abbr{PCC} \toMark{some days afterwards}, this time Molotov Cocktail attacks on the \abbr{ROTA} 's head-quarter \refMissing, which is located right beside the \locMissingSrc{Estation da Luz}, São Paulos main train station at the \locMissingSrc{Luz district}.

\startPersonal
The other day, I brought the July issue of \aQuoteInTextA{Caros Amigos}, inside an article about \abbr{PCC } called \aQuoteInText{Por dentro do PCC}\footnote{\toTranslate[From within PCC]{Por dentro do PCC}}\footnote{a shortened version of the article is \aLinkNewAlt[available online]{http://carosamigos.terra.com.br/index\_site.php?pag=revista\&id=145\&iditens=690}{http://bit.ly/o6NOs9} as well}. The article renders an interview with two anthropologist, Karina Biondi and Adalton Marques, who have conducted anthropological studies inside the prison system in Brazil\footnote{Karina Biondo: Junto e Misturado: uma ethnografia do PCC}\footnote{Adalton Marques: Crime, proceder, convívio-seguro: um experimento antropológico a partir de relacões enter ladrões}. In its introduction, the article states:
\stopPersonal

\startCitation
\language[pt]A Caros Amigos conversou com os dois antropólogos sobre os princípios e a organização do PCC, essa facção criminosa tão grande quanto pouco compreendida pela população do Estado com a maior população carcerária do Brasil \aQuoteB{}{2010}{36} \footnote{Caros Amigos spoke with the two anthropologists about the principles and the organisation of PCC, this criminal fraction, so large but little understood by the population of the state \bracket{of São Paulo} with the largest number of prisoners in Brasil.}
\stopCitation

\startPersonal
Me personally, I know little as well, that's why I am seeking for \toMark{more insights}. The notion of \aKeyword[crime]{streets+crime} and \aKeyword[jail]{streets+jail} is omnipresent in the narrations of the people, in the narrations of the streets but also as issues for actions, for instance as an \gotoTextMark[idea of the aRUAssa collective]{narratinginquiries_saopaulodiaries_aruacca_ideas} for film-making. \abbr{PCC} is one reality of São Paulo I stumbled across but I do not aim to argue for or against it.

For me, it is too complex to understand but I also do not want to neglect it because I have the feeling it is important to think about the reasons and manifestations of crime, the prison system, the police system, that all are produced in the city, that produce the city and affect social struggle.
\stopPersonal

Returning to \quote{crime}, \Ma talks about another invisible face of the streets. He explained that at some of the place we visited, houses have been converted from occupations of social movements to illegal apartment complexes. He says that occupations of social movements are sometimes infiltrated by \quote{the crime} \bracket[according to his notion]. In those cases the social movements are slowly drawn out by drug dealers for instance. Once the building has been entirely overtaken, rooms are prepared for renting and the building, initially occupied by social movement due to a lack of affordable housing, serves as an illegal apartment complex, generating profit. 

We have been in such a house, a former an industrial building, several floors high. Once this building was overtaken, walls have been brought up inside, establishing new rooms on each floor, ready to rent. In that house, probably 5 apartments are residing on each floor. The one where we have been was not very large, one room, a kitchen and sanitation.

In occupied buildings it is attempted to prevent such a development by strictly organizing the occupations. The \gotoTextMark[interview we made with the people]{text mark missing interview} of a hotel occupied by \abbrNew{Frente da Luta por Moradia}{FLM} at \locMissingSrc{Avenida Ipiranga} is bringing this matter up as well.

\subjectVeganLunch

\addReference
{
Dossiê, 2009. Mapas do extermínio: execuções extrajudiciais e mortes pela omissão do Estado de São Paulo. Available at: \goto{\hyphenatedurl{http://www.acatbrasil.org.br/down/DOSSIE\_pena de morte final.pdf}} [url(http://bit.ly/oLtVlG)] [Accessed August 24, 2011].
}

\addReference
{
o Trecheiro, 2009. Notícias do Povo da Rua. {\em o Trecheiro}, (180). Available at: \goto{\hyphenatedurl{http://www.rederua.org.br/pub/otrecheiro/2009/180\_trecheiro\_agosto\_2009.pdf}} [url(http://bit.ly/mXpSTr)] [Accessed August 24, 2011].
}

\addReference
{
Moncau, G. & Delmanto, J., 2010. Por dentro do PCC. {\em Caros Amigos}, (160). Available at: \goto{\hyphenatedurl{http://carosamigos.terra.com.br/index\_site.php?pag=revista\&id=145\&iditens=690}} [url(http://bit.ly/o6NOs9)] [Accessed August 26, 2011].
}

%---------------------------------------------------
% gets only displayed in unfinshed mode
\showImperfection

%---------------------------------------------------

\stopmode

\stoptext

\stopcomponent
