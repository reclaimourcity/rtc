\startcomponent component_NarratingInquiries_Intro
\product product_Thesis
\project project_MasterThesis

% definitions and macros
\environment envThesisAllEnvironments

\define[]\chapterNarratingInquiries{\chapter[narratinginquiries_intro]{Narrating Inquiries}}

\starttext

\startmode[tocLayout]
\chapterNarratingInquiries

Narrating Inquiries about the São Paulo experience.

\stopmode

\startmode[draft]
\chapterNarratingInquiries

Narrating diaries are what I sense a genuine expression of the sensible of the streets of São Paulo. Those diaries are subjective accounts of may situations, occurrences and thoughts. I do not try to strictly order them, chronologically or according to certain categories because they are also an account of the unpredictable born by the streets. As such I want them to apparently randomly float around as text as they cropped up to me on the streets. Narrating shall also give an account to the spaces that are created by the people through their actions, as individual subjects or as collectives or social movements. Those spaces often seem to be invisible to those outside a particular social struggle thus showing what the people do, and for what is what I think could be the purpose of narrating what I sensed. I probably cannot do more because I feel that I just started to get involved more when my time in the city came to an end, as I have already said \gotoTextMark[when asking who I am]{methodology_whoami_experience}. The diaries consists of personal notes, but also of public media accounts and movement content. I think I cannot draw too deep on each single bit of information, its is just too much and I will also leave out quite a bit \bracket{which is not lost but postponed to the times after this writing}. At some time I will just make use of flyers, pictures, links to websites or translated excerpts of movement manifests to catch a glimpse on what probably bears a whole new world, at another time I will just narrate dialogues or describe what we have said, what has been told me. I hope that this approach is amenable and too fuzzy. We will see.

%---------------------------------------------------
% gets only displayed in unfinished mode

\showImperfection

%---------------------------------------------------

\stopmode

\stoptext

\stopcomponent
