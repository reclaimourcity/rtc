\startcomponent component_Theorizing_Reconstruction
\product product_Thesis
\project project_MasterThesis

% definitions and macros
\environment envThesisAllEnvironments
\environment envCfgThesisImages

\define[]\sectionTheorizingReconstruction{\section[theorizing_reconstruction]{Re//Construct//Themes}}

\starttext

\startmode[tocLayout]
\sectionTheorizingReconstruction

Reconstruct theorizing themes.

\stopmode

\startmode[draft]
\sectionTheorizingReconstruction

\inright{entering the void}
The words are empty now and can be reshaped. How to start \reconstructing what has been stripped off meaning and content? I would like to start with the \void. The \void as imagined by \JRanciere is the space that is not recognized by the \police, that is not seen when \aQuoteInText{partitioning the sensible}. The \void is 

\startCitation
[...] pure chance or the complete absence of qualifications for governing [...] that state of exception where no oppositions can function, where there is no pre-determined principle of role allocation. \aQuoteB{Ranciere}{2001}{3}
\stopCitation

The absence of a \aQuoteInText{a pre-determined principle of role allocation} is what is makes social struggle emancipatory. Pre determined role allocation is what depoliticizes \participationDe, what keeps the structures of oppression and inequality intact. The space of emancipatory social struggle would then be the \void, where we are recognized as \politicalSubject, where we refuse to be treated and co-opted, paralysed and oppressed. 

The \void as space of the \poor, as the space of no predetermined role allocation, gathers all the people that are unaccounted in the space of the \police, that are unaccounted but that do care to act, that organize in movements, in collectives, individually, in order to disturb the order of the \police. One day the \void would be the space of all people, until then it is just the space of a group, the \poor, of those that do not accept the order imposed by the \police, that struggle to configure the \void as a space that abolishes particular roles and functions of people, roles and function invented and implemented by the \police while \aQuoteInText{partitioning the sensible}.

\startCitation
The principal function of politics is the configuration of its proper space. It is to disclose the world of its subjects and its operations. The essence of politics is the manifestation of dissensus, as the presence of two worlds in one \aQuoteB{Ranciere}{2001}{6}
\stopCitation

\spaceHalf

\inright{political struggle as opposition to partition}
For \JRanciere political struggle does not mean confrontation of different interest and opinions but it is rather the manifestation opposition between the logics of the world of the \police and the \void. Political struggle as dissenus makes the invisible visible, makes the unheard heard and by that disturbs the image of the partitioned world. 

\startCitation
There is politics as long as 'the people' is not identified with the race or a population,
inasmuch as the poor are not equated with a particular disadvantaged sector, and as long as the proletariat is not a group of industrial workers, etc... Rather, there is politics inasmuch as 'the people' refers to subjects inscribed as a supplement to the count of the parts of society, a specific figure of 'the part of those who have no-part.' [...] Political struggle is not a conflict between well defined interest groups; it is an opposition of logics that count the parties and parts of the community in different ways. \aQuoteB{Ranciere}{2001}{6}
\stopCitation

\spaceHalf

\inright{\maxDifference as manifestation of political struggle}
My \rhizomaticMap becomes now extended by tracings that leads to the \maxDifference in the sense imagined by \HLefebvre, \maxDifference in opposition to \minDifference, the \gotoTextMark[homogenisation of everyday life]{deconstruction_mindiff}. \MaxDifference is possible in the \void. The \void is the space where ones only qualification is to rule by not being qualified to rule. Thus it is not the the space of the professional, the expert, the powerful. It is the space where one can be, take part, by not being attributed to a role and a function. This is what lay at the core of \maxDifference. The maximum difference between the parts of the community, a multitude of social possibilities to shape everyday life, to shape the city, an utopia for self-determination.

\startCitation
\aQuoteB{Kipfer, Goonewardena, Schmid, Milgrom}{2008}{8}
Lefebvre’s primarily political understanding of the urban revolution as a dialectical transformation of minimal into maximal difference [...] This transformation can be achieved only by social struggles for political self-determination and a new spatial centrality, which help liberate difference from the alienating social constraints produced by capital, state, and patriarchy.
\stopCitation

\MaxDifference is not to be confused with diversity. Diversity as form of individualism or determined by powerful social groups is just another manifestation of \minDifference, of homogenisation.

\startCitation
The right to difference is thus simply the flip- side of asserting the right to the city (centrality/power). Affirming the right to the city/difference does not mean celebrating actually existing manifestation of diversity per se, however. The liberal-pluralist diversity refers to reified forms of minimal difference (individualism, group pluralism).\aQuoteB{Kipfer}{2008}{204}
\stopCitation

To the contrary, \maxDifference is produced difference, thus fully lived forms of plurality and individuality, an articulated identity based on rich social relations and not affected by any form of indifference. It is the quest for a unalienated, festive, creative, self-determined, fully lived urban society \aQuoteB{Kipfer}{2008}{203} that is not forced into a space that was produced only for the purpose of discrimination \aQuoteB{Kipfer et al}{2008}{293}.

\startCitation
However, oppositional strategies have counter-hegemonic potential only if the minimal differences of commodified festivality, multiculturalized ethnicity, and racialized suburban marginality are themselves transformed in the process of the political struggle. \aQuoteB{Kipfer et al}{2008}{296}
\stopCitation

\spaceHalf

\inright{the \rightToDifference and the city}
Taking part in the production of difference, in shaping the \void as a space of lived \maxDifference, taking part in shaping the city within the \void, is what asserts the \rightToTheCity. The \rightToTheCity is the right to be different, to self-determinately produce differences. This right is not of normative nature, thus a right granted by institutions (such as the right or obligation to vote) that finally do not prevent social, economical and cultural exclusion \aQuoteB{Gilbert and Dikeç}{2008}{258}. The \rightToDifference is an immanent human properties, \aQuoteInText{defined and redefined by political action, social relations [...] and the sharing of space} \aQuoteB{Gilbert and Dikeç}{2008}{258,259}. The continuous re-negotiation of those rights essentially means the active participation in societies self-management \aQuoteB{Gilbert and Dikeç}{2008}{260} where \aQuoteInText{each time a social group refuses passively to accept its conditions of existence, of life or of survival, each time such a group attempts not only to learn but to master its own conditions of existence} \aQuoteB{Lefebvre in Gilbert and Dikeç}{2008}{260}. 

\startCitation
Such alternative and oppositional claims for difference can take on very different forms and ways of expression: small-scale resistances, counter-projects, anti-imperial insurgencies, rebellions of the dispossessed in metropolitan centers such as the recent uprisings in Paris, as well as well-documented anti-globalization struggles and networked encounters. Struggles of peripheralized social groups against segregation and for empowerment can produce their own forms of centrality. Here, one can think of alternative social spaces created by sub- and counter-cultural groups or the oppositional centralities produced through mass mobilization (strikes, demonstrations, uprisings). \aQuoteB{Kipfer et al}{2008}{311}
\stopCitation

\ParticipationRe is already filled with new sense by now. Participation ultimately means political struggle. \ParticipationRe becomes then already an expression of \maxDifference, of taking part in the \void, to self-determinately produce difference. \ParticipationRe is political again, because it is not drawn on predetermined \minDifference, its is ones own expression, it means to rule by not being qualified to rule, in continuous social dialogue and \bracket{re}negotiation with the other, not individualistic or solely drawn on self interest. \Participation means then self-determination and self-organisation of the production of space, of the city, by those that life in it, that are in direct dialogue about their needs and desires, to make visible their needs and desires. 

Finally, \selfDetermination and \participation produces new political spaces, uncontested, not predetermined by others, by the \police. Those new spaces are in opposition to the partitioned space the \police creates, they formulate a dissensus that disturbs the \police order, they offer possibilities for practising mutual respect, for practising self-organisation, for seeing one another not in terms of roles but in terms \maxDifference and by that preventing to fall back into those categories, functions and roles that are necessary to maintain the \police order. \MaxDifference, \selfDetermination and \participation hence mean autonomy of the individual and the society that produces and indwells the \void.

\startCitation
Autonomy comes from autos-nomos: [to give to] oneself one’s laws... Autonomy does not consist in acting according to a law discovered in an immutable reason and discovered once and for all. It is the unlimited self-questioning about the law and its foundations and the capacity, in light of this interrogation, to make, to do, and to institute [and therefore also, to say]. Autonomy is the reflective capacity of a reason creating itself in an endless movement, both as individual and social reason... 

If the autonomous society is that society which self-institutes itself explicitly and lucidly, the one that knows that it it-self posits its institutions and its significations, this means that it knows as well that they have no source other than its own instituting and signification-giving activity, no extrasocial guarantee. \aQuoteB{Castoriadis}{1991}{163}
\stopCitation

\addReference
{
Gilbert, L. \ Dikeç, M., 2008. {\em Right to the city: Policits of citizenship}. In Space, Difference, Everyday Life: Reading Henri Lefebvre. New York: Routledge, pp. 250-261.
}
\addReference
{
Castoriadis, C., 1991. {\em Power, Politics, Autonomy}. In The Castoriadis Reader. Oxford University Press, USA, p. 163. Available at: \goto{\hyphenatedurl{http://brianholmes.files.wordpress.com/2008/12/continental\_drift.pdf}} [url(http://bit.ly/nlu4RT)].
}

\showImperfection

\stopmode

\stoptext

\stopcomponent
