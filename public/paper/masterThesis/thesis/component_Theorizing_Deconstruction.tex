\startcomponent component_Theorizing_Deconstruction
\product product_Thesis
\project project_MasterThesis

% definitions and macros
\environment envThesisAllEnvironments
\environment envCfgThesisImages

\define[]\sectionTheorizingDeconstruction{\section[theorizing_deconstruction]{De//Construct//Themes}}

\starttext

\startmode[tocLayout]
\sectionTheorizingDeconstruction

Deconstructing the meaning of theorizing themes.

\stopmode

\startmode[draft]
\sectionTheorizingDeconstruction

\imgIParticipateFlyer

\spaceHalf

\inright{deconstruction of participation}
The poster above is a wonderful opening for \deconstructing \participation. It says

\startCitation
I participate, You participate, he and she participates, We participate, You participate, They profit.
\stopCitation

and thereby precisely points to the deficiencies in the concept of \participation as it was meant back then and as it is meant today. This poster\footnote{A List of May 68 graffiti is documented at the \aLinkNewNoF[Bureau of Public Secrets]{http://www.bopsecrets.org/CF/graffiti.htm}} has been put on the streets in Mai 68, when students and workers gained momentum in 

\startCitation
the largest general strike that ever stopped the economy of an advanced industrial country, and the first wildcat general strike in history; revolutionary occupations and the beginnings of direct democracy; the increasingly complete collapse of state power for nearly two weeks; the resounding verification of the revolutionary theory of our time and even here and there the first steps toward putting it into practice; the most important experience of the modern proletarian movement that is in the process of constituting itself in its fully developed form in all countries, and the example it must now go beyond — this is what the French May 1968 movement was essentially, and this in itself already constitutes its essential victory. \aQuoteW{Bureau of Public Secrets}{1969}
\stopCitation

\quote{We all participate - We all benefit} is what one could excerpt out of this tiny insight. \Participation in putting revolutionary theory into praxis, overthrowing the capitalistic status quo for a while and genuinely participating in practising life with each other.

\quote{We participate - They profit} is what \participation is about in present \toMark{post-capitalistic?} times. We are invited to participate in many things. We are invited to crowdsource the net, the user is the producer, we can \refMissingSrc{Digg our most favourite blog post}, we participate in the construction of social communities\footnote{From Facebook, to Orkut and Google+}, we can conveniently participate in social campaigns with a \refMissingSrc{simple click}, the marginalized and oppressed are invited in participatory development projects that seems to end their poverty or at least reduce it, crisis management during times of catastrophes \refMissingSrc{incorporate citizen reports} from the ground in order to be more effective and precise in providing help, its seems that we can even participate in the planning of our direct environment, our neighbourhood, when local governments decide to give us a tiny share of \bracket{evaluation} power or at least allows us to select our most favourite proposal for a new shopping mall right around the corner, we are told to show civil courage to help the people on the streets around the corner, at work we can participate in unions to fight for better working conditions \bracket{at least where unions are established}, and last but not least it seems that we may even participate in \bracket{the illusion of} impacting public politics by putting a vote once in a few years. 

Even though this listing is relatively arbitrary and polemical, it may show what is currently understood by \participation. We participate in proposals of others, our participation becomes their profit. We are invited to participate by those that have the power to invite us, that already decided what we are supposed to do or that provides us with options for predetermined alternatives. This contemporary concept

\startCitation
highlights the fundamental point that participation without redistribution of power is an empty and frustrating process for the powerless. It allows the powerholders to claim that all sides were considered, but makes it possible for only some of those sides to benefit. It maintains the status quo. \aQuoteB{Arnstein}{1969}{216-224}
\stopCitation

This quote by \aQuoteInTextA{Sherry Arnstein} is excerpted from her \aQuoteInTextT{Ladder of Citizen Participation} that she has written from the standpoint of \participation of the excluded citizens in housing and urban planning decisions in the US. 

The \Ladder illustrates and denounces hierarchies within concrete projects, where the powerless and powerholders apparently \participate with the intention to genuinely propose ideas and articulate decisions about issues that bear direct constraints and necessities in the life of powerless.

\placefigure[force]{The \Ladder by \aQuote{Sherry Arnstein}{1969}}
{
\starttyping

	/BTEX \tfa\ss\color[green]{citizen control} /ETEX
			/BTEX \tfa\tt{citizen power} /ETEX			
	/BTEX \tfa\ss\color[green]{delegated power} /ETEX
			/BTEX \tfa\tt{citizen power} /ETEX			
	/BTEX \tfa\ss\color[green]{partnership} /ETEX			

	/BTEX \tfa\ss\color[green]{placation} /ETEX				
			/BTEX \tfa\tt{tokenism} /ETEX			
	/BTEX \tfa\ss\color[green]{consultation} /ETEX			
			/BTEX \tfa\tt{tokenism} /ETEX			
	/BTEX \tfa\ss\color[green]{informing} /ETEX			

	/BTEX \tfa\ss\color[green]{therapy} /ETEX
			/BTEX \tfa\tt{nonparticipation} /ETEX			
	/BTEX \tfa\ss\color[green]{manipulation} /ETEX
	
\stoptyping
}

The \Ladder serves as a basic starting point for \deconstructing \participation. Looking at \participationDe through the lenses of \powerStructures one could probably perceive more emerging issues caused by the \bracket{non}distribution of \powerDe. \Participation becomes \coopted and exploited by the powerholders to maintain the status quo, \participation becomes \depoliticized and does not aim to overcome structural inequalities beyond the local where \toMark{they take place}, \participation shrivels to a \technocraticMean to already predetermined ends.

\spaceHalf
\inright{1. non-distribution of \powerDe}
\ParticipationDe is a form of \refMissingSrc{tyranny} under current circumstances. It does not eliminate structural inequalities but rather reproduces the structural \powerHierarchies that are causing them. In this sense, the status quo is maintained or further enforced, the marginalized remain marginalized and the powerful remain in power. Maintaining \powerStructures is done as long as people are merely invited into predetermined participatory spaces set up and controlled by the power holders. 

\startCitation
[...] actions taken by the poor within the invited spaces of citizenship, however innovative, aim to cope with systems of hardship and are sanctioned by donors and
government interventions. \aQuoteB{Miraftab}{2006}{195}
\stopCitation

At \invitedSpaces, the powerless are manipulated and treated \aQuote{Sherry Arnstein}{1969}. Education and therapy are the means to provide the powerless with coping mechanisms for further survival in a persistent, structurally unequal, system. 

\toMark{At \invitedSpaces, the powerless are informed, consulted and placated \aQuote{Sherry Arnstein}{1969}.}

\spaceHalf

\inright{2. \coopting the idea of participation for single sided benefits}
\startCitation
[...]the ‘community’ has periodically been destroyed by underlying processes of development, only to be resurrected as the proper source of recovery through trustee-led intervention” \aQuoteB{Hickey and Mohan}{2004}{10}.
\stopCitation

As \aQuoteInTextA{Hickey and Mohan} put it, the very \powerStructures that produced and produces structural inequalities and exclusion are now proclaiming salvation by inviting the oppressed to take part in fixing their situation. \aQuoteInTextA{Hickey and Mohan} write from the standpoint of \bracket{international} development agency and address their failure of not having entailed structural transformation for the people, a transformation of \powerDe that would be finally directly exercised by the people.

A widening of their perception could conclude that those power structures that caused and causes exclusion and discrimination in society, no matter the form, are now even benefiting by apparently genuinely offering space for participation, by inviting the different groups and individuals that compose society, thus most of us, by pretending that all voices are being heard, that the grassroots are being involved, eventually in order to remain in control, to prevent dissent and self-determination and thereby further exercise power over the people in street situation, the inhabitants of favelas, the students at universities, the inhabitants of social housing, or the workers in precarious service industries for instance.

\startCitation
It is a kind of theft – to take away the valuable things of the people and to put them to work in a system that is against the people but in favor of the powerful and the rich. Not only the municipalities and the politicians but also many of the NGOs and ‘civil society’ structures and activists are guilty of playing a part in this ongoing theft against the people. It can make you feel like your struggle was useless. You fight for justice – for equality and for the world to be shared – and you end up with the promise of ‘service delivery’.“ \aQuoteW{Abahlali baseMjondolo}{2010}
\stopCitation

\spaceHalf
\inright{3. \depoliticization of participation}
This type of \participation is stripped off the political dimension, it is depoliticised because its aim is not learning and understanding how existing structures of power, oppression and injustice operate \aQuoteB{Hickey and Mohan}{2004}{11} in order overcome them. Once those political questions are excluded, how can \participation bear any idea of genuine social transformation that leads to a just society? It is then merely a method, a means to a different end, a \aQuoteInText{empty ritual}\aQuote{Arnstein}{1969} that does not discover new and uncontested spaces shaped by the people but that keeps them locked up in the current spaces of exclusion and powerlessness. \Participation becomes an obligation for every \quote{good citizen}, it becomes a must have for \quote{just} development projects, it is courteous for \quote{good governance practices} exercised by local, national and international institutions, but the structures of power never changes.

\startCitation
Western political thought has been that the central task of any political regime is education. The survival of the prevailing order depends upon depoliticizing young
people by making good citizens of them, by inviting, or even compelling them to participate. It is with good reason that we have long pathologized the figure of
the disengaged, apathetic youth, and groped frantically for therapeutic aids that might entice young people to participate (forgetting that time tends to make good
citizens of us all). And it is no wonder we have invested such hope in the potential of emerging media to engage young people and to encourage them to participate.
Participation, in the end, is truly much safer—and much easier to deal with—than politics. \aQuoteB{Barney}{2010}{145}
\stopCitation

\spaceHalf
\inright{4. participation as mere \technocraticMean to predetermined ends}
\Depoliticization reduces \participation to a mere method. Regimes of power \aQuoteInText{prove [...] participatory credentials} in their invitations to participate \aQuoteB{Hickey and Mohan}{2004}{16} but in turn just implement a \aQuoteInText{technical fix for complex problems of uneven development} \aQuoteB{Cleaver and Rahman in Hickey and Mohan}{2004}{59}. 

\Participation as such then renders an approach \toMark{that is rather technocratic than social}, where aim, methods and tool and principle decisions have already been determined beforehand and that does not intend to envision a broader view that contest superimposed structural inequalities nor does it intend to distribute power. \Participation merely sticks within the limited scope of a particular project \aQuoteB{Hickey and Mohan}{2004}{10}.

\startCitation
On the one hand, privatization, deregulation, unemployment and ‘precarization’ of labour (and ‘structural adjustment’ programmes at the periphery and semiperiphery of the world-system); on the other hand, attempts to bring people to ‘take part’ in the management of local-level state crisis (along with other measures like repression and ‘state of emergency’, as long as they are necessary and feasible). \aQuoteB{de Souza}{2006}{335}
\stopCitation

\spaceHalf
\inright{Summarizing}
Summarizing the previously said, the \deconstruction \participation reveals several deficiencies that do prevent the empowerment and self-determination of the people.

\placefigure[force]{Deconstructing participation.}
{
\starttyping

	/BTEX \tfa\ss\color[green]{does not distribute power} /ETEX

					/BTEX \tfa\ss\color[green]{co-opted for single sided benefits} /ETEX
	
						/BTEX \tfa\ss\color[green]{limited to the local} /ETEX			

/BTEX \tfa\ss\color[green]{carried out at the expense of the already powerless} /ETEX

		/BTEX \tfb\ss{PARTICIPATIONdeconstructed} /ETEX			

			/BTEX \tfa\ss\color[green]{a technocratic method} /ETEX			

		/BTEX \tfa\ss\color[green]{a mere invitation into predetermined spaces} /ETEX			

/BTEX \tfa\ss\color[green]{the maintenance of structural inequalities} /ETEX

						/BTEX \tfa\ss\color[green]{depoliticized} /ETEX
	
\stoptyping
}

Those deficiencies are not absolute nor do they occur all at once or have always equal shares. They shall primarily provide ground for critique of \participation as a mean for dis-empowerment and maintenance of the status quo. Deallocating these deficiencies from \participation could prepare the ground on which new meaning would be injected into its then empty shell. \aQuoteInTextA{Sherry Arnstein} remarks in her \aQuoteInTextT{Ladder of Citizen Participation}:

\startCitation
Obviously, the eight-rung ladder is a simplification, but it helps to illustrate the point that so many have missed - that there are significant gradations of citizen participation. Knowing these gradations makes it possible to cut through the hyperbole to understand the increasingly strident demands for participation from the have-nots as well as the gamut of confusing responses from the powerholders \aQuote{Arnstein}{1969}.
\stopCitation

Deficient \participationDe, \invitedSpaces, the powerless and power holders, are not just terms that represent a reality of development projects or other initiatives that explicitly addresses an excluded or discriminated group of people. \Participation pervades society and social relations independent from status or class, but it does not necessarily lead to a just society. 

\spaceHalf

\inright{invited participatory spaces are an expression of wider social power structures}
With \invitedSpaces I would not only refer to spaces determined by public and private institutions, the state, or \abbr{NGO}s. \InvitedSpaces could also be the spaces determined in struggle, by the grassroots or the civil society. They could be male dominated activist assemblies that structurally oppress female voices due to habits and apparently socially fixed norms, they could allow a vocal local elite to dominate discourse even in systems of horizontal decision making where everyone is supposed to have an equal voice, they could allow the majority society to oppress the needs and desires of minority groups. \PowerStructures starts to exists between people, determined by skills, different levels of education, their roles in predominant social discourses: about gender, \bracket{im}migration, race and colour, from where one is coming, in which area one is living. \PowerStructures exist thus not only between the poor and the rich, the have's and have not's.

\startCitation
In practice, because power relations between people are not addressed,
participation all too often involves only the voices of the vocal few and poor people
and women, in particular, tend to lose out, being marginalized and overlooked in
‘participatory’ processes. \aQuoteB{McEwan}{2005}{973}
\stopCitation

Deficient \participationDe means being detached from the processes that affects ones own life, being detached from the political space where those processes are shaped, being only invited to enter a predetermined participatory space that is not overlapping or touching the political one, thus remaining powerless with no means at hand for transformation. 

\startCitation
in the context of a hegemonic political and economic culture that not only accommodates participation but actually embraces, thrives, and insists upon it, and in light of proliferating technologies that effectively render routine participation obligatory, the ends that participation presently serves cannot be said to be unambiguously worthwhile. If we are looking for something to which we might attach our aspirations for a more just society, we might have to look for something other than mere participation. \aQuoteB{Barney}{2010}{144}
\stopCitation

\spaceHalf

\inright{deconstruction of the political space}
The answer to \quote{the other} seems to be located in the political space. The \politicalSpace denoted so often seems to be the space where we want to \participate in but where we are structurally excluded from. \Participation in the \politicalSpace seems to be necessary in order to shape the processes that affects our life's but apparently this space is not accessible, it is determined and owned by \powerStructures that prevents us from transforming those spaces or that intends to keep us under control and maintain the status quo.

So, how should we proceed? Even though I intended to dismiss categories where possible I made plenty of use them, mainly in form of binary expression, thus words that represent opposite roles and positions such as the powerless and the powerholders or just and injust. I partition the world as I perceive it into distinct elements. This is what \JRanciere calls \aQuoteInText{the partition in the sensible} and this is where I would like to start with the \deconstruction of \politicsDe. \aQuoteInText{The partition in the sensible} is formulated as one thesis in his \aQuoteInTextT{Ten Theses on Politics}.

\startCitation
The partition of the sensible is the cutting-up of the world and of 'world;' it is the nemeïn [distribution] upon which the nomoi [laws] of the community are founded. This partition should be understood in the double sense of the word: on the one hand, that which separates and excludes; on the other, that which allows participation [...]. A partition of the sensible refers to the manner in which a relation between a shared 'common' [un communpartagé] and the distribution of exclusive parts is determined through the sensible. This latter form of distribution, in turn, itself presupposes a partition between what is visible and what is not, of what can be heard from the inaudible. \aQuoteB{Rancière}{2001}{6}
\stopCitation 

% nemeïn	- verteilung	- distribution
% nomoi	- gesetz		- law

%Die Aufteilung des Fühlbaren ist die untergliederung der Welt und "Welt". Es ist die Verteilung auf der die Gesetze der Gemeinschaft basieren. Diese Aufteilung sollte in den beiden Sinnen des Wortes verstanden werden: das was trennt und ausschließt; das was Teilhabe erlaubt. Eine Aufteilung des Fühlbaren bezieht sich auf das Wesen das die Verbindung eines geteilten Gemeinsamen und einer Verteilungung von exklusiven Teilen durch dach Fühlbare festlegt. Die zweite Form der Verteilung selbst erwartet schon eine Aufteilung zwischen dem was sichtbar ist dem dem was nicht, von dem das aus dem nichthörbaren zu hören ist.

I partition what I sense according to a quality and meaning that I probably learned somewhere or that I invent: the majority society, a minority group, the oppressed, the streets, social movements, capitalism, the excluded, the invisible, the neglected, occupations, collectives, solidarity, city of extremes, non-athoritarian. Not only do I partition, the world that I am continuously learning to \bracket{re}discover is already partitioned and qualified, continuously reshaped, pointing out the differences between its partitions.

\startCitation
The clash between the 'rich' and the 'poor,' for instance, is the struggle over the very possibility of these words being coupled, of their being able to institute categories for another (ac)counting of the community. \aQuoteB{Rancière}{2001}{6}
\stopCitation

% (ac)counting	- zählen/ansehen als

\spaceHalf
\inright{partitioning of roles}
The sensible of São Paulo, for instance, is partitioned into a multitude of roles 

\startListKeywords
researcher, activist, observer, marginalized group, politicians, power holders, police, institutional agents, man, woman, slum dwellers, persons in street situation, citizens, a dealer, addicts, social movements, anarcho-punks, \catadores
\stopListKeywords

\spaceHalf
\inright{partitioning of locations}
that are bounded to a multitude of locations and spaces

\startListKeywords
the streets, self-determined and cultural places \bracket{\Ay, \Ocas}, São Paulo, neighbourhoods and city districts \bracket{\Bras, the \Centre, \Luz}, institutions \bracket{\Refeitorios, \Tendas, \Albergues}, public squares and spaces \bracket{\Se, \Republica}, Internet Cafés, \Mohinho, \Crackolandia, occupied houses, the periphery.
\stopListKeywords

\spaceHalf
\inright{partitioning of modes of actions}
where modes of actions and ways of being are being practised according to functions of locations and roles

\startListKeywords
repressive, empowering, militant, cultural, \selfDetermined, social, \emancipatory, \participatory, excluding, \coopting, treating, educating, coping, \participatory, \depoliticized	
\stopListKeywords

\startCitation
There are two ways of counting the parts of the community: The first only counts empirical parts -- actual groups defined by differences in birth, by different functions, locations, and interests that constitute the social body. The second counts 'in addition' a part of the no-part. [...]  We will call the first police and the second politics. [...] In this fittingness of functions, places, and ways of being, there is no place for a void. It is this exclusion of what 'there is not' that is the police-principle at the heart of statist [state] practices \aQuoteB{Rancière}{2000}{6}
\stopCitation

% fittingness	- angemessenheit
% statist		- state

\spaceHalf

\inright{what is the void?}
What does it mean that there is no place for the \aQuoteInText{void}? Maybe a look of the aim of social struggle is helpfully. In the sense that I understand social struggle, it should lead to emancipatory social transformation by overcoming the status quo. A transformation should lead to a just society where oppressive and discriminating \powerStructures have been abolished. This in turn would mean the abolishment of the partitions, symbols, roles and functions that are structuring our being in a \aQuoteInText{world lived out of necessity}. The \aQuoteInText{void} should then be the space emancipatory struggle is directed to, the space of the people that do not take part, that do disturb the structure of the ordered world. In the \aQuoteInText{void} a category such as \peopleInStreetSituation, military police, sem terra, sem teto, \abbr{NGO}s, \abbr{GCM}, public and wellfare politics and the like would not exist because they are expressions of structural inequalities, \bracket{\toMark{they are expressing that you are more powerfull then I, that I have more rights then you, that my role is predetermined by social norms that seems to be natural to me, that she gets good education and he not}}, and that social struggle is about the eliminating and overcomming those structural inequalities. 

\spaceHalf

\inright{contemporary politics is police}
Contemporary politics as associated today with \bracket{local or national} governments, states, politicians, parties, unions, internationally acting institutions, the \bracket{global} economy, political campaigns and discourses within \bracket{mainstream} civic society, deny the existence of what is not supposed to be, the \aQuoteInText{void} as \aQuoteInText{the space of the people}, because there they would cease to exist. 

What is understood as politics today is called \aQuoteInText{the police} in the thesis of \JRanciere. \aQuoteInText{The police} is not necessarily what we associate with the police on the streets \bracket{which is a part of it anyway}. The \aQuoteInText{the police} \aQuoteInText{partitions the sensible} the world a perceivable order, in a structure whose components have names and meanings, are qualified according to various standards.

\startCitation
The police is not a social function but a symbolic constitution of the social. The essence
of the police is neither repression nor even control over the living. Its essence is a certain
manner of partitioning the sensible. [...] a partition of the sensible characterized by the absence of a void or a supplement[...] \aQuoteB{Rancière}{2000}{6}.
\stopCitation

% constitution	- beschaffenheit
% supplement	- ergänzung

\spaceHalf

\inright{\politicalSpaces transform into \policeSpaces}
The spaces that have been previously been called \politicalSpaces are transformed then into \policeSpaces. The exclusion from taking part in \policeSpaces is probably inevitable then because \policeSpaces are the space in which the \aQuoteInText{poor} are not supposed to be recognized or seen.

\startCitation
Political litigiousness/struggle is that which brings politics into being by separating
it from the police that is, in turn, always attempting its disappearance either by crudely
denying it, or by subsuming that logic to its own. Politics is first and foremost an
intervention upon the visible and the sayable \aQuoteB{Rancière}{2000}{6}.
\stopCitation

% litigiousness	- streitsüchtig
% subsuming	- fassen, zusammenfassen

I think I need a break here i order structure things and my thoughts a bit. The \aQuoteInText{void} is that space that does not exist in the efforts of the \police to \aQuoteInText{partition the sensible}, thus the world we see and feel and perceive. The \aQuoteInText{void} is the space of those that are not supposed to he \aQuoteInText{take part}. Those that are not supposed to \aQuoteInText{take part} are the poor but the poor are not the material poor or what we perceive as poor. 

\startCitation
The 'poor,' however, does not designate an economically disadvantaged part of the population; it simply designates the category of peoples who do not count, those who have no qualifications to part-take in arche, no qualification for being taken into account. \aQuoteB{Rancière}{2000}{4}
\stopCitation

% designate	- bezeichnen

\spaceHalf

\inright{\participatorySpacesDe transform into \policeSpaces}
The \participatorySpacesDe referred to before should now turn into \policeSpaces as well. The praxis exercised there is that of the denial of the fact that the \aQuoteInText{poor} actually count. The praxis of those \participatorySpacesDe is eduction and treatment, mediation through information and consultation of the \aQuoteInText{poor} in order to prevent their part taking that should actually lead to a abolishment of structural inequalities that produced their roles and situations and functions and finally the entering of \participatorySpacesDe. 

\startCitation
If there is someone you do not wish to recognize as a political being, you begin by not seeing them as the bearers of politicalness, by not understanding what they say, by not hearing that it is an utterance coming out of their mouths. And the same goes for the opposition so readily invoked between the obscurity of domestic and private life, and the radiant luminosity of the public life of equals. In order to refuse the title of political subjects to a category -- workers, women, etc... -- it has traditionally been sufficient to assert that they belong to a 'domestic' space, to a space separated from public life; one from which only groans or cries expressing suffering, hunger, or anger could emerge, but not actual speeches demonstrating a shared aisthesis [perception] \aQuoteB{Rancière}{2000}{6}.
\stopCitation

% bearers 		- träger
% utterance 	- äußerung
% aisthesis 	- wahrnehmung	- perception

\spaceHalf

\inright{what comes next, after politics?}
Looking at São Paulo for a while, the \aQuoteInText{partitioning of the sensible} is in full pace. The police on the streets and public policies of cleansing aiming to expel the people in street situation from commercial districts, the economical poor life in the peripheries out of sight or cannot afford the expensive public transport or access to university. \Luz is supposed to be transformed into a shiny district by expelling the inhabitants of the \toMark{corticios} and by trying to push the crack scene out of sight and out of mind. Denying what is not supposed to be then means that emancipatory social struggle is attempted not to be recognized by the \police, either by repressing it or by co-opting its demands and propositions. People in street situation are send to \bracket{remote} \Albergues or to the \Tendas to receive the treatment to their poorness there or to just silent them, demands to the \rightToTheCity enter the agendas of institutions such as the \refMissingSrc{UN} or \refMissingSrc{UNESCO} or the proper state, in Brasil in guise of the \refMissingSrc{city statute}, the houses of the \favelas in \locMissingSrc{Zona Zul} are \refMissingSrc{demolished}, the thousands of housing occupants in the cities centres are \refMissingSrc{sent back to the peripheries} into social housing compounds literally located in a void lacking social networks, work opportunities, where public transport is too costly and time intensive to be utilizing in order to reach the central areas of the city. Out of sight, out of mind.

By using excerpts of the \aQuoteInText{Ten Thesis of Politics} by \JRanciere I tried to get to grasp what politics is not meant to be, that the current perception of \politicsDe and \politicalSpaces does not make them the prime spaces for transformation through social struggle because. That does not mean that I think that it is senseless to make the invisible visible there as well. 

\Deconstructing \politicsDe also means to \deconstruct the notion of the poor as a simple category of people economically or material poor, poor in opposition to the rich. For me, perceiving the \aQuoteInText{poor} as those people that are not qualified to take part in the shaping of ones own lived space is much more understandable because it does not reduce non-participation and discrimination to those parts \bracket{or \aQuoteInText{partitions}} of society that are categorized as the marginalized \bracket{as I did in the title of this thesis}. Instead the \aQuoteInText{poor} are we all, all those that cannot take part in ruling and being ruled \aQuoteB{Rancière}{2001}{2}.

\startCitation
The formulations according to which politics is the ruling of equals, and the citizen is the
one who part-takes in ruling and being ruled, articulate a paradox that must be thought
through rigorously. [...] This formulation speaks to us of a being who is at once the agent of an action and the one upon whom the action is exercised. [...] It contradicts the conventional 'cause-and-effect' model of action that has it that an agent endowed with a specific capacity produces an effect upon an object that is, in turn, characterized by its aptitude for receiving that effect. \aQuoteB{Rancière}{2001}{2}
\stopCitation

% endowed 	- ausgestattet sein
% aptitude 	- fähigkeit

So, the political as we know it is the \police that according to \aQuoteInTextA{Rancière} \aQuoteInText{partitions the sensible} world and denies the fact that there is a space, the \aQuoteInText{void}, where people do not need special qualifications to rule and being ruled. The political emits subjects, which are we, the \aQuoteInText{poor} that are now unheard but that care, that struggle, we that we have nothing in common but the qualification of not having the qualification to rule. 

\startCitation
That the distinguishing feature of politics is the existence of a subject who 'rules' by the
very fact of having no qualifications to rule; that the principle of beginnings / ruling is
irremediably divided as a result of this, and that the political community is specifically a
litigious community. \aQuoteB{Rancière}{2001}{8}
\stopCitation

% irremediably 	- unabänderlich/unheilbar
% litigious 	- strittig

At present, the \aQuoteInText{void} is the political space, the space of \gotoTextMark[those who care]{methodology_whoami_socialmovements}. Its an intermediate space that currently does not include all people but just a fraction. Probably the \aQuoteInText{void} and the community of those who care are an image of the future, a begin.

\spaceHalf

\inright{deconstructing space and difference}
Now, since it has been tried to deconstruct \participation and \politics, what is remaining? I think a look at São Paulo through the lens of spaces is worthwhile. They could probably extend the understanding of the \aQuoteInText{partition the sensible} into fragments. Those fragments are the everyday spaces we life in, that we shape and that shape us. Those spaces could probably reveal more then just a \aQuoteInText{statist practice}.

At the end, the struggles fought in São Paulo are urban struggles, struggles fought in urban spaces where a base level of chaos is prevailing, where the 

\HLefebvre

\aQuoteInB{Prigge}{2008}{49}
Lefebvre, however, connects the thesis of the dominance of the spatial to the present stage of capitalist societalization that is characterized, according to him, by the totalizing tendency of urbanization, and that, therefore, must cause an epistemo- logical shift. It is no longer the industrial and its disciplines focusing on capital and labor, classes and reproduction that constitute the episteme (the possibility of knowing the social formation), but the urban and its forms focused on everydayness and consumption, planning and spectacle, that expose the tendencies of social development in the second half of the twentieth century.

What does the city create? Nothing. It centralizes creation. Any yet it creates everything. Nothing exists without exchange, without union, without proximity, that is, without relationships. The city creates a situation, where different things occur one after another and do not exist separately but according to their differences. The urban, which is indifferent to each difference it contains, . . . itself unites them. In this sense, the city constructs, identifies, and sets free the essence of social relationships9 . . . We can say that the urban (as opposed to urbanism, whose ambiguity is gradually revealed) rises above the horizon, slowly occupies an epistemological field, and becomes the episteme of an epoch. History and the historic grow further apart.10

\aQuoteInB{Prigge}{2008}{66,67} 
The irreducible symbolic quality of spatial representations of daily social reality (the ideological hegemony of the spatial), the urban as the decisive episteme of the contemporary social structure (urbanity as the central apparatus [dispositif] of space) and the gulf between the spaces of subjective experience and objective perception (crisis of representation): how can this tangle of problems concerning an epistemology of space be unraveled? Lefebvre’s answer was a differentiated schema of the social production of space.

\aQuoteInB{Ronneberger}{2008}{135} 
According to Lefebvre, the reproduction of modern everydayness occurs through a threefold movement. First, societalization is accomplished through a “totalization of society.” Second, this process is accompanied by an “extreme individualization” which eventually leads to a “particularization.”9 The “bureaucratic society of controlled consumption” is grounded upon the parcelization of social praxis and the shredding of social contexts:

\aQuoteInB{Ronneberger}{2008}{136,137} 
Lefebvre relates the dominance of the spatial to the reproduction of capitalism. Space is presented as the result of a concrete production process. For Lefebvre, things are not separate from space. He considers space as social product; the production of space may reveal social relations. Each mode of production produces its own space, albeit not in a linear fashion. The relative fixity of spatial structures produces layering effects. Changing uses of space and processes of restructuring do not produce radical breaks in spatial patterns but develop within and against existing spatial structures. In principle, capitalist space is characterized by homogeneity and fragmentation. It is based on the separation and subsequent reconnection of space. On the one hand, the abstract logic of state and commodity produces homogenizing effects. On the other, capitalist valorization strategies subdivide, parcellize, and pulverize space. To define socio-spatial processes further, Lefebvre developed a differentiated schema that distinguishes three dimensions of the production of space.13 First, perceived space refers to the collective production of urban reality, the rhythms of work, residential, and leisure activities through which society develops and reproduces its spatiality. Conceived space is formed through knowledge, signs, and codes. Conceived space refers to “representations of space” by planners, architects, and other specialists who divide space into separate elements that can be recombined at will. The discourse of these specialists is oriented toward valorizing, quantifying, and administering space, thereby supporting and legitimating the modes of operation of state and capital. Finally, Lefebvre talks about lived and endured space: “spaces of representation.” Users of space experience lived space every day, through the mediation of images and symbols. Lived space offers possibilities of resistance. With this triadic model of the social production of space, Lefebvre tried to undermine dichotomies of structure and agency, discourse and practice.The schism between subjects’ perceived and lived spaces of activity and “objective” scientific- technological spatial structures is bridged by “ideologies of space.” According to Prigge,14 these ideologies articulate science with everyday life, render spatial practices coherent, guarantee the functioning of everyday life and prescribe modes of life.

\aQuoteInB{Kipfer}{2008}{201}
The production of space is not agent-less. It is, at least in part, a result of concerted strategies. Key agents in making the production of space hegemonic are specialists of urbanisme: architects, planners, developers, specialized academics. Given Lefebvre’s broader claim about the centrality of the urban as a mediation of neo-capitalist totality, urbanists take on a strategic role as organic intellectuals in an urbanizing, neo- capitalist order. Urbanists give meaning to the practical-material aspects of city- building with disciplinary, fragmented knowledge (savoir) and through symbolic forms (monuments, advertising images). Both forms of conceived space assume and promote the political passivity and the affective involvement of users of space.The specialized, state-dependent knowledge of architects, planners, and urban designers treats urban space as a reified, thing-like object that imposes itself on the urban inhabitants from the outside, as it were

\aQuoteInB{Kipfer}{2008}{201-202}
 Neo-capitalist urbanization is explosion/implosion. It undermines city centers by
scattering urban life into isolated parcels: bungalows (pavillons), districts of high-rise towers (grands ensembles), factory and university compounds, and resort towns on the beach. Demarcated by property divisions, transportation routes, and lines of functional and social segregation, these parcelized social spaces (planned in vulgar modernist fashion) represent forms of minimal difference. As specified in Le manifeste différentialiste, The Production of Space, and The Critique of Everyday Life(Volume III), minimal or “induced” difference is alienated particularity (individualism or group particularism) that tends toward “difference-as-sameness” and “formal identity.” Akin to the “diversity between villas in a suburb filled with villas” and the patriarchal “family cell,”63 minimally differential space dissociates everyday life, peripheralizes the working class, imposes much of the weight of reproduction onto women, and banishes new immigrants to “neocolonial” shantytowns and the worst public housing tracts.65 But minimally different spaces such as beach resorts extend “bourgeois hegemony to the whole of space.”66 Like bungalows, they promise a different, erotic appropriation of nature and body, embody hopes for non-instrumental human relationships, and nurture daydreams about freedom from repetitive drudgery even as they are managed with “identical plans” and strategies to foster predictable “rituals.”

\aQuoteInB{Kipfer}{2008}{203-204}
Asserting a (maximal) right to difference implies a two-pronged quest for revolutionary transformation. In the more general sense, asserting a right to difference means laying claim to adifferent, no longer capitalist world defined by use- value relationships and generalized autogestion: the self-determination of all aspects of life, from workplaces to territorial units.81 This general differential quest is tied to a commitment to strip existing social differences of the very alienating, often state-sanctioned aspects (productivism, sexism, racism) that make them minimal in the here and now. Differentialist theory is intended to help move from affirming oppositional manifestations of difference to transforming these manifestations in a dialectically humanist fashion

For Lefebvre, claims to “the right to the city” are the prism through which minimal difference may be transformed into maximal difference and fragments of abstract space may be connected in a quest to differential space. It helps to recall that Lefebvre treated the urban not only as a mediation of far and near order of totality. He also saw it as a (fleeting) form of centrality and difference, a point of convergence and “an ensemble of differences.”87 The right to difference is thus simply the flip- side of asserting the right to the city (centrality/power). Affirming the right to the city/difference does not mean celebrating actually existing manifestation of diversity per se, however. The liberal-pluralist diversity refers to reified forms of minimal difference (individualism, group pluralism).88 


\aQuoteInB{Kipfer, Schmid, Goonewardena, Milgrom}{2008}{291,292}
Centrality is defined by the association and encounter of whatever exists together in one space at the same time. Thus it corresponds to a logical form: the point of encounter, the place of coming together. This form has no specific content. Its logic stands for the simultaneity contained within it and from which it results: the simultaneity of everything that can be brought together at one point. Centrality eliminates peripheral elements and condenses wealth, means of action, knowledge, information, and culture. Ultimately it produces the highest power, the concentration of powers: the decision. In The Right to the City, long before the discussions about “global cities” and “world cities,” Lefebvre argued that urbanization signaled the advent of new forms of global centrality: today’s cities are centers of design and information, of organization and institutional decision-making on a global scale. They are centers of decision and power that unite all the constitutive elements of society in a limited territory.

Lefebvre’s conception of the urban incorporates relationships between centers and peripheries.This relationship is not fixed in physical form. Once central social spaces (imperial capitals at the global scale or historic city centers at the scale of urban regions) can lose their status as centers of decision-making and control. At the regional scale, central functions of city centers can “implode” socially and economically while cities “explode” into far-flung metropolitan agglomerations. In turn, new centralities can emerge: new business districts, newly powerful urban centers of finance and state control. In all cases, the formation of centralities is predicated on processes of peripheralization—displacement, enclosure, segregation, exclusion.This takes place through both “coercion” and “persuasion,” incorporation and domination.



\aQuoteInB{Kipfer, Schmid, Goonewardena, Milgrom}{2008}{}
Nevertheless, Lefebvre offers a view of the urbanization process that is distinct from
most others. He analyzes the urban not as accomplished reality but as possibility, a
potential that is inherent in existing urbanization but can be realized only through
fundamental social change: an urban revolution.The peculiar quality of this analysis
is that it does not limit itself to a critique of the urbanization process but thinks
through the implications of this process to lay bare its social possibilities.This reveals
a potential that Lefebvre programmatically calls “urban society.”


\startARemark{dann noch citizenship?}
\stopARemark


\addReference
{
Rancière, J., 2001. {\em Ten Theses on Politics}, Available at: \goto{\hyphenatedurl{http://www.ucd.ie/philosophy/staff/maevecooke/Ranciere.Ten.pdf}} [url(http://www.ucd.ie/philosophy/staff/maevecooke/Ranciere.Ten.pdf)].
}

\addReference
{
Arnstein, S.R., 1969. A Ladder of Citizen Participation. {\em JAIP}, 35(4), p.216-224. Available at: \goto{\hyphenatedurl{http://lithgow-schmidt.dk/sherry-arnstein/ladder-of-citizen-participation.html}} [url(http://lithgow-schmidt.dk/sherry-arnstein/ladder-of-citizen-participation.html)] [Accessed May 31, 2011].
}

\addReference
{
Barney, D., 2010. “Excuse us if we don’t give a fuck”:
The (Anti-)Political Career of Participation. {\em Jeunesse: Young People, Texts, Cultures}, 2(2), p.138-146. Available at: \goto{\hyphenatedurl{http://jeunessejournal.ca/index.php/yptc/article/view/78/71}} [url(http://jeunessejournal.ca/index.php/yptc/article/view/78/71)] [Accessed September 2, 2011].
}

\addReference
{
de Souza, M.L., 2006. Social movements as “critical urban planning” agents. {\em City}, 10(3), p.327-342. Available at: \goto{\hyphenatedurl{http://abahlali.org/files/Paper_CITY-2006.pdf}} [url(http://abahlali.org/files/Paper_CITY-2006.pdf)] [Accessed May 17, 2011].
}

\addReference
{
Abahlali baseMjondolo, 2010. The high cost of the right to the city. {\em Pambazuka News}. Available at: \goto{\hyphenatedurl{http://www.pambazuka.org/en/category/features/63126/print}} [url(http://www.pambazuka.org/en/category/features/63126/print)] [Accessed July 19, 2010].
}
\addReference
{
Hickey, S. \& Mohan, G., 2004. {\em Towards participation as transformation: critical themes and challenges}. In Participation: from tyranny to transformation? Exploring new approaches to to participation in development. London; New York: ZED Books Ltd; Distributed exclusively in the U.S. by Palgrave Macmillan, pp. 3-24.
}
\addReference
{
Hickey, S. \& Mohan, G., 2004. {\em Relocating participation within radical politics of development: critical modernism and citizenship}. In Participation: from tyranny to transformation? Exploring new approaches to to participation in development. London; New York: ZED Books Ltd; Distributed exclusively in the U.S. by Palgrave Macmillan, pp. 59-74.
}

\addReference
{
Miraftab, F., 2006. Feminist praxis, citizenship and informal politics: Reflections on South Africa’s anti-eviction campaign. {\em International Feminist Journal of Politics}, 8(2), p.194-218. Available at: \goto{\hyphenatedurl{http://www.urban.uiuc.edu/faculty/miraftab/miraftab/IFJP.pdf}} [url(http://www.urban.uiuc.edu/faculty/miraftab/miraftab/IFJP.pdf)] [Accessed September 7, 2011].
}

\addReference
{
McEwan, C., 2005. New spaces of citizenship? Rethinking
gendered participation and empowerment
in South Africa. {\em Political Geography}, (24), p.969-991. Available at: \goto{\hyphenatedurl{http://dro.dur.ac.uk/1204/1/1204.pdf}} [url(http://dro.dur.ac.uk/1204/1/1204.pdf)] [Accessed September 1, 2011].
}

%------------------------------
\showImperfection

%\aQuoteB{Barney}{2010}{144}

%The basic logic of citizenship is inclusion and participation; the basic logic of politics is exclusion and refusal. And this is why a culture of liberal democratic citizenship, a culture of the universalized invitation to participate, tends to produce politics only at and beyond its borders and margins. \aQuoteB{Barney}{2010}{144}

%\aQuoteB{Barney}{2010}{139}

%Participation, it would seem, is what citizenship is about. The prospect I would like to raise is that citizenship-as-participation is something altogether different from politics: that if participation, or taking part, is what citizenship is about, the possibility looms that neither citizenship nor participation necessarily conduces to politics. Indeed, as I suggest below, it may be the case that citizenship-as-participation is our best security against the possibility of politics.

%\aQuoteB{Barney}{2010}{144}

%at the very least they raise the question of the status of participation as both a political end and as a critical category under contemporary economic, social, and technological conditions. Participation is ambivalent, as open to stabilizing prevailing arrangements of power and injustice as it is to disrupting them. Thus, from a critical perspective, it would appear necessary to stipulate that participation is not an absolute, but rather only a contingent value: one whose worth is not intrinsic but rather derived from the ends it serves in any given context. Further, in the context of a hegemonic political and economic culture that not only accommodates participation but actually embraces, thrives, and insists upon it, and in light of proliferating technologies that effectively render routine participation obligatory, the ends that participation presently serves cannot be said to be unambiguously worthwhile. If we are looking for something to which we might attach our aspirations for a more just society, we might have to look for something other than mere participation. What we might actually need is politics, not just participation. It was suggested above that participation is what citizenship is about, but that citizenship-as-participation is something altogether different from politics. Much turns on what one thinks politics is, and on what one thinks politics is for.

%\aQuoteB{Miraftab}{2004}{1}

%While the former grassroots actions are geared mostly toward providing the poor with coping mechanisms and propositions to support survival of their informal membership, the grassroots activity of the latter challenges the status quo in the hope of larger societal change and resistance to the dominant power relations.

\stopmode

\stoptext

\stopcomponent
