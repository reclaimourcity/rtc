\startcomponent component_Report

\product product_cnrdReport
\project project_MasterThesis

% definitions and macros
\environment envCfgPaperLanguageEn
\environment envCnrdReportColors
\environment envCnrdReportImages

\starttext

\color[turquoise]{\bf{Name}}

Name

\color[turquoise]{\bf{First Name}}

Name

\color[turquoise]{\bf{Country of origin}}

Germany

\color[turquoise]{\bf{Kind of scholarship}} 

MSc field research

\color[turquoise]{\bf{Start of scholarship}} 

(06.2010)

\color[turquoise]{\bf{End of scholarship}}

(09.2010)

\color[turquoise]{\bf{Activities 2010}}

During my semesters in 2010 I primarily finished all necessary university courses in order to start my final master thesis. My focus area can be located in urban environments with special interest in urban (social) processes. These processes have been part of the lectures I participated in, but also in assignments I wrote for them. Thematically, the topics of my assignments varied from \quote{slum clearance in Cape Town} up to the \quote{The Right to City}.

The main effort in 2010 related to university activities has been the preparation of my final master thesis and the journey to carry out the empirical part of my thesis. Therefore I stayed six month in {\em São Paulo}, Brazil, and took part in the life of homeless people in the streets of the centre of the city. 

After my return in November 2010 I started to prepare the theoretical part of my thesis that I am writing right now in order to finish in May this year. 

\color[turquoise]{\bf{Assessment}}

Personally, the scholarship for my empirical work allowed me to grasp the real sense of the topic that I intended to form to a master thesis. During my stay abroad I realized that the purely assumptive and theoretical considerations I made about my topic in advance had to be adjusted drastically in order to write a thesis not just for the cause of the title, not just based on a short field trip where most of the experience I made would not have been visible, where I could hold myself completely accountable for, where I could ethically and socially argue with the people about my participation in their life, that could at least strengthen the corporation and information exchange about urban struggle on the grassroots level. I will lay out those aspects in more detail in the following paragraphs, but first of all I will briefly lay out in a non-chronically manner my stay in {\em São Paulo}.

My contact with people of the street population of the centre of the city was more coincidence than strictly planed. When I arrived, I pursued several paths with the intention to cooperate with people and groups in different marginalized situations in order to elaborate the potentials of mobile communication (mobile phones, cameras, etc) for the collaborative mapping of the urban environment of marginalized groups. 
Among others I asked in Favela inhabitant associations and community networks on the (still inner city) boundaries of the city {\em São Paulo} but also in the peripheries of the metropolitan region like {\em Zona Sul} (southern zone), which means basically a three hours journey with public transport from the centre. 

Due to their precarious living situation and even though great interest was expressed, a collaboration couldn't be concluded because the people just have been too busy with their own community struggle and couldn't hardly spend time to conduct the whole process of a collaborative project that was limited to just the short time frame I had available. 

The contact to the street population was then formed during my participation in an independent activist media project carried out by people from {\em Indymedia São Paulo} and people from the streets. This project is based on film making with low tech gadgets like mobile phones, small digital cameras and free software. In the course of this project, I got to know several of the people I stayed with later on.

Emblematic for our later collaboration, which consisted basically of observatory participation of the daily struggle, are the following points which mainly affected the way my master thesis has been evolved since then and my relation to this academic work.

First of all, building trust between us took a long period of time. Many preconceptions exists against scholars and university students, especially from the \quote{first world} because their work with marginalized people often just consists of {\em stealing} information by conducting interviews or questionnaires, that are incorporated into an academic work whose outcome only benefits the student or scholar while the information providers hardly see any improvement of their situation. That preconception could only be deconstructed by spending many hours, day and weeks with the people in order to discuss this issue and to think about the limited alternatives that exist in the frame of the \quote{fieldwork}.

\combPoliceMoinho

The daily conditions are so complex that it took a great deal of time to just slightly grasp and understand the people, the city, the processes of social and institutional repression against them but also their self-organisation. Even though I spent six month in {\em São Paulo}, this time was hardly enough to understand the real sense of the struggle of the street population. All prerequisites necessary for my original thesis proposal were not existent. The daily struggle is so hard, the city so aggressive, the society so segregative, that my project, even though libertarian by intention, just didn't make sense in its intended form.

The personal conclusion from my  time in {\em São Paulo}, which will be reflected in my current master thesis approach, is, that first of all, my collaboration with marginalized people first had to grasp the social conditions the people live in. This included an understanding of the people, their situation, the social conditions and institutional structures. This understanding could only be achieved by living with the people, becoming one of them, instead of spending some hours together to conduct interviews. A lot of trust is necessary before people start to give insights in their community, talking about the good and bad sides. A lot of time is necessary to personally experience the people's experience: the daily struggle for food, shelter, sanitation, work; the daily repression and violence of the police and state agents, the repression manifested in urban architecture but also in the social discourse and of course the aggression within their community. It took a lot of time to grasp the existence and meaning of different groups of street population, ranging from the little organized people that frequently participate in workshops and social centres and which organize themselves autonomously or in form of the urban social movement of street people ({\em MNPR}), to the lost people under the {\em Minoção} Highway and the destroyed ones in {\em Crackolândia} (crack land) around the central train station {\em Luz}. Further on there are the {\em Catadores} (the recyclers) which shape the urban scenery with their self constructed cars full of recycled materials and paper, there is the massive occurrence of organized crime and urban social movements that struggle for decent housing by conducting large scale occupations of speculative and abandoned real estate properties. 

\combAvIpirangaLuz

Under this conditions I had to ask how it would ever be possible to conduct my thesis project, when the city is already that aggressive, let alone the digital divide (the lack of skills, the lack of access to computers and mobile communication gadgets, the lack of acceptance) which exist and whose reduction would need an additional amount of time.

After the location of myself in this scenery, we had to ask what the purpose of my thesis project could be in order to facilitate the struggle of the people. It quickly became clear that such a time limited project for academic purposes would lead to just marginal (if at all) enforcement  in order to reach the real goal of leaving the streets and to gain a decent and just life. 

\combAvIpirangaMstc

This very brief overview roughly sketched some of the main factors that made me redirect my thesis from a more technical argumentation towards the social construction of participatory self-determined actions of marginalized groups in the case of street people in {\em São Paulo}.

For more information about the further evolution of my master thesis, check out \from[rtcplain].

\stoptext


\stopcomponent
