\startcomponent component_MethodologyMotivation
\product product_Thesis
\project project_MasterThesis

% definitions and macros
\environment envThesisAllEnvironments

\define[]\sectionMotivation{\section[methodology::motivation]{Motivations and Demands: being concrete}}

\starttext

\startmode[tocLayout]

\sectionMotivation

This section describes the personal motivation for conducting research and writing the thesis and its implications.

\stopmode

\startmode[draft]

\sectionMotivation

\startKeywords
non-authoritative, non-hierarchic , genuine participation, open access,  practicability, emancipatory, field research methodology
\stopKeywords

Personal motivations and demands on \bracket{my} research actions hold a pivotal role, as of now more than \bracket{academic} relevance \refMissing which will be addressed and lain out in a later chapter \refMissingSrc{Relevance Chapter}. My motivations and demands are of subjective nature. Subjectivity has already been briefly stressed in the introductory paragraph of the Methodology chapter \refMissingSrc{Methodology Introduction} and is a topic worth mentioning \bracket{briefly} again in order to make the demands on my research transparent.

\startCitation
As every investigating subject is different, her social position, and her political
values, should be explicitly clarified so to put a light on the question of subjectivity. Nonetheless this confession, necessary on one side, is not by itself sufficient to control the effects of the investigation, neither it is to clarify completely the author’s socio-political distortion. The way to Knowledge in precise science is usually filtered by a measuring tool, while in social sciences it is filter by a subjective observer. On one side it is proper to describe the measuring tool, on the other it is correct to reveal subjective positions. But none of these epistemological path will get to the understand of pure Reality.  \aQuote{Athenea Digital, 2006}{20}
\stopCitation

\spaceHalf

\inright{revelation of subjective positions}
Thus, my intention here is to reveal my subjective positions because I am the subjective observer and, as shown later on, I am also to a certain extend the observed subjective due to the fact that I did not just passively observed but also actively participated as observer. For me it is important to reveal those facts in order to make clear \toMark{why I decided to tie the current red line through this thesis }. 

\spaceHalf

\inright{subjective positions already at hand}
Some of my subjective positions have already been formulated in the last section \refMissingSrc{Experience}. They primarily summarize a basic reflection on this thesis and the corresponding research action(s) as ...

\spaceHalf

\textBoxedRoundMax[0.4]{...an attempt for genuine participation and acceptance of differences,}
\textBoxedRoundMax[0.4]{thus, an attempt for realizing research action(s) in a non-authoritarian and non-hierarchical manner.}
\textBoxedRoundMax[0.4]{... an attempt to shape a complementary component of the people's struggles on the streets.}
\textBoxedRoundMax[0.4]{...an attempt to find a non-elitist but common language, where academic and non-academic spaces may converge.}
\textBoxedRoundMax[0.4]{...an attempt to shape a space for exchange and raising of awareness about the people's struggles.}
\textBoxedRoundMax[0.4]{...an attempt to reflect on my personal practice and the contention of constraints and existing contradictions between different spheres of reality.}
\textBoxedRoundMax[0.4]{thus, an attempt to dissolve the separation of academic, political, social and private space.}
\textBoxedRoundMax[0.4]{... an attempt to avoid to represent or speak for anybody. I want to be together with the people and experience myself what they are experiencing.}
\textBoxedRoundMax[0.4]{thus, I am aware of my twofold role and its contradictions as scholar and activist and my alignment with the latter.}
\textBoxedRoundMax[0.4]{...an attempt to adapt the frame this thesis is embedded in on institutional level according to the principles I formulate here.}

\spaceHalf

\inright{strident communication as a building block towards social transformation is \aKeyword{relevant}}
My basic position, that this thesis is a complementary component of existing struggles of street people in São Paulo, does not mean that I expect concrete social change as its direct result and outcome. For me, provoking social transformation goes far beyond the scope of this work, far beyond the scope of the frame it is embedded in and constraint by. However, I think that every step towards an emancipatory and self-determined transformation of society is worthwhile to undertake, thus I hope this thesis may contribute at bit to undertake further steps into that direction and at least function as a strident mean of communication and distribution for that purpose. 

\startARemark
I could even imagine to say that this thesis could also be considered as another type of self-determined action, because I considered it as a component of a struggle and it was made possible by genuine participation between \quote{us}. I actually don't know in how far this kind of closed loop positively contributes to the thesis' form and content or in how far it would cancel out the narratives of other actions which are supposed to be included here in the first place.
\stopARemark

\spaceHalf

\inright{research is political}
The conviction to \toMark{actively support and provoke steps} that could illustrate and  pave new ways towards an emancipatory transformation of society, implies for me that this thesis' research is inherently political. It is political because the social space the city represents \bracket{here in the case of the street population in São Paulo}, is the space of struggle for transformation, the space where life and its multiple realities unfold, the space where my personal practice unfolds as well. This space is political.

\toMark{Emancipation and genuine participation could at best create spaces at all levels of the city that would not function according to the excluding logic of the currently existing ones}, according to those spaces that are representing and\bracket{re}producing social discrimination. Actions performed in those space could have the potential to either transform them or create new spaces where discrimination and its roots are not existing any more or are at least contested and progressively dismantled.  

\spaceHalf

\inright{strident communication means open access to information}
In order to be strident \bracket{here in the sense of the german \quote{plakativ}} and probably provoke ideas or reactions, this thesis and its content have to be freely accessible. The knowledge that accumulates here and which probably could be used elsewhere, have to be freely accessible. 

\spaceHalf

\startARemark
hier könnte/sollte ich nochmal kurz auf xxx eingehen und dessen konzept von politischen räumen
\stopARemark

\spaceHalf

\inright{research becomes relevant if it can contribute to an emancipatory social transformation}
A political self-conception of my research action(s) could become \aKeyword{relevant} from my point of view, if those action(s) contribute, even though just to a small extent, to an emancipatory social transformation. Hence, I would like to denote those additional positions as ...

\spaceHalf

\textBoxedRoundMax[0.4]{... relevant to consider my research action(s) as political. }
\textBoxedRoundMax[0.4]{... relevant to consider my research as small building block towards a a social transformation which should be strident in order to distribute information.}
\textBoxedRoundMax[0.4]{... an attempt to make the information this thesis provides freely accessible and usable and not just locked-in the academic space.}
\textBoxedRoundMax[0.4]{  thus, I prefer open access to all information, narratives and thoughts this thesis is composed of.}
\textBoxedRoundMax[0.4]{\toMark{... an attempt to turn something present but invisible, visible but not to \quote{invent} something completely new,}}
\textBoxedRoundMax[0.4]{  thus, I prefer to evoke a reaction and not just a sole analytical and systematically sound reproduction of \bracket{a} \quote{reality}.}
\textBoxedRoundMax[0.4]{... an attempt to decouple the question of relevance of research action(s) from the \bracket{western} scientific norm of being innovative, objective and systematic, in order to examine situations solely on an atomic level, which masks out the context those situations have been embedded in,}
\textBoxedRoundMax[0.4]{  thus, I prefer to work qualitative and event driven instead of systematic and quantitative, in order to make as many of contingent experiences as possible}
\spaceHalf

The hereby presented \bracket{subjective} motivations and demands, compose the basic layer this thesis is build upon. The very first position of non-authoritarian and non-hierarchical behaviour pervades in a sense all other positions. Therefore I would like to draw attention on this position in order to clarify its fundamental importance for me.

\startARemark
ich hab gerade noch das gefühl das sich dieser abschnitt eher wie ein politisches manifest anhört, owohl er natürlich eine rein persönliche schilderung meiner motivation darstellt...
\stopARemark

\stopmode

\stoptext

\stopcomponent
