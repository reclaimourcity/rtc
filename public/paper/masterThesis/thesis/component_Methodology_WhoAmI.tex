\startcomponent component_Methodology_WhoAmI
\product product_Thesis
\project project_MasterThesis

% definitions and macros
\environment envThesisAllEnvironments

\define[]\sectionMethodologyWhoAmI {\section[methodology::whoami]{Who am I?}}

\starttext

\startmode[tocLayout]

\sectionMethodologyWhoAmI

Here I define the thesis self-conception and the thesis approach to research.
\stopmode

\startmode[draft]
\sectionMethodologyWhoAmI

\startKeywords
who am i?, objectivity, subjectivity, standpoint, neutrality, research actions
\stopKeywords

So, \aKeyword[Who am I?]{who am i?} and why do I want to ask this question? 

\spaceHalf

\inright{research based on the narrators standpoint}
I personally consider important to make my personal attitude transparent in order to assess in how far this attitude affected and affects the course of this thesis. This thesis is not aimed to express objective narratives because objectivity is for me hard to achieve when expressing those experiences, actions and observations that represent the core of this thesis. The concept of objective observation, which from my perspective means neutrality, devoid of symbols and biased interpretations, probably the plain and \quote{real} nature of an object, is not realizable and probably not seminal for me either, because I cannot and don't want to disconnect myself from what I experienced, what people experienced, from my attitude that clearly influenced to large part the realization of my research action(s) in São Paulo. 

\spaceHalf

\startCitation
I have the use of the information that that which I see, the images, or that which I feel as pain, the prick of a pin, or the ache of a tired muscle—for these, too, are images created in their respective modes—that all this is neither objective truth nor is it all hallucination. There is a combining or marriage between an objectivity that is passive to the outside world and a creative subjectivity, neither pure solipsism nor its opposite.\aQuoteW{Brockman}{2004}\aLinkF{http://www.edge.org/documents/archive/edge149.html}
\stopCitation

\spaceHalf

The research actions that are assembled in this thesis are those that I volunteered to perform and experience or that just happened by incident, unplanned, unstructured, but never through external force or \aKeyword{other-directed} \footnote{guided by external standards}. By \aKeyword{other-directed} and external force I mean that nobody told me what I had to do, according to her or his demands, according to the structural demands of a project, without the possibility of negotiating according to our individual interests and limits. When I stayed with my people I was always asked if I am interested in joining them, in participating in their realities.

I could have chosen another frame, an existing academic or NGO project on the same topic, where I probably would have met the same people and visited the same places, but which perhaps would have resulted in totally different outcome, based on other \aKeyword[standpoints]{standpoint} and attitudes. Is the reality I experienced then more valid than that of others or vice versa? I think not, both have their legitimacy, they are probably motivated differently and therefore narrate different stories, probably describe the same realities from different standpoints based on the narrators individual reality and context. In the words of \aKeyword[Schrödinger]{schrödinger} I would then say

\spaceHalf

\inright{"The Fundamental Idea of Wave Mechanics", Nobel lecture, (12 December 1933)}
\startCitation
We cannot, however, manage to make do with such old, familiar, and seemingly indispensable terms as "real" or "only possible"; we are never in a position to say what really is or what really happens, but we can only say what will be observed in any concrete individual case. Will we have to be permanently satisfied with this...? On principle, yes. On principle, there is nothing new in the postulate that in the end exact science should aim at nothing more than the description of what can really be observed. The question is only whether from now on we shall have to refrain from tying description to a clear hypothesis about the real nature of the world. There are many who wish to pronounce such abdication even today. But I believe that this means making things a little too easy for oneself. \aQuoteB{Schrödinger}{1933}{316}
\stopCitation

By not aiming to reproduce narratives in an \aKeyword[objective]{objectivity} manner I do not mean to dismiss the idea of \aKeyword[neutral standpoints]{standpoint+neutral}. However, the content of the thesis shall reproduce the positions, ideas and thoughts of those that shared them with me, with whom I collaborated, my personal expression of that what I perceived and experienced.

\stopmode

\addReference
{
Schrödinger, E., 1933. {\em Nobel Lecture}. In The fundamental idea of wave mechanics. pp. 305-316. Available at: \goto{\hyphenatedurl{http://nobelprize.org/nobel_prizes/physics/laureates/1933/schrodinger-lecture.pdf}} [url(http://nobelprize.org/nobel_prizes/physics/laureates/1933/schrodinger-lecture.pdf)] [Accessed August 5, 2011]. 
}
\addReference
{
Brockman, J., 2004. About Bateson. {\em Edge}, (149). Available at: \goto{\hyphenatedurl{http://www.edge.org/documents/archive/edge149.html}} [url(http://www.edge.org/documents/archive/edge149.html)] [Accessed August 17, 2011].
}

%---------------------------------------------------
% gets only displayed in unfinshed mode

\showImperfection

%---------------------------------------------------
\stoptext

\stopcomponent
