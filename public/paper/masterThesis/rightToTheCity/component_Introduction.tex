\startcomponent component_Introduction

\product product_TheRightToTheCity
\project project_MasterThesis

% definitions and macros
\environment envCfgPaperLanguageEn
\environment envRttcUrl
\environment envRttcText


\starttext

\chapter[intro]{Introduction}

The explicit struggle for {\em right to the city} has been one of the major urban struggles in the last couple of years.

Nowadays, \inright{powerlessness of the citizen} cities are the living ambient for the majority of the world's population, from small towns to mega- and global cities. Simultaneously to the cities growth (and sometimes their shrinking as well), their literally \quote{explosion} as it can be observed in contemporary megacities and metropolitan regions, its citizens lost any possibility to shape the city they live in according to their ideas, wishes and needs. Decisions are made by a political elite, often driven by economic principles and reasoning, without or with very limited possibilities of participation by the citizens.

The \inright{urban processes and development} shape and structure of the city is changing, driven by or enforcing processes such as {\em gentrification, segregation,} or {\em rural-urban-international-migration}. Many western cities experience an expulsion and substitution of long time residents from their quarters and neighbourhoods by a new, young, dynamic and relatively rich elite. Mega-events such as the Football World-Cup or the Olympic Games are the justification for massive urban infrastructure investment (housing, transport, leisure) and the revitalization of whole city regions, in the centres and peripheries, which usually force many residents to leave the area (in the best case \quote{voluntarily} or by eviction in the worst case) due to increasing living costs or their \quote{inappropriate} social background and appearance. 

These and other factors and processes are often accompanied by high levels of repression in guise of police or private security agents which execute state led decisions and defend private interests, against the will of the citizen. 

Despite \inright{the right to the city} this development, a rising number of cities are scenes of an emerging struggle for the right to shape and influence the development of and the life in the cities, driven by the often excluded, marginalized or discriminated citizens. 


\stoptext
\stopcomponent
