\startcomponent component_MethodologyTheoryAndPracticeEmpirical
\product product_Thesis
\project project_MasterThesis

% definitions and macros
\environment envThesisAllEnvironments

\define[]\subsectionEmpirical{\subsubsection[methodology::motivation::empirical]{Empirical versus Theoretical}}

\starttext

\startmode[tocLayout]

\subsectionEmpirical

This sub-section argues whether this thesis is based on empirical information or not and what the alternatives are.

\stopmode

\startmode[draft]

\subsectionEmpirical

\startKeywords
empirical, theoretical, objective, observation, daily life, systematic, unsystematic, reproducibly, randomly
\stopKeywords

It has already been mentioned that the thesis structure and outcome is impacted by the experience made in São Paulo. This experience is composed of particular situations, events and narratives, bounded to the local context and the local people, not to the whole city but just a tiny fraction of it. Therefore it has to be seen in how far the situation(s) narrated in this thesis can and should be generalized, where it is reasonable and where unnecessary. 

\spaceHalf

\inright{empiricism: objective and reproducible}
One aspect that will be referred to later on in more detail, when introducing different types of research action(s), is the aspect of objectivity and reproducibility of observed situations and made experiences. Those notions are normally referred to particular types of research inquiries which are grounded on empiricism.

\startADefinition
In contrast to theory, \inright{empricism: a brief definition} empiricism observes situations and draws conclusions that cannot be explained and derived in a formal or logic manner, such as a mathematical formula. In order to derive conclusions, empiricism is supposed to be performed in a systematic and objective manner, grounded on \toMark{(reliable and analysable)} data and information, \toMark{ref missing}, the base of \toMark{western} science. This shall distinguish empiricism from observing randomly occurring daily life experiences, which are claimed to be non-producible in a systematic manner \toMark{ref missing}. Empiricism means therefore systematic inquiry of the daily life.
\stopADefinition

\spaceHalf

\inright{observation: open and non predictable}
In stark contrast, other types of research actions rely on observations that do not claim to be systematic and which question the objectivity of gathered data and reproducibility of situations. Those observations may occur when neither the concrete local context is known in advance by the research agent, nor the outcome of the research is clearly defined. Therefore, necessary knowledge has to be gained in the first place in order to formulate frame of research and action(s). This knowledge is composed of rather daily situations, realities and conditions, which are not predictable beforehand and which affect and construct the frame research is bounded by.

\spaceHalf

\inright{open local observations and narratives do not fit with empiricism}
Therefore, this thesis can not be considered empirical because the narratives of particular situations, experiences and observations made in São Paulo had to be made in an unsystematic manner due to the fact that the whole local context was unknown to me. Hence, the unawareness off the city of São Paulo, the thousands of realities it represents, the complexity of its social and political structures, the whole universe called São Paulo, automatically implies that unsystematic observation and participation is one of the viable options that allow to dig a bit deeper than just to scratch on the surface. 

For the sake of completeness, it should be mentioned that another option would have been the participation in an existing institutional project. For me this would have been antipodal to the intended non-hierarchical/non-authoritarian approach due to the fact that an existing project already represents a fixed frame, hierarchies are already established and existing (institutionalized) social networks have to be utilized. The fixed frame leaves little space for navigation and by that the thesis could have been degenerated to the mere achieving of the title and the fulfilment of project goals, no matter if reasonable and justifiable or not. Thus, project based thinking and working would have prevailed. This type of thinking was supposed to be pushed aside for and during this thesis.

I had to search for one way out of many that was viable for me and the ones I got to be acquainted with on that discovery. The role as observer and participator will be illustrated in more detail later on as well.

\spaceHalf

\inright{relevance of theoretical versus observed and experienced information}
In the end, if one argues about relevance in the context of this thesis, personal observation and participation contemplates those complex realities that are rather invisible in daily life and public discourse, while theory can primarily be used to translate those realities into another form. Thus, talking about those realities is relevant for the thesis, because then, this thesis and its information could become another mosaic that composes the struggle of those people that facilitated the insight in and cognition of the city of São Paulo and their realities and by that impacted and shaped the current form of this thesis. Relevance for the thesis means relevance for the people and vice versa. The question of relevance is crucial here because research actions were cooperative work and cannot be divided into research agent and research target. Thus research outcome, apart from putting everything into the form of a thesis, is cooperative work as well, thus work we have done together, and which is relevant for us. 

{\em In other words, this thesis has been shaped from and is drawn on concrete but unsystematic experiences, observations and actions, thus cannot be considered as a thesis of empirical origin or with empirical focus.}

\stopmode

\stoptext

\stopcomponent
