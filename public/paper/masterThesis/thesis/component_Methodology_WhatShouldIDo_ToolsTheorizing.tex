\startcomponent  component_Methodology_WhatShouldIDo_ToolsTheorizing

\product product_Thesis
\project project_MasterThesis

% definitions and macros
\environment envThesisAllEnvironments

% setup for table all cells
\setupTABLE[row][each]
[
	background=color,
	foreground=color,
	frame=on,
	rulethickness=2pt,
	corner=00,
	offset=2pt
]

\setupTABLE [y][first]
[	
	background=color,
	frame=off,
	rulethickness=2pt,
	foregroundcolor=black
]


\define[]\subsectionMethodologyToolsTheorizing {\subsection[methodology_whatshouldido_toolstheorizing]{Tools//Theorizing}}

\starttext

\startmode[tocLayout]

\subsectionMethodologyToolsTheorizing 

Describes the tools for theorizing.

\stopmode

\startmode[draft]

\subsectionMethodologyToolsTheorizing 

\startKeywords
citizenship, right to the city, spatial justice, participation, open access, media, theorizing, journals, resources, tools
\stopKeywords

In contrast to research actions in São Paulo, \aKeyword[theorizing]{thesis+theorizing} for this thesis remains to a large extend my individual work as it has \gotoTextMark[already been stated]{methodology_whoami_actionresearch_nocotheorizing}. This thesis theorizing draws on other theorizing manifold in form and expression. This thesis theorizing draws also on those \gotoTextMark[denotations]{methodology_whoami_actionresearch_notionsoftheorizing} made when introducing \abbrFull{AR} as overall research framework. 

\spaceHalf

\inright{mutual nurturing of academic and social space as objective}
\reference[methodology_whatshouldido_theorizing_nurturing]{}One of the \gotoTextMark[already defined]{methodology_whatdoiwant_guiding} \aKeyword[objectives]{objectives+general} of this thesis is the convergence and eventual overcoming of separateness of \aKeyword{theorizing} in academic and social space, thus the mutual nurturing of both spaces in order to facilitate the struggle for social and emancipatory transformation.

In order to reach this \aKeyword[objective]{objectives+general}, the \aKeyword[partial knowledge]{knowledge+partial+movement} from the streets must enter this thesis as well as the related \aKeyword[partial academic knowledge]{knowledge+partial+academia}. It will be proposed in the next paragraph which knowledge, theoretical discussions and considerations this could eventually be.

\spaceHalf

\inright{open accessibility of content and sources}
Further on, I would like to realize the demand of \aKeyword[open accessibility]{knowledge+open access} of produced content but also of used sources of information. Open access to sources shall give everyone the possibility to easily access and allow an individual or collective reflection on them. This has several consequences, mainly with respect to access of academic papers but also with respect to access to information from sources in São Paulo, such as newspapers or movement content. Therefore I would like to separately handle these types of access to information.

\placefigure[force]{Types of sources of knowledge and content}
{
\starttyping

				open access

			/BTEX \tfa\ss\color[green]{oa journals} /ETEX

				/BTEX \tfa\ss\color[green]{own content} /ETEX
							/BTEX \tfa\ss\color[green]{movement content} /ETEX

academia			/BTEX \tfa\ss\color[green]{newspapers	} /ETEX			society

					/BTEX \tfa\ss\color[green]{books} /ETEX

	/BTEX \tfa\ss\color[green]{traditional journals} /ETEX
	
				restricted access

\stoptyping
}

\spaceHalf

\inright{the movements standpoints affect the shape of this thesis}
I would also like to note again that this thesis is written from a particular \aKeyword[standpoint]{standpoint+movement}, from the \gotoTextMark[standpoint of the people on the streets]{methodology_whoami_actionresearch_standpoint} because I considered myself affiliated with them when I has been in São Paulo and due to the fact that I experienced the city to a certain extend through them and through their experience.

\spaceHalf

\inright{my personal standpoint affects the shape of this thesis}
My \aKeyword[personal standpoint]{standpoint+personal} may also be present in this thesis, which probably affects the sources of information I am going to select and to utilize for the thesis theorizing. Especially when we talk about \abbrFull{OA} to sources of knowledge and content, the knowledge in corresponding journals or other free sources does not represent the full spectrum of available knowledge \bracket{most of which is still locked up behind academic or corporate walls}. If then \abbr{OA} journals and other free sources organize their \aKeyword[knowledge production]{knowledge+production} and \aKeyword[distribution]{knowledge+distribution} according to other conventions \bracket{where knowledge is treated as a common resource that shall benefit all, for instance}, the accessible content may reflect these modes of access and production standards. The \aQuoteInTextT{Manifest}\aLinkNewF{http://www.rhizomes.net/files/manifesto.html} of the \aQuoteInTextA{Rhizomes Journal} \aLinkNewF{http://www.rhizomes.net} illustrates this succinctly.

\spaceHalf

\startCitation
Rhizomes oppose the idea that knowledge must grow in a tree structure from previously accepted ideas. New thinking need not follow established patterns. [...] We are not interested in publishing texts that establish their authority merely by affirming what is already believed. Instead, we encourage migrations into new conceptual territories resulting from unpredictable juxtapositions \aQuoteWA{Rhizomes}.
\stopCitation

\spaceHalf

Such concepts may affect the knowledge accessible for me and by that the ground I draw my argumentation upon. By mentioning this I once more would like to render transparent the question of \bracket{academic} \aKeyword[objectivity]{knowledge+objective} and \aKeyword[neutrality]{knowledge+neutral} versus \aKeyword[partial knowledge]{knowledge+partial} that is produced according to different \aKeyword[standpoints]{standpoint}, which is the perspective I am committed to by choosing \abbrFull{AR} as overall \aKeyword[research framework].

\subject{Which knowledge and content is relevant then?}

Within the considerations of \abbrFull {AR} as \gotoTextMark[research framework]{methodology_whoami_actionresearch}, \gotoTextMark[several themes]{methodology_whoami_actionresearch_topics} have already been mentioned. These themes are expressed in various flavours in the demands of \aKeyword[urban social movements]{social movement+urban} in São Paulo. 

The following passages are taken from manifests and flyers of \abbrNew{Frente da Luta pr Moradia}{FLM}, \abbrNew{Movimento Nacional Catadores de Rua}{MNCR}\abbrNew{Movimento Nacional da Populacão de Rua}{MNPR} and Rede de Extremo Sul de São Paulo. They shall briefly illustrate some of the actual discourses that are pushed forward from the movements standpoint. 

\refMissingSrc{flyers with demands of MNCR, MNPR, FLM, etc}

\spaceHalf

While being together and discussing with the people, I comprehend that the mentioned themes are related to the concrete praxis that urban social movements and collectives exhibit, may it be through self-determination and participation in actions, through the question of \aKeyword[Who we are?]{who we are} and the related citizenship discourse or through the struggle for \aKeyword{access to the city} and its assertion of the \aKeyword{right to the city}. 

\spaceHalf

\placefigure[force]{Themes of this thesis theorizing}
{
\starttyping		

				/BTEX \tfa\ss\color[green]{right to the city} /ETEX

/BTEX \tfa\ss\color[green]{self-determination} /ETEX						/BTEX \tfa\ss\color[green]{citizenship} /ETEX
							
		/BTEX \tfa\ss\color[green]{participation} /ETEX

					/BTEX \tfa\ss\color[gray]{spatial justice} /ETEX

\stoptyping
}

\spaceHalf

Thus by following the discourses of urban social movements in São Paulo and \refMissingSrc{succinct research on academic and movement discourse in general...}

\spaceHalf

\textBoxedRoundMaxObj[0.4]
{
 ...I would propose to take the following themes into account, as core of this thesis \aKeyword[theorizing]{theorizing+overview}: \aKeyword[the right to the city]{theorizing+the right to the city}, \aKeyword[self-determination]{theorizing+self-determination}, \aKeyword[participation]{theorizing+participation}, \aKeyword[citizenship]{theorizing+citizenship}, \aKeyword[spatial justice(probably)]{theorizing+spatial justice}
}

\textBoxedRoundMaxObj[0.4]
{
...I would like to examine these themes on an abstract level as my \aKeyword[contribution to movement theorizing and struggle]{objectives+contribution} and in order to provide access to related content and knowledge that could be applied in further movement theorizing.
}

\spaceHalf

For me, those themes are inherently connected to the city as social space and metaphor for society, thus we that live in the cities, the way we are organized \bracket{on all levels} in our \aKeyword[theorizing+lived space]{lived space}, how our \aKeyword[lived space]{theorizing+lived space} is organized, how the notion of \aKeyword[citizenship]{theorizing+citizenship} is currently used and how \aKeyword[self-determination]{theorizing+self-determination} and \aKeyword[participation]{theorizing+participation} in the \aKeyword{production of the city} asserts \aKeyword[access]{theorizing+access to the city} and the \aKeyword[right to the city]{theorizing+right to the city}, which oppose the contemporary \aKeyword{other-directed} praxis of city production. Therefore, ...

\spaceHalf

\textBoxedRoundMaxObj[0.4]
{
... I would like to examine the mentioned themes from the \aKeyword{standpoint} of self-determined and \aKeyword{emancipatory praxis} because those standpoints are inherent to this thesis but often also part of the praxis of social movements and collectives in São Paulo.
}

\spaceHalf

What I think is relevant for this thesis theorizing is the examination of the prospects that self-determined and participatory production of the city may provide. Therefore...

\spaceHalf
\textBoxedRoundMaxObj[0.4]
{
... I would like to consider the new \bracket{social, political, lived} spaces that could be constructed while examining the selected themes.
}

\subject{Channels of academic knowledge}

\inright{open access journals, blogs and other open sources}
As sources of \aKeyword[academic knowledge]{knowledge+academic} I will mainly use \aKeyword[open access journals]{open access+journal}\footnote{Traditionally, a journal serves as publication channel of academic papers and research results. It is fed by scholars and serves academic agents. A traditional journal claims to provide high quality standards through peer review of publications by specialists, profound in the different topics. It is thus like a library of specialized publications, where only a selected and approved number of publications enters and where access is restricted mainly to academic and research agents which still have to pay a high fee for their library card} and papers that are freely available on the internet and whose licence allows reuse, such as \aKeyword[creative commons]{open licence+creative commons}\footnote{An open licence allows authors to keep their property rights for their product instead of transferring them to a publisher. An open licence also gives an authors the freedom to share with others and grant others the right to reuse instead of denying them any right that goes beyond the right for consumption.} or the like. All sources will be listed in the thesis \gotoTextMark[reference chapter]{refBib} with the link to their download addresses. 

I will also make use of articles and essays available on scholar's websites, blogs and other online platforms if I find them useful for this thesis. In certain cases, such as books and other printed media, no online access may be possible. I will try to minimize this kind of sources wherever possible as long as I think that their exclusion can be compensated with an equivalent that is open accessible. Even though I have access to a certain number of closed scientific journals due to my status as student, I will only make use of them if the provided information are freely accessible.

During the course of literature selection and research I discovered an increasing number of academic \abbrFull{OA} journals, in social sciences for instances. Besides a larger number of still very academically aligned \abbr{OA} journals, a smaller number of open access journals is emerging, which are theorizing for instance \quote{for and about social movements} (\aQuoteInTextA{Interface Journal}). There, one can already perceive the convergence of academia and social movements because published articles are written from the standpoint of a movement, as reflection on the peoples struggle but also from the standpoint of activists rooted in academic and movement space.

\subject{Channels of peoples knowledge}

\inright{digital movement content}
The scene of movement and people theorizing is quite different. In São Paulo, knowledge and content are disseminated through different channels. Movements and collectives make their content freely accessible online, on own websites, blogs or social media platforms such as Flickr or Vimeo. Thus movement content is produced not only in text form but spans a wide range of mediums. Photographic documentations of events are accessible at Flickr, alternative media coverage of events and issues are posted to Indymedia Brasil, communities of the peripheries of São Paulo feed their own blog with reports from their sites.

\spaceHalf

\inright{concrete movement content}
Besides virtual channels, movement content can always be found at the local level, at self-organized and social centres, at events and actions in the city, or distributed by vendors of street papers for instance. When being together with the people, we frequented many spaces in the city, day by day, for various purposes, in order to fetch food, to participate in a workshop about poetry, to conduct an interview in a occupation. Being at those space always meant the discovering of various publications made by movements and collectives. Publications have different forms and content, the call for demonstration on a small printed flyer, a handout with background information about a particular struggle, a manifest of an occupation, or even DVD's with collectively made films. 

In contrast to the readily available concrete and digital content, theorizing whose outcome is content, takes place in various settings. 

\spaceHalf

\inright{theorizing through discussion}
I participated in various assemblies and workshops of different movements and collectives and got a glimpse on the relevant topics of the peoples struggles. But not only \quote{official} assemblies and meetings provide space for discussion and theorizing, also our time spend together discussing in the \locMissingSrc{park} below the train rails in \locMissingSrc{Brás} or at the \refMissingSrc{public Piano} in the \locMissingSrc{Luz} train station or the nearby \locMissingSrc{Parque da Luz} offered plenty of space for exchange of experiences and arguments about the situations we are residing in. 

On some occasion, movements and academia joined and shared the same space. During the \toTranslate[First Colloquium of Autonomous Territories]{Primeiro Colóquio Território Autônomo}\aLinkNewF{https://territorioautonomo.wordpress.com/} in Rio de Janeiro, the question was raised how social movements and academia could support each other and cooperate in struggle from a spatial, libertarian and autonomous perspective \aQuoteW{Primeiro Colóquio Território Autônomo}{2010} \aLinkNewF{https://territorioautonomo.wordpress.com/convite/}.

\spaceHalf

\inright{theorizing through experience and perception}
Besides discussions and dialogues, the concrete experience and perception of the city is tremendous and important for me personally, because through my lived experience I could make up my own mind and understand the arguments of the people that made and make those experience in a much more intense fashion and on a more frequent base. I perceived our walks through the centre, our rambling through the streets for two days and nights, our visit to the new occupations for conducting interviews, or the young guy on crack I ran into and talked to several times, as a kind of liberation from the abstraction and depersonalization of theoretical and research papers on similar topics. As I argued \gotoTextMark[elsewhere]{methodology_whoami_experience}, being on the streets directed my vision to those topics that now enter this thesis and that helped me to understand what the purpose of this thesis could be.

\subject{Channels of other knowledge}
\inright{newspapers}
Besides movement and academic theorizing and content, other sources of knowledge and content have not been left aside. Newspapers from São Paulo such as \refMissingSrc{Folia de São Paulo} for instance were \refMissingSrc{frequently reporting} about issues related to the streets, often about conflicts or the corresponding \refMissingSrc{political agendas} to handle the \quote{problem of street populations}. Published orders of political agents are to a certain extend part of the movements and peoples \refMissingSrc{formulation of critiques and demands}. To a certain extend, newscasts are accessible online for free or are gathered on \refMissingSrc{websites and blogs affiliated with the streets and its struggles}. 

\spaceHalf

\inright{more on social movements and alternative content }
For theorizing I also incorporate content from zines and movements about urban struggles and urban development that are not rooted in São Paulo, such as the \aLinkNew[Abahlali baseMjondolo]{http://www.abahlali.org/} a slum dwellers movements from South Africa or any other source of inspiration that helps me to realize this thesis. Especially this methodology chapter with its immanent question about the \gotoTextMark[meaning of knowledge]{methodology_whoami_actionresearch_knowledge} and the process of knowledge production and formulation is drawn on several free sources that are related to militant ethnography \aLinkNewF{http://periferiesurbanes.org/?p=165}\aLinkNewF{http://periferiesurbanes.org/?p=2136}\aLinkNewF{http://periferiesurbanes.org/?p=2553}, action research and feminist struggle.

\spaceHalf

\inright{more on open journals}
Finally, \abbr{OA} journals that draw on different forms of knowledge production and theorizing about topics related to this thesis are taken into account as well because they provide even more differing perspectives, from even more differing standpoints.

\startARemark{bin noch nicht ganz glücklich mit diesem letzten teil, hört sich so wischiwaschi an}
\stopARemark

\subject{The main setting of knowledge and information sources}

The sources of content and knowledge can now be assembled into \aKeyword[pools of open knowledge and content]{open knowledge pools+general}. The mentioned sources may only render very broad pools that are utilized for this thesis and that are actually extended by a relatively large number of individual sources, too many to mention here. 

I will also make a distinction between offline and online access because some sources are most easily accessible online because their main distribution platform with the highest outreach is the internet, such as \abbr{OA} journals, while others are just available offline, such as street papers, flyers and the like, because they are primarily addressed to the local people. Even though the virtual world provides plenty of inspirations and content for reuse, being on the streets often provides just temporary means for entering the virtual space and much information can only be found in printed form, offline, distributed at social or cultural centres, at demonstrations or other urban spaces.

As mentioned before, several pools of open sources are utilized in this thesis. A \aKeyword[pool of \abbr{OA} journals]{open knowledge pools+open access journals}, mainly a resource of academic theorizing, disconnected from the streets in São Paulo. A \aKeyword[pool of sources for movement theorizing]{open knowledge pools+movement theorizing} mainly related to the social struggles in São Paulo, and a \aKeyword[pool of mixed sources]{open knowledge pools+mixed content}, not necessarily related to strict movement or academic theorizing, located in São Paulo but also detached from any concrete place, covering the themes of this thesis from different perspectives, according to different conventions and objectives.

\spaceHalf

% setup for table all cells
\setupTABLE[row][each]
[
	backgroundcolor=black,
	foregroundcolor=white,
	framecolor=green,
]

\setupTABLE [y][first]
[	
	backgroundcolor=green,
	foregroundcolor=black,
]

\placetable[force]{Accessed Open Access Journals \bracket{online}}
{
\bTABLE
\bTR
	\bTD[nc=3, align={middle,lohi}] {\tt\bf Accessed \abbrFull{OA} Journals \bracket{online}} \eTD 
\eTR 
\bTR 
	\bTD \color[green]{\tt Interfaces}\aLinkNewF{http://interfacejournal.nuim.ie} a journal for and about social movements. \eTD 
	\bTD \color[green]{\tt Justice Spatiale - Spatial Justice} \aLinkNewF{http://www.jssj.org} a journal about spatial justice and spatial inequality on from local to global scales. \eTD 
	\bTD \color[green]{\tt International Journal of Communication}\aLinkNewF{http://ijoc.org/ojs/index.php/ijoc/index} a Journal centred in communication, networks and society. \eTD 
\eTR
\bTR 
	\bTD \color[green]{\tt Social Science Open Access Repository}\aLinkNewF{http://www.ssoar.info/} a repository of articles and papers centred in social science.  \eTD 
	\bTD \color[green]{\tt Forum Qualitative Sozial Forschung - Forum Qualitative Social Research}\aLinkNewF{http://www.qualitative-forschung.de} a Journal that addresses qualitative research. \eTD 
	\bTD \color[green]{\tt Techné}\aLinkNewF{http://scholar.lib.vt.edu/ejournals/SPT/} a Journal about research in philosophy and technology. \eTD 
\eTR
\bTR 
	\bTD \color[green]{\tt eScholarship}\aLinkNewF{http://escholarship.org} a repository provided by the University of California \eTD 
	\bTD \color[green]{\tt Scientific Commons}\aLinkNewF{http://en.scientificcommons.org/} a repository of articles and papers. \eTD 
	\bTD \color[green]{\tt Kommunikation@Gesellschaft}\aLinkNewF{http://www.ssoar.info/de/portale/kommunikationgesellschaft.html} a Journal about society, media and communication. \eTD 
\eTR
\eTABLE
}

\spaceHalf

% setup for table all cells
\setupTABLE[row][each]
[
	backgroundcolor=black,
	foregroundcolor=white,
	framecolor=orange,
]

\setupTABLE [y][first]
[	
	backgroundcolor=orange,
	foregroundcolor=black
]

\placetable[force]{Accessed Movement Content \bracket{online and offline}}
{
\bTABLE
\bTR
	\bTD[nc=3, align={middle,lohi}] {\tt\bf Accessed Movement Content \bracket{online and offline}} \eTD 
\eTR
\bTR
	\bTD \color[orange]{\tt Indymedia Brazil}\aLinkNewF{http://midiaindependente.org} an open platform for self-publishing of independent and critical media\eTD
	\bTD \color[orange]{\tt Passa Palavra}\aLinkNewF{http://passapalavra.info}\eTD
	\bTD \color[orange]{\tt Ocas}\aLinkNewF{http://www.blogdaocas.blogspot.com} a street paper in São Paulo\eTD
\eTR
\bTR
	\bTD \color[orange]{\tt Forum Centro Vivo} a forum about urban reform in Brazil\eTD
	\bTD \color[orange]{\tt Flyers, Posters, Handouts} made by movements in São Paulo\eTD
	\bTD \color[orange]{\tt  Narrations and Poems} made by people in São Paulo\eTD
\eTR 
\bTR
	\bTD \color[orange]{\tt Photos and Videos} made by people in São Paulo\eTD
	\bTD \color[orange]{\tt  Own media} such as audio and video recordings\eTD
	\bTD \color[orange]{\tt }\eTD
\eTR
\eTABLE
}

\spaceHalf

% setup for table all cells
\setupTABLE[row][each]
[
	backgroundcolor=black,
	foregroundcolor=white,
	framecolor=yellow,
]

\setupTABLE [y][first]
[	
	backgroundcolor=yellow,
	foregroundcolor=black,
]

\placetable[force]{Other Content Resources \bracket{online and offline}}
{
\bTABLE
\bTR
	\bTD[nc=3, align={middle,lohi}] {\tt\bf Other Content Resources \bracket{online and offline}} \eTD 
\eTR
\bTR
	\bTD \color[yellow]{\tt Reclaiming Spaces}\aLinkNewF{http://www.reclaiming-spaces.org}\eTD
	\bTD \color[yellow]{\tt Occupied London}\aLinkNewF{http://www.occupiedlondon.org/}\eTD
	\bTD \color[yellow]{\tt Republicart}\aLinkNewF{http://www.republicart.net/}\eTD
\eTR
\bTR
	\bTD \color[yellow]{\tt Rhizomes}\aLinkNewF{http://www.rhizomes.net/}\eTD
	\bTD \color[yellow]{\tt [Instituto Pólis}\aLinkNewF{http://www.polis.org.br/}\eTD
	\bTD \color[yellow]{\tt Books, Blogs, Web-Platforms and Services}\eTD
\eTR
\bTR
	\bTD \color[yellow]{\tt São Paulo Newspapers} such as \refMissingSrc{newspapers}\eTD
	\bTD \eTD
	\bTD \eTD
\eTR
\eTABLE
}
\spaceHalf

\inright{selection of a publishing licence}
In order to allow reproduction, reuse and access to this thesis, all content will be published under an \aKeyword{open licence} \aLinkNewF{http://creativecommons.org/licenses/by-sa/3.0/}. The licence is not restricted to non-commercial use only because I think that commercial users shall provide their content in an open access manner as well if they make use of free content. This would allow access and reuse of commercial content as well, which is normally restricted\footnote{ you are free \startitemize[packed] \item to Share - to copy, distribute and transmit the work \item to Remix - to adapt the work \item to make commercial use of the work \stopitemize }.

\spaceHalf

\textBoxedRoundMaxObj[0.4]
{
This thesis and all further online content is published under a \abbrNew{Creative Commons Attribution-ShareAlike 3.0 Unported}{cc by-sa 3.0} \aKeywordInVi{Creative Commons Attribution-ShareAlike 3.0 Unported} licence. 
}

%--------------------------------------
% only active in "unfinished" mode

\addReference
{
FQS, Forum Qualitative Sozialforschung / Forum: Qualitative Social Research. {\em Forum Qualitative Sozialforschung / Forum: Qualitative Social Research}. Available at: \goto{\hyphenatedurl{http://www.qualitative-research.net/index.php/fqs/index}} [url(http://www.qualitative-research.net/index.php/fqs/index)] [Accessed August 6, 2011].
}

\addReference
{
IJoC, International Journal of Communication. {\em International Journal of Communication}. Available at: \goto{\hyphenatedurl{http://ijoc.org/ojs/index.php/ijoc}} [url(http://ijoc.org/ojs/index.php/ijoc)] [Accessed February 24, 2010]. 
}

\addReference
{
Interface, Interface: a journal for and about social movements. {\em Interface}. Available at: \goto{\hyphenatedurl{http://interfacejournal.nuim.ie/}} [url(http://interfacejournal.nuim.ie/)] [Accessed August 6, 2011].
}

\addReference
{
jssi, justice spatiale / spatial justice. {\em justice spatiale / spatial justice}. Available at: \goto{\hyphenatedurl{http://www.jssj.org/}} [url(http://www.jssj.org/)] [Accessed August 6, 2011]. 
}

\addReference
{
Kommunikation@Gesellschaft, 2009. Kommunikation@Gesellschaft. {\em Kommunikation@Gesellschaft}. Available at: \goto{\hyphenatedurl{http://www.ssoar.info/de/portale/kommunikationgesellschaft.html}} [url(http://www.ssoar.info/de/portale/kommunikationgesellschaft.html)] [Accessed January 6, 2010]. 
}

\addReference
{
Reclaiming Spaces, reclaiming spaces: project. {\em reclaiming spaces}. Available at: \goto{\hyphenatedurl{http://www.reclaiming-spaces.org/project/}} [url(http://www.reclaiming-spaces.org/project/)] [Accessed September 27, 2010]. 
}

\addReference
{
republicart, republicart. {\em republicart}. Available at: \goto{\hyphenatedurl{http://www.republicart.net/index.htm}} [url(http://www.republicart.net/index.htm)] [Accessed November 15, 2010]. 
}

\addReference
{
Rhizomes, Rhizomes. {\em Rhizomes}. Available at: \goto{\hyphenatedurl{http://www.rhizomes.net/files/masthead.html}} [url(http://www.rhizomes.net/files/masthead.html)] [Accessed July 19, 2011]. 
}

\addReference
{
Scientific Commons, 2010. Scientific Commons / A Community for Scientific Information. {\em Scientific Commons / A Community for Scientific Information}. Available at: \goto{\hyphenatedurl{http://en.scientificcommons.org/}} [url(http://en.scientificcommons.org/)] [Accessed January 6, 2010]. 
}

\addReference
{
SSOAR, Social Science Open Access Repository. {\em Social Science Open Access Repository}. Available at: \goto{\hyphenatedurl{http://www.ssoar.info/}} [url(http://www.ssoar.info/)] [Accessed August 6, 2011]. 
}

\addReference
{
Techné, Techné: Research in Philosophy and Technology. {\em Techné: Research in Philosophy and Technology}. Available at: \goto{\hyphenatedurl{http://scholar.lib.vt.edu/ejournals/SPT/}} [url(http://scholar.lib.vt.edu/ejournals/SPT/)] [Accessed January 4, 2010]. 
}

\addReference
{
University of California, eScholarship. {\em University of California}. Available at: \goto{\hyphenatedurl{http://escholarship.org/}} [url(http://escholarship.org/)] [Accessed April 5, 2010]. 
}

\showImperfection

\stopmode

\stoptext

\stopcomponent
