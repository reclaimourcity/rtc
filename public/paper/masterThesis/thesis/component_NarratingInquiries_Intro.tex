\startcomponent component_NarratingInquiries_Intro
\product product_Thesis
\project project_MasterThesis

% definitions and macros
\environment envThesisAllEnvironments

\define[]\chapterNarratingInquiries{\chapter[narratinginquiries_intro]{Narrating Inquiries}}

\starttext

\startmode[tocLayout]
\chapterNarratingInquiries

Narrating Inquiries about the São Paulo experience.

\stopmode

\startmode[draft]
\chapterNarratingInquiries

Narrating diaries are what I sense a genuine expression of the sensible of the streets of São Paulo. Those diaries are subjective accounts of many situations, occurrences and thoughts. I do not try to strictly order them, chronologically or according to certain categories, because they are also an account of the unpredictable born by the streets. As such I want them to apparently randomly float around as they cropped up to me on the streets, further constructing the \rhizomaticMap. Narrating shall also give an account to the spaces that are created by the people through their actions, as individual subjects or as collectives or social movements. Those spaces often seem to be invisible to those outside a particular social struggle, thus showing what the people do, is what I think could be the purpose of narrating. I probably cannot do more because I felt that I just started to get involved more when my time in the city slowly came to an end, as I have already said \gotoTextMark[when asking who I am]{methodology_whoami_experience}. These diaries consists of personal notes, but also of public media accounts and movement content. I think I cannot draw too deep on each single bit of information, its is just too much, there are too much tracings to be included in my map, thus I will also leave out quite a bit \bracket{which is not lost but postponed to the times after this writing}. At some pages I will just leave tracings in form of flyers, pictures, links to websites or translated excerpts of movement manifests in order to catch a glimpse on what probably bears a whole new world, a whole new \rhizomaticMap. At another time I will just narrate dialogues or describe what we have said, what has been told to me. I hope that this approach is amenable and not too fuzzy...

%---------------------------------------------------
% gets only displayed in unfinished mode

\showImperfection

%---------------------------------------------------

\stopmode

\stoptext

\stopcomponent
