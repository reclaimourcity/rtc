\startcomponent component_NarratingInquiries_SaoPauloDiaries_RepublicaWithJuvenil
\product product_Thesis
\project project_MasterThesis

% definitions and macros
\environment envThesisAllEnvironments

\define[]\subsectionSaoPauloDiariesRepublica{\subsection[narratinginquiries_saopaulodiaries_republica]{Around//Praça de República}}

\define[]\subjectSmashed{\subject{Sad Encounter}}
\define[]\subjectInformalStreetlife{\subject{Informal Streetlife}}
\define[]\subjectInternetForFree{\subject{Internet for free}}
\define[]\subjectSaturdays{\subject{Saturday's}}

\starttext

\startmode[tocLayout]
\subsectionSaoPauloDiariesRepublica
This section contains narratives experienced around Praça de República in the centre of São Paulo.

\subjectSmashed
How we met a battered women that disappeared for several days.

\subjectInformalStreetlife
About informal work on the streets in the centre.

\subjectInternetForFree
About spots with free access to the internet

\subjectSaturdays
About a saturday in the park and from there to the center
\stopmode

\startmode[draft]
\subsectionSaoPauloDiariesRepublica

Everytime I met \Juvenil at \aLinkE{ocas} we finally walked by foot from \aLocationE{bras} to \aLocationE{republica} in the centre. \Juvenil walks everyday, in the morning from his \aLinkE{albergue} to several spots in the city, often to \aLinkE{ocas} in \aLocationE{bras}, to temporal work or to his \aLinkE{syndicate}. In the evening he stays at \aLocationE{republica}, right in the centre, before he returns, still by foot, to his \aLinkE{albergue}. 

During this walks we often encountered unexpected situations and other people from the streets. Especially around \aLocationE{republica} at night, another type of citizens is representing the streets, certainly different from the people that frequent the same region by day on their ways to their offices, shops and working places.

\aLocationE{republica} \inright{Praça de República} and its adjacent and diverging streets at night means the encounter of informal DVD vendors (\toTranslate{DVD vendors}), informal food, \toMark{Cachaça} and softdrink distributors, circulating military and civil police, street population, \aLinkE{Catadores}, drag kings and queens, normal pedestrians and other inhabitants of the centre, all passing, meeting and doing business there. 

At night, when the shops are closing, pretty much garbage is dumped in the pedestrian areas forming big piles, often \isCorrect{to the benefit} of the \toMark{Catadores} which are even more present at night then during the day, gathering there and loading their \toTranslate{catador karren} with Cardboard (\toMark{Papelão}) or other recyclable materials.

\startPersonal
Immer wenn \Juvenil und ich uns in \aLinkE{ocas} getroffen haben sind wir von dort zum \aLocationE{republica} zu Fuß gelaufen. Immer der gleiche Weg über das \aLocationE{viaduto25demarco} nach \aLocationE{se} und von dort über das \aLocationE{viadutocha}, durch die Füßgängerzone zum \aLocationE{republica}.

\Juvenil geht eigentlich jeden Tag zu Fuß, egal wohin. Morgens geht es los von seiner \toMark{albergue}, ins Zentrum, ins Internetcafé, zu temporärer Arbeit, in sein \aLinkE{sindicato} oder nach aLocationE{ocas}. Im Laufe des Tages gibt es Essen in einem der \aLinkE{casadeconvivencia} im Zentrum oder in den angrenzenden Vierteln. Dorthin zu Fuß ist die Norm. Abends hängt \Juvenil meistens am \aLocationE{republica} rum, bevor es zurück in die \toMark{albergue} geht.

Während unserer Streifzüge durch die Straßen sind wir oft in unerwartete Situationen geraten und haben viele Andere von den Straßen getroffen. Besonders Nachts transformiert sich das Zentrum, um \aLocationE{se}, \aLocationE{luz}, oder \aLocationE{republica} mit seinen angrenzenden und abzweigenden Straßen und Wegen auf denen andere Bewohner und Besucher\_Innen des Zentrums das Bild prägen . Sie passen nicht in das tägliche Gewusel von rennenden Menschen auf ihrem Weg zur Arbeit, zum Shoppen oder ins Büro.

Während Abends die Geschäfte im Zentrum schließen, sammeln sich um \aLocationE{republica} die DVD Verkäufer mit Ihren mobilen und informellen Ständen voller \isCorrect{Kopien} aktueller Filme und Musik, die informellen \toMark{cachaca}, Alkohol und Erfrischungsgetränke VerkäuferInnen mit Ihren Ständen, ergänzt durch das Angebot von fahrbaren \isCorrect{Kochwägen}, dazwischen die \toMark{catadores} die alles brauchbare wie \toMark{papelao} aus den riesigen Müllbergen extrahieren, die die Shops tagtäglich im Zentrum anhäufen. Militär und Zivil Polizei ist immer präsent auf ihren Patrouillen zu Fuß oder in \isCorrect{viadutos}, auf der Suche nach aller Arten von informellem Business.

Die Straße wird zum Treffpunkt der Straßenbewohner\_Innen, Drag Queens und Kings, Kinderbanden, Bewohner\_Innen der Region, Jugendlichen auf Ihrem Weg zur \aLocationE{galleriaderock}, \isCorrect{nigerianischen} oder \isCorrect{bolivianischen} Immigranten, uns allen, die vorbeigehen, sich treffen und stehen bleiben und Teil dieser abendlichen Szenerie werden.
\stopPersonal

\subjectSmashed

\startPersonal

Ich hab \Juvenil an einem Montag in \aLocationE{ocas} getroffen. Abends sind wir Richtung \aLocationE{republica} gegangen. Auf dem schon oft gegangen Weg durchs Zentrum über \isCorrect{rua25demaio}, treffen wir eine Freundin von \Juvenil, eine scheinbar mittelalte Frau. Sie scheint total am Ende, ihr Gesicht ist blau und grün und geschwollen, und Tränen überströmt. 

Sie bittet \Juvenil um Hilfe, denn sie weiß nicht wo sie ist und wie sie zu ihrem Schlafplatz kommen soll, der ganz in der Nähe zu sein scheint. \Juvenil kenn den Ort auf jeden Fall, er ist wohl ca. 15 minuten zu Fuß entfernt. Sie bittet Ihn sie zu begleiten aber \Juvenil möchte nicht mit Ihr gehen sondern erklärt Ihr stattdessen den Weg. Zuerst ist sie sehr enttäuscht und bittet Ihn mehrere Male eindringlich, nach einiger Diskussion ist dann erstmal alles ok. Sie beginnt zu erzählen dass ihr Mann sie gerade \toMark{extrem verprügelt hat} und sie total verloren ist. Wir beide stehen nur da hören uns ihre Geschichte an, ansonsten sind wir ratlos und wissen nicht was wir tun könnten. \Juvenil versucht sie zu trösten und dann macht sie sich auf den Weg

Nachdem sie uns beide umarmte und schließlich verließ, erzählt \Juvenil das die beiden schon seit zehn Jahren befreundet sind. Damals ist sie mit Anfang zwanzig aus \aLocationE{parana} nach São Paulo gekommen, so wie er. Seitdem lebt sie auf der Straße, ist zum Crack Junkie geworden, und mit einem Mann verheiratet der sie regelmäßig verprügelt. \Juvenil kennt Ihren Schlafort gut, denn sie ist übernachtet dort seit jeher. Er kennt auch ihren Mann und befürchtete auf Ihn zu treffen wenn er sie dorthin begleiten würde. Für Ihn ist gegenseitige Hilfe auf der Straße wichtig und notwendig aber er zieht eine Grenze wenn er eigene Probleme bekommen könnte, da seine Situation für ihn schon prekär genug ist. 

Nach diesem traurigen Zusammentreffen gehen wir beide zu den in der Nähe stehenden \toMark{informellen Schnapsverkäufern} und teilen uns ein paar Pinnchen \toMark{cachaça}. \Juvenil läßt das eben erlebte nicht so recht los. Er macht sich sorgen was noch alles mit deiner Freundin passieren kann und entschließt sich morgen als erstes zu Ihr zu gehen um zu schauen wie es Ihr geht. Für \Juvenil gehört das Treffen von Freunden von der Straße zum Alltag, mit all diesen traurigen und schönen Momenten. 

\Juvenil bleibt an diesem Abend besorgt und erzählt noch wie er vor einigen Tagen auf dem Weg durch \aLocationE{crackolandia} ausgeraubt wurde. Einer der \toMark{Crack Junkies} die um \aLocationE{luz} die Straßen bewohnen verfolgte Ihn und zwang Ihn seine Jacke abzugeben. Danach sollte er auch noch seine Papiere rausrücken, was er aber nach viel Bitten nicht musste. \Juvenil meinte das der Verlust seiner Jacke nicht schlimm wäre im Vergleich zum Verlust seiner Papiere. Deren Verlust und Ersatz hätte viele Konsequenzen, bürokratische aber auch die Gefahr auf der Straße in eine Bullenkontrolle zu kommen. Außerdem hat \Juvenil mehrere Ausweise die Ihm zB. Zugang in die \toTranslate{casas de convievencia} gewähren und so einen notwendigen Bestandteil zum erfolgreichen täglichen Überleben darstellen.

Wir uns am nächsten Tag wieder. \Juvenil ist sehr besorgt denn er ist morgens seine Freundin besuchen gegangen und fand sie nicht dort wo sie gewöhnlich anzutreffen ist. Nun macht er sich große Vorwürfe am Tag davor nicht für eine halbe Stunde mit Ihr gegangen zu sein. Er macht sich Vorwürfe weil er wusste das es Ihr bei unserem Zusammentreffen richtig mies ging und er zu egoistisch war und nicht in Ihren Konflikt mit reingezogen werden wollte. Nun ist sie nicht da und er weiß ob etwas mit Ihr passiert. 

\Juvenil schmeißt deshalb all seine Pläne für die nächsten Tage über den Haufen und entschließt sich seine Freundin in der Stadt suchen zu gehen. Er will verschiedene Schlafplätze auf der Straße und \toTranslate{casas de convivencia} aufsuchen um nach Ihr zu fragen. 

\Juvenil verschwindet danach für drei Tage auf den Straßen São Paulos, zu Fuß auf der Suche nach seiner verschwundenen Freundin. 

\stopPersonal

\startPersonal

Die Situation im Zentrum war für mich ziemlich krass. Wir beide stehen dort und hören uns die Horrorgeschichte einer zerschlagenen Frau an, während um uns die Leute strömen und nur marginal von uns dreien Notiz nehmen. Vor allem die Hoffnungslosigkeit der Situation von \Juvenils Freundin hat mich traurig gemacht da ich auch nicht wusste wie wir Ihr helfen können. 

Auch die Geschwindigkeit und Plötzlichkeit mit der unsere vorher fröhliche Wanderung durch die Stadt innerhalb eines Sekundenbruchteils umschlug in hilfloses Dabeistehen und Erfahren einer Geschichte die sich überall in der Stadt zu jeder Zeit wiederholt und von der sonst keine Menschen mitbekommen hätten, war erschreckend. Dazu die Tatsache das \Juvenil in seiner eigenen, schon prekären Situation, noch die Kraft findet für mehrere Tage durch die Stadt auf die Suche nach einem Menschen zu gehen, ist für mich krass und zeugt für mich von einer total großen Wertschätzung der Leute untereinander, und einer anderen Wahrnehmung der Stadt und Ihrer Räume, in der es klar ist wo Menschen zu finden sind oder wo die Menschen der Straße untereinander Wissen wo sich die anderen Befinden oder potentiell aufhalten können. So wandelt sich eine Sache, die sonst eventuell nur einen Telefonanruf bedeutet, in eine Aufgabe die mehrere Tage in Anspruch nehmen kann, zusätzlich zum eigenen Überleben.

\stopPersonal

\subjectInformalStreetlife

\reference[narratinginquiries_saopaulodiaries_republica_streetlife]{}Almost every time we spent some time at \aLocationE{republica}, heavy police presence was 

\subjectSaturdays

\startPersonal
An einem Samstag treffe ich \Bob, \Juvenil und zwei anderen im \aLocationE{parkOcas} unter den Metro-Schienen der \isCorrect{gelben Linie}. \Juvenil \toMark{war betrunken und} hat als Erstes eine Runde mit meinem Fahrrad gedreht. Der Rest blieb im Park. In diesem \aLocation{parkOcas} wohnen einige Menschen, hier treffen sich Jugendliche oder wir um \toMark{Weed zu rauchen und} abzuhängen oder es wird Fußball auf dem \isCorrect{eingezäunten Sandplatz} gespielt.

Wir haben uns irgendwo an Tische gesetzt in deren Nähe 3 Jungs abhingen. Später kam ein Vierter dazu, der erst mit uns redete und später zu denen anderen ging. Die die öfter hier sind kennen sich oft und so begrüßen sie sich zuerst und tauschen Infos aus was in letzter Zeit passiert ist und ob alles in Ordnung ist, oder auch nicht.

Später rauchen die Vier \toMark{weed}, genauso wie wir. Während wir uns danach unterhalten kommt eine \toTranslate{pm} Patrouille die Straße entlang gefahren und hält in unserer Nähe. Es steigen zwei \isCorrect{Bullen} aus, gehen auf die vier Jungs zu und fordern sie ohne Zögern mit gezogenen Pistolen auf sich mit dem Gesicht zur Wand zu stellen. Alles passiert schnell und ohne Geschrei oder Gewalt. Bei den Vier wird \toMark{weed} gesucht, Personalien werden kontrolliert und mit der Basis über Funk abgechekt. Die \toTranslate{pm} fand nichts, verwies die Vier aber aus dem \aLocationE{parkOcas}. Die \isCorrect{bullen} warten danach noch einige Zeit, ohne dabei Anstalten zu machen uns auch noch zu kontrollieren, und fahren dann davon. Das alles dauerte ca. 20 Minuten. 
\stopPersonal

\startPersonal
Das Verhalten der Polizei kriminalisiert Menschen, oft Jugendliche, die sich in öffentlichen Plätzen treffen und verallgemeinert als potentielle Drogenkonsumenten betrachtet werden. Die Polizei verhält sich repressiv und demonstriert ihre Machtposition in dem sie ihre Kontrollen direkt unter Androhung von Waffengewalt durchführt. Interessant fanden wir dass vermutlich unser bunt gemixtes Aussehen, \Bob, dunkelhäutig mit Dreads und eher Reggaslastig gekleidet, \Carlinos, auch dunkelhäutig, auch in ne Reggaerichtung gehend und ich als Weißer in schwarzen abgerockten Klamotten und die anderen beiden eher Jung, in Bermudas, T-Shirt und \toTranslate{Flip-Flops} und unterschiedlichen \isCorrect{Alterserscheinungen} die \toTranslate{pm} nicht dazu veranlasst hat uns auch zu  kontrollieren, wir also nicht in das Bild der Gruppe von jugendlichen Drogenkonsumenten passten.
\stopPersonal

\startPersonal
Unsere Diskussion handelte derweil von folgendem: Wie kann die Situation auf der Straße beendet werden?

\Carlinos würde eine Kooperative gründen, mit 25000 Mitgliedern, mit all denen die auf der Straße leben. Jede\_r einzelne würde gefragt werden was er\_sie machen wollen würde. Die Kooperative würde den einzelnen Geld geben um die eignen Ideen umzusetzen, zB. wohnen, Kunst, Musik, etc. Er würde aber gleichzeitig \quote{einen teil der Methoden aus der Militär Diktatur} anwenden, also die Leute in den Knast stecken, die die  gegebenen Möglichkeiten missbrauchen. 

Weiter meint er das der \toTranslate{assistente social} und die Politiker von der Misere der Leute leben und keinerlei Interesse hätten etwas an der Situation der Straßenbewohner\_innen zu ändern, sprich mit den Menschen eine Möglichkeit zum Verlassen der Strasse zu überlegen und umzusetzen. Ein Beispiel sind die \toTranslate{tendas} in denen den Menschen nur Fernsehen und Essen angeboten würde, nur um sie abzulenken, also passiver Konsum statt aktive Änderung. Deshalb macht seiner Meinung nach eine Kooperation mit dem Staat wenig Sinn.

\Carlinos findet das der Glauben wichtig ist und das die Menschen eine gute Bildung benötigen die für alle zugänglich sein muss. Er sagt aber auch das nicht jede\_r von sich aus von der Strasse weg will oder kann. Das muss respektiert werden und in diesem Fall ist die Gesellschaft als Kollektiv verantwortlich diese Menschen nicht allein zu lassen. Das wäre seiner Meinung nach eine von vielen Möglichkeiten die Strasse zu verlassen, auch im Hinblick auf die für die Menschen in \aLocationE{carackolandia} (crackland), die schwer in selbstorganisation aus ihrer zerstörten Umwelt raus kommen werden.
\stopPersonal

\startPersonal
In \aLocationE{carackolandia} sammeln und ergänzen sich die krassesten Effekte des Ausschlusses von Menschen durch die Stadt an sich: \aLinkE{Crack}, Junkies aller Altersklassen, extrem organisierte \quote{Kriminalität} vom Dealer bis zu Laboratorien, das Leben auf der Straße, tägliche und massive  Sichtbarkeit von Polizei deren einziges Tun darin besteht die Menschen tagsüber öffentlichkeitswirksam durch die Gegend zu hetzten um große Ansammlungen zu verhindern oder aufzulösen. 

Abends ändert sich dann das Bild \aLocationE(crackolandias) komplett. \aLocationE{santaeffigenia}, eines der Viertel in der Gegend um \aLocationE{luz}, bei Tageslicht extrem wusselig mit Myriaden von Geschäften und Menschen auf den Straßen, einem Clustering von einzelnen Arten von Businessen pro Straße oder per Block, Fahrräder und Motorräder, Elektonikläden, Musikläden, Laptop und Handyshops, alles lässt sich finden und bestimmt tagsüber das Bild der Gegend. 

Abends, nach 7 Uhr, nachdem die meisten Läden geschlossen wurden, sammeln sich hunderte von Crackjunkies in den gleichen Straßen die wenige Stunden vorher von ganz andere Menschen genutzt wurde. Es wird offen und ohne Angst Crack verkauf, 3-5 Reais pro \toTranslate{crackpedra} (Crack Stein) und die Menschen schießen sich total ab.

Das ganze Treiben wird aus der Ferne von obligatorischen  {\GCM} Wägen und Bullen betrachtet, die aber meistens nicht einschreiten. 

\aLocationE{crackolandia} ist ein Ort den die Leute von der Strasse meiden sofern sie nicht Teil der Crackszene sind. Es gibt deswegen auch unter den Straßenbewohner\_innen eine klare Abgrenzungen und Aufteilung des urbanen Raums, mit unterschiedlichen Gruppen von Straßenbewohnern\_innen in unterschiedlichen Gegenden. \aLocationE{crackolandia} ist ein Beispiel von extremer \toMark{segregation}, \aLocationE{miocao}, \aLocationE{se} oder \aLocationE{republica} sind andere Beispiele.
\stopPersonal

\startPersonal
Zurück zu unserer Diskussion: Alle scheinen irgendwas in Richtung Bildung machen zu wollen. \Carlinos will Psychologie studieren und Musik machen, \Bob will Musik und anderes \quote{irgendwas} machen. 
\stopPersonal

\startPersonal
Einige Zeit nachdem die Bullen weg sind, kommt \Juvenil zurück, immer noch gut betrunken. Alle wollen zurück nach \aLocationE{ocas} gehen, nur er nicht. Deshalb machen wir uns zusammen auf den Fußweg zurück ins Zentrum. 

Im Zeitraffer sind das etwas folgende Punkte in der zentralen Gegend São Paulos: von \aLocationE{parkOcas} in Richtung \aLocationE{viadutocha}, darüber und an den \aLocationE{tendas} vorbei, über \aLocationE{se}, dann ins \aLocationE{refeitoriarederua} in \aLocationE{bixiga}, nahe der \aLocationE{ruaaugusta}, durchs Zentrum zurück zu \aLocationE{republica} und schließlich auf den Markt der dort jeden Samstag stattfindet. Dann trennen sich für diesen Tag unsere Wege. 

Während unseres Weges vom \aLocationE{parkOcas} wunderte sich \Juvenil das die {\PM} keinen der 4 Jungs die kontrolliert wurden, mitgenommen haben. Das sei sonst die Regel weil sie immer jemanden mitnehmen (müssen). Die {\PM} hält bei solchen Aktionen die Knarren immer direkt ins Gesicht der ihrer Ansicht nach \quote{Verdächtigen}. 

Außerdem war heute für \Juvenil Brasilien, São Paulo, die Paulistas, die Politiker und die Polizei eine einzige große Scheisse. \quote{Os paulistas podem se fuder, fuder mesmo e o resto do povo também}. \Juvenil meint das sich São Paulo nie ändern würde, höchstens immer schlechter werden. 

Er erzählte das er von einem Organisator seine \aLinkE{sindicato} raus geschmissen wurde, was wohl der Grund für sein heutiges Saufen und seine Wut sein könnte. Seiner Meinung nach erzählen einige der anderen viel, machen aber letztendlich nix weil sie alle Drogen nehmen, Maconha (Weed), Cachaça (Zuckerrohr-Schnaps), usw.  

Er will einen Kurs machen für den er eine Prüfung im Oktober bestehen muss. Er spricht viel über \toMark{Inclusão Digital}, die unter anderem notwendig ist um sich zB. für einen Kurs anzumelden, einen Lebenslauf zu schicken oder den Unterrichtsstoff aus dem Internet zu bekommen. Das erlernen der Tools (er benutzt Orkut, Facebook, Yaho und Googlemail) hat lange gedauert, etwas, das heute seiner Meinung nach viel einfacher für die ist die mit \toMark{NM/DK} aufwachsen. 

Für ihn ist das Internet wichtig (obwohl er auch schon mal seine ganzen privaten Daten angibt um an einer Umfrage teilzunehmen). Da er keinen eigenen Computer besitzt sucht und kennt er alle Möglichkeiten Computer und Internet umsonst zu benutzen. Einige Orte wären zB die \aLocationE{galeriaolida}, \aLocationE{ocas}, \aLocationE{ay}, \aLocationE{metrose}, \aLocationE{culturalbancobrasil?} 

.... in der \aLocationE{galeriaolida} zB. muss er immer seine Zeit im Voraus ausmachen und dann pünktlich sein da er pro Tag nur eine Stunde an den Computern sein darf (mit 10 Minuten Tolleranz). Er sagte das eine jährliche Vollmitgliedschaft in der \aLocationE{galeriaolida} 170R kosten würde, die er sich nicht leisten kann. In der \aLocationE{galeriaolida} ist er meistens Samstags und Montags Abends, ansonst \aLocationE{ocas} ....

Nach dem \aLocationE{viadutocha} gehen wir auf den \aLocationE{se} und treffen dort \Gibson der uns für den nächsten Tag, ein Sonntag, zum Essen ins \aLocationE{refeitoriarederua}, wohin wir gerade schon unterwegs sind, einlädt. \Gibson war zusammen mit einem älteren, ziemlich betrunkenen, Freund und einem Jugendlichen, höchstens 18 Jahre, schon mit vielen Anzeichen beginnender Zerstörung. 

Auf der anderen Seite von \aLocationE{se} trafen wir einen weiteren Bekannten von der Straße, total zerstört. Er konnte nicht richtig reden und gab nur Laute von sich. \Juvenil meinte später, dass das die Realität auf der Strasse ist, die die Menschen alle verändert. Aber trotzdem seien sie alle Menschen und dürfen nicht auf der Seite liegen gelassen werden. Auch wenn es nicht möglich ist mit einigen von Ihnen zu reden, muss es immer versucht werden, und das macht \Juvenil immer.

Auf dem Weg von \aLocationE{se} Richtung \aLocationE{refeitoriarederua} erzählt \Juvenil, dass er alles was in der Stadt zum Überleben notwendig ist, am Anfang, als er vor 16 Jahren als Immigrant aus \aLocationE{parana} nach São Paulo kam, gelernt hat. In einer seiner Erinnerungen suchte er zu jener Zeit einmal einen Ort im Zentrum und fragte die Bullen die er unterwegs traf. Die schickten ihn aber nach \aLocationE{Tucuruvi} in der nördlichen Zone (Zona Norte) der Stadt. Mit dem Bus und seinen letzten Reais investiert in die Fahrt kam er dort an, fragte rum, aber niemand kannte den Ort den er suchte, bis eine Frau im sagte das er ins Zentrum zurück müsse da sich dort der Ort befindet den er suchte. Die Bullen haben ihn also absichtlich in die falsche Richtung geschickt obwohl er sich schon ganz in der Nähe befand als er die Bullen fragte.
\stopPersonal

\startPersonal
Wir kommen \aLocationE{refeitoriarederua} spät an. Das Essen wird drinnen schon verteilt und alle Menschen haben sich schon einen Platz gesucht. \Juvenil hatte fetten Hunger weil er gestern und heute noch nichts gegessen hatte, auch eine der täglichen Erfahrungen die er als Arbeitsloser der in einer \aLinkE{albergue} (statt auf der Straße) schläft macht. 

Er meinte das die Menschen in den \aLinkE{albergues} um 5 Uhr morgens aufstehen müssen, auf einen ersten Kaffee warten und dann alle auf die Straßen verschwinden, viele als Ziel die \toMark{nucleosconvivencia}. Dort warten sie dann auf Frühstück oder Mittagessen. Danach bis zum Abend warten, wenn sie wieder in die \aLinkE{albergues} reingelassen werden.  

das \aLocationE{refeitoriarederua} ist in \aLocationE{bixiga}, in der Nähe von \aLocationE{ruaaugusta} und war schon voll mit Leuten, fast alle Tische besetzt. Samstags und an den Feiertagen gibt es Saft und ein Brötchen, heute mit Wurst und Käse. \Juvenil meinte das sich die Leute sogar darum manchmal streiten und das viele Menschen die dorthin kommen nicht auf der Strasse leben, sondern auch in \aLinkE{albergues}. 

Die Stimmung drinnen war mehr als gedrückt, nur wenige Menschen redeten miteinander, der überwiegende teil Männer, nur ein oder zwei Frauen. Einer der Mitarbeiter war cool und nett zu uns. \Juvenil wollte zuerst Wasser trinken, aber nicht das gefilterte sondern aus der Leitung :D 

Nachdem Essen sind wir schnell wieder verschwunden, Zigarettchen rauchend Richtung \aLocationE{republica} an einer \aLinkE{Besetzung} auf \aLocationE{ruamartinsfontes}  vorbei, der Verlängerung von \aLocationE{ruaaugusta}. Unterwegs treffen wir immer wieder Leute die \Juvenil kennt. Auf der \aLocationE{25demaio } haben wir mit einem der \toTranslate{ouro} Verkäufer über Fußball gequatscht. \Juvenil meinte das die Paulistanos immer denken das sie die besten sind, und der Rest \toTranslate{bichos}. Und in Rio denken sie genauso. Das eigene Team ist immer am besten, der Rest sind nur \toTranslate{bichos}. 

Am \aLocationE{republica} angekommen sind wir über den Markt den es dort immer Samstags gibt, gegangen und haben etwas gegessen, er \isCorrect{tamora}, ich ein Stück Kuchen. Dort haben wir mit einem der \toMark{securities} geredet, der eigentlich aus Rio kam. Der erzählte irgendwas von einem reichen deutschen Typen der eine Brasilianerin geheiratet hat und hier ein fettes Business mit Holz oder so, am Start hätte. \Juvenil hat seine \isCorrect{tamora} aufgehoben, für später, ich hab meinen leckeren Kuchen aber direkt aufgegessen und trennten sich unsere Wege für diesen Tag. Ein trauriger und schneller Tag.

Zum Schluss meinte \Juvenil noch das er jederzeit ein Interview mit mir machen würde, weil es seiner Meinung nach immer etwas bringt, bzw. hilft und einen Nutzen hat.
\stopPersonal

%---------------------------------------------------
% gets only displayed in unfinshed mode

\showImperfection

%---------------------------------------------------

\stopmode

\stoptext

\stopcomponent